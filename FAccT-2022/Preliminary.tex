% !TEX root =main.tex







\section{Preliminaries} \label{preliminaries}


\subsection{Informal Thread Model and Assumptions}\label{Notations-and-Assumptions}

The PwDR scheme consists of six types of parties. Below, we informally explain each type of party's role and the security assumption we make about each of them. We will provide a formal definition of the  PwDR scheme in Section \ref{sec::def}. 
%
\begin{itemize}
%
\item[$\bullet$] Customer $\mathcal{C}$: it is a regular customer of a bank. We call a customer  a victim after it falls victim to an APP fraud. We assume a victim is corrupted by a non-colluding active (or malicious) adversary. %; for instance, an unqualified victim  tries to make itself appear qualified for reimbursement. 
%
\item[$\bullet$] Bank $\mathcal{B}$: it is a regular bank that provides a standard online banking system. We assume it is corrupted by a non-colluding active adversary. We assume  any change to the source code of the online banking system is transparent and  can be detected. 

%Note that in the real-world a bank is  not usually an active adversary and cares about its reputation, such as a ``rational'' or ``covert'' adversary which is weaker than the active one. However, to ensure our solution offers  a strong security guarantee, we assume the adversary is strong too, i.e., active one. 

%, e.g., the bank  uses a cryptographic commitment to commit to the source code. 
%
\item[$\bullet$] Smart contract $\mathcal{S}$: it is a standard  smart contract of a public  blockchain (e.g., Ethereum). It mainly acts as a tamper-proof public bulletin board to store different parties' messages.  We do not assume that a smart contract itself can offer any privacy. 
%
\item[$\bullet$] Certificate generator $\mathcal{G}$: it is a trusted third party (e.g., hospital, registry office) which provides signed digital certificates (e.g., certificate of   divorce, disability) to customers. %Its involvement is more implicit than the other  parties.
%
\item[$\bullet$]  A committee of arbiters $\{\mathcal{D}_{\st 1},..., \mathcal{D}_{\st n}\}$: it consists of  trusted third-party authorities, auditors, or regulators (e.g.,  financial conduct authority, financial ombudsman service). Given a set of complaints, they compile the complaints    and provide  their binary verdicts. If needed, they are authorised to access the banking  system's backend software to carry out investigations. We assume all arbiters have interacted with each other once,  to agree on (a) a secret key, $\bar k_{\st 0}$, and (b) a pair of keys $({pk}_{\st\mathcal {D}}, {sk}_{\st\mathcal {D}})$  of an asymmetric key encryption.
%
\item[$\bullet$]  Dispute resolver $\mathcal{DR}$: it is an aggregator of arbiters' votes (e.g., public court). Given a collection of votes, it extracts and announces the final verdict. We assume it is corrupted by a non-colluding passive adversary. We assume $\mathcal C$ and $\mathcal B$  use a secure channel when they  send a message directly to $\mathcal{DR}$. 
%
\end{itemize}



\subsection{Notations}
We use $\mathtt{Enc}(.)$  and $\mathtt{Dec}(.)$ to denote the encrypting and decrypting algorithms of   a semantically-secure symmetric key encryption scheme respectively. We also use  ${ \tilde {\mathtt{Enc}}}({pk}_{\st\mathcal {D}}, .)$ and   ${\tilde{\mathtt {Dec}}}({sk}_{\st\mathcal {D}}, .)$ to denote the encrypting and decrypting algorithms of   a semantically secure asymmetric key encryption scheme  which has  the following key generating algorithm:  $\tilde{\mathtt{keyGen}}(1^{\st\lambda})\rightarrow({sk}_{\st\mathcal {D}}, {pk}_{\st\mathcal {D}})$ and its  public key, ${pk}_{\st\mathcal {D}}$, is known to everyone.  We denote the banking system's internal payment  algorithm by  $\mathtt{pay}(.)$.  We use $in_{\st p}$ to denote the inputs of this algorithm.  We also use $\phi$ to denote a null value. In Appendix \ref{sec:notation-table}, we provide a notation table. 


%We use $\mathtt{Enc}(.)$  and $\mathtt{Dec}(.)$ to denote the encrypting and decrypting algorithms of   a semantically secure symmetric key encryption scheme. In this work, we use a committee of honest arbiters $\mathcal{D}:\{\mathcal{D}_{\st 1},..., \mathcal{D}_{\st n}\}$. Each arbiter,  given a set of  inputs, provides a binary verdict.    We  assume  $\mathcal{D}_{\st i}$'s share a pair of keys $({pk}_{\st\mathcal {D}}, {sk}_{\st\mathcal {D}})$ of an asymmetric key encryption. The encryption scheme has  key generating  $\tilde{\mathtt{keyGen}}(1^{\st\lambda})\rightarrow({sk}_{\st\mathcal {D}}, {pk}_{\st\mathcal {D}})$,   encrypting $\mathtt{ \tilde {Enc}}({pk}_{\st\mathcal {D}}, .)$ and  decrypting $\mathtt{\tilde{Dec}}({sk}_{\st\mathcal {D}}, .)$ algorithms, where its public key is known to everyone.  Also, we assume all arbiters have interacted with each other to agree on a secret key, $\bar k_{\st 0}$.  In this work, a third party dispute resolver, $\mathcal{DR}$, is involved, which can be potentially semi-honest, i.e., a passive adversary. We assume bank and customer are potentially malicious, i.e., active adversary. We use $\phi$ to denote a null value.  Our proposed solution is built upon the existing online banking system. We assume the banking system has  algorithm  $\mathtt{pay}(.)$ that  transfers money from the customer's account to a payee's account that is specific by the customer.  We denote the inputs of this algorithm $in_{\st p}$. We assume the source code of the online banking system is static, and any change to the source code is transparent and  can be detected, e.g., the bank  uses a cryptographic commitment to commit to the source code. Moreover, we assume the online banking  system is secure. In Appendix \ref{sec:notation-table}, we provide a notation table. 





% Similar to the \emph{optimistic} fair cryptographic protocols that aim  at efficiency, e.g., in \cite{AsokanSW97,eurocrypt/AsokanSW98,BaoDM98,DongCCR13}, we assume the existence of a trusted third party arbiter   which remains offline most of the time but can be invoked to resolve a part of dispute. We emphasise that if the bank and customers behave honestly, then the arbiter is never involved. Even the (offline) presence of such an arbiter threatens adversarial behaviour and acts as a deterrence.  %The idea is akin to deterrence in criminology,  i.e.,  the threat of punishment will deter people from committing crimes.














\subsection{Digital Signature}\label{subsec:DS}

A digital signature is a scheme for verifying the authenticity of digital messages or documents. Below, we restate its formal definition, taken from \cite{DBLP:books/crc/KatzLindell2014}. 


\begin{definition}\label{sec::def}
A signature scheme  involves three algorithms, $\mathtt{Signature}:=(\mathtt{Sig.keyGen}, $ $\mathtt{Sig.sign}, \mathtt{Sig.ver})$, that are defined as follows.

\begin{itemize} 
\item[$\bullet$] $\mathtt{Sig.keyGen}(1^{\st \lambda})\rightarrow (sk,pk)$.  A probabilistic algorithm run by  a  signer. It takes as input a security parameter. It outputs a key pair: $(sk,pk)$, consisting of secret: $sk$, and public: $pk$ keys. 
\item[$\bullet$] $\mathtt{Sig.sign}(sk, pk, u)\rightarrow sig$. An algorithm run by the signer. It takes as input  key pair: $(sk,pk)$ and a message: $u$. It outputs a signature: $sig$.
\item[$\bullet$]  $\mathtt{Sig.ver}( pk, u, sig)\rightarrow h\in\{0,1\}$. A deterministic algorithm run by a verifier. It takes as input  public key: $pk$,  message: $u$, and signature: $sig$. It checks the signature's validity.   If the verification passes, then it outputs $1$; otherwise, it outputs $0$. 
\end{itemize}
\end{definition}

A digital signature scheme should meet the following properties:

\begin{itemize} 
\item[$\bullet$]  \textit{Correctness.} For every input $u$ it holds that:
%
$$Pr\Big[\  \  \mathtt{Sig.ver}( pk, u, \mathtt{Sig.sign}(sk, pk, u))=1\ : \
\mathtt{Sig.keyGen}(1^{\st \lambda})\rightarrow (sk, pk)  \Big]=1$$
%
\item[$\bullet$] \textit{Existential unforgeability under chosen message attacks.} A probabilistic polynomial time (PPT) adversary that obtains $pk$ and has access to a signing  oracle for messages of its choice, cannot create a valid pair $(u^{\st *},sig^{\st *})$ for a new message $u^{\st *}$  (that was never a query to the signing oracle), except with a small probability, $\sigma$. More formally: 




\small{
$$ \Pr\left[
  \begin{array}{l}
  
  u^{\st *}\not\in Q\ \wedge \\
   \mathtt{Sig.ver}( pk,  u^{\st *}, sig^{\st *}) =1\\
  
  
    
%(M(u^{\scriptscriptstyle *},k)\neq \sigma \lor Q(\text{aux},k)\neq q) \wedge\\ (a=1 \ \vee b=1)
\end{array} : 
    \begin{array}{l}
   
    \mathtt{Cer.keyGen}(1^{\st \lambda})\rightarrow (sk,pk) \\
  \mathcal{A}^{\mathtt{Sig.sign}(k,)}(pk)\rightarrow(u^{\st *}, sig^{\st *})

     
\end{array}    \right]\leq \mu(\lambda)$$
}
where $Q$ is the set of queries that $\mathcal{A}$ sent to the certificate generator oracle.
\end{itemize}



%
%An application of a digital signature is in digital certificate, which is a  document digitally signed  by a certificate generator. Given a certificate and its parameters,  anyone can check whether it has been correctly generated by a valid generator. There is a case  where 
%a \emph{hard copy} certificate is used.  In this case,  the process $\mathtt{Sig.keyGen}(.)$ outputs a blank legitimate stamped certificate as a private parameter: $sk$, and the description of a standard legitimate certificate as a public parameter: $pk$. Moreover, the process $\mathtt{Sig.sign}(k, u)$ takes $k$ and the file $u$ on which a certificate should be generated and outputs a stamped certificate with the information printed on it. The process $\mathtt{Sig.ver}( pk, u, sig)$ takes the public parameter, the file,  and the hard copy of the certificate and outputs $1$ if it is valid and $0$ if it is not. Note, when  a hard copy certificate is considered, it is not possible to precisely define the success probability of the adversary, as it depends on the technology available to the adversary to generate a blank stamped certificate that looks like a legitimate one. In the real world however, this probability is usually small (but it may not be negligible). In this paper, we mainly consider a digital certificate; however, our solution can adopt hard copy certificates as well with the above caveat  regarding the adversary's success probability. 



\subsection{Smart Contract}\label{subsec:SC} Cryptocurrencies, such as Bitcoin \cite{bitcoin} and Ethereum \cite{ethereum}, beyond offering a decentralised currency,  support  computations on  transactions. In this setting, often a certain computation logic is encoded in a computer program, called a \emph{``smart contract''}. Although Bitcoin, the first decentralised cryptocurrency, supports smart contracts, the functionality of Bitcoin's smart contracts is very limited, due to the use of the underlying programming language that does not support arbitrary tasks. To address this limitation, Ethereum, as a generic smart contract platform, was designed. Thus far, Ethereum has been the most predominant cryptocurrency framework that lets users define arbitrary smart. This framework allows users to create an account with a unique account number or address. Such users are often called external account holders, which can send (or deploy) their contracts to the framework’s blockchain. In this framework, a contract's code and its related data  are held by every node in the blockchain's network. Ethereum smart contracts are often written in a high-level Turing-complete programming language called ``Solidity''.The program execution's  correctness  is  guaranteed by the security of the underlying blockchain components. To prevent  a denial of service attack, the framework requires a transaction creator to pay a  fee, called \emph{``gas''}. %, depending on the complexity of the contract running on  it.  


% !TEX root =main.tex

\vspace{-2.2mm}


\subsection{Commitment Scheme}\label{subsec:commit}


A commitment scheme involves two parties,  \emph{sender} and  \emph{receiver}, and includes  two phases: \emph{commit} and  \emph{open}. In the commit phase, the sender  commits to a message: $x$ as $\mathtt{Com}(x,r)=\mathtt{Com}_{\scriptscriptstyle x}$, that involves a secret value,  $r$. In the open phase, the sender sends the opening $\ddot{x}:=(x,r)$ to the receiver which verifies its correctness: $\mathtt{Ver}(\mathtt{Com}_{\scriptscriptstyle x},\ddot{x})\stackrel{\scriptscriptstyle ?}=1$ and accepts if the output is $1$. Informally, a commitment scheme must satisfy two properties, (a) \textit{hiding}: it is infeasible for an adversary to learn any information about the committed  message, and (b) \textit{binding}: it is infeasible for an adversary to open a commitment  to different values  than the one  used in the commit phase. We provide more detail in Appendix \ref{subsec:commit-long}. 



% A commitment scheme involves two parties,  \emph{sender} and  \emph{receiver}, and includes  two phases: \emph{commit} and  \emph{open}. In the commit phase, the sender  commits to a message: $x$ as $\mathtt{Com}(x,r)=\mathtt{Com}_{\st x}$, that involves a secret value: $r\stackrel{\st\$}\leftarrow \{0,1\}^{\st\lambda}$. In the end of the commit phase,  the commitment $\mathtt{Com}_{\st x}$ is sent to the receiver. In the open phase, the sender sends the opening $\ddot{x}:=(x,r)$ to the receiver who verifies its correctness: $\mathtt{Ver}(\mathtt{Com}_{\st x},\ddot{x})\stackrel{\st ?}=1$ and accepts if the output is $1$.  A commitment scheme must meet two properties: (a) \textit{hiding}: it is infeasible for an adversary  to learn any information about the committed  message $x$, until the commitment $\mathtt{Com}_{\st x}$ is opened, and (b) \textit{binding}: it is infeasible for an adversary  to open a commitment $\mathtt{Com}_{\st x}$ to different values $\ddot{x}':=(x',r')$ than that was  used in the commit phase, i.e.,  to find  $\ddot{x}'$, \textit{s.t.} $\mathtt{Ver}(\mathtt{Com}_{\st x},\ddot{x})=\mathtt{Ver}(\mathtt{Com}_{\st x},\ddot{x}')=1$, where $\ddot{x}\neq \ddot{x}'$.  There exist efficient non-interactive  commitment schemes both in (a) the standard model, e.g., Pedersen scheme \cite{Pedersen91}, and (b)  the random oracle model using the well-known hash-based scheme such that committing  is : $\mathtt{H}(x||r)=\mathtt{Com}_{\st x}$ and $\mathtt{Ver}(\mathtt{Com}_{\st x},\ddot{x})$ requires checking: $\mathtt{H}(x||r)\stackrel{\st ?}=\mathtt{Com}_{\st x}$, where $\mathtt{H}:\{0,1\}^{\st *}\rightarrow \{0,1\}^{\st\lambda}$ is a collision resistant hash function; i.e., the probability to find $x$ and $x'$ such that $\mathtt{H}(x)=\mathtt{H}(x')$ is negligible in the security parameter, $\lambda$.
% !TEX root =main.tex



%\vspace{-3mm}
\subsection{Statement Agreement Protocol}\label{SAP}

Recently, a scheme called ``Statement Agreement Protocol'' (SAP)  has been proposed in \cite{cryptoeprint:2021:1145}. It   lets two mutually distrusted parties, e.g., $\mathcal{B}$ and $\mathcal{C}$, \emph{efficiently} agree on a private statement, $\pi$. Informally, the SAP  satisfies the following four properties: (1) neither party can convince a third-party verifier that it has agreed with its counter-party on a different statement than the one both parties previously agreed on, (2) after they agree on a statement,  an honest party can (almost) always prove to the verifier that it has the agreement, (3) the privacy of the statement is preserved (from the public), and (4) after both parties reach an agreement, neither can deny it.
%
%(1) neither party can convince  a third-party  verifier that it has agreed with its counter-party on a different statement than the one both parties previously agreed on, (2) after they  agree on a statement,  an honest party can (almost) always  prove to the verifier that it has the agreement, (3) the privacy of the statement is preserved (from the public), and (4) after both parties reach an agreement, neither can  deny it.  
%
The SAP uses a  smart contract and commitment scheme. It  assumes that each party  has a blockchain public address,  $adr_{\st\mathcal{R}}$ (where $\mathcal{R}\in\{\mathcal{B,C}\}$). Below, we restate  the  SAP, taken from \cite{cryptoeprint:2021:1145}. 


%
 \begin{enumerate}
 %
 \item\textbf{Initiate}. $\mathtt{SAP.init}(1^{\st\lambda}, adr_{\st\mathcal{B}}, adr_{\st\mathcal{C}}, \pi )$ 

 The following steps are taken   by  $\mathcal B$.
 
  \begin{enumerate}
  \item\label{SAP::deploy-contract}  Deploys a smart contract that  explicitly states both parties'  addresses, $adr_{\st\mathcal{B}}$ and  $adr_{\st\mathcal{C}}$. Let $adr_{\st\text{SAP}}$ be the deployed contract's address. 

   \item  Picks a random value $r$, and commits to the statement, $\mathtt{Com}(\pi, r)=g_{\st \mathcal{B}}$.
   \item Sends $adr_{\st\text{SAP}}$ and $\ddot{\pi}:=(\pi, r)$  to  $\mathcal C$, and $g_{\st\mathcal B}$ to the contract. 
   %\item Sends $g_{\st\mathcal C}$ to the contract, using its account. 
    \end{enumerate}
     
     \vspace{1mm}
     
    \item\textbf{Agreement}. $\mathtt{SAP.agree}(\pi, r, g_{\st \mathcal{B}}, adr_{\st\mathcal{B}}, adr_{\st\text{SAP}})$

     The following steps are taken   by  $\mathcal C$.
     %
     \begin{enumerate}
 %
   \item Checks  if $g_{\st \mathcal{B}}$ was  sent  from $adr_{\st \mathcal{B}}$, and checks locally $\mathtt{Ver}(g_{\st\mathcal B}, \ddot{\pi})=1$.
   %
   \item If the checks pass, it sets $b=1$,    computes locally $\mathtt{Com}(\pi, r)=g_{\st\mathcal C}$, and sends $g_{\st\mathcal C}$ to the contract. Otherwise, it sets $b=0$ and $g_{\st\mathcal C}=\bot$.
 %
   %\item  If $b=1$, then sends $g_{\st\mathcal S}$ to the contract. % using its account.
    \end{enumerate}
     
          \vspace{1mm}
     
   \item\textbf{Prove}. For either $\mathcal B$ or $\mathcal C$ to prove, it sends $\ddot{\pi}:=(\pi, r)$  to the smart contract. 
   
        \vspace{1mm}
   
 \item\textbf{Verify}. $\mathtt{SAP.verify}(\ddot{\pi}, g_{\st\mathcal B},g_{\st\mathcal C}, adr_{\st\mathcal{B}}, adr_{\st\mathcal{C}})$
 
 
 The following steps are taken   by  the smart contract.
   \begin{enumerate}
   
\item\label{SAP::check-adr} Ensures $g_{\st\mathcal B}$ and $g_{\st\mathcal C}$ were sent from   $adr_{\st \mathcal{B}}$ and  $adr_{\st \mathcal{C}}$  respectively. 
  
   \item\label{SAP::check-commit} Ensures $\mathtt{Ver}(g_{\st\mathcal B},\ddot{\pi})=\mathtt{Ver}(g_{\st\mathcal C},\ddot{\pi}) =1$.
   
   \item Outputs $s=1$, if the checks, in steps \ref{SAP::check-adr} and \ref{SAP::check-commit}, pass. It outputs $s=0$, otherwise.
    \end{enumerate}
 \end{enumerate}

  
  
  
  
%  \item\textbf{Agreement}.
%  \begin{enumerate}
%   \item $\mathcal S$ picks a random value: $r$, and commits to the statement: $\mathtt{H}(x||r)=g_{\st S}$
%   \item $\mathcal S$ sends $r$  to the client and sends $g_{\st\mathcal S}$ to the contract. 
%   \item $\mathcal C$ checks: $\mathtt{H}(x||r)\stackrel{?}=g_{\st \mathcal S}$. If the equation  holds, it computes $\mathtt{H}(x||r)=g_{\st\mathcal C}$
%   \item $\mathcal C$   stores $g_{\st\mathcal C}$ in the contract. 
%    \end{enumerate}
%   \item\textbf{Prove}. For either $\mathcal C$ or $\mathcal S$ to prove, it has agreement on $x$ with its counter-party, it sends $\mu=(x, r)$  to the contract. 
% \item\textbf{Verify}. Given $\mu$, the contract does the following. 
%   \begin{enumerate}
%
%   \item checks if $\mathtt{H}(x||r)=g_{\st\mathcal C}=g_{\st\mathcal S}$
%   \item outputs $1$, if the above equation holds; otherwise, it outputs $0$
%    \end{enumerate}
% \end{enumerate}



%
% \begin{enumerate}
% \item\textbf{Setup}.  Both parties agree on a  smart contract and deploy it, such that the parties public keys, $pk_{\st C}$ and $pk_{\st S}$, are encoded in the contract.
%
%  
%  \item\textbf{Agreement}.
%  \begin{enumerate}
%   \item The server picks a random value: $r$, and commits to the statement: $H(s||r)=y_{\st S}$.
%   \item The server sends $r$  to the client and sends $y_{\st S}$ to the contract. 
%   \item The client checks: $H(s||r)\stackrel{?}=y_{\st S}$. If the equation  holds, it computes $H(s||r)=y_{\st C}$.
%   \item The client   stores $y_{\st C}$ in the contract. 
%    \end{enumerate}
%   \item\textbf{Prove}. For either $C$ or $S$ to prove, it has agreement on $s$ with its counter-party, it sends $\mu=(s, r)$, in a signed transaction, to the contract. 
% \item\textbf{Verify}. Given $\mu$, the contract does the following. 
%   \begin{enumerate}
%   \item verifies the public keys related to  signatures of $y_{\st C}$ and $y_{\st S}$ match $pk_{\st C}$ and $pk_{ \st S}$ respectively.
%   \item checks if $H(s||r)=y_{\st C}=y_{\st S}$.
%   \item outputs 1, if the above equation holds; otherwise, it outputs 0.
%    \end{enumerate}
% \end{enumerate}
 
 %Note that the above protocol is one-off, which means after first party
 
 
 
 
 %In Appendix \ref{sec:Discussion-on-the-SAP}, we discuss the SAP's security and explain why naive solutions are not suitable.
 

 
 
 
 %The SAP protocol might be of independent interest. 
 
 
% \begin{remark}
% The verification algorithm can also be executed \emph{off-chain}. In particular, given  statement $\ddot{x}$, anyone can read $(g_{\st\mathcal C},g_{\st\mathcal S},adr_{\st\mathcal{C}}, adr_{\st\mathcal{S}})$ from the SAP contract and locally run $\mathtt{SAP.verify}(\ddot{x}, g_{\st\mathcal C},g_{\st\mathcal S},adr_{\st\mathcal{C}}, adr_{\st\mathcal{S}})$ to check the statement's correctness.  This relieves  the verifier from the  transaction  and contract's execution costs. 
% \end{remark}
% 
%  \begin{remark}
%One may simply let each party  sign the statement and send it to the other party, so later on each party can send both signatures to the contract which verifies them. However, this would not work,  as the party who first receives the other party's signature  may refuse  to send its  signature, that prevents the other party from proving that it has  agreed on the statement with its counter-party. Alternatively, one may want to use a protocol for a fair exchange of digital signature (or fair contract signing) such as \cite{BonehN00,DBLP:conf/fc/GarayJ02}. In this case, after both parties have the other party's signature, they can sign the statement themselves and send the two signatures to the contract which first checks the validity of both  signatures. Although this satisfies the above security requirements, it yields two main efficiency and practical issues: (a) it imposes very high computation costs, as  protocols for a fair exchange of signature involve generic zero-knowledge proofs and require a high number of modular exponentiations. And (b) it is impractical because protocols for fair exchange of signature protocol support only certain signature schemes (e.g., RSA, Rabin, or Schnorr) that are not directly supported by the most predominant  smart contract framework,  Ethereum, that only supports  Elliptic Curve Digital Signature Algorithm (EDCSA).
% \end{remark}







\subsection{Pseudorandom Function}


Informally, a pseudorandom function is a deterministic function that takes a key of length $\Lambda$ and an input; and outputs a value  indistinguishable from that of  a truly random function.  In this paper, we use the pseudorandom function:   $\mathtt {PRF}: \{0,1\}^{\st \Lambda}\times \{0,1\}^{\st *} \rightarrow  \mathbb{F}_p$, where $p$ is a large prime number, $|p|=\lambda$, and $(\Lambda,\lambda)$ are the security parameters. In practice, a pseudorandom function can be obtained from an efficient block cipher. We refer readers to \cite{DBLP:books/crc/KatzLindell2014} for a formal definition of a pseudorandom function.


\subsection{Bloom Filter}


A Bloom filter \cite{DBLP:journals/cacm/Bloom70} is a compact data structure that allows us to 
efficiently check an  element membership. It is an array of $m$ bits (initially all set to zero), that  represents $n$  elements. It is accompanied by $k$ independent hash functions. To insert an element, all the  hash values of the element are computed and their corresponding bits in the filter are set to $1$. To check an element membership, all its hash values are re-computed and checked whether all are set to $1$ in the filter. If all the corresponding bits are $1$, then the element is probably in the filter; otherwise, it is not. In Bloom filters,  it is possible that an element is not in the set, but the membership query indicates it is, i.e., false positives. In this work, we ensure that the false positive probability is negligible, e.g.,  $2^{\st - 40}$. Also, we require that a Bloom filter uses \emph{cryptographic} hash functions. In Appendix \ref{sec::bloom-filter-}, we explain how the Bloom filter's parameters can be set.

%Informally, a digital certificate's security requires that no one can generate a valid certificate that was not previously produced by the certificate generator. The security of a digital certificate relies on the security of the digital signature scheme used.  Below, we present the formal definition of the digital signature.



%In the case where a digital certificate is considered, then $\mathtt{Cer}$ definition is equivalent to the definition of a digital signature scheme. In this case, it holds $\sigma=neg(\lambda)$, where $neg$ is a negligible function and $\lambda$ is a security parameter. Now, we briefly explain the procedures' input/output when a hard copy certificate is considered.  The process $\mathtt{Cer.genPar}(.)$ outputs a blank legitimate stamped certificate as a private parameter: $sk$, and the description of a standard legitimate certificate as a public parameter: $pk$. Morevoer, the process $\mathtt{Cer.genCrt}(k, u)$ takes $k$ and the file $u$ on which a certificate should be generated and outputs a stamped certificate with the information printed on it. The process $\mathtt{Cer.verCrt}( pk, u, crt)$ takes the public parameter, the file,  and the hard copy of the certificate and outputs $1$ if it is valid and $0$ if it is not. In the case where a hard copy certificate is considered, it is not possible to precisely define the probability $\sigma$, as it depends on the technology available to the adversary to generate a blank stamped certificate that looks like a legitimate one. In the real world however, this probability is usually small (but it may not be negligible).




%\input{PoR-def}

 

%% !TEX root =main.tex

%\break

\clearpage

\section{Notations}\label{sec:notation-table}

We summarise our notations in Table \ref{table:notation-table}.


\begin{table*}[!htbp]
\begin{scriptsize}
\begin{center}
\footnotesize{
\caption{ \small{Notation Table}.}\label{commu-breakdown-party} 
\renewcommand{\arraystretch}{1}
\scalebox{1.1}{
% 1st table
\begin{tabular}{|c|c|c|c|c|c|c|c|c|c|c|c|c|c|} 

\hline 

\cellcolor{gray!15} \scriptsize \textbf{Symbol}&\cellcolor{gray!15} \scriptsize \textbf{Description}  \\
    \hline
    
     \hline

%Generic
%\multirow{25}{*}{\rotatebox[origin=c]{90}{\scriptsize \textbf{Generic}}}

 \cellcolor{white!20}\scriptsize$\mathtt{Enc}(.)$&\cellcolor{white!20}\scriptsize \text{Encryption algorithm of symmetric key encryption  }\\   
 %
  \cellcolor{gray!20}\scriptsize$\mathtt{Dec}(.)$&\cellcolor{gray!20}\scriptsize \text{Decryption algorithm of symmetric key encryption  }\\   
  %
  \cellcolor{white!20}\scriptsize${\tilde{\mathtt{Enc}}}(.)$&\cellcolor{white!20}\scriptsize \text{Encryption algorithm of asymmetric key encryption  }\\   
  %
  \cellcolor{gray!20}\scriptsize${\tilde{\mathtt{Dec}}}(.)$&\cellcolor{gray!20}\scriptsize \text{Decryption algorithm of asymmetric key encryption  }\\   
  %
    \cellcolor{white!20}\scriptsize$\tilde{\mathtt{keyGen}}(.)$&\cellcolor{white!20}\scriptsize \text{Key generator algorithm of asymmetric key encryption } \\
%
   \cellcolor{gray!20}\scriptsize${\mathtt{Sig.keyGen}}(.)$&\cellcolor{gray!20}\scriptsize \text{Key generator algorithm of digital signature scheme} \\
 %
\cellcolor{white!20}\scriptsize$\mathtt{verStat}(.)$ &\cellcolor{white!20}\scriptsize  Algorithm to determine $\mathcal{B}$'s message status \\ 
%
\cellcolor{gray!20}\scriptsize$\mathtt{checkWarning}(.)$ &\cellcolor{gray!20}\scriptsize  Algorithm to check a warning’s effectiveness \\ 
%
\cellcolor{white!20}\scriptsize$\mathtt{pay}(.)$ &\cellcolor{white!20}\scriptsize $\mathcal{B}$'s internal algorithm to transfers money\\   
%
 \cellcolor{gray!20}\scriptsize$\mathtt{Com}(.)$ &\cellcolor{gray!20}\scriptsize  Commitment's commit\\
  %
\cellcolor{white!20}\scriptsize$\mathtt{Ver}(.)$ &\cellcolor{white!20}\scriptsize  Commitment's verify\\   
%                    
\cellcolor{gray!20}\scriptsize$\mathtt{H}(.)$ &\cellcolor{gray!20}\scriptsize Hash function\\
%
\cellcolor{white!20}\scriptsize$\mathtt{PRF}(.)$ &\cellcolor{white!20}\scriptsize  Pseudorandom function \\ 
%
\cellcolor{gray!20}\scriptsize$\mathcal{C}$ &\cellcolor{gray!20}\scriptsize Customer  \\  
%
\cellcolor{white!20}\scriptsize$\mathcal{B}$ &\cellcolor{white!20}\scriptsize Bank  \\
%  
\cellcolor{gray!20}\scriptsize$\mathcal{D}_{\st 1},... , \mathcal{D}_{\st n}$ &\cellcolor{gray!20}\scriptsize Arbiters  \\  
%
\cellcolor{white!20}\scriptsize$\mathcal{DR}$ &\cellcolor{white!20}\scriptsize Dispute resolver  \\  
%
\cellcolor{gray!20}\scriptsize$\mathcal{S}$ &\cellcolor{gray!20}\scriptsize Smart contract  \\  
%
\cellcolor{white!20}\scriptsize$\mathcal{G}$ &\cellcolor{white!20}\scriptsize Certificate generator  \\  
%
\cellcolor{gray!20}\scriptsize{SAP} &\cellcolor{gray!20}\scriptsize  Statement agreement protocol\\ 
%
\cellcolor{white!20}\scriptsize{PVE} &\cellcolor{white!20}\scriptsize  Private verdict encoding protocol\\ 
%
\cellcolor{gray!20}\scriptsize{FVD} &\cellcolor{gray!20}\scriptsize  Final verdict decoding protocol\\ 
%
\cellcolor{white!20}\scriptsize{GPVE} &\cellcolor{white!20}\scriptsize  Generic private verdict encoding protocol\\ 
%
\cellcolor{gray!20}\scriptsize{GFVD} &\cellcolor{gray!20}\scriptsize  Generic final verdict decoding protocol\\ 
%
\cellcolor{white!20}\scriptsize$\oplus$ &\cellcolor{white!20}\scriptsize  XOR operator \\ 
%
\cellcolor{gray!20}\scriptsize$\Delta$ &\cellcolor{gray!20}\scriptsize  time parameter \\ 
%
\cellcolor{white!20}\scriptsize$add_{\st I}$ &\cellcolor{white!20}\scriptsize  Address of $I$\\ 
%
\cellcolor{gray!20}\scriptsize$aux, aux'$ &\cellcolor{gray!20}\scriptsize  Auxiliary information\\ 
%
\cellcolor{white!20}\scriptsize$\mathtt{BF}$ &\cellcolor{white!20}\scriptsize  Bloom filter\\ 
%
\cellcolor{gray!20}\scriptsize$z_{\st 1}$ &\cellcolor{gray!20}\scriptsize  $\mathcal{C}$'s complaint about $\mathcal{B}$'s message status\\ 
%
\cellcolor{white!20}\scriptsize$z_{\st 2}$ &\cellcolor{white!20}\scriptsize  $\mathcal{C}$'s complaint about a warning's effectiveness\\
% 
\cellcolor{gray!20}\scriptsize$z_{\st 3}$ &\cellcolor{gray!20}\scriptsize  $\mathcal{C}$'s complaint about payment message  inconsistency\\ 
%
 {   }\scriptsize$t_{\st i}$ &\scriptsize  time point\\ 



                     
 \hline
  


\end{tabular}

% 2nd table
\begin{tabular}{|c|c|c|c|c|c|c|c|c|c|c|c|c|c|} 
    \hline
\cellcolor{gray!15} \scriptsize \textbf{Symbol}&\cellcolor{gray!15} \scriptsize \textbf{Description}  \\
    \hline
    
\hline


%---------------------------------
%\multirow{10}{*}{\rotatebox[origin=c]{90}{\scriptsize  \textbf{Generic}}}
%

\cellcolor{gray!20}\scriptsize$\bm{l}$ &\cellcolor{gray!20}\scriptsize  $\mathcal{C}$'s payees list\\ 
%
\cellcolor{white!20}\scriptsize$\hat{\bm{l}}$ &\cellcolor{white!20}\scriptsize  $\mathcal{C}$'s encoded payees list\\ 
%


\cellcolor{gray!20}\scriptsize$ k_{\st 0}$ &\cellcolor{gray!20}\scriptsize   A secret key of $\mathtt{PRF}$\\ 
%               
\cellcolor{white!20}\scriptsize$f$ &\cellcolor{white!20}\scriptsize New payee’s detail  \\ 
 %                            
\cellcolor{gray!20}\scriptsize$in_{\st f}$ &\cellcolor{gray!20}\scriptsize Payment detail \\ 
%
  \cellcolor{white!20}\scriptsize$\hat a$ &\cellcolor{white!20}\scriptsize Encoded $a$ \\     
%
\cellcolor{gray!20}\scriptsize$\hat {m}^{\st(\mathcal{C})}_{\st 1}$ &\cellcolor{gray!20}\scriptsize $\mathcal{C}$'s encoded update request\\     
%
\cellcolor{white!20}\scriptsize$\hat {m}^{\st(\mathcal{C})}_{\st 2}$ &\cellcolor{white!20}\scriptsize $\mathcal{C}$'s encoded payment request\\     
%      
\cellcolor{gray!20}\scriptsize$\hat {m}^{\st(\mathcal{B})}_{\st 1}$ &\cellcolor{gray!20}\scriptsize $\mathcal{B}$'s encoded warning message\\     
%
\cellcolor{white!20}\scriptsize$\hat {m}^{\st(\mathcal{B})}_{\st 2}$ &\cellcolor{white!20}\scriptsize $\mathcal{B}$'s encoded payment message\\     
%          
     \cellcolor{gray!20}\scriptsize$sk$ and $pk$ &\cellcolor{gray!20}\scriptsize Secret  and public keys\\     
%
   \cellcolor{white!20}\scriptsize$sk_{\st\mathcal{D}}$ &\cellcolor{white!20}\scriptsize Arbiters' secret key\\     
%    
\cellcolor{gray!20}\scriptsize$pp$ &\cellcolor{gray!20}\scriptsize Public parameter\\  
%
\cellcolor{white!20}\scriptsize$j$ &\cellcolor{white!20}\scriptsize  Arbiter's index,  $1\leq j\leq n$ \\ 
%
\cellcolor{gray!20}\scriptsize$T, T_{\st 1},T_{\st 2}$ &\cellcolor{gray!20}\scriptsize  Tokens, where  $T:=( T_{\st 1},T_{\st 2})$\\ 
%
\cellcolor{white!20}\scriptsize$w_{\st 1}$ &\cellcolor{white!20}\scriptsize  Output of $\mathtt{verStat}(.)$\\ 
%
\cellcolor{gray!20}\scriptsize$(w_{\st 2},w_{\st 3})$ &\cellcolor{gray!20}\scriptsize  Output of $\mathtt{checkWarning}(.)$\\ 
%
\cellcolor{white!20}\scriptsize$\bar{w}_{\st j}$ &\cellcolor{white!20}\scriptsize  Output of $\mathtt{PVE}(.)$\\ 
%
\cellcolor{gray!20}\scriptsize$v$ &\cellcolor{gray!20}\scriptsize  Output of $\mathtt{FVD}(.)$\\ 
%
\cellcolor{white!20}\scriptsize$e$ &\cellcolor{white!20}\scriptsize  Threshold\\ 
%            
 \cellcolor{gray!20}\scriptsize$w_{\st i,j}$ &\cellcolor{gray!20}\scriptsize  Arbiter's plain verdict\\ 
%     
\cellcolor{white!20}\scriptsize$o$ &\cellcolor{white!20}\scriptsize  Offset\\  
 %
  \cellcolor{gray!20}\scriptsize$r_{\st j}$ &\cellcolor{gray!20}\scriptsize  Pseudorandom value\\   
 %              
\cellcolor{white!20}\scriptsize$\lambda$ &\cellcolor{white!20}\scriptsize Security parameter\\  
%
\cellcolor{gray!20}\scriptsize$\mu$ &\cellcolor{gray!20}\scriptsize Negligible function\\  


\cellcolor{white!20}\scriptsize$in_{\st p}$ &\cellcolor{white!20}\scriptsize The input of $\mathtt{pay}(.)$\\                    
%
\cellcolor{gray!20}\scriptsize$\pi$ &\cellcolor{gray!20}\scriptsize Private statement\\        
  %
  \cellcolor{white!20}\scriptsize$Pr$ &\cellcolor{white!20}\scriptsize Probability\\   

%
\cellcolor{gray!20}\scriptsize$\phi$ &\cellcolor{gray!20}\scriptsize  Null\\ 
%
\cellcolor{white!20}\scriptsize$n$ &\cellcolor{white!20}\scriptsize  Total number of arbiters\\  
%           
\cellcolor{gray!20}\scriptsize$z$ &\cellcolor{gray!20}\scriptsize  $\mathcal{C}$'s complaint, where $z:=(z_{\st 1}, z_{\st 2}, z_{\st 3})$\\ 
%
\cellcolor{white!20}\scriptsize $g:=(g_{\st 1}, g_{\st 2})$&\cellcolor{white!20}\scriptsize  Commitment values \\    
%
%***********************

\hline 
      

           
              
\end{tabular}\label{table:notation-table}
%
}}
\end{center}
\end{scriptsize}
\end{table*}








%%%%%%%%%%%%%%%%%%%%%%%%%%%%%%%%%%%%%%%%%%%























