%!TEX root = main.tex





\vspace{-3mm}
\section{ Definition of Payment with Dispute Resolution Scheme}\label{sec::def}

In this section, we outline a formal definition of the payment with dispute resolution  ($\mathsf{pwdr}$) notion. We refer readers to Appendix \ref{sec::long-def} for the definition's  full version. 
%
%In this section, we outline a formal definition of a  PwDR scheme. We first provide the scheme's syntax.  Then, we   formally define its correctness  and security properties. We refer readers to Appendix \ref{sec::long-def} for a full version of the formal definition. 
%
\vspace{-3mm}
\begin{definition}\label{def-a::sytax} A $\mathsf{pwdr}$    involves six   types of entities; namely,  bank $\mathcal{B}$, customer $\mathcal{C}$,  smart contract $\mathcal{S}$,  certificate generator $\mathcal{G}$,   set of auditors $\mathcal{D}:\{\mathcal{D}_{\st 1},..., \mathcal{D}_{\st n}\}$, and  dispute resolver $\mathcal{DR}$. It also includes the following    algorithms.  

%\item \item[$\bullet$] $\mathtt{setup}(1^{\st \lambda})\rightarrow k:=(sk,pk)$. %%% this might be needed when other security properties, e.g., privacy, are added to the scheme. 
%


\item [$\bullet$]  $\mathtt{keyGen}(1^{\st \lambda})\rightarrow (sk,pk)$. It  generates and outputs a pair of secret keys $sk:=(sk_{\st\mathcal {G}},sk_{\st\mathcal {D}})$ and public keys $pk:=(pk_{\st\mathcal {G}},pk_{\st\mathcal {D}})$. 
%
\item[$\bullet$] $\mathtt{bankInit}(1^{\st \lambda})\rightarrow (T, pp, \bm{l})$. It  outputs an encoding-decoding token $T$ (where  $T:=(T_{\st 1},T_{\st 2})$,  each $T_{\st i}$  contains  a secret value $\ddot\pi_{\st i}$ and its  public witness $g_{\st i}$),  set of  public parameters  $pp$  (including a threshold parameter $e$),  and    empty list $\bm{l}$.
%
\item[$\bullet$] $\mathtt{customerInit} (1^{\st \lambda}, T, pp)\rightarrow a$. It is an   initiation algorithm. It checks the correctness of the elements in $T$ and $pp$. If the checks pass, it outputs $1$. Otherwise, it outputs $0$. 
%
\item [$\bullet$] $\mathtt{genUpdateRequest}(T, f, {\bm l})\rightarrow \hat m^{\st\mathcal{(C)}}_{\st 1}$.  It is an update request  algorithm. It uses the new payee's detail $f$ and encoding algorithm $\mathtt{Encode}(T_{\st 1},.)$ to generate an encoded update request $\hat  m^{\st\mathcal{(C)}}_{\st 1}$.  It outputs  $\hat  m^{\st\mathcal{(C)}}_{\st 1}$.
%
\item [$\bullet$] $\mathtt{insertNewPayee}(\hat{m}^{\st\mathcal{(C)}}_{\st 1}, {\bm l})\rightarrow  \hat{\bm l}$. It is  an algorithm that inserts a new payee's detail into ${\bm l}$ and outputs an updated list $\hat{\bm l}$.
%
\item  [$\bullet$] $\mathtt{genWarning}(T, \hat{\bm l}, {aux})\rightarrow \hat m^{\st\mathcal{(B)}}_{\st1}$. It is a warning generating algorithm that outputs an encoded (warning) message $\hat m^{\st\mathcal{(B)}}_{\st1}$, with the help of auxiliary data ${aux}$ and $\mathtt{Encode}(T_{\st 1},.)$, where the plaintext message is either pass or warning string. 
%
\item  [$\bullet$] $\mathtt{genPaymentRequest}(T, in_{\st f}, \hat{\bm{l}}, \hat m^{\st\mathcal{(B)}}_{\st1})\rightarrow \hat m^{\st\mathcal{(C)}}_{\st2}$. It an algorithm that generates an encoded payment request  $\hat m^{\st\mathcal{(C)}}_{\st2}$, with the help of new payment's detail $in_{\st f}$ and $\mathtt{Encode}(T_{\st 1},.)$. It outputs  $\hat m^{\st\mathcal{(C)}}_{\st2}$.  
%
\item  [$\bullet$] $\mathtt{makePayment}(T, \hat m^{\st\mathcal{(C)}}_{\st2})\rightarrow \hat m^{\st\mathcal{(B)}}_{\st2}$. It  generates and outputs an encoded message $\hat m^{\st\mathcal{(B)}}_{\st2}$ for confirmation of payment. 
%


%\item  [$\bullet$] $\mathtt{genComplaint}(\hat m^{\st\mathcal{(B)}}_{\st 1}, \hat m^{\st\mathcal{(B)}}_{\st2}, T, pk, {aux}_{\st f})\rightarrow (\hat z, \hat{\ddot \pi})$. It  generates complaints with the help of auxiliary data ${aux}_{\st f}$. It 
% generates and outputs  encoded complaints $\hat z$ using  $\mathtt{Encode}(T_{\st 1},.)$ and encoded secret  parameters $\hat{\ddot \pi}$  using another encoding algorithm $\bar{\mathtt{Encode}}(pk_{\st \mathcal{D}},.)$.   
 
 
 \item  [$\bullet$] $\mathtt{genComplaint}(\hat m^{\st\mathcal{(B)}}_{\st 1}, \hat m^{\st\mathcal{(B)}}_{\st2}, T, pk, {aux}_{\st f})\rightarrow (\hat z, \hat{\ddot \pi})$. It  generates complaints with the help of auxiliary data ${aux}_{\st f}$.  If $\mathcal{C}$ wants to complain that (i)   pass message should have been a warning  or (ii) no  message was provided,   it sets $z_{\st 1}$ to ``challenge message''. If its complaint is about the warning's effectiveness,  it sets $z_{\st 2}$ to a combination of an evidence $u\in aux_{\st f}$, the evidence's certificate $sig\in aux_{\st f}$, the certificate's public parameter,   and ``challenge warning'', where the certificate is obtained from  $\mathcal{G}$ via a query, $Q$. If its complaint is  about the   payment, it sets $z_{\st 3}$ to ``challenge payment''. It  generates and outputs (i)  encoded complaints $\hat z$ using  $\mathtt{Encode}(T_{\st 1},.)$, and (ii) encoded secret  parameters $\hat{\ddot \pi}$  using another encoding algorithm $\bar{\mathtt{Encode}}(pk_{\st \mathcal{D}},.)$.   
%
\item  [$\bullet$] $\mathtt{verComplaint}( \hat z, \hat{\ddot{\pi}}, g, \hat{\bm m}, \hat{\bm{l}}, j, sk_{\st\mathcal D}, aux, pp)\rightarrow \hat{\bm{w}_{\st j}}$. It compiles $j$-th auditor's complaints. It initially sets parameters as  ${w}_{\st 1,j}= {w}_{\st 2, j}= {w}_{\st 3, j}= {w}_{\st 4, j}=0$. If the complaint in $ z_{\st 1}$ is valid, it sets  $w_{\st 1,j}=1$. If the certificate in $z_{\st 2}$ is valid,  it sets $ w_{\st 3,j}=1$. It  checks the warning's effectiveness, by running  algorithm $\mathtt{checkWarning}(.)$. If  it is not effective, i.e., $\mathtt{checkWarning}({m}^{\st\mathcal{(B)}}_{\st1})= 0$,  it sets  $w_{\st 2,j}=1$.  Also, if  the payment was indeed made,    it sets $w_{\st 4,j}=1$. It outputs encoded verdicts $\hat{\bm{w}}_{\st j}= [\hat {w}_{\st 1,j}, \hat{w}_{\st 2, j}, \hat{w}_{\st 3, j}, \hat{w}_{\st 4, j}]$ for $j$-th auditor. %, where $g:=(g_{\st 1},g_{\st 2})$. % and  each decoded verdict ${w}_{\st 1,j}$  is $1$ if the related complaint is valid, and  is $0$ otherwise.
%
\item  [$\bullet$] $\mathtt{resDispute}(T_{\st2}, \hat{\bm{w}}, pp)\rightarrow {\bm v}$. It  aggregates all  encoded verdicts $\hat{\bm{w}}=[\hat{\bm{w}}_{\st 1},...,\hat{\bm{w}}_{\st n}]$ and outputs $\bm{v}=[v_{\st 1},..., v_{\st 4}]$, where $v_{\st i}=1$ if at least $e$   verdicts ${w}_{\st i,j}$ is $1$; otherwise, $v_{\st i}=0$. If $v_{\st 4}=1$ and (i) either $v_{\st 1}=1$ (ii) or $v_{\st 2}=1$ and $v_{\st 3}=1$,  then  $\mathcal C$  is reimbursed.
%
\end{definition}



%
% $\hat{m}_{\st 1}^{\st(\mathcal C)}, \hat {m}_{\st 2}^{\st(\mathcal C)}$, and $\hat{m}_{\st 2}^{\st(\mathcal B)}$ which results in ${m}_{\st 1}^{\st(\mathcal C)},  {m}_{\st 2}^{\st(\mathcal C)}$, and ${m}_{\st 2}^{\st(\mathcal B)}$ respectively. It extracts from $\hat {\bm w}$ decoded combined verdicts $[v_{\st 1},...,v_{\st 4}]$. It sets $\bar v_{\st 0}=1$ and proceeds, if  $m_{\st 1}^{\st(\mathcal C)}$ and $m_{\st 2}^{\st(\mathcal C)}$ are  not empty.  Otherwise, it aborts. It sets $\bar v_{\st 1}=1$, if $v_{\st 1}$ indicates that the majority of the auditors believe that ``pass'' or missing message should have been a warning message.  It sets $\bar v_{\st 2}=1$, if $v_{\st 2}$ indicates that
%the majority of the auditors believe that the warning was ineffective.  It sets $\bar v_{\st 3}=1$, if $v_{\st 3}$ indicates that  the majority of the auditors believe that $\mathcal C$ did not provide an invalid certificate. It sets $\bar v_{\st 4}=1$, if  $v_{\st 4}$ (or ${m}_{\st 2}^{\st(\mathcal B)}$) indicates that  the majority of the auditors believe that the payment was made. It outputs $\bar{\bm v}=[\bar v_{\st 0},..., \bar v_{\st 4}]$.  Note, $\mathcal{C}$ is reimbursed only if $\bar v_{\st 0}=\bar v_{\st 3}=\bar v_{\st 4}=1$ and $\bar v_{\st 1}$ or $\bar v_{\st 2}$ is $1$. $\mathcal{DR }$ sends the output to  $\mathcal{C}$ and $\mathcal{B}$. 





%\item  [$\bullet$] $\mathtt{resDispute}(\hat{m}_{\st 1}^{\st(\mathcal C)}, \hat {m}_{\st 2}^{\st(\mathcal C)}, \hat{m}_{\st 2}^{\st(\mathcal B)}, T, \hat {\bm w})\rightarrow c$. It is a deterministic algorithm run by $\mathcal {DR}$. It takes as input, $\mathcal C$'s encoded  update request  $\hat m_{\st 1}^{\st(\mathcal C)}$, $\mathcal C$'s encoded payment request  $\hat m_{\st 2}^{\st(\mathcal C)}$,  $\mathcal S$'s encoded payment message $\hat m_{\st 2}^{\st(\mathcal B)}$,  token $T$, and auditors' encoded verdicts  $\hat{ \bm{w}}$. Checks if the token's parameters match,  i.e., $\mathtt{match}({\ddot{\pi}}, g)=1$. If the check fails, it aborts; otherwise, it continues. It decodes $\hat{m}_{\st 1}^{\st(\mathcal C)}, \hat {m}_{\st 2}^{\st(\mathcal C)}$, and $\hat{m}_{\st 2}^{\st(\mathcal B)}$ which results in ${m}_{\st 1}^{\st(\mathcal C)},  {m}_{\st 2}^{\st(\mathcal C)}$, and ${m}_{\st 2}^{\st(\mathcal B)}$ respectively. It extracts from $\hat {\bm w}$ decoded combined verdicts $[v_{\st 1},...,v_{\st 4}]$. It proceeds if  $m_{\st 1}^{\st(\mathcal C)}$ and $m_{\st 2}^{\st(\mathcal C)}$ are  not empty.  Otherwise, it aborts. It sets $c=1$ (i.e., $\mathcal{C}$ must be reimbursed), if $\mathcal C$ requested a money transfer, $v_{\st 4}$ (or ${m}_{\st 2}^{\st(\mathcal B)}$) indicates that the payment was made, $v_{\st 3}$ indicates that $\mathcal C$ did not provide an invalid certificate, and one the following  conditions holds (i) $v_{\st 2}$ indicates that the warning was ineffective, or (ii) $v_{\st 1}$ indicates that  ``pass'' or missing message should have been a warning message. Otherwise, it sets $c=0$. It returns $c$.



%\mathtt{Encode}(T_{\st 1},  z)\rightarrow \hat z, \
%{\bar{ \mathtt{Encode}}}(pk_{\st\mathcal{D}}, \ddot \pi)

%Note, algorithms $\mathtt{genWarning}(.)$-$\mathtt{resDispute}(.)$ implicitly take another input, $state_{\st\mathcal S}$, which is $\mathcal{S}$'s current state, i.e., data registered in $\mathcal{S}$. For the sake of simplicity, we avoided explicitly passing this input  to these processes.  
\vspace{-2.5mm}
A $\mathsf{pwdr}$  has two  properties,  \emph{correctness} and \emph{security}. Correctness requires that the payment journey is completed (in the absence of a fraudster) without the need for (i) the honest customer to complain and (ii)  the honest bank to reimburse.  A $\mathsf{pwdr}$  is secure if it meets three main properties,  (a) security against a malicious victim, (b) security against a malicious bank, and (c) privacy. 
%
%Below, we present a brief formal definition of   them. 
%
 Informally, security against a malicious victim states that an APP fraud victim who is not qualified for the reimbursement should not be reimbursed. 
%
Specifically, a corrupt victim cannot (a) make at least threshold auditors, $\mathcal{D}_{\st j}$s,  conclude that $\mathcal{B}$ should have provided a warning, although $\mathcal{B}$ has done so, or (b) make $\mathcal{DR}$ conclude that   the pass message was incorrectly given or a vital warning message was missing despite only less than threshold  $\mathcal{D}_{\st j}$s  believe so, or (c) persuade at least threshold $\mathcal{D}_{\st j}$s to  conclude that the warning was ineffective although it was effective, or (d)  make $\mathcal{DR}$ believe that the warning message was ineffective although only less than threshold $\mathcal{D}_{\st j}$s   believe that, or (e)  convince  $\mathcal{D}_{\st j}$s to accept an invalid certificate, or  (f) make $\mathcal{DR}$ believe that at least  threshold $\mathcal{D}_{\st j}$s accepted the certificate  although they did not, except for a negligible probability. Below, we formally state it. 
%%%%%%%%%%%%%
%
\vspace{-2mm}
\begin{definition}[Security against a malicious victim]\label{def-a::Security-against-malicious-victim} A $\mathsf{pwdr}$  is secure against a malicious victim, if for any security parameter $\lambda$,  auxiliary data $aux$, and   PPT adversary $\mathcal{A}$, there is a negligible function $\mu(\cdot)$, s.t. for  experiment $\mathsf{Exp}_{\st 1}^{\mathcal{A}}$:
%


\vspace{-.7mm}

{\small{
\begin{center}
\begin{mybox}[colback=white,  width=119mm, height=27.2mm, drop fuzzy shadow southwest]{{{$\mathsf{Exp}_{\st 1}^{\st\mathcal{A}}(1^{\st\lambda}\text{, }  
{aux})$}}}

%\fbox{\begin{minipage}{3.8 in}
\vspace{-2.2mm}
$$
  \begin{array}{l}
%  \mathsf{Exp}_{\st 1}^{\mathcal{A}}(\mathsf{In}:=(1^{\st\lambda}, p, \bm{l})): \\
  
    \mathtt{keyGen}(1^{\st\lambda})\rightarrow (sk,pk),\ \
  % 
  \mathtt{bankInit}(1^{\st \lambda})\rightarrow (T, pp, \bm{l}), \ \ \
  %
      \mathcal{A}( 1^{\st \lambda}, T, pp, \bm{l} )\rightarrow \hat m^{\st\mathcal{(C)}}_{\st 1}, \\
%   
\mathtt{insertNewPayee}(\hat{m}^{\st\mathcal{(C)}}_{\st 1}, {\bm l})\rightarrow \hat{\bm l}, \ \ \
%
\mathtt{genWarning}(T, \hat{\bm l}, {aux})\rightarrow \hat m^{\st\mathcal{(B)}}_{\st1}, \ \ \
%
   \mathcal{A}(T,  \hat{\bm{l}}, \hat m^{\st\mathcal{(B)}}_{\st1})\rightarrow \hat m^{\st\mathcal{(C)}}_{\st2}, \\
%
   \mathtt{makePayment}(T, \hat m^{\st\mathcal{(C)}}_{\st2})\rightarrow \hat m^{\st\mathcal{(B)}}_{\st2}, \ \ \
   %
\mathcal{A}(\hat m^{\st\mathcal{(B)}}_{\st 1}, \hat m^{\st\mathcal{(B)}}_{\st2}, T, pk)\rightarrow (\hat z, \hat{\ddot \pi}),\\


 \forall j, j\in [n]:
\Big(\mathtt{verComplaint}( \hat z, \hat{\ddot{\pi}}, g, \hat{\bm{m}}, \hat{\bm{l}}, j, sk_{\st\mathcal D}, aux, pp)\rightarrow \hat{\bm{w}}_{\st j}= [\hat w_{\st 1,j}, \hat w_{\st 2, j}, \hat w_{\st 3, j}, \hat w_{\st 4, j}]\Big), \\
\mathtt{resDispute}(T_{\st 2}, \hat {\bm w}, pp)\rightarrow \bm v=[v_{\st 1},..., v_{\st 4}]
   \end{array} 
$$
\end{mybox}
%
\end{center}
}}
%
\vspace{-2.38mm}
it holds that:
\vspace{-1mm}
%
%%%%%%%%%%%%%%%%%%%%%%%%%%%
{\small{
$$ \Pr\left[
  \begin{array}{l}
%
%\Bigg(
%
  \Big((m^{\st\mathcal{(B)}}_{\st1}=warning) \wedge (\sum\limits_{\st j=1}^{\st n}w_{\st 1,j}\geq {e})\Big) \
  %
  \ \ \vee\ \ \Big((\sum\limits_{\st j=1}^{\st n}w_{\st 1,j}<{e}) \wedge ( {v}_{\st 1}=1)\Big)
 % \Bigg)
  \\
  %
\vee
  %\Bigg(
    \Big((\mathtt{checkWarning}(m^{\st\mathcal{(B)}}_{\st1})= 1) \wedge (\sum\limits_{\st j=1}^{\st n}w_{\st 2,j}\geq {e})\Big)\
  %
 \ \ \vee\ \ \Big((\sum\limits_{\st j=1}^{\st n}w_{\st 2,j}< {e}) \wedge ({v}_{\st 2}=1)\Big)
  %\Bigg)
  \\
  %
  \vee 
  %\Bigg(
  \Big(u\notin Q \wedge\mathtt{Sig.ver}( pk,  u, sig) =1\Big) \
  %
  \ \ \vee\ \ \Big((\sum\limits_{\st j=1}^{\st n}w_{\st 3,j}< {e}) \wedge ( {v}_{\st 3}=1)\Big)
  %
  %\Bigg)
  \\
  %
%\text{s.t.}\\
%y_{\st j}= E^{\st -1}_{\st 2}(y^{\st *}_{\st j},t_{\st qp})\\
%E^{\st -1}_{\st 2},t_{\st qp}\in en\\

\end{array} :
    \begin{array}{l}
    \mathsf{Exp}_{\st 1}^{\st\mathcal{A}}(\mathsf{input})\\
\end{array}    \right]\leq \mu(\lambda),$$
}}
where  $\hat {\bm m}=[\hat {m}^{\st (\mathcal{C})}_{\st 1}, \hat {m}^{\st (\mathcal{C})}_{\st 2}, \hat {m}^{\st (\mathcal{B})}_{\st 1}, \hat {m}^{\st (\mathcal{B})}_{\st 2}]$, $(w_{\st 1,j},..,w_{\st 3,j})$ are the decoding of  $(\hat w_{\st 1,j},..,\hat w_{\st 3,j})\in \hat {\bm w}$,  and $\mathsf{input}:=(1^{\st\lambda}, {aux})$.


%$\mathtt{checkWarning}(.)$ determines a warning's effectiveness,   $\mathsf{input}:=(1^{\st\lambda}, {aux})$,   $ sk_{\st \mathcal{D}}\in sk$, and $n$ is the total number of auditors. %The probability is taken over the uniform choice of $sk$, randomness used in the blockchain's primitives (e.g., in signatures), randomness used during the encoding,   and  the randomness of $\mathcal{A}$. 
\end{definition}

\vspace{-2mm}

Security against a malicious bank requires that a malicious bank cannot disqualify an honest victim from being reimbursed. 
%
Specifically,  a corrupt bank cannot  (a) make $\mathcal{DR}$ conclude that the  ``pass'' message was correctly given or an important warning was not missing although at least threshold  $\mathcal{D}_{\st j}$s  do not believe so, or (b) convince $\mathcal{DR}$  that the warning message was effective although at least threshold $\mathcal{D}_{\st j}$s do not believe so, or (c) make $\mathcal{DR}$ believe that less than threshold $\mathcal{D}_{\st j}$s did  not accept the certificate although at least threshold of them did that, or (d) make $\mathcal{DR}$ believe that no payment was made, although at least threshold $\mathcal{D}_{\st j}$s believe the opposite, except for a negligible probability. 
%%%%%%%%%%%

\vspace{-1.5mm}
\begin{definition}[Security against a malicious bank]\label{def-a::Security-against-malicious-bank} A $\mathsf{pwdr}$ scheme is secure against a malicious bank, if for any  $\lambda$,  $ {aux}$, and   PPT adversary $\mathcal{A}$, there exists a negligible function $\mu(\cdot)$, such that for an experiment $\mathsf{Exp}_{\st 2}^{\mathcal{A}}$:
%
\vspace{-.6mm}
{\small{
\begin{center}
\begin{mybox}[colback=white,  width=119mm, height=31.6mm, drop fuzzy shadow southwest]{$\mathsf{Exp}_{\st 2}^{\st\mathcal{A}}(1^{\st\lambda}\text{, }  {aux})$}
%\fbox{\begin{minipage}{3.8 in}
\vspace{-2.2mm}
$$
  \begin{array}{l}
%
 \mathtt{keyGen}(1^{\st\lambda})\rightarrow (sk,pk), \ \ \
%
\mathcal{A}(1^{\st \lambda})\rightarrow (T, pp, \bm{l}, f,  in_{\st f}, {aux}_{\st f}), \ \ \
%
\mathtt{customerInit} (1^{\st \lambda}, T, pp)\rightarrow a, \\
%
\mathtt{genUpdateRequest}(T, f, {\bm l})\rightarrow \hat m^{\st\mathcal{(C)}}_{\st 1}, \ \ \
%
\mathtt{insertNewPayee}(\hat{m}^{\st\mathcal{(C)}}_{\st 1}, {\bm l})\rightarrow  \hat{\bm l}, \ \ \
%
\mathcal{A}(T, \hat{\bm l}, {aux})\rightarrow \hat m^{\st\mathcal{(B)}}_{\st1}, \\
%
\mathtt{genPaymentRequest}(T, in_{\st f}, \hat{\bm{l}}, \hat m^{\st\mathcal{(B)}}_{\st1})\rightarrow \hat m^{\st\mathcal{(C)}}_{\st2}, \ \ \
%
\mathcal{A}(T, \hat m^{\st\mathcal{(C)}}_{\st2})\rightarrow \hat m^{\st\mathcal{(B)}}_{\st2}, \\
%
\mathtt{genComplaint}(\hat m^{\st\mathcal{(B)}}_{\st 1}, \hat m^{\st\mathcal{(B)}}_{\st2}, T, pk, {aux}_{\st f})\rightarrow (\hat z, \hat{\ddot \pi}), \\
%
 \forall j, j\in [n]:
\Big(\mathtt{verComplaint}( \hat z, \hat{\ddot{\pi}}, g, \hat{\bm m}, \hat{\bm{l}}, j, sk_{\st\mathcal D}, aux, pp)\rightarrow \hat{\bm{w}_{\st j}}= [\hat {w}_{\st 1,j}, \hat{w}_{\st 2, j}, \hat{w}_{\st 3, j}, \hat{w}_{\st 4, j}]
\Big),
\\
%
\mathtt{resDispute}(T_{\st 2}, \hat {\bm w}, pp)\rightarrow \bm v=[v_{\st 1},...,   v_{\st 4}]\\
   \end{array} 
$$
\end{mybox}
%\end{minipage}}
\end{center}
}}
%
\vspace{-2mm}
it holds that:
%
{\small{
$$ \Pr\left[
  \begin{array}{l}
  
 
\Big( (\sum\limits_{\st j=1}^{\st n}w_{\st 1,j}\geq e) \wedge ( v_{\st 1}=0)\Big) \
 %
 
\ \ \vee\ \ \Big(( \sum\limits_{\st j=1}^{\st n}w_{\st 2,j}\geq e) \wedge ( v_{\st 2}=0)\Big) \\
 %
\ \ \vee \ \ \Big(( \sum\limits_{\st j=1}^{\st n}w_{\st 3,j}\geq e) \wedge ( v_{\st 3}=0)\Big) \
 %
\ \ \vee \ \  \Big(( \sum\limits_{\st j=1}^{\st n}w_{\st 4,j}\geq e) \wedge ( v_{\st 4}=0)\Big) \\
 %
\end{array} :
    \begin{array}{l}
    \mathsf{Exp}_{\st 2}^{\st\mathcal{A}}(\mathsf{input})\\
\end{array}    \right]\leq \mu(\lambda),$$
}}
where $\hat {\bm m}=[\hat {m}^{\st (\mathcal{C})}_{\st 1}, \hat {m}^{\st (\mathcal{C})}_{\st 2}, \hat {m}^{\st (\mathcal{B})}_{\st 1}, \hat {m}^{\st (\mathcal{B})}_{\st 2}]$, $(w_{\st 1,j},..,w_{\st 3,j})$ are the  decoding of   $(\hat w_{\st 1,j},..,\hat w_{\st 3,j})\in \hat {\bm w}$, and $\mathsf{input}:=(1^{\st\lambda},  {aux})$. %, $ sk_{\st \mathcal{D}}\in sk$, $n$ is the total number of auditors. %The probability is taken over the uniform choice of $sk$, randomness used in the blockchain's primitives, randomness used during the encoding, and  the randomness of $\mathcal{A}$. 
\end{definition}


%A careful reader may ask why the following two conditions (some forms of which were in the events of Definition \ref{def::Security-against-malicious-victim}) are not added to the above events list: (a) $\mathcal{B}$ makes at least threshold committee members conclude that it  has provided a warning, although $\mathcal{B}$ has not (i.e., $ m^{\st\mathcal{(B)}}_{\st1}\neq warning \wedge \sum\limits_{\st j=1}^{\st n}w_{\st 1,j}<  e$), and (b) $\mathcal{B}$ persuades at least threshold $\mathcal{D}_{\st j}$s  to conclude that the warning was effective although it was not (i.e., $\mathtt{checkWarning}(m^{\st\mathcal{(B)}}_{\st1})= 0 \wedge \sum\limits_{\st j=1}^{\st n}w_{\st 2,j}< e$). The answer is that  $\mathcal{B}$ does not generate a complaint  and interact directly with $\mathcal{D}_{\st j}$s; therefore, we do not need to add these two events to the above events' list. 


\vspace{-2mm}

Informally, a $\mathsf{pwdr}$ scheme is privacy-preserving if it protects the privacy of (1)  customers', bank's, and auditors' sensitive messages from the scheme's non-participants,  and (2) each auditor's verdict from $\mathcal{DR}$.%  which sees the final verdict. 




% !TEX root =main.tex




\begin{definition}[Privacy]\label{def-a::privacy} A PwDR preserves privacy if  the following two properties are satisfied.
\begin{enumerate}

\item For any PPT  adversary $\mathcal{A}_{\st 1}$,  security parameter $\lambda$, and  auxiliary information $aux$, there exists a negligible function $\mu(\cdot)$, such that for any  experiment $\mathsf{Exp}_{\st 3}^{\mathcal{A}_{\st 1}}$:



\begin{center}
\begin{mybox}[colback=white,  width=121mm, height=35mm, drop fuzzy shadow southwest]{$\mathsf{Exp}_{\st 3}^{\st\mathcal{A}_{\st 1}}(1^{\st\lambda}\text{, }  {aux})$}
%\fbox{\begin{minipage}{3.8 in}
\vspace{-1.2mm}
$$
  \begin{array}{l}
%
 \mathtt{keyGen}(1^{\st\lambda})\rightarrow (sk,pk), \ \ \
%
  \mathtt{bankInit}(1^{\st \lambda})\rightarrow (T, pp, \bm{l}), \ \ \
%
\mathtt{customerInit} (1^{\st \lambda}, T, pp)\rightarrow a, \\
%
\mathcal{A}_{\st 1}(1^{\st \lambda}, pk, a, pp, g, \bm{l})\rightarrow \Big((f_{\st 0}, f_{\st 1}),(in_{\st f_{\st_{\st 0}}},in_{\st f_{\st _1}}),(\text{aux}_{\st f_{0}},\text{aux}_{\st f_{\st1}})\Big), \\
%
\gamma\stackrel{\st \$}\leftarrow\{0,1\}, \ \ \
%
\mathtt{genUpdateRequest}(T, f_{\st \gamma}, {\bm l})\rightarrow \hat m^{\st\mathcal{(C)}}_{\st 1}, \ \ \
%
\mathtt{insertNewPayee}(\hat{m}^{\st\mathcal{(C)}}_{\st 1}, {\bm l})\rightarrow  \hat{\bm l}, \\
%
\mathtt{genWarning}(T, \hat{\bm l}, \text{aux})\rightarrow \hat m^{\st\mathcal{(B)}}_{\st1}, \ \ \
%
\mathtt{genPaymentRequest}(T, in_{\st f_{_\gamma}}, \hat{\bm{l}}, \hat m^{\st\mathcal{(B)}}_{\st1})\rightarrow \hat m^{\st\mathcal{(C)}}_{\st2}, \\
%
\mathtt{makePayment}(T, \hat m^{\st\mathcal{(C)}}_{\st2})\rightarrow \hat m^{\st\mathcal{(B)}}_{\st2}, \ \ \
%
\mathtt{genComplaint}(\hat m^{\st\mathcal{(B)}}_{\st 1}, \hat m^{\st\mathcal{(B)}}_{\st2}, T, pk, \text{aux}_{\st f_{_\gamma}})\rightarrow (\hat z, \hat{\ddot \pi}), \\
%
 \forall j, j\in [n]:
\Big(\mathtt{verComplaint}( \hat z, \hat{\ddot{\pi}}, g, \hat{\bm m}, \hat{\bm{l}}, j, sk_{\st\mathcal D}, aux, pp)\rightarrow \hat{\bm{w}}_{\st j}
\Big)
, \ \ \
\mathtt{resDispute}(T_{\st 2}, \hat {\bm w}, pp)\rightarrow \bm v
\\
%
%\mathtt{resDispute}(\hat{m}_{\st 1}^{\st(\mathcal C)}, \hat {m}_{\st 2}^{\st(\mathcal C)}, \hat{m}_{\st 2}^{\st(\mathcal B)}, T, \hat {\bm w})\rightarrow \bar{\bm v}\\
%
   \end{array} 
$$
\end{mybox}
%\end{minipage}}
\end{center}
%
it holds that:
%
%%%%%%%%%%%%%%%%%%%%%%%%%%%%%
$$ \Pr\left[
  \begin{array}{l}
  %
\mathcal{A}_{\st 1}(g, \hat {\bm m}, \hat{\bm l}, \hat z, \hat{\ddot \pi}, \hat{\bm{w}})\rightarrow \gamma
\end{array} :
    \begin{array}{l}
    \mathsf{Exp}_{\st 3}^{\st\mathcal{A}_{\st 1}}(\mathsf{input})\\
\end{array}    \right]\leq \frac{1}{2}+\mu(\lambda).$$
%}
%where   $g:=(g_{\st 1}, g_{\st 2})\in T$,   $\hat {\bm m}=[\hat {m}^{\st (\mathcal{C})}_{\st 1}, \hat {m}^{\st (\mathcal{C})}_{\st 2}, \hat {m}^{\st (\mathcal{B})}_{\st 1}, \hat {m}^{\st (\mathcal{B})}_{\st 2}], \hat {\bm w}=[\hat {\bm w}_{\st 1},.., \hat{\bm w}_{\st n}]$, $\mathsf{input}:=(1^{\st\lambda}, aux)$, $ sk_{\st \mathcal{D}}\in sk$, $n$ is the total number of arbiters. The probability is taken over the uniform choice of $sk$, randomness used in the blockchain's primitives, and  the randomness of $\mathcal{A}_{\st 1}$. 




\item For any PPT  adversaries $\mathcal{A}_{\st 2}$ and  $\mathcal{A}_{\st 3}$, security parameter $\lambda$, and  auxiliary information $aux$, there exists a negligible function $\mu(\cdot)$, such that for any  experiment $\mathsf{Exp}_{\st 4}^{\mathcal{A}_{\st 2}}$:

%%%%%%%%%%%%%%%--EXp_4%%%%%%%%%%%%%%%
\begin{center}
\begin{mybox}[colback=white,  width=121mm, height=38mm, drop fuzzy shadow southwest]{$\mathsf{Exp}_{\st 4}^{\st\mathcal{A}_{\st 2}}(1^{\st\lambda}\text{, }  {aux})$}
%\fbox{\begin{minipage}{3.8 in}
\vspace{-1.2mm}
$$
  \begin{array}{l}
%
 \mathtt{keyGen}(1^{\st\lambda})\rightarrow (sk,pk), \ \ \
%
  \mathtt{bankInit}(1^{\st \lambda})\rightarrow (T, pp, \bm{l}), \ \ \
%
\mathtt{customerInit} (1^{\st \lambda}, T, pp)\rightarrow a, \\
%
\mathcal{A}_{\st 2}(1^{\st \lambda}, pk, a, pp, \bm{l})\rightarrow (f, in_{\st f}, \text{aux}_{\st f}), \ \ \
%
\mathtt{genUpdateRequest}(T, f, {\bm l})\rightarrow \hat m^{\st\mathcal{(C)}}_{\st 1}, \\
%
\mathtt{insertNewPayee}(\hat{m}^{\st\mathcal{(C)}}_{\st 1}, {\bm l})\rightarrow  \hat{\bm l}, \ \ \
%
\mathcal{A}_{\st 2}(T, \hat{\bm l}, \text{aux})\rightarrow  m^{\st\mathcal{(B)}}_{\st1}, \ \ \
%
\mathtt{Encode}(T_{\st 1}, m^{\st\mathcal{(B)}}_{\st1})\rightarrow \hat m^{\st\mathcal{(B)}}_{\st1}, \\
%
\mathtt{genPaymentRequest}(T, in_{\st f}, \hat{\bm{l}}, \hat m^{\st\mathcal{(B)}}_{\st1})\rightarrow \hat m^{\st\mathcal{(C)}}_{\st2}, \\
%

\mathcal{A}_{\st 2}(T,  pk, \text{aux}_{\st f}, \hat m^{\st\mathcal{(B)}}_{\st 1}, \hat m^{\st\mathcal{(C)}}_{\st2})\rightarrow ( m^{\st\mathcal{(B)}}_{\st2},  z, {\ddot \pi}),\ \ \
%
\mathtt{Encode}(T_{\st 1}, m^{\st\mathcal{(B)}}_{\st 2})\rightarrow \hat m^{\st\mathcal{(B)}}_{\st2}, \\
%
\mathtt{Encode}(T_{\st 1},  z)\rightarrow \hat z, \ \ \
%
{\bar{ \mathtt{Encode}}}(pk_{\st\mathcal{D}}, \ddot \pi)\rightarrow  \hat{\ddot \pi}, \\
%
%
%\mathtt{makePayment}(T, \hat m^{\st\mathcal{(C)}}_{\st2})\rightarrow \hat m^{\st\mathcal{(B)}}_{\st2}\\
%%
%\mathtt{genComplaint}(\hat m^{\st\mathcal{(B)}}_{\st 1}, \hat m^{\st\mathcal{(B)}}_{\st2}, T, pk, \text{aux}_{\st f_{_\gamma}})\rightarrow (\hat z, \hat{\ddot \pi})\\
%
 \forall j, j\in [n]:
\Big(\mathtt{verComplaint}( \hat z, \hat{\ddot{\pi}}, g, \hat{\bm m}, \hat{\bm{l}}, j, sk_{\st\mathcal D}, aux, pp)\rightarrow \hat{\bm{w}}_{\st j}
\Big)
, \ \ \
\mathtt{resDispute}(T_{\st 2}, \hat {\bm w}, pp)\rightarrow \bm v
\\
%
%\mathtt{resDispute}(\hat{m}_{\st 1}^{\st(\mathcal C)}, \hat {m}_{\st 2}^{\st(\mathcal C)}, \hat{m}_{\st 2}^{\st(\mathcal B)}, T, \hat {\bm w})\rightarrow \bar{\bm v}\\
%
   \end{array} 
$$
\end{mybox}
%\end{minipage}}
\end{center}
%
it holds that:
%
%%%%%%%%%%%%%%%%%%%%%%%%%%%%%
$$ \Pr\left[
  \begin{array}{l}
  
 
\mathcal{A}_{\st 3}(T_{\st 2}, pk, pp, g, \hat{\bm m}, \hat{\bm l},  \hat z, \hat{\ddot \pi}, \hat{\bm{w}}, \bm v)\rightarrow w_{\st j}
\end{array} :
    \begin{array}{l}
    \mathsf{Exp}_{\st 4}^{\st\mathcal{A}_{\st 2}}(\mathsf{input})\\
\end{array}    \right]\leq {Pr}'+\mu(\lambda),$$
%}
where  $g:=(g_{\st 1}, g_{\st 2})\in T$,   $\hat {\bm m}=[\hat {m}^{\st (\mathcal{C})}_{\st 1}, \hat {m}^{\st (\mathcal{C})}_{\st 2}, \hat {m}^{\st (\mathcal{B})}_{\st 1}, \hat {m}^{\st (\mathcal{B})}_{\st 2}], \hat {\bm w}=[\hat {\bm w}_{\st 1},.., \hat{\bm w}_{\st n}]$, and $\mathsf{input}:=(1^{\st\lambda}, aux)$. Let arbiter $\mathcal{D}_{\st i}$ output $0$ and $1$ with probabilities $Pr_{\st i,0}$ and $Pr_{\st i,1}$ respectively. Then, $Pr'$ is defined as  $Max\{Pr_{\st 1,0},Pr_{\st 1,1}, ..., Pr_{\st n,0}, $ $Pr_{\st n,1} \}$. %In the above privacy definition, the probability is taken over the uniform choice of $sk$, the probability that each  $\mathcal{D}_{\st j}$ outputs $0$ or $1$, the randomness used in the blockchain's primitives,  the randomness used during the encoding, and  the randomness of $\mathcal{A}_{\st 1}$ in $\mathsf{Exp}_{\st 3}^{\mathcal{A}_{\st 1}}$ and $\mathcal{A}_{\st 2}$ in $\mathsf{Exp}_{\st 4}^{\mathcal{A}_{\st 2}}$. 


\end{enumerate}
\end{definition}


\vspace{-3.5mm}
\begin{definition}[Security]\label{def-a::PwDR-security}
A $\mathsf{pwdr}$ scheme is secure if it meets security against a malicious victim,  security against a malicious bank, and preserves privacy with respect to definitions \ref{def-a::Security-against-malicious-victim}, \ref{def-a::Security-against-malicious-bank}, and \ref{def-a::privacy} respectively. 
\end{definition}


%Definition 13 (RC-PoR-P Security). A RC-PoR-P scheme is secure if it satisfies security against a malicious server, security against a malicious client, and preserves privacy, w.r.t. Definitions 10-12.

%\begin{mybox}[colback=white]{Title}
%This is my box.
%\end{mybox}

%{\small
%\begin{center}
%\fbox{\begin{minipage}{3.8 in}
%$$
%  \begin{array}{l}
%  \mathsf{Exp}_{\st 1}^{\mathcal{A}}(\mathsf{In}:=(1^{\st\lambda}, p, \bm{l})): \\
%  
%  \ \  \mathtt{keyGen}(1^{\st\lambda})\rightarrow (sk,pk)\\
%  
%   \ \   \mathcal{A}(1^{\st\lambda}, p, \bm{l}, pk)\rightarrow m^{\st\mathcal{(C)}}_{\st 1}\\
%   
%   
%    \ \ \mathtt{insertNewPayee}(m^{\st\mathcal{(C)}}_{\st 1}, {\bm l})\rightarrow {\bm l'}\\
%   
%   
%   \ \ \mathtt{genWarning}( {\bm l'}, {aux})\rightarrow m^{\st\mathcal{(B)}}_{\st1}\\
%   
%
%   \ \   \mathcal{A}(in_{\st p}, \bm{l}', m^{\st\mathcal{(B)}}_{\st1})\rightarrow m^{\st\mathcal{(C)}}_{\st 2}\\
%
%\ \ \mathtt{makePayment}(m^{\st\mathcal{(C)}}_{\st2})\rightarrow m^{\st\mathcal{(B)}}_{\st2}\\
%  
%
%
% \ \   \mathcal{A}(m^{\st\mathcal{(B)}}_{\st1},m^{\st\mathcal{(B)}}_{\st 2}, aux')\rightarrow z:=(k,x,y)\\
%
%
%\ \ \forall i, i\in [n]:\\
%\ \  \Big(\mathtt{verComplaint}( z, m^{\st\mathcal{(B)}}_{\st1},i)\rightarrow (d_{\st i}, v_{\st i}, \bar v_{\st i},  w_{\st i})\Big)\\
%
%\ \ \mathtt{resDispute}(m_{\st 1}^{\st(\mathcal C)}, m_{\st 2}^{\st(\mathcal C)}, m_{\st 1}^{\st(\mathcal B)}, m_{\st 2}^{\st(\mathcal B)}, z, \bm{d}, \bm{v}, \bm{\bar v}, \bm{w})\rightarrow c\\
%   \end{array} 
%$$
%\end{minipage}}
%\end{center}
%}













