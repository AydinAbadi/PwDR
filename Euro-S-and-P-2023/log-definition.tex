%!TEX root = main.tex






\section{Full Version of Definition of Payment with Dispute Resolution Scheme}\label{sec::long-def}



%We first present a formal definition of a certificate scheme. We present  a generic definition that can capture both digital and hard copy certificates to reflect the real-world setting in which both kinds of certificates are being used. 


%\begin{definition}\label{sec::def}
%A certificate scheme  involves three sub-processes, $\mathtt{Cer}:=(\mathtt{Cer.genPar}, $ $\mathtt{Cer.gen}, \mathtt{Cer.ver})$, that are defined as follows.
%
%\begin{itemize} 
%\item[$\bullet$] $\mathtt{Cer.genPar}(1^{\st \lambda})\rightarrow (sk,pk)$.  A process executed by  a certificate generator. It takes as input a security parameter. It outputs a parameter pair: $(sk,pk)$, consisting of secret: $sk$, and public: $pk$ parameters. 
%\item[$\bullet$] $\mathtt{Cer.genCrt}(sk, pk, u)\rightarrow crt$. A process executed by  a certificate generator. It takes as input  parameter pair: $(sk,pk)$ and a file: $u$. It outputs a certificate: $crt$.
%\item[$\bullet$]  $\mathtt{Cer.verCrt}( pk, u, crt)\rightarrow h\in\{0,1\}$. A process run by a verifier. It takes as input  public parameter: $pk$,  file: $u$, and certificate: $crt$. It checks the certificate validity.   If the verification passes, then it outputs $1$; otherwise, it outputs $0$. 
%\end{itemize}
%\end{definition}
%
%The $\mathtt{Cer}$ should meet the following properties:
%
%\begin{itemize} 
%\item[$\bullet$]  \textit{Correctness.} For every input $u$ it holds that:
%
%$$Pr\Big[\  \  \mathtt{Cer.verCrt}( pk, u, \mathtt{Cer.genCrt}(sk, pk, u))=1\ : \
%\mathtt{Cer.genPar}(1^{\st \lambda})\rightarrow (sk, pk)  \Big]=1$$
%
%\item[$\bullet$] \textit{Existential unforgeability under chosen message attacks.} A (PPT) adversary that obtains $pk$ and has access to a certificate generation oracle for messages of its choice, cannot create a valid pair $(u^{\st *},crt^{\st *})$ for a new file $u^{\st *}$  (that was never a query to the signing oracle), except with a small probability, $\sigma$. More formally: 
%
%
%
%
%\small{
%$$ \Pr\left[
%  \begin{array}{l}
%  
%  u^{\st *}\not\in Q\ \wedge \\
%   \mathtt{Cer.verCrt}( pk,  u^{\st *}, crt^{\st *}) =1\\
%  
%  
%    
%%(M(u^{\scriptscriptstyle *},k)\neq \sigma \lor Q({aux},k)\neq q) \wedge\\ (a=1 \ \vee b=1)
%\end{array} : 
%    \begin{array}{l}
%   
%    \mathtt{Cer.genPar}(1^{\st \lambda})\rightarrow (sk,pk) \\
%  \mathcal{A}^{\mathtt{Cer.genCrt}(k,)}(pk)\rightarrow(u^{\st *},crt^{\st *})
%
%     
%\end{array}    \right]\leq \sigma$$
%}
%where $Q$ is the set of queries that $\mathcal{A}$ sent to the certificate generator oracle.
%\end{itemize}
%
%
%
%In the case where a digital certificate is considered, then $\mathtt{Cer}$ definition is equivalent to the definition of a digital signature scheme. In this case, it holds $\sigma=neg(\lambda)$, where $neg$ is a negligible function and $\lambda$ is a security parameter. Now, we briefly explain the procedures' input/output when a hard copy certificate is considered.  The process $\mathtt{Cer.genPar}(.)$ outputs a blank legitimate stamped certificate as a private parameter: $sk$, and the description of a standard legitimate certificate as a public parameter: $pk$. Morevoer, the process $\mathtt{Cer.genCrt}(k, u)$ takes $k$ and the file $u$ on which a certificate should be generated and outputs a stamped certificate with the information printed on it. The process $\mathtt{Cer.verCrt}( pk, u, crt)$ takes the public parameter, the file,  and the hard copy of the certificate and outputs $1$ if it is valid and $0$ if it is not. In the case where a hard copy certificate is considered, it is not possible to precisely define the probability $\sigma$, as it depends on the technology available to the adversary to generate a blank stamped certificate that looks like a legitimate one. In the real world however, this probability is usually small (but it may not be negligible). 


In this section, we present a full formal definition of payment with dispute resolution. To allow this section to be self-contained, we repeat some content from Section \ref{sec::def}. Below, we first provide the scheme's syntax.  Then, we formally define its correctness and security properties. 


%\label{def::sytax}

\begin{definition}
%\begin{definition} 
A payment with dispute resolution ($\mathsf{pwdr}$) involves  six   types of entities; namely,  bank $\mathcal{B}$, customer $\mathcal{C}$,  smart contract $\mathcal{S}$,  certificate generator $\mathcal{G}$,   set of auditors $\mathcal{D}:\{\mathcal{D}_{\st 1},..., \mathcal{D}_{\st n}\}$, and  dispute resolver $\mathcal{DR}$. It includes the following   algorithms $\mathsf{pwdr}:=(\mathtt{keyGen}, $ $\mathtt{bankInit}, \mathtt{customerInit},  \mathtt{genUpdateRequest}, \\ \mathtt{insertNewPayee}, \mathtt{genWarning},$ $ \mathtt{genPaymentRequest}, $ $\mathtt{makePayment}, $ $ \mathtt{genComplaint}, $ $\mathtt{verComplaint},$ $ \mathtt{resDispute})$. Below, we define these algorithms.  

%\item \item[$\bullet$] $\mathtt{setup}(1^{\st \lambda})\rightarrow k:=(sk,pk)$. %%% this might be needed when other security properties, e.g., privacy, are added to the scheme. 
%

%\vspace{2mm}


\item [$\bullet$]  $\mathtt{keyGen}(1^{\st \lambda})\rightarrow (sk,pk)$. It is a probabilistic  algorithm run independently by  $\mathcal{G}$ and one of the auditors, $\mathcal{D}_{\st j}$. It takes as input a security parameter $1^{\st \lambda}$. It outputs a pair of secret keys $sk:=(sk_{\st\mathcal {G}}, sk_{\st\mathcal {D}})$ and public keys $pk:=(pk_{\st\mathcal {G}},pk_{\st\mathcal {D}})$, where $sk_{\st\mathcal {D}}$ may contain multiple secret keys. The public key pair, $pk$, is sent to all participants. 

%\vspace{2mm}



\item[$\bullet$] $\mathtt{bankInit}(1^{\st \lambda})\rightarrow (T, pp, \bm{l})$. It is  run by  $\mathcal{B}$. It takes as input  security parameter $1^{\st \lambda}$.  It sets private parameters   $\ddot\pi_{\st 1}$ and $\ddot\pi_{\st 2}$. It generates an encoding-decoding token $T$, where  $T:=(T_{\st 1},T_{\st 2})$,  each $T_{\st i}$  contains  a secret parameter $\ddot\pi_{\st i}$ and its public witness $g_{\st i}$.    Given a  value and its witness anyone can check if they match. It also generates a set of (additional) public parameters,  $pp$, one of which is $e$ that is a threshold parameter.  It also generates an  empty list, $\bm{l}$. It outputs $T, pp$, and $\bm l$.  $\mathcal{B}$ sends $(\ddot\pi_{\st 1}, \ddot\pi_{\st 2})$ to $\mathcal{C}$ and sends $(g_{\st 1}, g_{\st 2},  pp, \bm {l})$ to $\mathcal{S}$.



%\vspace{2mm}

\item[$\bullet$] $\mathtt{customerInit} (1^{\st \lambda}, T, pp)\rightarrow a$. It is a deterministic  algorithm run by  $\mathcal{C}$. It takes as input security parameter $1^{\st \lambda}$,  token $T$, and  set $pp$ of public parameters. It checks the correctness of the elements in $T$ and $pp$. If the checks pass, it outputs $1$. Otherwise, it outputs $0$. 

%\vspace{2mm}


\item [$\bullet$] $\mathtt{genUpdateRequest}(T, f, {\bm l})\rightarrow \hat m^{\st\mathcal{(C)}}_{\st 1}$.  It is a deterministic algorithm run by $\mathcal{C}$.  It takes as input token $T$,  new payee's detail $f$, and payees' list $ {\bm l}$. It  generates $m^{\st\mathcal{(C)}}_{\st 1}$ which is an  update request to the payees' list. It uses $T_{\st 1}\in T$ and encoding algorithm $\mathtt{Encode}(T_{\st 1}, .)$ to encode $m^{\st\mathcal{(C)}}_{\st 1}$ which results in $\hat {m}^{\st\mathcal{(C)}}_{\st 1}$. It outputs $\hat  m^{\st\mathcal{(C)}}_{\st 1}$. Party $\mathcal{C}$ sends the output to $\mathcal{S}$.
%

%\vspace{2mm}


\item [$\bullet$] $\mathtt{insertNewPayee}(\hat{m}^{\st\mathcal{(C)}}_{\st 1}, {\bm l})\rightarrow  \hat{\bm l}$. It is a deterministic algorithm run by $\mathcal{S}$. It takes as input $\mathcal{C}$'s encoded update request  $\hat{m}^{\st\mathcal{(C)}}_{\st 1}$, and $\mathcal{C}$'s payees' list ${\bm l}$. It inserts the new payee's detail into ${\bm l}$ and outputs an updated list, $\hat{\bm l}$.
%

%\vspace{2mm}


\item  [$\bullet$] $\mathtt{genWarning}(T, \hat{\bm l}, {aux})\rightarrow \hat m^{\st\mathcal{(B)}}_{\st1}$. It is run by $\mathcal{B}$. It takes as input token $T$, $\mathcal{C}$'s encoded   payees' list $ \hat{\bm l}$, and auxiliary information: ${aux}$, e.g., a set of policies. Using $T_{\st 1}\in T$, it decodes and checks  all elements of the list, e.g., whether they comply with the policies. If the check  passes,  it sets $m^{\st\mathcal{(B)}}_{\st1}=\text{``pass''}$; otherwise, it sets $m^{\st\mathcal{(B)}}_{\st1}={warning}$, where the $warning$ is a string  containing a warning detail along with
the string ``warning''. It uses  $T_{\st 1}$ and $\mathtt{Encode}(T_{\st 1}, .)$ to encode $m^{\st\mathcal{(B)}}_{\st1}$ which yields $\hat m^{\st\mathcal{(B)}}_{\st1}$. It outputs  $\hat m^{\st\mathcal{(B)}}_{\st1}$.  Party $\mathcal{B}$ sends $\hat m^{\st\mathcal{(B)}}_{\st1}$ to $\mathcal{S}$.
%

%\vspace{2mm}


\item  [$\bullet$] $\mathtt{genPaymentRequest}(T, in_{\st f}, \hat{\bm{l}}, \hat m^{\st\mathcal{(B)}}_{\st1})\rightarrow \hat m^{\st\mathcal{(C)}}_{\st2}$. It is  run by $\mathcal{C}$. It takes as input token $T$, a payment detail $in_{\st f}$, encoded payees' list $\hat{\bm{l}}$, and encoded warning message,  $\hat m^{\st\mathcal{(B)}}_{\st1}$. Using $T_{\st 1}\in T$, it decodes  $\hat{\bm{l}}$ and $\hat m^{\st\mathcal{(B)}}_{\st1}$ yielding ${\bm{l}}$ and $m^{\st\mathcal{(B)}}_{\st1}$ respectively. It checks the warning.  It sets   $m^{\st\mathcal{(C)}}_{\st2}=\phi$, if it does not want to proceed. Otherwise, it  sets $m^{\st\mathcal{(C)}}_{\st2}$ according to the content of $in_{\st f}$ and  ${\bm{l}}$ (e.g., the amount of payment and payee's detail). It uses $T_{\st 1}$ and  $\mathtt{Encode}(T_{\st 1}, .)$ to encode $m^{\st\mathcal{(C)}}_{\st2}$ resulting in $\hat m^{\st\mathcal{(C)}}_{\st2}$. It outputs $\hat m^{\st\mathcal{(C)}}_{\st2}$.  Party $\mathcal{C}$ sends $\hat m^{\st\mathcal{(C)}}_{\st2}$ to $\mathcal{S}$. 
%

%\vspace{2mm}


\item  [$\bullet$] $\mathtt{makePayment}(T, \hat m^{\st\mathcal{(C)}}_{\st2})\rightarrow \hat m^{\st\mathcal{(B)}}_{\st2}$. It is a deterministic algorithm run by $\mathcal{B}$. It takes as input token $T$,  and encoded payment  detail $\hat m^{\st\mathcal{(C)}}_{\st2}$. Using $T_{\st 1}\in T$, it decodes $\hat m^{\st\mathcal{(C)}}_{\st2}$ and  checks the result's validity, e.g., ensures it is well-formed or $\mathcal{C}$ has enough credit. If the check  passes,  it makes the payment and  sets $m^{\st\mathcal{(B)}}_{\st2}=\text{``paid''}$. Otherwise, it sets $m^{\st\mathcal{(B)}}_{\st2}=\phi$. It uses $T_{\st 1}$ and  $\mathtt{Encode}(T_{\st 1}, .)$ to encode $m^{\st\mathcal{(B)}}_{\st2}$ yielding $\hat m^{\st\mathcal{(B)}}_{\st2}$. It outputs $\hat m^{\st\mathcal{(B)}}_{\st2}$. Party $\mathcal{B}$ sends $\hat m^{\st\mathcal{(B)}}_{\st2}$ to $\mathcal{S}$.
%

%\vspace{2mm}


\item  [$\bullet$] $\mathtt{genComplaint}(\hat m^{\st\mathcal{(B)}}_{\st 1}, \hat m^{\st\mathcal{(B)}}_{\st2}, T, pk, {aux}_{\st f})\rightarrow (\hat z, \hat{\ddot \pi})$. It is run by $\mathcal{C}$. It takes as input   the encoded warning message $\hat m^{\st\mathcal{(B)}}_{\st1}$, encoded payment message $\hat m^{\st\mathcal{(B)}}_{\st2}$,  token $T$, public key $pk$, and  auxiliary information ${aux}_{\st f}$. It initially sets fresh strings $(z_{\st 1}, z_{\st 2}, z_{\st 3})$ to null. Using $T_{\st 1}\in T$, it decodes  $\hat m^{\st\mathcal{(B)}}_{\st 1}$ and $\hat m^{\st\mathcal{(B)}}_{\st2}$ and checks the results' content. If it wants to complain that (i)   “pass” message should have been a warning  or (ii) no  message was provided,   it sets $z_{\st 1}$ to ``challenge message''. If its complaint is about the warning's effectiveness,  it sets $z_{\st 2}$ to a combination of an evidence $u\in aux_{\st f}$, the evidence's certificate $sig\in aux_{\st f}$, the certificate's public parameter,   and ``challenge warning'', where the certificate is obtained from  $\mathcal{G}$ via a query, $Q$. In certain cases, the certificate might be empty.  If its complaint is  about the   payment, it sets $z_{\st 3}$ to ``challenge payment''.  It uses $T_{\st 1}$ and  $\mathtt{Encode}(T_{\st 1}, .)$ to encode $z:=(z_{\st 1}, z_{\st 2}, z_{\st 3})$ and uses $pk_{\st\mathcal D}$ and another encoding algorithm  $\bar{\mathtt{Encode}}(pk_{\st\mathcal D}, .)$ to encode  $\ddot \pi:=(\ddot \pi_{\st 1},\ddot \pi_{\st 2})\in T$. This  results in $\hat z$ and $\hat {\ddot \pi}$ respectively. It outputs $(\hat z, \hat {\ddot \pi})$. Party $\mathcal{C}$ sends the pair to $\mathcal{S}$.
%
%%
%\item  [$\bullet$] $\mathtt{genComplaint}( m^{\st\mathcal{(S)}}_{\st1},m^{\st\mathcal{(S)}}_{\st2}, {aux}')\rightarrow z:=(k,x,y)$. It is run by $\mathcal{C}$. It takes as input  the warning message: $m^{\st\mathcal{(S)}}_{\st1}$, payment message: $m^{\st\mathcal{(S)}}_{\st2}$, and some auxiliary information: ${aux}'$. If $\mathcal{C}$ wants to complain that (i)  the “pass” message (given by B) should have been a warning  or (ii) no  message was provided at all, then it sets $k$ to ``challenge message''. Otherwise, it sets $k=\phi$. If $\mathcal{C}$'s complaint is about the warning effectiveness, then it sets $x$ to a combination of an evidence  and ``challenge warning''.  Otherwise, it sets $x=\phi$. If its complaint is  about the   payment, it sets $y$ to {``challenge payment''}; otherwise, it sets $y=\phi$. It sends $z:=(k,x, y)$ to  $\mathcal{S}$.
%

%\vspace{2mm}


\item  [$\bullet$] $\mathtt{verComplaint}( \hat z, \hat{\ddot{\pi}}, g, \hat{\bm m}, \hat{\bm{l}}, j, sk_{\st\mathcal D}, aux, pp)\rightarrow \hat{\bm{w}_{\st j}}$. It is run by every auditor $\mathcal{D}_{\st  j}$. It takes as input  $\mathcal{C}$'s encoded complaint $\hat z$,  encoded private parameters $\hat{\ddot{\pi}}$,  the tokens' public parameters $g:=(g_{\st 1}, g_{\st 2})$,  encoded  messages $\hat{\bm m}=[\hat{m}_{\st 1}^{\st(\mathcal C)}$, $\hat{m}_{\st 2}^{\st(\mathcal C)}, \hat{m}_{\st 1}^{\st(\mathcal B)}, \hat{m}_{\st 2}^{\st(\mathcal B)}]$, encoded payees' list $ \hat{\bm{l}}$,  the  auditor's index $j$, secret key set $sk_{\st\mathcal D}$,   auxiliary information $aux$, and set of public parameters $pp$.  It uses a secret key in  $sk_{\st\mathcal D}$ to decode $\hat{\ddot{\pi}}$ that yields  $\ddot{\pi}:=(\ddot{\pi}_{\st 1}, \ddot{\pi}_{\st 2})$. It uses $\ddot{\pi}_{\st 1}$ to decode  $\hat z,  \hat{\bm{m}}$, and  $\hat{\bm{l}}$ that results in $z:=(z_{\st 1}, z_{\st 2}, z_{\st 3}), {\bm m}$, and  ${\bm{l}}$ respectively.  It checks if $\ddot{\pi}_{\st i}$ matches  $g_{\st i}$. If the check fails, it aborts.  It  checks  if ${m}_{\st 1}^{\st(\mathcal C)}$ and ${m}_{\st 2}^{\st(\mathcal C)}$ are non-empty; it aborts if the checks fail. It  sets  fresh parameters $(w_{\st 1,j}, w_{\st 2,j},  w_{\st 3,j},   w_{\st 4,j})$ to $0$. If ``challenge message'' $\in z_{\st 1}$, given $\bm l$,  it checks whether  ``pass''  message (in ${m}^{\st\mathcal{(B)}}_{\st1}$) was given correctly or the missing message did not play any role in preventing the fraud.  If either checks passes, it sets  $w_{\st 1,j}=0$;  otherwise,  it sets $w_{\st 1,j}=1$. If ``challenge warning'' $\in z_{\st 2}$,   it verifies  the certificate in $z_{\st 2}$.  If it is invalid,  it sets $ w_{\st 3,j}=0$.  If it is valid, it sets  $w_{\st 3,j}=1$. It  determines the effectiveness of the warning (in ${m}^{\st\mathcal{(B)}}_{\st1}$), by running an algorithm which  determines that, i.e., $\mathtt{checkWarning}(.)\in aux$. If  it is effective, i.e., $\mathtt{checkWarning}({m}^{\st\mathcal{(B)}}_{\st1})= 1$,  it sets its verdict to $0$, i.e., $w_{\st 2,j}=0$; otherwise, it sets $w_{\st 2,j}=1$.  If ``challenge payment'' $\in z_{\st 3}$,    it checks if the payment was made (with the help of ${m}_{\st 2}^{\st(\mathcal B)}$). If the check passes,  it sets $w_{\st 4,j}=1$; otherwise, it sets $w_{\st 4,j}=0$.  It uses (another) secret key in $sk_{\st\mathcal D}$ and $\ddot \pi_{\st 2}$ to encode  $\bm{w}_{\st j} = [w_{\st 1,j}, w_{\st 2, j}, w_{\st 3, j}, w_{\st 4, j}]$ yielding  $\hat{\bm{w}}_{\st j}= [\hat {w}_{\st 1,j}, \hat{w}_{\st 2, j}, \hat{w}_{\st 3, j}, \hat{w}_{\st 4, j}]$. It outputs $\hat{\bm{w}}_{\st j}$. Party $\mathcal{D}_{\st  j}$ sends $\hat{\bm{w}}_{\st j}$ to  $\mathcal{S}$. 
%

%\vspace{2mm}


\item  [$\bullet$] $\mathtt{resDispute}(T_{\st2}, \hat {\bm w}, pp)\rightarrow {\bm v}$. It is a deterministic algorithm run by $\mathcal {DR}$. It takes as input token $T_{\st 2}$,   auditors' encoded verdicts  $\hat{ \bm{w}}=[\hat{\bm{w}}_{\st 1},..., \hat{\bm{w}}_{\st n}]$, and  public parameters set $pp$. It checks if the token's parameters match. If the check fails, it aborts. It uses $\ddot \pi_{\st 2}\in T_{\st 2}$  to decode $\hat {\bm w}$ and from the result it extracts final verdicts $\bm v= [v_{\st 1},...,v_{\st 4}]$.  The extraction procedure ensures each $v_{\st i}$ is set to $1$ only if at least  $e$ auditors' original verdicts (i.e., $w_{\st i,j,}$) is $1$, where $e\in pp$.  It outputs $\bm v$. If $v_{\st 4}=1$ and (i) either $v_{\st 1}=1$ (ii) or $v_{\st 2}=1$ and $v_{\st 3}=1$, then  $\mathcal C$  is reimbursed.
\end{definition}



%
% $\hat{m}_{\st 1}^{\st(\mathcal C)}, \hat {m}_{\st 2}^{\st(\mathcal C)}$, and $\hat{m}_{\st 2}^{\st(\mathcal B)}$ which results in ${m}_{\st 1}^{\st(\mathcal C)},  {m}_{\st 2}^{\st(\mathcal C)}$, and ${m}_{\st 2}^{\st(\mathcal B)}$ respectively. It extracts from $\hat {\bm w}$ decoded combined verdicts $[v_{\st 1},...,v_{\st 4}]$. It sets $\bar v_{\st 0}=1$ and proceeds, if  $m_{\st 1}^{\st(\mathcal C)}$ and $m_{\st 2}^{\st(\mathcal C)}$ are  not empty.  Otherwise, it aborts. It sets $\bar v_{\st 1}=1$, if $v_{\st 1}$ indicates that the majority of the auditors believe that ``pass'' or missing message should have been a warning message.  It sets $\bar v_{\st 2}=1$, if $v_{\st 2}$ indicates that
%the majority of the auditors believe that the warning was ineffective.  It sets $\bar v_{\st 3}=1$, if $v_{\st 3}$ indicates that  the majority of the auditors believe that $\mathcal C$ did not provide an invalid certificate. It sets $\bar v_{\st 4}=1$, if  $v_{\st 4}$ (or ${m}_{\st 2}^{\st(\mathcal B)}$) indicates that  the majority of the auditors believe that the payment was made. It outputs $\bar{\bm v}=[\bar v_{\st 0},..., \bar v_{\st 4}]$.  Note, $\mathcal{C}$ is reimbursed only if $\bar v_{\st 0}=\bar v_{\st 3}=\bar v_{\st 4}=1$ and $\bar v_{\st 1}$ or $\bar v_{\st 2}$ is $1$. $\mathcal{DR }$ sends the output to  $\mathcal{C}$ and $\mathcal{B}$. 





%\item  [$\bullet$] $\mathtt{resDispute}(\hat{m}_{\st 1}^{\st(\mathcal C)}, \hat {m}_{\st 2}^{\st(\mathcal C)}, \hat{m}_{\st 2}^{\st(\mathcal B)}, T, \hat {\bm w})\rightarrow c$. It is a deterministic algorithm run by $\mathcal {DR}$. It takes as input, $\mathcal C$'s encoded  update request  $\hat m_{\st 1}^{\st(\mathcal C)}$, $\mathcal C$'s encoded payment request  $\hat m_{\st 2}^{\st(\mathcal C)}$,  $\mathcal S$'s encoded payment message $\hat m_{\st 2}^{\st(\mathcal B)}$,  token $T$, and auditors' encoded verdicts  $\hat{ \bm{w}}$. Checks if the token's parameters match,  i.e., $\mathtt{match}({\ddot{\pi}}, g)=1$. If the check fails, it aborts; otherwise, it continues. It decodes $\hat{m}_{\st 1}^{\st(\mathcal C)}, \hat {m}_{\st 2}^{\st(\mathcal C)}$, and $\hat{m}_{\st 2}^{\st(\mathcal B)}$ which results in ${m}_{\st 1}^{\st(\mathcal C)},  {m}_{\st 2}^{\st(\mathcal C)}$, and ${m}_{\st 2}^{\st(\mathcal B)}$ respectively. It extracts from $\hat {\bm w}$ decoded combined verdicts $[v_{\st 1},...,v_{\st 4}]$. It proceeds if  $m_{\st 1}^{\st(\mathcal C)}$ and $m_{\st 2}^{\st(\mathcal C)}$ are  not empty.  Otherwise, it aborts. It sets $c=1$ (i.e., $\mathcal{C}$ must be reimbursed), if $\mathcal C$ requested a money transfer, $v_{\st 4}$ (or ${m}_{\st 2}^{\st(\mathcal B)}$) indicates that the payment was made, $v_{\st 3}$ indicates that $\mathcal C$ did not provide an invalid certificate, and one the following  conditions holds (i) $v_{\st 2}$ indicates that the warning was ineffective, or (ii) $v_{\st 1}$ indicates that  ``pass'' or missing message should have been a warning message. Otherwise, it sets $c=0$. It returns $c$.





%Note, algorithms $\mathtt{genWarning}(.)$-$\mathtt{resDispute}(.)$ implicitly take another input, $state_{\st\mathcal S}$, which is $\mathcal{S}$'s current state, i.e., data registered in $\mathcal{S}$. For the sake of simplicity, we avoided explicitly passing this input  to these processes.  

Informally, $\mathsf{pwdr}$ has two  properties; namely, \emph{correctness} and \emph{security}. Correctness requires that, in the absence of a fraudster, the payment journey is completed without the need for the honest customer to complain and the honest bank to reimburse the customer. Below, we formally state it.  

\vspace{-6mm}

\begin{flushleft}
\begin{definition}[Correctness] A $\mathsf{pwdr}$ scheme is correct if the key generation algorithm produces keys $\mathtt{keyGen}(1^{\st \lambda})\rightarrow (sk,pk)$ such that for any payee's detail $f$, payment's detail $in_{\st f}$, and auxiliary information $(aux, aux_{\st f})$, if $ \mathtt{bankInit}(1^{\st \lambda})\rightarrow (T, pp, \bm{l})$, $\mathtt{customerInit} (1^{\st \lambda}, T, pp)\rightarrow a,$ $\mathtt{genUpdateRequest}(T, f, {\bm l})\rightarrow \hat m^{\st\mathcal{(C)}}_{\st 1},$ $\mathtt{insertNewPayee}(\hat{m}^{\st\mathcal{(C)}}_{\st 1}, {\bm l})\rightarrow  \hat{\bm l},$  $\mathtt{genWarning}(T, \hat{\bm l}, {aux})\rightarrow \hat m^{\st\mathcal{(B)}}_{\st1},$  $\mathtt{genPaymentRequest}(T, in_{\st f}, \hat{\bm{l}}, \hat m^{\st\mathcal{(B)}}_{\st1})\rightarrow \hat m^{\st\mathcal{(C)}}_{\st2},$  $\mathtt{makePayment}(T,  \hat {m}^{\st\mathcal{(C)}}_{\st2})\rightarrow \hat m^{\st\mathcal{(B)}}_{\st2},$ $\mathtt{genComplaint}(\hat m^{\st\mathcal{(B)}}_{\st 1}, \hat m^{\st\mathcal{(B)}}_{\st2}, T, pk, {aux}_{\st f})\rightarrow (\hat z, \hat{\ddot \pi}),$\\ $ 
\forall j\in [n]: \Big(\mathtt{verComplaint}( \hat z, \hat{\ddot{\pi}}, g, \bm{\hat{m}}, \hat{\bm{l}}, j, sk_{\st\mathcal D}, aux, pp)\rightarrow \hat{\bm{w}_{\st j}}\Big),$ $  
 \mathtt{resDispute}(T_{\st 2}, \hat {\bm w}, pp)\rightarrow {\bm v}$, then $(z_{\st 1}=z_{\st 2}=z_{\st 3}=\phi) \wedge({\bm{v}}=0)$, where $g:=(g_{\st 1}, g_{\st 2})\in T$,   $\hat {\bm m}=[\hat {m}^{\st (\mathcal{C})}_{\st 1}, \hat {m}^{\st (\mathcal{C})}_{\st 2}, \hat {m}^{\st (\mathcal{B})}_{\st 1}, \hat {m}^{\st (\mathcal{B})}_{\st 2}]$,   $\hat {\bm w}=[{\hat{\bm w}}_{\st 1},...,{\hat{\bm w}}_{\st n}]$,  and $z:=(z_{\st 1}, z_{\st 2}, z_{\st 3})$ is the result of  decoding  $\hat{z}$.
\end{definition}
\end{flushleft}



%Informally, security against a malicious victim requires that the victim of the APP scam which is not qualified for the reimbursement should not be reimbursed  (despite  it tries to be). More specifically, a corrupt victim cannot (a) make committee members, $\mathcal{D}_{\st j}$s,  conclude that $\mathcal{B}$ should have provided a warning, although $\mathcal{B}$ has done so, or (b) make $\mathcal{DR}$ conclude that the  ``pass'' message was incorrectly given or a vital warning message was missing despite the majority of $\mathcal{D}_{\st j}$s do not believe so, or (c) persuade $\mathcal{D}_{\st j}$s to  conclude that the warning was ineffective although it was effective, or (d)  make $\mathcal{DR}$ believe that the warning message was ineffective although the majority of $\mathcal{D}_{\st j}$s do not believe that, or (e)  convince  $\mathcal{D}_{\st j}$s to accept an invalid certificate, or (f) make $\mathcal{DR}$ believe that the majority of $\mathcal{D}_{\st j}$s accepted the certificate  although they did not, except for a negligible probability. 

\vspace{-1mm}

A $\mathsf{pwdr}$ scheme is secure if it satisfies three main properties; namely, (a) security against a malicious victim, (b) security against a malicious bank, and (c) privacy. Below, we formally define them. Intuitively, security against a malicious victim requires that the victim of an APP fraud that is not qualified for the reimbursement should not be reimbursed  (despite it tries to be). More specifically, a corrupt victim cannot (a) make at least the threshold of the committee members, $\mathcal{D}_{\st j}$s,  conclude that $\mathcal{B}$ should have provided a warning, although $\mathcal{B}$ has done so, or (b) make $\mathcal{DR}$ conclude that   the pass message was incorrectly given or a vital warning message was missing despite only less than the threshold of $\mathcal{D}_{\st j}$s  believing so, or (c) persuade at least the threshold of $\mathcal{D}_{\st j}$s to  conclude that the warning was ineffective although it was effective, or (d)  make $\mathcal{DR}$ believe that the warning message was ineffective although only less than the threshold of $\mathcal{D}_{\st j}$s   believe that, or (e)  convince  $\mathcal{D}_{\st j}$s to accept an invalid certificate, or  (f) make $\mathcal{DR}$ believe that at least the  threshold of $\mathcal{D}_{\st j}$s accepted the certificate  although they did not, except for a negligible probability. Below, we formally state it. 


% (f) make $\mathcal{DR}$ believe that all $\mathcal{D}_{\st j}$s accepted the certificate  although they did not, except for a negligible probability. 



\vspace{-1mm}



\begin{definition}[Security against a malicious victim]\label{def::Security-against-malicious-victim} A $\mathsf{pwdr}$ scheme is secure against a malicious victim, if for any security parameter $\lambda$,  auxiliary information $aux$, and   probabilistic polynomial-time adversary $\mathcal{A}$, there exists a negligible function $\mu(\cdot)$, such that for an experiment $\mathsf{Exp}_{\st 1}^{\mathcal{A}}$:

%for functions $ M, Q ,E$, and $D$, if   for every $j$ (where $1\leq j\leq z$), and any probabilistic polynomial time adversary $\mathcal{A}$, there exists a negligible function $\mu(\cdot)$, such that for any security parameter $\lambda$ and experiment $\mathsf{Exp}_{\st 2}^{\mathcal{A},j}$:



\begin{center}
{\small{
\begin{mybox}[colback=white, width=80mm, height=52mm, left=0mm, drop fuzzy shadow southwest]{$\mathsf{Exp}_{\st 1}^{\st\mathcal{A}}(1^{\st\lambda}\text{, }  {aux})$}
%\fbox{\begin{minipage}{3.8 in}
$$
  \begin{array}{l}
%  \mathsf{Exp}_{\st 1}^{\mathcal{A}}(\mathsf{In}:=(1^{\st\lambda}, p, \bm{l})): \\
  
    \mathtt{keyGen}(1^{\st\lambda})\rightarrow (sk,pk) \\
  
  
  \mathtt{bankInit}(1^{\st \lambda})\rightarrow (T, pp, \bm{l}) \\
  
      \mathcal{A}( 1^{\st \lambda}, T, pp, \bm{l} )\rightarrow \hat m^{\st\mathcal{(C)}}_{\st 1} \\
   
\mathtt{insertNewPayee}(\hat{m}^{\st\mathcal{(C)}}_{\st 1}, {\bm l})\rightarrow \hat{\bm l} \\

\mathtt{genWarning}(T, \hat{\bm l}, {aux})\rightarrow \hat m^{\st\mathcal{(B)}}_{\st1} \\



   \mathcal{A}(T,  \hat{\bm{l}}, \hat m^{\st\mathcal{(B)}}_{\st1})\rightarrow \hat m^{\st\mathcal{(C)}}_{\st2}\\
   
   
   \mathtt{makePayment}(T, \hat m^{\st\mathcal{(C)}}_{\st2})\rightarrow \hat m^{\st\mathcal{(B)}}_{\st2}\\
   


  
\mathcal{A}(\hat m^{\st\mathcal{(B)}}_{\st 1}, \hat m^{\st\mathcal{(B)}}_{\st2}, T, pk)\rightarrow (\hat z, \hat{\ddot \pi})\\


 \forall j, j\in [n]:\\
\hsp\Big(\mathtt{verComplaint}( \hat z, \hat{\ddot{\pi}}, g, \hat{\bm{m}}, \hat{\bm{l}}, j, sk_{\st\mathcal D}, aux, pp)\rightarrow \hat{\bm{w}}_{\st j}\Big)\\


\mathtt{resDispute}(T_{\st 2}, \hat {\bm w}, pp)\rightarrow \bm v=[v_{\st 1},..., v_{\st 4}]\\

   \end{array} 
$$
\end{mybox}
%\end{minipage}}
}}
\end{center}


it holds that the following probability is $\mu(\lambda)$:

{\small{
$$ \Pr\left[
  \begin{array}{l}
%
%\Bigg(
%
\hs  \Big((m^{\st\mathcal{(B)}}_{\st1}=warning) \wedge (\sum\limits_{\st j=1}^{\st n}w_{\st 1,j}\geq {e})\Big)
  %
  \vee\\
  \hs  \Big((\sum\limits_{\st j=1}^{\st n}w_{\st 1,j}<{e}) \wedge ( {v}_{\st 1}=1)\Big)
 % \Bigg)
 
  %
\vee\\
  %\Bigg(
  \hs   \Big((\mathtt{checkWarning}(m^{\st\mathcal{(B)}}_{\st1})= 1) \wedge\\
  \hs  (\sum\limits_{\st j=1}^{\st n}w_{\st 2,j}\geq {e})\Big)
  %
  \vee\\
  \hs  \Big((\sum\limits_{\st j=1}^{\st n}w_{\st 2,j}< {e}) \wedge ({v}_{\st 2}=1)\Big)
  %\Bigg)
 
  %
  \vee \\
  %\Bigg(
\hs   \Big(u\notin Q \wedge\mathtt{Sig.ver}( pk,  u, sig) =1\Big) 
  %
  \vee\\
  \hs  \Big((\sum\limits_{\st j=1}^{\st n}w_{\st 3,j}< {e}) \wedge ( {v}_{\st 3}=1)\Big)
  %
  %\Bigg)
  \\
  %
 


  
  




%\text{s.t.}\\
%y_{\st j}= E^{\st -1}_{\st 2}(y^{\st *}_{\st j},t_{\st qp})\\
%E^{\st -1}_{\st 2},t_{\st qp}\in en\\
\end{array} \hs:
    \begin{array}{l}
   \hs  \mathsf{Exp}_{\st 1}^{\st\mathcal{A}}(\mathsf{input})\\
\end{array}    \right]$$
}}
where $g:=(g_{\st 1}, g_{\st 2})\in T$,  $\hat {\bm m}=[\hat {m}^{\st (\mathcal{C})}_{\st 1}, \hat {m}^{\st (\mathcal{C})}_{\st 2}, \hat {m}^{\st (\mathcal{B})}_{\st 1}, \hat {m}^{\st (\mathcal{B})}_{\st 2}]$, $(w_{\st 1,j},...,w_{\st 3,j})$ are the result of decoding   $(\hat w_{\st 1,j},...,\hat w_{\st 3,j})\in \hat{\bm w}_{\st j}\in \hat{\bm w}$,  $\mathtt{checkWarning}(.)$ determines a warning's effectiveness,   $\mathsf{input}:=(1^{\st\lambda}, {aux})$,   $ sk_{\st \mathcal{D}}\in sk$, and $n$ is the total number of auditors. The probability is taken over the uniform choice of $sk$, randomness used in the blockchain's primitives (e.g., in signatures), randomness used during the encoding,   and  the randomness of $\mathcal{A}$. 
\end{definition}

%\vspace{-3mm}

Note, in the above experiment, we did not explicitly pass payment detail $(f, in_{\st f})$ and auxiliary information ${aux}_{\st f}$ on to the adversary, to let the adversary pick them (on the client's behalf).  Intuitively, security against a malicious bank requires that a malicious bank should not be able to disqualify an honest victim of an APP fraud from being reimbursed. Specifically,  a corrupt bank cannot  (a) make $\mathcal{DR}$ conclude that the  ``pass'' message was correctly given or an important warning message was not missing despite at least the threshold of $\mathcal{D}_{\st j}$s  do not believe so, or (b) convince $\mathcal{DR}$  that the warning message was effective although at least the threshold of $\mathcal{D}_{\st j}$s do not believe so, or (c) make $\mathcal{DR}$ believe that less than the threshold of $\mathcal{D}_{\st j}$s did not accept the certificate although at least the threshold of them did that, or (d) make $\mathcal{DR}$ believe that no payment was made, although at least the threshold of $\mathcal{D}_{\st j}$s believe the opposite, except for a negligible probability. Below, we formally state it.

\begin{definition}[Security against a malicious bank]\label{def::Security-against-malicious-bank} A $\mathsf{pwdr}$ scheme is secure against a malicious bank, if for any security parameter $\lambda$, auxiliary information $ {aux}$, and   probabilistic polynomial-time adversary $\mathcal{A}$, there exists a negligible function $\mu(\cdot)$, such that for an experiment $\mathsf{Exp}_{\st 2}^{\mathcal{A}}$:

%\

\vspace{-2mm}

\begin{center}
{\small{
\begin{mybox}[colback=white,   width=80mm, height=55mm, left=-1mm, drop fuzzy shadow southwest]{$\mathsf{Exp}_{\st 2}^{\st\mathcal{A}}(1^{\st\lambda}\text{, }  {aux})$}
%\fbox{\begin{minipage}{3.8 in}
$$
  \begin{array}{l}
%
 \mathtt{keyGen}(1^{\st\lambda})\rightarrow (sk,pk)\\
%
\mathcal{A}(1^{\st \lambda})\rightarrow (T, pp, \bm{l}, f,  in_{\st f}, {aux}_{\st f})\\
%
\mathtt{customerInit} (1^{\st \lambda}, T, pp)\rightarrow a\\
%
\mathtt{genUpdateRequest}(T, f, {\bm l})\rightarrow \hat m^{\st\mathcal{(C)}}_{\st 1}\\
%
\mathtt{insertNewPayee}(\hat{m}^{\st\mathcal{(C)}}_{\st 1}, {\bm l})\rightarrow  \hat{\bm l}\\
%
\mathcal{A}(T, \hat{\bm l}, {aux})\rightarrow \hat m^{\st\mathcal{(B)}}_{\st1}\\
%
\mathtt{genPaymentRequest}(T, in_{\st f}, \hat{\bm{l}}, \hat m^{\st\mathcal{(B)}}_{\st1})\rightarrow \hat m^{\st\mathcal{(C)}}_{\st2}\\
%
\mathcal{A}(T, \hat m^{\st\mathcal{(C)}}_{\st2})\rightarrow \hat m^{\st\mathcal{(B)}}_{\st2}\\
%
\mathtt{genComplaint}(\hat m^{\st\mathcal{(B)}}_{\st 1}, \hat m^{\st\mathcal{(B)}}_{\st2}, T, pk, {aux}_{\st f})\rightarrow (\hat z, \hat{\ddot \pi})\\
%
 \forall j, j\in [n]:\\
\hsp\Big(\mathtt{verComplaint}( \hat z, \hat{\ddot{\pi}}, g, \hat{\bm m}, \hat{\bm{l}}, j, sk_{\st\mathcal D}, aux, pp)\rightarrow \hat{\bm{w}}_{\st j}
\Big)
\\
%
\mathtt{resDispute}(T_{\st 2}, \hat {\bm w}, pp)\rightarrow \bm v=[v_{\st 1},...,   v_{\st 4}]\\
   \end{array} 
$$
\end{mybox}
%\end{minipage}}
}}
\end{center}
%
it holds that:
%
{\small{
$$ \Pr\left[
  \begin{array}{l}
  
 
\Big( (\sum\limits_{\st j=1}^{\st n}w_{\st 1,j}\geq e) \wedge ( v_{\st 1}=0)\Big) \\
 %
 
 \vee \Big(( \sum\limits_{\st j=1}^{\st n}w_{\st 2,j}\geq e) \wedge ( v_{\st 2}=0)\Big) \\
 %
 \vee  \Big(( \sum\limits_{\st j=1}^{\st n}w_{\st 3,j}\geq e) \wedge ( v_{\st 3}=0)\Big) \\
 %
  \vee  \Big(( \sum\limits_{\st j=1}^{\st n}w_{\st 4,j}\geq e) \wedge ( v_{\st 4}=0)\Big) \\
 %
\end{array} :
    \begin{array}{l}
    \mathsf{Exp}_{\st 2}^{\st\mathcal{A}}(\mathsf{input})\\
\end{array}    \right]$$
}}
where $g:=(g_{\st 1}, g_{\st 2})\in T$, $\hat {\bm m}=[\hat {m}^{\st (\mathcal{C})}_{\st 1}, \hat {m}^{\st (\mathcal{C})}_{\st 2}, \hat {m}^{\st (\mathcal{B})}_{\st 1}, \hat {m}^{\st (\mathcal{B})}_{\st 2}]$, $(w_{\st 1,j},...,w_{\st 4,j})$ are the result of decoding   $(\hat w_{\st 1,j},...,\hat w_{\st 4,j})\in \hat{\bm w}_{\st j}\in \hat {\bm w}$, $\mathsf{input}:=(1^{\st\lambda},  {aux})$, $ sk_{\st \mathcal{D}}\in sk$, $n$ is the total number of auditors. The probability is taken over the uniform choice of $sk$, randomness used in the blockchain's primitives, randomness used during the encoding, and  the randomness of $\mathcal{A}$. 
\end{definition}


%A careful reader may ask why the following two conditions (some forms of which were in the events of Definition \ref{def::Security-against-malicious-victim}) are not added to the above events list: (a) $\mathcal{B}$ makes at least threshold committee members conclude that it  has provided a warning, although $\mathcal{B}$ has not (i.e., $ m^{\st\mathcal{(B)}}_{\st1}\neq warning \wedge \sum\limits_{\st j=1}^{\st n}w_{\st 1,j}<  e$), and (b) $\mathcal{B}$ persuades at least threshold $\mathcal{D}_{\st j}$s  to conclude that the warning was effective although it was not (i.e., $\mathtt{checkWarning}(m^{\st\mathcal{(B)}}_{\st1})= 0 \wedge \sum\limits_{\st j=1}^{\st n}w_{\st 2,j}< e$). The answer is that  $\mathcal{B}$ does not generate a complaint  and interact directly with $\mathcal{D}_{\st j}$s; therefore, we do not need to add these two events to the above events' list. 

Now we move on to privacy. Informally, a $\mathsf{pwdr}$ is privacy-preserving if it protects the privacy of (1) the customers', bank's, and auditors' messages (except public parameters) from non-participants of the scheme, including other customers, and (2) each auditor's verdict from $\mathcal{DR}$  which sees the final verdict. Below, we formally state it.

% !TEX root =main.tex




\begin{definition}[Privacy]\label{def::privacy} A $\mathsf{pwdr}$ preserves privacy if  the following two properties are satisfied.
\begin{enumerate}

\item For any probabilistic polynomial-time  adversary $\mathcal{A}_{\st 1}$,  security parameter $\lambda$, and  auxiliary information $aux$, there exists a negligible function $\mu(\cdot)$, such that for any  experiment $\mathsf{Exp}_{\st 3}^{\mathcal{A}_{\st 1}}$:



\begin{center}
{\small{
\begin{mybox}[colback=white,  width=88mm, height=71mm, drop fuzzy shadow southwest]{$\mathsf{Exp}_{\st 3}^{\st\mathcal{A}_{\st 1}}(1^{\st\lambda}\text{, }  {aux})$}
%\fbox{\begin{minipage}{3.8 in}
$$
  \begin{array}{l}
%
 \mathtt{keyGen}(1^{\st\lambda})\rightarrow (sk,pk)\\
%
  \mathtt{bankInit}(1^{\st \lambda})\rightarrow (T, pp, \bm{l})\\
%
\mathtt{customerInit} (1^{\st \lambda}, T, pp)\rightarrow a\\
%
\mathcal{A}_{\st 1}(1^{\st \lambda}, pk, a, pp, g, \bm{l})\rightarrow \Big((f_{\st 0}, f_{\st 1}),(in_{\st f_{\st_{\st 0}}},in_{\st f_{\st _1}}),(\text{aux}_{\st f_{0}},\text{aux}_{\st f_{\st1}})\Big)\\
%
\gamma\stackrel{\st \$}\leftarrow\{0,1\}\\
%
\mathtt{genUpdateRequest}(T, f_{\st \gamma}, {\bm l})\rightarrow \hat m^{\st\mathcal{(C)}}_{\st 1}\\
%
\mathtt{insertNewPayee}(\hat{m}^{\st\mathcal{(C)}}_{\st 1}, {\bm l})\rightarrow  \hat{\bm l}\\
%
\mathtt{genWarning}(T, \hat{\bm l}, \text{aux})\rightarrow \hat m^{\st\mathcal{(B)}}_{\st1}\\
%
\mathtt{genPaymentRequest}(T, in_{\st f_{_\gamma}}, \hat{\bm{l}}, \hat m^{\st\mathcal{(B)}}_{\st1})\rightarrow \hat m^{\st\mathcal{(C)}}_{\st2}\\
%
\mathtt{makePayment}(T, \hat m^{\st\mathcal{(C)}}_{\st2})\rightarrow \hat m^{\st\mathcal{(B)}}_{\st2}\\
%
\mathtt{genComplaint}(\hat m^{\st\mathcal{(B)}}_{\st 1}, \hat m^{\st\mathcal{(B)}}_{\st2}, T, pk, \text{aux}_{\st f_{_\gamma}})\rightarrow (\hat z, \hat{\ddot \pi})\\
%
 \forall j, j\in [n]:\\
\Big(\mathtt{verComplaint}( \hat z, \hat{\ddot{\pi}}, g, \hat{\bm m}, \hat{\bm{l}}, j, sk_{\st\mathcal D}, aux, pp)\rightarrow \hat{\bm{w}}_{\st j}
\Big)
\\
\mathtt{resDispute}(T_{\st 2}, \hat {\bm w}, pp)\rightarrow \bm v
\\
%
%\mathtt{resDispute}(\hat{m}_{\st 1}^{\st(\mathcal C)}, \hat {m}_{\st 2}^{\st(\mathcal C)}, \hat{m}_{\st 2}^{\st(\mathcal B)}, T, \hat {\bm w})\rightarrow \bar{\bm v}\\
%
   \end{array} 
$$
\end{mybox}
%\end{minipage}}
}}
\end{center}

it holds that:

%%%%%%%%%%%%%%%%%%%%%%%%%%%%%
$$ \Pr\left[
  \begin{array}{l}
  
 
\mathcal{A}_{\st 1}(g, \hat {\bm m}, \hat{\bm l}, \hat z, \hat{\ddot \pi}, \hat{\bm{w}})\rightarrow \gamma
\end{array} :
    \begin{array}{l}
    \mathsf{Exp}_{\st 3}^{\st\mathcal{A}_{\st 1}}(\mathsf{input})\\
\end{array}    \right]\leq \frac{1}{2}+\mu(\lambda).$$
%}}
%where   $g:=(g_{\st 1}, g_{\st 2})\in T$,   $\hat {\bm m}=[\hat {m}^{\st (\mathcal{C})}_{\st 1}, \hat {m}^{\st (\mathcal{C})}_{\st 2}, \hat {m}^{\st (\mathcal{B})}_{\st 1}, \hat {m}^{\st (\mathcal{B})}_{\st 2}], \hat {\bm w}=[\hat {\bm w}_{\st 1},.., \hat{\bm w}_{\st n}]$, $\mathsf{input}:=(1^{\st\lambda}, aux)$, $ sk_{\st \mathcal{D}}\in sk$, $n$ is the total number of auditors. The probability is taken over the uniform choice of $sk$, randomness used in the blockchain's primitives, and  the randomness of $\mathcal{A}_{\st 1}$. 




\item For any probabilistic polynomial-time  adversaries $\mathcal{A}_{\st 2}$ and  $\mathcal{A}_{\st 3}$, security parameter $\lambda$, and  auxiliary information $aux$, there exists a negligible function $\mu(\cdot)$, such that for any  experiment $\mathsf{Exp}_{\st 4}^{\mathcal{A}_{\st 2}}$:

%%%%%%%%%%%%%%%--EXp_4%%%%%%%%%%%%%%%
\begin{center}
{\small{
\begin{mybox}[colback=white,  width=88mm, height=76mm, drop fuzzy shadow southwest]{$\mathsf{Exp}_{\st 4}^{\st\mathcal{A}_{\st 2}}(1^{\st\lambda}\text{, }  {aux})$}
%\fbox{\begin{minipage}{3.8 in}
$$
  \begin{array}{l}
%
 \mathtt{keyGen}(1^{\st\lambda})\rightarrow (sk,pk)\\
%
  \mathtt{bankInit}(1^{\st \lambda})\rightarrow (T, pp, \bm{l})\\
%
\mathtt{customerInit} (1^{\st \lambda}, T, pp)\rightarrow a\\
%
\mathcal{A}_{\st 2}(1^{\st \lambda}, pk, a, pp, \bm{l})\rightarrow (f, in_{\st f}, \text{aux}_{\st f})\\
%
\mathtt{genUpdateRequest}(T, f, {\bm l})\rightarrow \hat m^{\st\mathcal{(C)}}_{\st 1}\\
%
\mathtt{insertNewPayee}(\hat{m}^{\st\mathcal{(C)}}_{\st 1}, {\bm l})\rightarrow  \hat{\bm l}\\
%
\mathcal{A}_{\st 2}(T, \hat{\bm l}, \text{aux})\rightarrow  m^{\st\mathcal{(B)}}_{\st1}\\
%
\mathtt{Encode}(T_{\st 1}, m^{\st\mathcal{(B)}}_{\st1})\rightarrow \hat m^{\st\mathcal{(B)}}_{\st1}\\
%
\mathtt{genPaymentRequest}(T, in_{\st f}, \hat{\bm{l}}, \hat m^{\st\mathcal{(B)}}_{\st1})\rightarrow \hat m^{\st\mathcal{(C)}}_{\st2}\\
%

\mathcal{A}_{\st 2}(T,  pk, \text{aux}_{\st f}, \hat m^{\st\mathcal{(B)}}_{\st 1}, \hat m^{\st\mathcal{(C)}}_{\st2})\rightarrow ( m^{\st\mathcal{(B)}}_{\st2},  z, {\ddot \pi})\\
%
\mathtt{Encode}(T_{\st 1}, m^{\st\mathcal{(B)}}_{\st 2})\rightarrow \hat m^{\st\mathcal{(B)}}_{\st2}\\
%
\mathtt{Encode}(T_{\st 1},  z)\rightarrow \hat z\\
%
\mathtt{\bar{ Encode}}(pk_{\st\mathcal{D}}, \ddot \pi)\rightarrow  \hat{\ddot \pi}\\
%
%
%\mathtt{makePayment}(T, \hat m^{\st\mathcal{(C)}}_{\st2})\rightarrow \hat m^{\st\mathcal{(B)}}_{\st2}\\
%%
%\mathtt{genComplaint}(\hat m^{\st\mathcal{(B)}}_{\st 1}, \hat m^{\st\mathcal{(B)}}_{\st2}, T, pk, \text{aux}_{\st f_{_\gamma}})\rightarrow (\hat z, \hat{\ddot \pi})\\
%
 \forall j, j\in [n]:\\
\Big(\mathtt{verComplaint}( \hat z, \hat{\ddot{\pi}}, g, \hat{\bm m}, \hat{\bm{l}}, j, sk_{\st\mathcal D}, aux, pp)\rightarrow \hat{\bm{w}}_{\st j}
\Big)
\\
\mathtt{resDispute}(T_{\st 2}, \hat {\bm w}, pp)\rightarrow \bm v
\\
%
%\mathtt{resDispute}(\hat{m}_{\st 1}^{\st(\mathcal C)}, \hat {m}_{\st 2}^{\st(\mathcal C)}, \hat{m}_{\st 2}^{\st(\mathcal B)}, T, \hat {\bm w})\rightarrow \bar{\bm v}\\
%
   \end{array} 
$$
\end{mybox}
%\end{minipage}}
}}
\end{center}


\vspace{-3mm}

it holds that:

%%%%%%%%%%%%%%%%%%%%%%%%%%%%%
$$ \Pr\left[
  \begin{array}{l}
  
 
\mathcal{A}_{\st 3}(T_{\st 2}, pk, pp, g, \hat{\bm m}, \hat{\bm l},  \hat z, \hat{\ddot \pi}, \hat{\bm{w}}, \bm v)\rightarrow w_{\st j}
\end{array} :
    \begin{array}{l}
    \mathsf{Exp}_{\st 4}^{\st\mathcal{A}_{\st 2}}(\mathsf{input})\\
\end{array}    \right]\leq {Pr}'+\mu(\lambda),$$
%}
where  $g:=(g_{\st 1}, g_{\st 2})\in T$,   $\hat {\bm m}=[\hat {m}^{\st (\mathcal{C})}_{\st 1}, \hat {m}^{\st (\mathcal{C})}_{\st 2}, \hat {m}^{\st (\mathcal{B})}_{\st 1}, \hat {m}^{\st (\mathcal{B})}_{\st 2}], \hat {\bm w}=[\hat {\bm w}_{\st 1},.., \hat{\bm w}_{\st n}]$, $\mathsf{input}:=(1^{\st\lambda}, aux)$, $T_{\st 1}\in T$, $pk_{\st \mathcal{D}}\in pk$, $sk_{\st \mathcal{D}}\in sk$, $n$ is the total number of auditors. Moreover, $Pr'$ is defined as follows. Let auditor $\mathcal{D}_{\st i}$ output $0$ and $1$ with probabilities $Pr_{\st i,0}$ and $Pr_{\st i,1}$ respectively. Then, $Pr'$ is defined as  $Max\{Pr_{\st 1,0},Pr_{\st 1,1}, ..., Pr_{\st n,0}, $ $Pr_{\st n,1} \}$. In the above privacy definition, the probability is taken over the uniform choice of $sk$, the probability that each  $\mathcal{D}_{\st j}$ outputs $0$ or $1$, the randomness used in the blockchain's primitives,  the randomness used during the encoding, and  the randomness of $\mathcal{A}_{\st 1}$ in $\mathsf{Exp}_{\st 3}^{\mathcal{A}_{\st 1}}$ and $\mathcal{A}_{\st 2}$ in $\mathsf{Exp}_{\st 4}^{\mathcal{A}_{\st 2}}$. 


\end{enumerate}
\end{definition}

\begin{definition}[Security]\label{def::PwDR-security}
A $\mathsf{pwdr}$ is secure if it meets security against a malicious victim,  security against a malicious bank, and preserves privacy with respect to definitions \ref{def::Security-against-malicious-victim}, \ref{def::Security-against-malicious-bank}, and \ref{def::privacy} respectively. 
\end{definition}


%Definition 13 (RC-PoR-P Security). A RC-PoR-P scheme is secure if it satisfies security against a malicious server, security against a malicious client, and preserves privacy, w.r.t. Definitions 10-12.

%\begin{mybox}[colback=white]{Title}
%This is my box.
%\end{mybox}

%{\small
%\begin{center}
%\fbox{\begin{minipage}{3.8 in}
%$$
%  \begin{array}{l}
%  \mathsf{Exp}_{\st 1}^{\mathcal{A}}(\mathsf{In}:=(1^{\st\lambda}, p, \bm{l})): \\
%  
%  \ \  \mathtt{keyGen}(1^{\st\lambda})\rightarrow (sk,pk)\\
%  
%   \ \   \mathcal{A}(1^{\st\lambda}, p, \bm{l}, pk)\rightarrow m^{\st\mathcal{(C)}}_{\st 1}\\
%   
%   
%    \ \ \mathtt{insertNewPayee}(m^{\st\mathcal{(C)}}_{\st 1}, {\bm l})\rightarrow {\bm l'}\\
%   
%   
%   \ \ \mathtt{genWarning}( {\bm l'}, {aux})\rightarrow m^{\st\mathcal{(B)}}_{\st1}\\
%   
%
%   \ \   \mathcal{A}(in_{\st p}, \bm{l}', m^{\st\mathcal{(B)}}_{\st1})\rightarrow m^{\st\mathcal{(C)}}_{\st 2}\\
%
%\ \ \mathtt{makePayment}(m^{\st\mathcal{(C)}}_{\st2})\rightarrow m^{\st\mathcal{(B)}}_{\st2}\\
%  
%
%
% \ \   \mathcal{A}(m^{\st\mathcal{(B)}}_{\st1},m^{\st\mathcal{(B)}}_{\st 2}, aux')\rightarrow z:=(k,x,y)\\
%
%
%\ \ \forall i, i\in [n]:\\
%\ \  \Big(\mathtt{verComplaint}( z, m^{\st\mathcal{(B)}}_{\st1},i)\rightarrow (d_{\st i}, v_{\st i}, \bar v_{\st i},  w_{\st i})\Big)\\
%
%\ \ \mathtt{resDispute}(m_{\st 1}^{\st(\mathcal C)}, m_{\st 2}^{\st(\mathcal C)}, m_{\st 1}^{\st(\mathcal B)}, m_{\st 2}^{\st(\mathcal B)}, z, \bm{d}, \bm{v}, \bm{\bar v}, \bm{w})\rightarrow c\\
%   \end{array} 
%$$
%\end{minipage}}
%\end{center}
%}













