%!TEX root = main.tex


\section{Introduction}\label{sec::intro}

An  ``Authorised Push Payment" (APP) fraud is a type of cyber-crime where a fraudster tricks a victim into making an authorised online payment into an account controlled by the fraudster. It is defined by the ``Financial Conduct Authority” (FCA) as \textit{``a transfer of funds by person $A$ to person $B$, other than a transfer initiated by or through person $B$, where: (1) $A$ intended to transfer the funds to a person other than $B$ but was instead deceived into transferring the funds to $B$; or (2) A transferred funds to $B$ for what they believed were legitimate purposes but which were in fact fraudulent''} \cite{FCA-Glossary}. The APP fraud has various variants, such as romance, investment, CEO, or invoice  frauds \cite{overview-of-payment-fraud}. The amount of money lost to  APP frauds is   substantial. According to a report produced by ``UK Finance",   only in the first half of 2021, a total of £$355$ million was lost to APP frauds, which has increased by  $71\%$  compared to losses reported in the same period in 2020; for the first time, the amount of money stolen via  APP frauds overtook even   card fraud losses \cite{2021-Half-Year-Fraud-Update}. The UK Finance report suggests that  online  payment is the type of payment method the victims used to make the authorised push payment in  $98\%$  of cases. The true cost of APP frauds often extends beyond the immediate financial loss,  to additional fees imposed by  investigating the event, remediation of internal banks' processes, and  dealing with the emotional fallout of the fraud.  APP fraud is a \emph{global} phenomenon. According to  the FBI's report, victims of APP frauds reported to it at least a total of  \$$419$ million losses\footnote{We  excluded the losses to ``romance'' and ``government impersonation'', accounting for \$$710$ million, from the above estimation, as  the FBI's report defines them  broadly, that could cover also  frauds unrelated to APP ones. Thus, it is likely that the total sum of losses due to APP frauds reported to the FBI is  higher than our estimation.}, in 2020 \cite{internet-crime-report}. Recently, Interpol  warned  its member countries  about a concerning  variant of APP fraud called investment fraud via dating software \cite{interpol-notce}, also known as ``pig butchering'' in China  \cite{pig-butchering}. According to  Europol’s notice, at least five variants of APP fraud are among the seven most common types of online financial fraud \cite{europol-notice}. 





Although the amount of money lost to  APP frauds and the number of cases have been significantly  increasing, the victims are not receiving  enough protection.  In the first half of 2021, only $42\%$ of the stolen funds returned to victims of  APP frauds in the UK \cite{2021-Half-Year-Fraud-Update}. Previous years had even lower rates than that. Despite  the UK's financial authorities and regulators (unlike other countries) have provided specific guidelines to financial institutes to prevent  APP frauds occurrence and improve victims' protection, these guidelines are  ambiguous and    open to interpretation. Furthermore,  there exists  no  mechanism in place via which honest victims can  \emph{prove} their innocence. The APP frauds are not specific to the regular online banking systems, they will eventually find their way into other payment systems, such as cryptocurrency. To date, the APP fraud problem has been overlooked by the information security and cryptography research communities.

To  facilitate    victims' reimbursement, in this work, we put forward the notion of ``Payment with Dispute Resolution'' (PwDR), formally define it, and  propose a provably secure protocol which instantiates it.  The PwDR lets an honest victim (of an APP fraud)  independently prove its innocence to a  (potentially semi-honest) third-party dispute resolver, in order to be reimbursed.  We identify three crucial properties that a PwDR scheme should possess; namely, (a) security against a malicious victim: a malicious victim  which is not qualified for the reimbursement should not be reimbursed, (b) security against a malicious bank: a malicious bank should not be able to disqualify an honest victim  from being reimbursed, and (c) privacy: the customer’s and bank’s messages remain confidential from non-participants of the scheme, and a party which resolves dispute  learns as little information as possible.  The  protocol makes black-box use of a standard  online banking system, meaning that it does not require significant changes to the existing online banking systems and can rely on their security. It is accompanied by our lightweight threshold voting protocol, which can be of independent interest. We perform a rigorous cost analysis of the protocol via both asymptotic and runtime  evaluation (via a prototype implementation). Our cost analysis indicates that the protocol is indeed efficient. The customer's and bank's computation and communication complexity are constant, $O(1)$. It only takes $0.09$ milliseconds for a dispute resolver to settle a dispute between the two parties. We  make  the implementation source code publicly available. We hope that our result lays the foundation for future solutions that will protect victims of this concerning  fraud. In summary,  our contributions are three-fold:
\begin{enumerate}
\item we put forward the notion of Payment with Dispute Resolution (PwDR), identify its core security properties, and  formally define the PwDR scheme.
\item we propose an efficient candidate construction,  and formally prove its security.
\item  we perform a rigorous cost analysis of the construction.     
\end{enumerate}

%We hope that our result can serve as a foundation for further solutions to protect an APP fraud victims and combat this type of fraud. 
  



%\noindent\textbf{Summary of Our Contributions.} We (i) put forth the notion of Payment with Dispute Resolution (PwDR), identify its core security properties, and  formally define the PwDR, (ii) propose an efficient candidate construction  and formally prove its security, and (iii) perform a rigorous cost analysis of the construction.     


\vspace{2mm}

The rest of this paper has been organised as follows.  In Section \ref{sec::background}, we provide a background on how  APP frauds drew regulators' attention and outline the existing related guidelines. In Section \ref{preliminaries}, we explain the thread model and  the  tools we use. In Section \ref{sec:: challenges}, we briefly explain the challenges that  we need to overcome when designing  the PwDR scheme.  In Section \ref{sec::def}, we provide a formal definition of the PwDR scheme. In Section \ref{sec::PwDR-Protocol}, we give an overview of  the PwDR protocol, present a few subroutines (including our threshold voting protocols) along with the detailed PwDR protocol. In Section \ref{sec::proof}, we formally analyse the security of this protocol. In Section \ref{sec::eval}, we evaluate the PwDR protocol's costs, while in Section \ref{sec::Future-Research}, we provide a set of future research directions. In Section \ref{sec::related-work}, we provide the related work and in Section \ref{sec::conclusion} we conclude the paper. We provide a notation table and more detail about Bloom filters in appendices \ref{sec:notation-table} and \ref{sec::bloom-filter-}, respectively. In Appendix \ref{sec::Variant-1-Theorem-proof}, we provide the main security theorem and related  proof of the first variant of   our threshold voting protocol.  In Appendix \ref{sec::Generic-Verdict-Encoding-Decoding-Protocols}, we provide the full protocol of the second variant of our  threshold voting scheme. In Appendix \ref{sec::Variant-2-Theorem-proof}, we provide the latter protocol's main theorem and proof. In Appendix \ref{sec:: Further-Discussion-on-the-Encoding-decoding-Protocol}, we provide further discussion on these threshold voting protocols. 
