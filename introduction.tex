%!TEX root = main.tex


\section{Introduction}

An  ``Authorised Push Payment" (APP) fraud is a type of cyber-crime where a fraudster tricks a victim into making an authorised online payment into an account controlled by the fraudster. It is defined by the ``Financial Conduct Authority” (FCA) as \textit{``a transfer of funds by person $A$ to person $B$, other than a transfer initiated by or through person $B$, where: (1) $A$ intended to transfer the funds to a person other than $B$ but was instead deceived into transferring the funds to $B$; or (2) A transferred funds to $B$ for what they believed were legitimate purposes but which were in fact fraudulent''} \cite{FCA-Glossary}. The amount of money lost to  APP frauds is   substantial. According to  ``UK Finance" report,   only in the first half of 2021, a total of £$355.3$ million was lost to APP frauds, which has increased by  $71\%$  compared to losses reported in the same period in 2020; for the first time, the amount of money stolen via the APP frauds overtook even   card fraud losses \cite{2021-Half-Year-Fraud-Update}. The UK Finance report suggests that  online  payment is the type of payment method the victims used to make the authorised push payment in  $98\%$  of cases. The true cost of APP frauds often extends beyond the immediate financial loss,  to additional fees imposed by  investigating the event, remediation of internal banks' processes, and  dealing with the emotional fallout of the fraud. %The level of fraud in the UK is such that it is now a national security threat.


Although the amount of money lost to  App frauds and the number of cases have been significantly  increasing, the victims are not receiving  enough protection.   In the first half of 2021, only $42\%$ of the stolen funds returned to victims of  APP frauds \cite{2021-Half-Year-Fraud-Update}. Previous years had a lower rate than that. Despite   financial authorities and regulators have provided  guidelines to financial institutes to prevent  APP frauds occurrence and improve victims' protection, these guidelines are  ambiguous and    open to interpretation. Furthermore,  there exists  no  mechanism in place via which honest victims can  \emph{prove} their innocence. The APP frauds are not specific to the regular online banking systems, they will eventually find their way in other payment systems, such as cryptocurrency. To date, the APP fraud problem has been overlooked by the information security and cryptography research communities.


In this work, we formally define a scheme called ``Payment with Dispute Resolution'' (PwDR),  propose a protocol which instantiates it, and  prove the protocol's security.  The PwDR lets an honest victim (of an APP fraud)  independently prove its innocence to a third-party dispute resolver, in order to be reimbursed.  We identify three crucial properties that a PwDR scheme should possess; namely, (a) security against a malicious victim: a malicious victim  which is not qualified for the reimbursement should not be reimbursed, (b) security against a malicious bank: a malicious bank should not be able to disqualify an honest victim  from being reimbursed, and (c) privacy: the customer’s and bank’s messages remain confidential from non-participants of the scheme, and a party which resolves dispute  learns as little information as possible.  The  protocol makes black-box use of a standard  online banking system, meaning that it does not require significant changes to the existing online banking systems and can rely on their security. It is accompanied by our lightweight threshold voting protocol, which can be of independent interest. We perform a rigorous cost analysis of the protocol via both asymptotic and runtime  evaluation (via a prototype implementation). Our cost analysis indicates that the protocol is indeed efficient. The customer's and bank's computation and communication complexity are constant, $O(1)$. It only takes $0.09$ milliseconds for a dispute resolver to settle a dispute between the two parties. We  make  the implementation source code publicly available. We hope that our result can serve as a foundation for future solutions that protect victims of this concerning  fraud. 



 


%We hope that our result can serve as a foundation for further solutions to protect an APP fraud victims and combat this type of fraud. 
  





\vspace{2mm}

\noindent\textbf{Summary of Our Contributions.} We (i) put forth the notion of Payment with Dispute Resolution (PwDR), identify its core security properties, and  formally define the PwDR, (ii) propose an efficient candidate construction  and formally prove its security, and (iii) perform a rigorous cost analysis of the construction.     




