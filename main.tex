%\documentclass[10pt]{article}
%\usepackage[usenames]{color} %used for font color
%\usepackage{amssymb} %maths
%\usepackage{amsmath} %maths
%\usepackage[utf8]{inputenc} %useful to type directly diacritic characters
%


\documentclass[orivec]{llncs}
\usepackage[table]{xcolor}
\usepackage{amssymb}
\usepackage[most]{tcolorbox}
\newtcolorbox{mybox}[2][]{%
  attach boxed title to top center
               = {yshift=-10pt},
               %width=85mm,%
                  %height=52mm,
  %colback      = black,
  colframe     =black,
  %fonttitle    = \bfseries,
  colbacktitle = black,
  title        = #2,#1,
  enhanced,
}

\usepackage{pgfplots}
\usepackage{pgfplots, pgfplotstable}

\usepackage{hyperref}

\usepackage{float}
\newcommand{\st}{\scriptscriptstyle}
\newcommand{\mb}{\mathbf}
\newcommand{\resizeT}{\scalebox{1}}
\newcommand{\resizeS}{\scalebox{.6}}
\newcommand{\resizeSS}{\scalebox{.38}}
\newcommand{\ignore}[1]{}
\newcommand{\PRF}{\mathtt{PRF}}
\newcommand{\TZ}[1]{\textcolor{magenta}{\textbf{Thomas: }{#1}}}

\usepackage{boxedminipage}
\usepackage{enumitem}
\usepackage{makeidx}  % allows for indexgeneration
\usepackage{amsfonts,amsmath,amssymb,graphicx,setspace,tipx}
\usepackage{amsmath}
\usepackage{float}
\usepackage{amssymb}
\usepackage[ruled,linesnumbered]{algorithm2e}
\usepackage{subfig}
\usepackage{graphicx}
\usepackage{framed}
\usepackage{esvect}
\usepackage{tikz}
\usepackage{latexsym}

\usepackage{tikz}
\usepackage{pgfplots}
\usepackage{pgfplots, pgfplotstable}
\usepackage{blkarray}% http://ctan.org/pkg/blkarra
\usepackage{mathtools}
\usepackage{amsmath}
\usepackage{bm}
\usepackage{adjustbox}
\usepackage{blindtext}
\usepackage{multicol}
\usepackage[disable]{todonotes}
\usepackage{prettyref}
\newrefformat{fig}{Figure~\ref{#1}}
\newrefformat{apndx}{Appendix~\ref{#1}}
\newrefformat{equ}{Equation~\eqref{#1}}
\let\proof\relax
\let\endproof\relax
\usepackage{amsthm}
\makeatletter
\newtheorem*{rep@theorem}{\rep@title}
\newcommand{\newreptheorem}[2]{%
\newenvironment{rep#1}[1]{%
 \def\rep@title{#2 \ref{##1}}%
 \begin{rep@theorem}}%
 {\end{rep@theorem}}}
\makeatother
\newreptheorem{theorem}{Theorem}
\usepackage[title]{appendix}
\usepackage{multirow}
\usepackage{adjustbox}
\usepackage{blindtext}
\usepackage{multicol}
\usepackage{tablefootnote}

%%%%%
\newcounter{counter}


%%%%%
\usepackage{fullpage}
\begin{document}



\title{Protecting Victims of \\ Authorised Push Payment Fraud}
\author{}
\institute{}

\date{}
\maketitle{}


\begin{abstract}
An ``Authorised Push Payment'' (APP) fraud occurs when fraudsters deceive a victim  to make  a payment  to a bank account controlled by them.  Total financial losses due to  APP frauds have been  growing. Although  authorities   provided guidelines to  improve victims’ protection, the guidelines are  vague and  the victims are not receiving sufficient protection. To protect the  victims,  we propose the notion of ``Payment with Dispute Resolution'' (PwDR) and formally define it. The PwDR lets an honest victim  prove its innocence to a third-party dispute resolver  (to be reimbursed) while preserving the involved parties' privacy. We also propose a  candidate protocol  and formally prove its security. The protocol  makes black-box use of a standard online banking system. We evaluate its    asymptotic cost and   runtime  via a prototype implementation. Our evaluation shows that the protocol is  efficient. It imposes only $O(1)$ overheads to the customer and bank.  Also, a dispute resolver can  settle a dispute between the two  in $0.09$ milliseconds.


\end{abstract}

%!TEX root = main.tex


\section{Introduction}

An  ``Authorised Push Payment" (APP) fraud is a type of cyber-crime where a fraudster tricks a victim into making an authorised online payment into an account controlled by the fraudster. It is defined by the ``Financial Conduct Authority” (FCA) as \textit{``a transfer of funds by person $A$ to person $B$, other than a transfer initiated by or through person $B$, where: (1) $A$ intended to transfer the funds to a person other than $B$ but was instead deceived into transferring the funds to $B$; or (2) A transferred funds to $B$ for what they believed were legitimate purposes but which were in fact fraudulent''} \cite{FCA-Glossary}. The amount of money lost to  APP frauds is   substantial. According to  ``UK Finance" report,   only in the first half of 2021, a total of £$355.3$ million was lost to APP frauds, which has increased by  $71\%$  compared to losses reported in the same period in 2020; for the first time, the amount of money stolen via  APP frauds overtook even   card fraud losses \cite{2021-Half-Year-Fraud-Update}. The UK Finance report suggests that  online  payment is the type of payment method the victims used to make the authorised push payment in  $98\%$  of cases. The true cost of APP frauds often extends beyond the immediate financial loss,  to additional fees imposed by  investigating the event, remediation of internal banks' processes, and  dealing with the emotional fallout of the fraud. %The level of fraud in the UK is such that it is now a national security threat.


Although the amount of money lost to  App frauds and the number of cases have been significantly  increasing, the victims are not receiving  enough protection.   In the first half of 2021, only $42\%$ of the stolen funds returned to victims of  APP frauds \cite{2021-Half-Year-Fraud-Update}. Previous years had even lower rates than that. Despite   financial authorities and regulators have provided  guidelines to financial institutes to prevent  APP frauds occurrence and improve victims' protection, these guidelines are  ambiguous and    open to interpretation. Furthermore,  there exists  no  mechanism in place via which honest victims can  \emph{prove} their innocence. The APP frauds are not specific to the regular online banking systems, they will eventually find their way in other payment systems, such as cryptocurrency. To date, the APP fraud problem has been overlooked by the information security and cryptography research communities.


In this work, we formally define a scheme called ``Payment with Dispute Resolution'' (PwDR),  propose a protocol which instantiates it, and  prove the protocol's security.  The PwDR lets an honest victim (of an APP fraud)  independently prove its innocence to a third-party dispute resolver, in order to be reimbursed.  We identify three crucial properties that a PwDR scheme should possess; namely, (a) security against a malicious victim: a malicious victim  which is not qualified for the reimbursement should not be reimbursed, (b) security against a malicious bank: a malicious bank should not be able to disqualify an honest victim  from being reimbursed, and (c) privacy: the customer’s and bank’s messages remain confidential from non-participants of the scheme, and a party which resolves dispute  learns as little information as possible.  The  protocol makes black-box use of a standard  online banking system, meaning that it does not require significant changes to the existing online banking systems and can rely on their security. It is accompanied by our lightweight threshold voting protocol, which can be of independent interest. We perform a rigorous cost analysis of the protocol via both asymptotic and runtime  evaluation (via a prototype implementation). Our cost analysis indicates that the protocol is indeed efficient. The customer's and bank's computation and communication complexity are constant, $O(1)$. It only takes $0.09$ milliseconds for a dispute resolver to settle a dispute between the two parties. We  make  the implementation source code publicly available. We hope that our result lays the foundation for future solutions that will protect victims of this concerning  fraud. 



 


%We hope that our result can serve as a foundation for further solutions to protect an APP fraud victims and combat this type of fraud. 
  





\vspace{2mm}

\noindent\textbf{Summary of Our Contributions.} We (i) put forth the notion of Payment with Dispute Resolution (PwDR), identify its core security properties, and  formally define the PwDR, (ii) propose an efficient candidate construction  and formally prove its security, and (iii) perform a rigorous cost analysis of the construction.     





%!TEX root = main.tex

\section{Background}


\subsection{Existing Guideline: Contingent Reimbursement Model Code}

In  2016, the UK's consumer protection organisation, called ``Which?'', submitted a super-complaint to the FCA about  APP frauds. It raised its concerns that, despite  APP frauds victims' rate is  growing, the victims do not have enough protection \cite{Which?-super-complaint}.  Since then, the FCA has been collaborating with financial institutes  to develop several initiatives that
could help prevent these frauds, and improve the response when they  occur. As a result,  the ``Contingent Reimbursement Model'' (CRM) code has been proposed. The  CRM code  lays out a set of requirements and explains under which circumstances customers should be reimbursed by their trading financial institutes when they fall victim to an APP fraud \cite{CRM-code}. So far,  there are at least nine firms, comprising nineteen brands (e.g., Barclays, HSBC,  Lloyds) signed up to the CRM code. One of the tangible outcomes of the code is a service called ``Confirmation of Payee" (CoP)  offered by the CRM code signatories \cite{CoP}. This service checks the money recipient's account name, once it is inserted by the sender customer into the online banking platform. If there is not an exact match, CoP provides a warning to the  customers about the risks of making the payment. In the case where a customer ignores such a warning, makes a payment, and later falls to an APP fraud,  it may not be reimbursed, according to the CRM code.

Although the CRM code is a vital guideline  towards reducing the occurrence of such frauds and protecting  the frauds' victims, it is still  vague and leaves a huge room for interpretation. For instance, in 2020, the ``Financial Ombudsman Service'' (that settles complaints between consumers and businesses)  highlighted that firms  are applying the CRM code inconsistently and in some cases incorrectly, which resulted in failing to reimburse victims in circumstances anticipated by this code \cite{Financial-Ombudsman-Service-response}.  As another example, one of the conditions in the CRM code that allows a bank to avoid reimbursing the customer is clause R2(1)(e) which states: \textit{``The Customer has been grossly negligent. For the avoidance of doubt the provisions of R2(1)(a)-(d) should not be taken to define gross negligence in this context''}.  Nevertheless, neither the CRM code  nor the ``Payment Services Regulations'' \cite{Regulations}   explicitly define under which circumstances the customer is considered ``grossly negligent'' in the context of  APP frauds. In particular, in the CRM code, the only terms that discuss customers' misbehaviour are  the provisions of R2(1)(a)-(d); however, as stated above, they should be excluded from the definition of the term gross negligence. On the  other hand,  in the Payment Services Regulations, this term is used three times, i.e.,  twice in regulation 75 and once in regulation 77. But in all  three cases, it is used for frauds related to \emph{unauthorised payments} which are  different types of frauds from  APP ones. Hence, there is a pressing need for an accurate solution to help and protect  APP frauds victims. 

\subsection{Different But Related Area: Central Bank Digital Currency}

Due to the growing interest in digital payments, some central banks around the world have been exploring or even piloting the idea of ``Central Bank Digital Currency" (CBDC) \cite{CBDC}.  The idea behind CBDC is that a central bank issues digital money/token (i.e., a representation of banknotes and coins) to the public where this digital money is regulated by the nation's monetary authority or central bank, similar to regular fiat currencies. CBDC can offer various features such as efficiency, transparency, programmable money, transactions' traceability, or financial inclusion to name a few \cite{CBDC,CBDC-core-features}. Researchers have already discussed that (in CBDC) users transactions' privacy and regulatory oversight can coexist, e.g., in \cite{abs-2103-00254,WustKCC19}. In such a setting, users' amount of payments and even with whom they transact can remain confidential, while  various regulations can be accurately encoded  into cryptographic algorithms (and ultimately into a software) which are executed on users' transaction history by  authorities, e.g., to ensure compliance with ``Anti-Money Laundering'' or ``Countering the Financing of Terrorism''.  CBDC is still in its infancy and has not been adopted by banks yet. The adoption of such a model requires fundamental changes to and further digitalisations of the current banking  infrastructure. Therefore, unlike CBDC which is more futuristic, the focus of our work is on the existing regular  online banking systems.  Nevertheless, when CBDC becomes mainstream, APP frauds  might happen to the CBDC users as well. In this case, (a variant of) our result can be integrated into  such a framework to protect  APP frauds victims.  






% !TEX root =main.tex







\section{Preliminaries} \label{preliminaries}


\subsection{Informal Thread Model and Assumptions}\label{Notations-and-Assumptions}

The PwDR scheme consists of six types of parties. Below, we informally explain each type of party's role and the security assumption we make about each of them. We will provide a formal definition of the  PwDR scheme in Section \ref{sec::def}. 
%
\begin{itemize}
%
\item[$\bullet$] Customer $\mathcal{C}$: it is a regular customer of a bank. We call a customer  a victim after it falls victim to an APP fraud. We assume a victim is corrupted by a non-colluding active (or malicious) adversary. %; for instance, an unqualified victim  tries to make itself appear qualified for reimbursement. 
%
\item[$\bullet$] Bank $\mathcal{B}$: it is a regular bank that provides a standard online banking system. We assume it is corrupted by a non-colluding active adversary. We assume  any change to the source code of the online banking system is transparent and  can be detected. Note that in the real-world a bank is  not usually an active adversary and cares about its reputation, such as a ``rational'' or ``covert'' adversary which is weaker than the active one. However, to ensure our solution offers  a strong security guarantee, we assume the adversary is strong too, i.e., active one. 

%, e.g., the bank  uses a cryptographic commitment to commit to the source code. 
%
\item[$\bullet$] Smart contract $\mathcal{S}$: it is a standard  smart contract of a public  blockchain (e.g., Ethereum). It mainly acts as a tamper-proof public bulletin board to store different parties' messages.  We do not assume that a smart contract itself can offer any privacy. 
%
\item[$\bullet$] Certificate generator $\mathcal{G}$: it is a trusted third party (e.g., hospital, registry office) which provides signed digital certificates (e.g., certificate of death, divorce, disability) to customers. Its involvement is more implicit than the other  parties.
%
\item[$\bullet$]  A committee of arbiters $\{\mathcal{D}_{\st 1},..., \mathcal{D}_{\st n}\}$: it consists of  trusted third-party authorities, auditors, or regulators (e.g.,  financial conduct authority, prudential regulation authority, financial ombudsman service). Given a set of complaints, they compile the complaints    and provide  their binary verdicts. If needed, they are authorised to access the banking  system's backend software to carry out investigations. We assume all arbiters have interacted with each other once,  to agree on (a) a secret key, $\bar k_{\st 0}$, and (b) a pair of keys $({pk}_{\st\mathcal {D}}, {sk}_{\st\mathcal {D}})$  of an asymmetric key encryption.
%
\item[$\bullet$]  Dispute resolver $\mathcal{DR}$: it is an aggregator of arbiters' votes (e.g., public court). Given a collection of votes, it extracts and announces the final verdict. We assume it is corrupted by a non-colluding passive adversary. 
%
\end{itemize}



\subsection{Notations}
We use $\mathtt{Enc}(.)$  and $\mathtt{Dec}(.)$ to denote the encrypting and decrypting algorithms of   a semantically secure symmetric key encryption scheme respectively. We also use  $\mathtt{ \tilde {Enc}}({pk}_{\st\mathcal {D}}, .)$ and   $\mathtt{\tilde{Dec}}({sk}_{\st\mathcal {D}}, .)$ to denote the encrypting and decrypting algorithms of   a semantically secure asymmetric key encryption scheme  which has  the following key generating algorithm:  $\tilde{\mathtt{keyGen}}(1^{\st\lambda})\rightarrow({sk}_{\st\mathcal {D}}, {pk}_{\st\mathcal {D}})$ and its  public key, ${pk}_{\st\mathcal {D}}$, is known to everyone.  We denote the banking system's internal payment  algorithm by  $\mathtt{pay}(.)$ that  transfers money from the customer's account to a payee's account that is specified by the customer.  We use $in_{\st p}$ to denote the inputs of this algorithm.  We also use $\phi$ to denote a null value. In Appendix \ref{sec:notation-table}, we provide a notation table. 


%We use $\mathtt{Enc}(.)$  and $\mathtt{Dec}(.)$ to denote the encrypting and decrypting algorithms of   a semantically secure symmetric key encryption scheme. In this work, we use a committee of honest arbiters $\mathcal{D}:\{\mathcal{D}_{\st 1},..., \mathcal{D}_{\st n}\}$. Each arbiter,  given a set of  inputs, provides a binary verdict.    We  assume  $\mathcal{D}_{\st i}$'s share a pair of keys $({pk}_{\st\mathcal {D}}, {sk}_{\st\mathcal {D}})$ of an asymmetric key encryption. The encryption scheme has  key generating  $\tilde{\mathtt{keyGen}}(1^{\st\lambda})\rightarrow({sk}_{\st\mathcal {D}}, {pk}_{\st\mathcal {D}})$,   encrypting $\mathtt{ \tilde {Enc}}({pk}_{\st\mathcal {D}}, .)$ and  decrypting $\mathtt{\tilde{Dec}}({sk}_{\st\mathcal {D}}, .)$ algorithms, where its public key is known to everyone.  Also, we assume all arbiters have interacted with each other to agree on a secret key, $\bar k_{\st 0}$.  In this work, a third party dispute resolver, $\mathcal{DR}$, is involved, which can be potentially semi-honest, i.e., a passive adversary. We assume bank and customer are potentially malicious, i.e., active adversary. We use $\phi$ to denote a null value.  Our proposed solution is built upon the existing online banking system. We assume the banking system has  algorithm  $\mathtt{pay}(.)$ that  transfers money from the customer's account to a payee's account that is specific by the customer.  We denote the inputs of this algorithm $in_{\st p}$. We assume the source code of the online banking system is static, and any change to the source code is transparent and  can be detected, e.g., the bank  uses a cryptographic commitment to commit to the source code. Moreover, we assume the online banking  system is secure. In Appendix \ref{sec:notation-table}, we provide a notation table. 





% Similar to the \emph{optimistic} fair cryptographic protocols that aim  at efficiency, e.g., in \cite{AsokanSW97,eurocrypt/AsokanSW98,BaoDM98,DongCCR13}, we assume the existence of a trusted third party arbiter   which remains offline most of the time but can be invoked to resolve a part of dispute. We emphasise that if the bank and customers behave honestly, then the arbiter is never involved. Even the (offline) presence of such an arbiter threatens adversarial behaviour and acts as a deterrence.  %The idea is akin to deterrence in criminology,  i.e.,  the threat of punishment will deter people from committing crimes.














\subsection{Digital Signature}\label{subsec:DS}

A digital signature is a scheme for verifying the authenticity of digital messages or documents. Below, we restate its formal definition, taken from \cite{DBLP:books/crc/KatzLindell2014}. 


\begin{definition}\label{sec::def}
A signature scheme  involves three algorithms, $\mathtt{Signature}:=(\mathtt{Sig.keyGen}, $ $\mathtt{Sig.sign}, \mathtt{Sig.ver})$, that are defined as follows.

\begin{itemize} 
\item[$\bullet$] $\mathtt{Sig.keyGen}(1^{\st \lambda})\rightarrow (sk,pk)$.  A probabilistic algorithm run by  a  signer. It takes as input a security parameter. It outputs a key pair: $(sk,pk)$, consisting of secret: $sk$, and public: $pk$ keys. 
\item[$\bullet$] $\mathtt{Sig.sign}(sk, pk, u)\rightarrow sig$. An algorithm run by the signer. It takes as input  key pair: $(sk,pk)$ and a message: $u$. It outputs a signature: $sig$.
\item[$\bullet$]  $\mathtt{Sig.ver}( pk, u, sig)\rightarrow h\in\{0,1\}$. A deterministic algorithm run by a verifier. It takes as input  public key: $pk$,  message: $u$, and signature: $sig$. It checks the signature's validity.   If the verification passes, then it outputs $1$; otherwise, it outputs $0$. 
\end{itemize}
\end{definition}

A digital signature scheme should meet the following properties:

\begin{itemize} 
\item[$\bullet$]  \textit{Correctness.} For every input $u$ it holds that:
%
$$Pr\Big[\  \  \mathtt{Sig.ver}( pk, u, \mathtt{Sig.sign}(sk, pk, u))=1\ : \
\mathtt{Sig.keyGen}(1^{\st \lambda})\rightarrow (sk, pk)  \Big]=1$$
%
\item[$\bullet$] \textit{Existential unforgeability under chosen message attacks.} A probabilistic polynomial time (PPT) adversary that obtains $pk$ and has access to a signing  oracle for messages of its choice, cannot create a valid pair $(u^{\st *},sig^{\st *})$ for a new message $u^{\st *}$  (that was never a query to the signing oracle), except with a small probability, $\sigma$. More formally: 




\small{
$$ \Pr\left[
  \begin{array}{l}
  
  u^{\st *}\not\in Q\ \wedge \\
   \mathtt{Sig.ver}( pk,  u^{\st *}, sig^{\st *}) =1\\
  
  
    
%(M(u^{\scriptscriptstyle *},k)\neq \sigma \lor Q(\text{aux},k)\neq q) \wedge\\ (a=1 \ \vee b=1)
\end{array} : 
    \begin{array}{l}
   
    \mathtt{Cer.keyGen}(1^{\st \lambda})\rightarrow (sk,pk) \\
  \mathcal{A}^{\mathtt{Sig.sign}(k,)}(pk)\rightarrow(u^{\st *}, sig^{\st *})

     
\end{array}    \right]\leq \mu(\lambda)$$
}
where $Q$ is the set of queries that $\mathcal{A}$ sent to the certificate generator oracle.
\end{itemize}



%
%An application of a digital signature is in digital certificate, which is a  document digitally signed  by a certificate generator. Given a certificate and its parameters,  anyone can check whether it has been correctly generated by a valid generator. There is a case  where 
%a \emph{hard copy} certificate is used.  In this case,  the process $\mathtt{Sig.keyGen}(.)$ outputs a blank legitimate stamped certificate as a private parameter: $sk$, and the description of a standard legitimate certificate as a public parameter: $pk$. Moreover, the process $\mathtt{Sig.sign}(k, u)$ takes $k$ and the file $u$ on which a certificate should be generated and outputs a stamped certificate with the information printed on it. The process $\mathtt{Sig.ver}( pk, u, sig)$ takes the public parameter, the file,  and the hard copy of the certificate and outputs $1$ if it is valid and $0$ if it is not. Note, when  a hard copy certificate is considered, it is not possible to precisely define the success probability of the adversary, as it depends on the technology available to the adversary to generate a blank stamped certificate that looks like a legitimate one. In the real world however, this probability is usually small (but it may not be negligible). In this paper, we mainly consider a digital certificate; however, our solution can adopt hard copy certificates as well with the above caveat  regarding the adversary's success probability. 



\subsection{Smart Contract}\label{subsec:SC} Cryptocurrencies, such as Bitcoin \cite{bitcoin} and Ethereum \cite{ethereum}, beyond offering a decentralised currency,  support  computations on  transactions. In this setting, often a certain computation logic is encoded in a computer program, called a \emph{``smart contract''}. Although Bitcoin, the first decentralised cryptocurrency, supports smart contracts, the functionality of Bitcoin's smart contracts is very limited, due to the use of the underlying programming language that does not support arbitrary tasks. To address this limitation, Ethereum, as a generic smart contract platform, was designed. Thus far, Ethereum has been the most predominant cryptocurrency framework that lets users define arbitrary smart. This framework allows users to create an account with a unique account number or address. Such users are often called external account holders, which can send (or deploy) their contracts to the framework’s blockchain. In this framework, a contract's code and its related data  are held by every node in the blockchain's network. Ethereum smart contracts are often written in a high-level Turing-complete programming language called ``Solidity''.The program execution's  correctness  is  guaranteed by the security of the underlying blockchain components. To prevent  a denial of service attack, the framework requires a transaction creator to pay a  fee, called \emph{``gas''}. %, depending on the complexity of the contract running on  it.  


% !TEX root =main.tex

\vspace{-3.67mm}


\subsection{Commitment Scheme}\label{subsec:commit}

\vspace{-1.2mm}
A commitment scheme involves two parties,  \emph{sender} and  \emph{receiver}, and includes  two phases: \emph{commit} and  \emph{open}. In the commit phase, the sender  commits to a message $x$ as $\mathtt{Com}(x,r)=\mathtt{Com}_{\scriptscriptstyle x}$, that involves a secret value,  $r$. In the open phase, the sender sends the opening $\ddot{x}:=(x,r)$ to the receiver which verifies its correctness: $\mathtt{Ver}(\mathtt{Com}_{\scriptscriptstyle x},\ddot{x})\stackrel{\scriptscriptstyle ?}=1$ and accepts if the output is $1$.  A commitment scheme  satisfies two properties, (a) \textit{hiding}: it is infeasible for an adversary to learn any information about the   message, and (b) \textit{binding}: it is infeasible for an adversary to open a commitment  to different values  than the one  used in the commit phase. We provide more detail about the commitment scheme  in Appendix \ref{subsec:commit-long}. 



% A commitment scheme involves two parties,  \emph{sender} and  \emph{receiver}, and includes  two phases: \emph{commit} and  \emph{open}. In the commit phase, the sender  commits to a message: $x$ as $\mathtt{Com}(x,r)=\mathtt{Com}_{\st x}$, that involves a secret value: $r\stackrel{\st\$}\leftarrow \{0,1\}^{\st\lambda}$. In the end of the commit phase,  the commitment $\mathtt{Com}_{\st x}$ is sent to the receiver. In the open phase, the sender sends the opening $\ddot{x}:=(x,r)$ to the receiver who verifies its correctness: $\mathtt{Ver}(\mathtt{Com}_{\st x},\ddot{x})\stackrel{\st ?}=1$ and accepts if the output is $1$.  A commitment scheme must meet two properties: (a) \textit{hiding}: it is infeasible for an adversary  to learn any information about the committed  message $x$, until the commitment $\mathtt{Com}_{\st x}$ is opened, and (b) \textit{binding}: it is infeasible for an adversary  to open a commitment $\mathtt{Com}_{\st x}$ to different values $\ddot{x}':=(x',r')$ than that was  used in the commit phase, i.e.,  to find  $\ddot{x}'$, \textit{s.t.} $\mathtt{Ver}(\mathtt{Com}_{\st x},\ddot{x})=\mathtt{Ver}(\mathtt{Com}_{\st x},\ddot{x}')=1$, where $\ddot{x}\neq \ddot{x}'$.  There exist efficient non-interactive  commitment schemes both in (a) the standard model, e.g., Pedersen scheme \cite{Pedersen91}, and (b)  the random oracle model using the well-known hash-based scheme such that committing  is : $\mathtt{H}(x||r)=\mathtt{Com}_{\st x}$ and $\mathtt{Ver}(\mathtt{Com}_{\st x},\ddot{x})$ requires checking: $\mathtt{H}(x||r)\stackrel{\st ?}=\mathtt{Com}_{\st x}$, where $\mathtt{H}:\{0,1\}^{\st *}\rightarrow \{0,1\}^{\st\lambda}$ is a collision resistant hash function; i.e., the probability to find $x$ and $x'$ such that $\mathtt{H}(x)=\mathtt{H}(x')$ is negligible in the security parameter, $\lambda$.
% !TEX root =main.tex



%\vspace{-3mm}
\subsection{Statement Agreement Protocol}\label{SAP}

Recently, a scheme called ``Statement Agreement Protocol'' (SAP)  has been proposed in \cite{cryptoeprint:2021:1145}. It   lets two mutually distrusted parties, e.g., $\mathcal{B}$ and $\mathcal{C}$, efficiently agree on a private statement, $\pi$. Informally, the SAP  satisfies the following four  properties: (1) neither party can convince  a third-party  verifier that it has agreed with its counter-party on a different statement than the one both parties previously agreed on, (2) after they  agree on a statement,  an honest party can (almost) always  prove to the verifier that it has the agreement, (3) the privacy of the statement is preserved (from the public), and (4) after both parties reach an agreement, neither can  deny it.  The SAP uses a  smart contract and commitment scheme. It  assumes that each party  has a blockchain public address,  $adr_{\st\mathcal{R}}$ (where $\mathcal{R}\in\{\mathcal{B,C}\}$). Below, we restate  the  SAP, taken from \cite{cryptoeprint:2021:1145}. 


%
 \begin{enumerate}
 %
 \item\textbf{Initiate}. $\mathtt{SAP.init}(1^{\st\lambda}, adr_{\st\mathcal{B}}, adr_{\st\mathcal{C}}, \pi )$ 

 The following steps are taken   by  $\mathcal B$.
 
  \begin{enumerate}
  \item\label{SAP::deploy-contract}  Deploys a smart contract that  explicitly states both parties'  addresses, $adr_{\st\mathcal{B}}$ and  $adr_{\st\mathcal{C}}$. Let $adr_{\st\text{SAP}}$ be the deployed contract's address. 

   \item  Picks a random value $r$, and commits to the statement, $\mathtt{Com}(\pi, r)=g_{\st \mathcal{B}}$.
   \item Sends $adr_{\st\text{SAP}}$ and $\ddot{\pi}:=(\pi, r)$  to  $\mathcal C$, and $g_{\st\mathcal B}$ to the contract. 
   %\item Sends $g_{\st\mathcal C}$ to the contract, using its account. 
    \end{enumerate}
     %
    \item\textbf{Agreement}. $\mathtt{SAP.agree}(\pi, r, g_{\st \mathcal{B}}, adr_{\st\mathcal{B}}, adr_{\st\text{SAP}})$

     The following steps are taken   by  $\mathcal C$.
     %
     \begin{enumerate}
 %
   \item Checks  if $g_{\st \mathcal{B}}$ was  sent  from $adr_{\st \mathcal{B}}$, and checks locally $\mathtt{Ver}(g_{\st\mathcal B}, \ddot{\pi})=1$.
   %
   \item If the checks pass, it sets $b=1$,    computes locally $\mathtt{Com}(\pi, r)=g_{\st\mathcal C}$, and sends $g_{\st\mathcal C}$ to the contract. Otherwise, it sets $b=0$ and $g_{\st\mathcal C}=\bot$.
 %
   %\item  If $b=1$, then sends $g_{\st\mathcal S}$ to the contract. % using its account.
    \end{enumerate}
     
     
     
   \item\textbf{Prove}. For either $\mathcal B$ or $\mathcal C$ to prove, it sends $\ddot{\pi}:=(\pi, r)$  to the smart contract. 
   
   
   
 \item\textbf{Verify}. $\mathtt{SAP.verify}(\ddot{\pi}, g_{\st\mathcal B},g_{\st\mathcal C}, adr_{\st\mathcal{B}}, adr_{\st\mathcal{C}})$
 
 
 The following steps are taken   by  the smart contract.
   \begin{enumerate}
   
\item\label{SAP::check-adr} Ensures $g_{\st\mathcal B}$ and $g_{\st\mathcal C}$ were sent from   $adr_{\st \mathcal{B}}$ and  $adr_{\st \mathcal{C}}$  respectively. 
  
   \item\label{SAP::check-commit} Ensures $\mathtt{Ver}(g_{\st\mathcal B},\ddot{\pi})=\mathtt{Ver}(g_{\st\mathcal C},\ddot{\pi}) =1$.
   
   \item Outputs $s=1$, if the checks, in steps \ref{SAP::check-adr} and \ref{SAP::check-commit}, pass. It outputs $s=0$, otherwise.
    \end{enumerate}
 \end{enumerate}

  
  
  
  
%  \item\textbf{Agreement}.
%  \begin{enumerate}
%   \item $\mathcal S$ picks a random value: $r$, and commits to the statement: $\mathtt{H}(x||r)=g_{\st S}$
%   \item $\mathcal S$ sends $r$  to the client and sends $g_{\st\mathcal S}$ to the contract. 
%   \item $\mathcal C$ checks: $\mathtt{H}(x||r)\stackrel{?}=g_{\st \mathcal S}$. If the equation  holds, it computes $\mathtt{H}(x||r)=g_{\st\mathcal C}$
%   \item $\mathcal C$   stores $g_{\st\mathcal C}$ in the contract. 
%    \end{enumerate}
%   \item\textbf{Prove}. For either $\mathcal C$ or $\mathcal S$ to prove, it has agreement on $x$ with its counter-party, it sends $\mu=(x, r)$  to the contract. 
% \item\textbf{Verify}. Given $\mu$, the contract does the following. 
%   \begin{enumerate}
%
%   \item checks if $\mathtt{H}(x||r)=g_{\st\mathcal C}=g_{\st\mathcal S}$
%   \item outputs $1$, if the above equation holds; otherwise, it outputs $0$
%    \end{enumerate}
% \end{enumerate}



%
% \begin{enumerate}
% \item\textbf{Setup}.  Both parties agree on a  smart contract and deploy it, such that the parties public keys, $pk_{\st C}$ and $pk_{\st S}$, are encoded in the contract.
%
%  
%  \item\textbf{Agreement}.
%  \begin{enumerate}
%   \item The server picks a random value: $r$, and commits to the statement: $H(s||r)=y_{\st S}$.
%   \item The server sends $r$  to the client and sends $y_{\st S}$ to the contract. 
%   \item The client checks: $H(s||r)\stackrel{?}=y_{\st S}$. If the equation  holds, it computes $H(s||r)=y_{\st C}$.
%   \item The client   stores $y_{\st C}$ in the contract. 
%    \end{enumerate}
%   \item\textbf{Prove}. For either $C$ or $S$ to prove, it has agreement on $s$ with its counter-party, it sends $\mu=(s, r)$, in a signed transaction, to the contract. 
% \item\textbf{Verify}. Given $\mu$, the contract does the following. 
%   \begin{enumerate}
%   \item verifies the public keys related to  signatures of $y_{\st C}$ and $y_{\st S}$ match $pk_{\st C}$ and $pk_{ \st S}$ respectively.
%   \item checks if $H(s||r)=y_{\st C}=y_{\st S}$.
%   \item outputs 1, if the above equation holds; otherwise, it outputs 0.
%    \end{enumerate}
% \end{enumerate}
 
 %Note that the above protocol is one-off, which means after first party
 
 
 
 
 %In Appendix \ref{sec:Discussion-on-the-SAP}, we discuss the SAP's security and explain why naive solutions are not suitable.
 

 
 
 
 %The SAP protocol might be of independent interest. 
 
 
% \begin{remark}
% The verification algorithm can also be executed \emph{off-chain}. In particular, given  statement $\ddot{x}$, anyone can read $(g_{\st\mathcal C},g_{\st\mathcal S},adr_{\st\mathcal{C}}, adr_{\st\mathcal{S}})$ from the SAP contract and locally run $\mathtt{SAP.verify}(\ddot{x}, g_{\st\mathcal C},g_{\st\mathcal S},adr_{\st\mathcal{C}}, adr_{\st\mathcal{S}})$ to check the statement's correctness.  This relieves  the verifier from the  transaction  and contract's execution costs. 
% \end{remark}
% 
%  \begin{remark}
%One may simply let each party  sign the statement and send it to the other party, so later on each party can send both signatures to the contract which verifies them. However, this would not work,  as the party who first receives the other party's signature  may refuse  to send its  signature, that prevents the other party from proving that it has  agreed on the statement with its counter-party. Alternatively, one may want to use a protocol for a fair exchange of digital signature (or fair contract signing) such as \cite{BonehN00,DBLP:conf/fc/GarayJ02}. In this case, after both parties have the other party's signature, they can sign the statement themselves and send the two signatures to the contract which first checks the validity of both  signatures. Although this satisfies the above security requirements, it yields two main efficiency and practical issues: (a) it imposes very high computation costs, as  protocols for a fair exchange of signature involve generic zero-knowledge proofs and require a high number of modular exponentiations. And (b) it is impractical because protocols for fair exchange of signature protocol support only certain signature schemes (e.g., RSA, Rabin, or Schnorr) that are not directly supported by the most predominant  smart contract framework,  Ethereum, that only supports  Elliptic Curve Digital Signature Algorithm (EDCSA).
% \end{remark}







\subsection{Pseudorandom Function}


Informally, a pseudorandom function is a deterministic function that takes a key of length $\Lambda$ and an input; and outputs a value  indistinguishable from that of  a truly random function.  In this paper, we use the pseudorandom function:   $\mathtt {PRF}: \{0,1\}^{\st \Lambda}\times \{0,1\}^{\st *} \rightarrow  \mathbb{F}_p$, where $p$ is a large prime number, $|p|=\lambda$, and $(\Lambda,\lambda)$ are the security parameters. In practice, a pseudorandom function can be obtained from an efficient block cipher. We refer readers to \cite{DBLP:books/crc/KatzLindell2014} for a formal definition of a pseudorandom function.


\subsection{Bloom Filter}


A Bloom filter \cite{DBLP:journals/cacm/Bloom70} is a compact data structure that allows us to 
efficiently check an  element membership. It is an array of $m$ bits (initially all set to zero), that  represents $n$  elements. It is accompanied by $k$ independent hash functions. To insert an element, all the  hash values of the element are computed and their corresponding bits in the filter are set to $1$. To check an element membership, all its hash values are re-computed and checked whether all are set to $1$ in the filter. If all the corresponding bits are $1$, then the element is probably in the filter; otherwise, it is not. In Bloom filters,  it is possible that an element is not in the set, but the membership query indicates it is, i.e., false positives. In this work, we ensure that the false positive probability is negligible, e.g.,  $2^{\st - 40}$. Also, we require that a Bloom filter uses \emph{cryptographic} hash functions. In Appendix \ref{sec::bloom-filter-}, we explain how the Bloom filter's parameters can be set.

%Informally, a digital certificate's security requires that no one can generate a valid certificate that was not previously produced by the certificate generator. The security of a digital certificate relies on the security of the digital signature scheme used.  Below, we present the formal definition of the digital signature.



%In the case where a digital certificate is considered, then $\mathtt{Cer}$ definition is equivalent to the definition of a digital signature scheme. In this case, it holds $\sigma=neg(\lambda)$, where $neg$ is a negligible function and $\lambda$ is a security parameter. Now, we briefly explain the procedures' input/output when a hard copy certificate is considered.  The process $\mathtt{Cer.genPar}(.)$ outputs a blank legitimate stamped certificate as a private parameter: $sk$, and the description of a standard legitimate certificate as a public parameter: $pk$. Morevoer, the process $\mathtt{Cer.genCrt}(k, u)$ takes $k$ and the file $u$ on which a certificate should be generated and outputs a stamped certificate with the information printed on it. The process $\mathtt{Cer.verCrt}( pk, u, crt)$ takes the public parameter, the file,  and the hard copy of the certificate and outputs $1$ if it is valid and $0$ if it is not. In the case where a hard copy certificate is considered, it is not possible to precisely define the probability $\sigma$, as it depends on the technology available to the adversary to generate a blank stamped certificate that looks like a legitimate one. In the real world however, this probability is usually small (but it may not be negligible).




%\input{PoR-def}

 

%\input{notation-Table.tex}


%!TEX root = main.tex

%\section{Notations and Assumptions}
%
%
%
%
%
%
%There are three main algorithms involved in the protocol; namely, $\mathtt{setupNewPayee}(.)$, $\mathtt{ammendExistingPayee}(.)$, and $\mathtt{pay}(.)$.
%The first two algorithms are executed by a smart contract while the third one is executed off-chain by a bank. We denote the inputs of algorithms $\mathtt{setupNewPayee}(.), \mathtt{ammendExistingPayee}(.)$, and $\mathtt{pay}(.)$ by $in_{\st s}, in_{\st a}$, and $in_{\st p}$ respectively. A null value is denoted by $\phi$. To determine the effectiveness of a warning we define a deterministic function   $\mathtt{f}(v_{\st 1}, ..., v_{\st n})\rightarrow b$  that takes $n$ binary inputs (or verdicts)  and outputs a binary value, $b$. For the sake of simplicity, we let this function output $1$ if the majority of the inputs are $1$; otherwise, it outputs $0$.  If $b=1$, it is said the warning is effective; otherwise (i.e.,   $b=0$), it is not effective. We assume the source code of the online banking system is static, is not updated, and any change to the source code can be detected (e.g., the bank can use a cryptographic commitment to commit to the source code). Moreover, we assume the online banking  system is secure.  %Moreover, we assume a customer does not collude with the APP scammer. 

%In our scheme, the bank requires to send a confirmation message to the smart contract after it transfers the customer's money. Since the transfer of money is performed off-chain, there is a possibility that  disputes occur between   the customer and bank in the case where there is an inconsistency between the money transfer and the confirmation (e.g., the bank transferred the money but did not send the confirmation). To allow such dispute to be resolved, we define a deterministic function   $\mathtt{madePayment}(w_{\st 1}, ..., w_{\st n})\rightarrow d$,  which takes $n$ binary verdicts of arbiters and outputs a binary value, $d$. This function outputs $1$ if the majority of the inputs are $1$; otherwise, it outputs $0$.  If $d=1$, it is said the payment has been made; otherwise (i.e.,   $d=0$), it has not been made. We assume the source code of the online banking system is static, is not updated, and any change to the source code can be detected (e.g., the bank can use a cryptographic commitment to commit to the source code).






% !TEX root =main.tex


\section{Challenges to Overcome}\label{sec:: challenges}


Our starting point in defining and designing a payment with dispute resolution scheme is the CRM code, as this code (although vaguely) sets out the primary requirements a victim must meet to be reimbursed.  To design such a scheme, we need to address several challenges. The rest of this section outlines these challenges. 


%Our starting point in defining and  designing a payment with dispute resolution scheme is the CRM code, as this code (although vaguely) sets out the primary requirements a victim must meet to be reimbursed.  To design such a scheme, we need to address  several challenges. The rest of this section  outlines these challenges. 

%In this section, we provide an overview of a Payment with Dispute Resolution (PwDR) scheme. Simply put, a PwDR scheme  allows a customer to interact with its bank (via the online banking platform) to transfer a certain amount of money from its account to another account in a transparent manner; meanwhile, it  offers a distinct  feature. Namely, when an APP fraud takes place, it lets an honest customer raise a dispute and \emph{prove} to a third-party dispute resolver that it has acted honestly, so it can be reimbursed. It offers other features too. For instance,  an honest bank can also prove it has acted honestly; it lets the parties prove their innocence  without their counter-party's collaboration,  it  ensures the message exchanged between a bank and customer remains confidential, and even the party which resolves the dispute between the two learns as little information as possible.  The PwDR scheme can be considered as an extension of the existing   online banking system. Our starting point in defining and  designing the PwDR scheme is the CRM code, as this code (although vaguely) sets out the primary requirements a victim must meet to be reimbursed.  To design such a scheme, we need to address  several challenges. The rest of this section  highlights these challenges. 







%\subsection{Challenge 1: Lack of Transparent Logs} 
%In the current online  banking system, during a payment journey, the messages exchanged between customer and bank are usually logged by the bank and are not accessible to the customer without the bank's collaboration. Even if the bank provides access to the transaction logs, there is no guarantee that the logs have remained intact. Due to the lack of a transparent logging mechanism, a customer or bank can wrongly claim that (a) it has sent a certain message or warning to its counter-party or (b) it has never  received a certain message, e.g., due to hardware or software failure.  Thus, it would be hard for an honest party (especially a customer) to prove its innocence. To address this challenge, our scheme uses a standard smart contract (as a public bulletin board) to which  each party needs to send (a copy of) all outgoing messages, e.g., payment requests, warnings, and  confirmation of payments. 



\subsection{Challenge 1: Lack of Transparent Logs} 
In the current online banking system, during a payment journey, the messages exchanged between customer and bank are usually logged by the bank and are not accessible to the customer without the bank's collaboration. Even if the bank provides access to the transaction logs, there is no guarantee that the logs have remained intact. Due to the lack of a transparent logging mechanism, a customer or bank can wrongly claim that (a) it has sent a certain message or warning to its counter-party or (b) it has never received a certain message.  Thus, it would be hard for an honest party to prove its innocence. To address this challenge, our scheme will use a  smart contract to which each party sends its messages.


\subsection{Challenge 2: Lack of Effective Warning's Accurate Definition in Banking}\label{sec::Lack-of-Effective-Warning-Definition}


One of the determining factors in the process of allocating liability to an APP fraud victim is following ``warning(s)'', according to the CRM code. However, there exists no publicly available study on the effectiveness of banks' warnings. So, we cannot hold a customer accountable for becoming the fraud's victim,  even if the related warnings are ignored. Furthermore, currently,  banks assess whether their own warnings are effective. But, in a fair process, such an assessment is conducted by a third party.  To address these challenges, we let a warning's effectiveness be determined on a case-by-case basis after an APP fraud occurs. The protocol lets a victim challenge a certain warning whose effectiveness will be assessed by a \emph{committee}, i.e., a  set of auditors. In this setting, each auditor provides its (encoded) verdict to the smart contract, from which a dispute resolver retrieves all verdicts to learn the final one. The scheme ensures that the final verdict is in the customer's favour if at least a threshold of the auditors voted so. Thus, unlike the traditional setting where a central party determines a warning's effectiveness, which is error-prone, we let a collection of auditors determines it.





\subsection{Challenge 3: Linking Off-chain Payments with a Smart Contract}\label{sec::Linking Off-chain-Payments-with-contract}
 Recall that an APP fraud occurs when a payment is made. In the case where a  bank sends  (to the smart contract) a confirmation of payment message, it is not possible to automatically validate such a claim, as the money transfer occurs outside of the blockchain network. To address this challenge, our scheme lets a customer raise a dispute and report it to the smart contract when it detects an inconsistency. In this case, the above auditors investigate and provide their verdicts to the smart contract. Then, dispute resolver $\mathcal{DR}$ extracts them and announces the final verdict. 



\subsection{Challenge 4: Preserving Privacy}
 Although the use of a public logging mechanism is essential in resolving disputes transparently, if it does not use a  privacy-preserving mechanism, then parties' privacy would be violated. To protect the privacy of the bank's and customers' messages against the public, our scheme lets them provably agree on encoding-decoding tokens with which they can encode their messages. Later, either party can provide the token to a third party  (e.g., $\mathcal{D}_{\st i}$) which checks the token's correctness, and decodes the messages. To protect the privacy of the committee members' verdicts from $\mathcal{DR}$, the scheme ensures that  $\mathcal{DR}$ can learn only the final verdict without being able to link a verdict to a specific auditor or even learn the number of yes/$1$ and no/$0$ votes. To this end, we develop and use novel threshold voting protocols. 











 
 
 
 

\input{Definition}





\section{Payment with Dispute Resolution Protocol}

In this section, first we  provide an outline of the PwDR protocol. Then, we present a few subroutines that will be used in this protocol. After that, we describe the PwDR protocol in detail. 




\subsection{An Overview of  the PwDR Protocol}


In this section, we provide an overview of the  PwDR protocol. For the sake of simplicity, we assume $\mathcal{C}$ wants to transfer a certain amount of money to a new payee. Initially, only for once,  customer  $\mathcal{C}$ and bank  $\mathcal{B}$   agree on a smart contract  $\mathcal{S}$. They also use the SAP to provably agree on two private statements that include two  secret keys. The keys will be used to encrypt  messages sent to $\mathcal{S}$ and will be used  by $\mathcal{D}_{\st j}$s and $\mathcal{DR}$ to decrypted related messages. When $\mathcal{C}$ wants to transfer money to a new payee, it   signs into the standard online banking system. Then, it generates an update  request that specifies the new payee's detail,  encrypts the request, and sends the result to  $\mathcal{S}$. After that, $\mathcal{B}$ decrypts and checks the request's validity, e.g.,  whether it  meets its internal policy or  the requirements of the ``Confirmation of Payee'' \cite{CoP}. Depending on the  request's content, $\mathcal{B}$ generates a pass or warning message. It encrypts the message and sends the result to $\mathcal{S}$. Then, $\mathcal{C}$ checks $\mathcal{B}$'s message and depending on the content of this message, it decides whether it wants to proceed to made  payment. If it decides to do so, then it sends an encrypted payment detail to $\mathcal{S}$. This  allows $\mathcal{B}$ to decrypt the message and locally transfer the amount of money specified in $\mathcal{C}$'s message. Once the money is transferred, $\mathcal{B}$ sends an encrypted  ``paid" message to $\mathcal{S}$. 


%Initially, the customer  $\mathcal{C}$ and bank  $\mathcal{B}$   agree on a smart contract  $\mathcal{S}$. They also use SAP to provably agree on two private statements; each statement includes a secret key, one is used to encrypt  messages that should be decrypted by all parties except $\mathcal{DR}$ and the other one is used to encrypt  messages that are supposed to be decrypted by $\mathcal{DR}$. For the sake of simplicity, we assume $\mathcal{C}$ wants to transfer a certain amount of money to a new payee. To do so, it   signs into the standard online banking system.  It generates an update  request that specifies the new payee's detail, it encrypts the request and sends the result to  $\mathcal{S}$. After that, $\mathcal{B}$ decrypts and checks this request's validity, e.g.,  whether it  meets its internal policy or  the requirements of the Confirmation of Payee (CoP) \cite{CoP}. Depending on the  request's content, $\mathcal{B}$ generates a pass or warning message. It encrypts the message and sends the result to $\mathcal{S}$. Then, $\mathcal{C}$ checks $\mathcal{B}$'s message and depending on the content of this message, it decides whether it wants to proceed to made  payment. If it decides to do so, then it sends an encrypted payment detail to $\mathcal{S}$. This  allows $\mathcal{B}$ to decrypt the message and locally transfer the amount of money specified in $\mathcal{C}$'s message. Once the money is transferred, $\mathcal{B}$ sends encrypted  ``paid" message to $\mathcal{S}$. 






%When it decides to invoke either  (a) algorithm $\mathtt{setupNewPayee}(.)$ to setup a new payee, or (b) algorithm $\mathtt{ammendExistingPayee}(.)$ to modify the existing payee's detail,  it sends the inputs of the related algorithm to $\mathcal{S}$ which runs the associated algorithm.  



%After a fixed time period, $\mathcal{B}$ checks the validity of the inputs that are registered by $\mathcal{C}$ in $\mathcal{S}$.  If the check fails (e.g., the input does not meet  the requirements of the Confirmation of Payee \cite{CoP}), then it sends  to $\mathcal{S}$ a warning message that includes the waring's detail; otherwise, it sends to $\mathcal{S}$ the ``pass'' message. Note that  $\mathcal{C}$ comes across with the same warning when it interacts with the online payment system's user interface.   Depending on the content of $\mathcal{B}$'s message, $\mathcal{C}$ decides whether it wants to proceed to made  payment. If it decides to make payment, then it sends the payment's detail to $\mathcal{S}$. This  allows $\mathcal{B}$ to locally transfer the amount of money specified in $\mathcal{C}$'s message. 


%Once the money is transferred, $\mathcal{B}$ sends the message ``paid" to $\mathcal{S}$. 


Once $\mathcal{C}$ realises that it has fallen victim to an APP fraud, it  raises a dispute. In particular, it  generates  an encrypted complaint that could  challenge the effectiveness of the warning and/or any payment inconsistency (as explained in Section \ref{sec::Linking Off-chain-Payments-with-contract}). It can also  include in the complaint an  evidence/certificate, e.g., to claim that it falls into  the vulnerable customer category as defined in \cite{CRM-code}. $\mathcal{C}$ encrypts the complaint. It also   encrypts the secret key (under arbiters' public key) that it uses to encrypt the messages. It sends to $\mathcal{S}$ the ciphertexts along with a proof asserting the secret key's correctness.  Upon receiving $\mathcal{C}$'s complaint, each committee member verifies the proof. If the verification passes, it decrypts and compiles $\mathcal{C}$'s complaint to generate a (set of) verdict. Then, each committee member encodes its verdict and sends the  encryption of the encoded verdict to $\mathcal{S}$. To resolve a dispute between $\mathcal{C}$ and $\mathcal{B}$, either of them can  invoke $\mathcal{DR}$. To do so, they  directly send to it one of the above secret keys and a proof asserting that key was generated correctly.   $\mathcal{DR}$ verifies the proof. If the verification passes it locally decrypts the encrypted encoded verdicts (after retrieving them from $\mathcal{S}$) and then combines the result to find out the final verdict.  If the final verdict indicates the legitimacy of  $\mathcal{C}$'s complaint, then $\mathcal{C}$ must be reimbursed.   Note, the verdicts are encoded in such a way that even after decrypting them, the dispute resolver cannot link a verdict to a committee member or even figure out how many $1$ or $0$ verdicts have been provided  (except when all verdicts are $0$). However, it can find out whether at least threshold committee members voted  in faviour of $\mathcal{C}$. Shortly, we present  novel verdict encoding-decoding protocols that offer the above features. 




%  Upon receiving $\mathcal{C}$'s message, $\mathcal{S}$ carries out the following primary checks: (a)   $\mathcal{C}$ requested a payment, (b) a warning was provided, (c)   the warning was effective, and (d)   the (off-chain) payment has been made. $\mathcal{S}$ concludes $\mathcal{C}$ should be reimbursed if both checks (a) and (d)  pass while either (b) or (c)  fails. 

%Before, we provide the  PwDR protocol in detail, we present two subroutines that will be needed and invoked by the PwDR protocol. 


%!TEX root = main.tex

%\subsection{The PwDR Protocol's Subroutines}\label{sec::PwDR-Subroutines}


%In this section, we present a few subroutines that will be  called  by  the PwDR protocol. 



%\subsubsection{Determining Bank's Message Status.}
%
%  As we stated earlier, in the payment journey the customer may receive a ``pass'' message or even nothing at all, e.g., due to the system failure. In such cases,  a victim of  an APP fraud may complain that the pass or missing message should have been a warning one that  ultimately would have prevented the victim from falling into the APP fraud. To assist the committee members to deal with and process such complaints deterministically, we propose a process  called $\mathtt{verStat}(.)$ that outputs $1$ in the case where a pass message was given correctly or the missing message could not play prevent the scam, and output $0$, otherwise. The process is defined in figure \ref{fig:verstat}.
  
  
% !TEX root =main.tex

\vspace{-2mm}
\subsection{A Subroutines for Determining Bank's Message Status}

As we stated earlier, in the payment journey the customer may receive a ``pass'' message or even nothing at all, e.g., due to a system failure. In such cases,  a victim  must be able to complain that if the pass or missing message was   a warning, then it   would have prevented it  from falling victim. To assist the committee members to deal with  such complaints deterministically, we propose    $\mathtt{verStat}(.)$ algorithm, which is run locally by each committee member. This algorithm is presented in Figure \ref{fig:verStat}. 

%To assist the committee members deterministically  compile a victim's  complaint about a warning's  effectiveness, we propose an algorithm,  called $\mathtt{checkWarning}(.)$  run locally by each committee member. It also allows the victims to provide a certificate/evidence as part of their complaints.   This algorithm is presented in Figure \ref{fig:checkWarning}.

%
%It outputs $0$ if a pass message was given correctly or the missing message could not  prevent the fraud, and outputs $1$ otherwise. 
%
 \vspace{-2mm}
\begin{figure}[!htbp]
\setlength{\fboxsep}{1pt}
\begin{center}
    \begin{tcolorbox}[enhanced,width=5.5in, height=42.5mm,
    drop fuzzy shadow southwest,
    colframe=black,colback=white]
\small{
    \vspace{-2.5mm}
\underline{$\mathtt{verStat}(add_{\st\mathcal{S}}, m^{\st\mathcal{(B)}},  \bm l, \Delta, aux)\rightarrow w_{\st 1}$}\\
%
\vspace{-2.5mm}
\begin{itemize}
\item\noindent\textit{Input}. $add_{\st\mathcal{S}}$: the address of smart contract $\mathcal{S}$, $m^{\st\mathcal{(B)}}$:  $\mathcal{B}$'s warning message,  $\bm l$:  customer's payees' list, $\Delta$: a time parameter, and $aux$: auxiliary information, e.g., bank's policy. 
%
\item\noindent\textit{Output}. $ w_{\st 1}=0$: if the ``pass'' message had been given correctly or the missing message did not play any role in preventing the fraud; $ w_{\st 1}=1$: otherwise. 
\end{itemize}
\begin{enumerate}
\item reads the content of   $\mathcal{S}$. It checks if $m^{\st\mathcal{(B)}}=$``pass''  or the encrypted warning message was not sent on time (i.e., never sent or sent after    $t_{\st 0}+\Delta$).  If one of the checks passes, it proceeds; Otherwise, it aborts. 
\item checks the validity of  customer's most recent payees' list $\bm l$, with the help of the auxiliary information, $aux$. 
\begin{itemize}
\item[$\bullet$]  if $\bm l$ contains an invalid element,  it sets $ w_{\st 1}=1$.
\item [$\bullet$] otherwise, it sets $ w_{\st 1}=0$.
\end{itemize}
\item returns $ w_{\st 1}$.
\vspace{-1.4mm}
\end{enumerate}
}
\end{tcolorbox}
\end{center}
\vspace{-2mm}
\caption{Algorithm to Determine a Bank's Message Status} 
\label{fig:verStat}
\end{figure}




%\begin{figure}[!htbp]
%\setlength{\fboxsep}{1pt}
%\begin{center}
%\begin{boxedminipage}{13.3cm}
%\small{
%\underline{$\mathtt{verStat}(add_{\st\mathcal{S}}, m^{\st\mathcal{(B)}},  \bm l, \Delta, aux)\rightarrow w_{\st 1}$}\\
%%
%\begin{itemize}
%\item\noindent\textit{Input}. $add_{\st\mathcal{S}}$: the address of smart contract $\mathcal{S}$, $m^{\st\mathcal{(B)}}$:  $\mathcal{B}$'s warning message,  $\bm l$:  customer's payees' list, $\Delta$: a time parameter, and $aux$: auxiliary information, e.g., bank's policy. 
%%
%\item\noindent\textit{Output}. $ w_{\st 1}=0$: if the ``pass'' message had been given correctly or the missing message did not play any role in preventing the fraud; $ w_{\st 1}=1$: otherwise. 
%\end{itemize}
%\begin{enumerate}
%\item reads the content of   $\mathcal{S}$. It checks if $m^{\st\mathcal{(B)}}=$``pass''  or the encrypted warning message was not sent on time (i.e., never sent or sent after    $t_{\st 0}+\Delta$).  If one of the checks passes, it proceeds; Otherwise, it aborts. 
%\item checks the validity of  customer's most recent payees' list $\bm l$, with the help of the auxiliary information, $aux$. 
%\begin{itemize}
%\item[$\bullet$]  if $\bm l$ contains an invalid element,  it sets $ w_{\st 1}=1$.
%\item [$\bullet$] otherwise, it sets $ w_{\st 1}=0$.
%\end{itemize}
%\item returns $ w_{\st 1}$.
%\vspace{1mm}
%
%\end{enumerate}
%
%}
%\end{boxedminipage}
%\end{center}
%\caption{Algorithm to Determine a Bank's Message Status} 
%\label{fig:verStat}
%\end{figure}
%%%%%%%%%%%%%%%%%%%%%%%%%%%%%%%%%%%%%%

  % !TEX root =main.tex



\subsubsection{Checking Warning's Effectiveness.}



To help the committee members deterministically and accurately compile a victim's  complaint about the effectiveness of a warning (that bank provides during the payment journey), we propose a process,  called $\mathtt{checkWarning}$.  This process is run locally by each committee member. It also allows the victims to provide a certificate/evidence as part of their complaints.   The process outputs a pair $(w_{\st 2}, w_{\st 3})$. It  sets $w_{\st 2}=1$  if the given warning message is effective, and sets $w_{\st 2}=0$, if it is not. It sets $w_{\st 3}=1$ if the certificate that the victim provided is valid (or empty) and sets $w_{\st 3}=0$ if it is invalid.  This process is presented in figure \ref{fig:checkWarning}.


\begin{figure}[!htbp]
\setlength{\fboxsep}{0.7pt}
\begin{center}
\begin{boxedminipage}{12.3cm}
\small{
\underline{$\mathtt{checkWarning}(add_{\st\mathcal{S}}, z, m^{\st\mathcal{(B)}},  aux')\rightarrow (w_{\st 2},  w_{\st 3})$}\\
%
\begin{itemize}
%
\item \noindent\textit{Input}. $add_{\st\mathcal{S}}$: the address of smart contract $\mathcal{S}$, $z$:  $\mathcal{C}$'s complaint, $m^{\st\mathcal{(B)}}$:  $\mathcal{B}$'s warning message,  and $aux'$: auxiliary information, e.g., guideline on warnings' effectiveness. 
%
\item\noindent\textit{Output}. $w_{\st 3}=1$: if the certificate in $z$ is valid or no certificate is provided; $ w_{\st 3}=0$: if the certificate is invalid. Also, 
$w_{\st 2}=1$: if the given warning message is effective; $w_{\st 2}=0$: if the  warning message is ineffective.

\end{itemize}


\begin{enumerate}
%
\item parse $z= m||sig||pk||\text{``challenge warning''}$. If the certificate $sig$ is  empty, then it  sets $w_{\st 3}=0$ and proceeds to  step \ref{check-m}. Otherwise, it:
%
\begin{enumerate} 
%
\item verifies the certificate: $\mathtt{Sig.ver}(pk, m, sig)\rightarrow h$. 
%
\item if  the certificate is rejected (i.e., $h=0$),  it sets $w_{\st 3}=1$. It  goes to step \ref{return}. 
%
\item otherwise (i.e., $h=1$), it sets $w_{\st 3}=0$ and moves onto the next step. 
\end{enumerate}
%
%\item  reads the content of $m$ in $\mathcal{S}$.
%
\item\label{check-m} checks if ``warning'' $\in m^{\st\mathcal{(B)}}$.  If the check is passed, it proceeds to the next step. Otherwise, it aborts. 
%
\item checks the warning's effectiveness, with the assistance of the evidence $m$ and auxiliary information $aux'$. 
%
\begin{itemize}
\item[$\bullet$]  if it is effective,  it sets $w_{\st 2} = 1$.
\item [$\bullet$] otherwise, it sets $w_{\st 2} = 0$.
\end{itemize}
\item\label{return} returns $(w_{\st 2}, w_{\st 3})$.
\end{enumerate}
 
}
\end{boxedminipage}
\end{center}
\caption{Process to Check Warning's Effectiveness} 
\label{fig:checkWarning}
\end{figure}
%%%%%%%%%%%%%%%%%%%%%%%%%%%%%%%%%%%%%%

  % !TEX root =main.tex

\subsection{Subroutines for  Encoding-Decoding Verdicts}


In this section, we present verdict encoding and decoding protocols. They let a third party $\mathcal{I}$, e.g., $\mathcal{DR}$, find out whether threshold  arbiters voted $1$, while satisfying the following  requirements.  The protocols should (1) generate unlinkable verdicts, (2)  not require arbiters to interact with each other for each customer, and (3) be  efficient. Since, the second and third requirements are self-explanatory,  we only explain the first one.  Informally, the first property requires  that the protocols should generate encoded verdicts and final verdict in a way that $\mathcal{I}$,  given the encoded verdicts and final verdict, should not be able to (a)   link a  verdict to an arbiter (except when all arbiters' verdicts are $0$), and (b) find out the total number of $1$ or $0$ verdicts when they provide different verdicts.  In this section, we present two variants of verdict encoding and decoding protocols, where each variant contains two protocol. The first variant is highly efficient and suitable for the case where the threshold is $1$. The second variant is  generic and works for any threshold. The latter variant is less efficient than the former one, but it is still efficient. Below, we explain each variant. 


\subsubsection{Variant 1: Highly Efficient Verdict  Encoding-Decoding Protocols.}

Here, we present two efficient verdict encoding and decoding protocols; namely, Private Verdict Encoding (PVE) and Final Verdict Decoding (FVD) protocols, that lets $\mathcal{I}$ find out whether at least one arbiter voted $1$, while satisfying the above requirements.  At a high level, the protocols work as follows.  The arbiters only once for all customers agree on a secret key of a pseudorandom function. This key will let each of them   generate a pseudorandom masking values such that if all masking values are ``XOR''ed, they would cancel out each other and result $0$.\footnote{This is similar to the idea used in the XOR-based secret sharing \cite{Schneier0078909}.} Each arbiter represents its verdict by (i) representing it as a parameter which is set to either $0$ if the verdict is $0$ or a random value if the verdict is $1$, and then (ii) masking this parameter by the above  pseudorandom value.  It sends the result to $\mathcal{I}$.  To decode the final verdict and find out whether any arbiter voted $1$, $\mathcal{I}$  does XOR all encoded verdicts. This removes the masks and XORs are verdicts' representations.  If the result is $0$, then    all arbiters must have voted $0$; therefore,  the final verdict is $0$. However, if the result is not $0$ (i.e., a random value), then at least one of the arbiters voted $1$, so  the final verdict is $1$. We present the encoding  and decoding protocols in figures \ref{fig:PVE} and \ref{fig:FVD} respectively.
 
 
 Not that the protocols' correctness holds with an overwhelming probability. In particular, if two arbiters  represent their verdict by an identical random value, then when they are XORed they would cannel out each other which can affect the result's correctness. The same holds if the XOR of  multiple verdicts' representations results in a value that can cancel out another verdict's representation. Nevertheless, the probability that such an event occurs is negligible in the security parameter $|p|=\lambda$, i.e., at most   $\frac{1}{2^{\st \lambda}}$. It is evident that PVE and FVD protocols meet properties (2) and (3). The primary reason they also meet  property (1) is that each masked verdict reveals nothing about the verdict (and its representation) and  given the final verdict, $\mathcal{I}$ cannot distinguish between the case where there is exactly one arbiter that voted  $1$ and the case where multiple arbiters voted $1$, as in both cases $\mathcal{I}$   extracts only a single random value, which reveals nothing about the number of arbiters which voted $0$ or $1$. We will use this variant in the PwDR protocol. 
 

\subsubsection{Variant 2: Generic Verdict  Encoding-Decoding Protocols.} Now, we present two  efficient generic verdict  encoding-decoding protocols, denoted by GPVE and GFVD,  that let $\mathcal{I}$ find out whether at least $e$ arbiters voted $1$, where $e$ can be any integer in the range $[1, n]$. This variant is  built upon the previous one; however, it uses a novel idea that relies on  Bloom filter and combinatorics.  At a high level, the encoding protocol works as follows.  The arbiters (similar to Variant 1) agree on a secret key of a pseudorandom function. As before, each arbiter will use this key to  generate a pseudorandom masking value such that if all arbiters' masking values are ``XOR''ed, they would cancel out each other. Then, each arbiter represents its verdict by a parameter. In particular, if its verdict is $0$, then  it  sets the parameter to either $0$. However, if   its verdict is $1$, it sets the parameter to a fresh \emph{pseudorandom} value $\alpha_{\st j}$ (instead of a random value used in Variant 1),  where this  pseudorandom value is derived from the above key. Therefore, there would be a set $A=\{\alpha_{\st 1},..., \alpha_{\st n}\}$ from which  $\mathcal{D}_{\st j}$ would pick $\alpha_{\st j}$ to represent its verdict if its verdict is $1$. Next, each arbiter masks its verdict representation by its masking  value. It sends the result to 
 $\mathcal{I}$.
 

Arbiter $\mathcal{D}_{\st n}$ also generates a Bloom filter that contains the combinations of set $A$'s elements, regardless of whether a certain arbiter's  vote is $0$ or $1$. More specifically,  for every integer $i$ in the range $[e,n]$, computes the combinations, without repetition, of $i$ elements from set $A=\{\alpha_{\st 1},..., \alpha_{\st n}\}$, where when multiple elements are taken at a time (i.e., $i>1$), the elements are XORed with each other. Let $W=\{(\alpha_{\st 1}\oplus ... \oplus \alpha_{\st e}),  (\alpha_{\st 2}\oplus  ... \oplus \alpha_{\st e+1}), ..., (\alpha_{\st 1}\oplus ... \oplus \alpha_{\st n})\}$ be the result. For instance, when $n=3$ and $e=2$,  we would have $W=\{(\alpha_{\st 1}\oplus \alpha_{\st 2}),  (\alpha_{\st 2}\oplus  \alpha_{\st 3}), (\alpha_{\st 1}\oplus \alpha_{\st 3}), (\alpha_{\st 1}\oplus \alpha_{\st 2} \oplus \alpha_{\st n})\}$. After that, it generates an empty Bloom filter and  inserts all elements of $W$ into this Bloom filter. Let $\mathtt{BF}$ be the Bloom filter that encodes $W$'s elements. It sends $\mathtt{BF}$ to $\mathcal{I}$. To decode and extract the final verdict, as in Variant 1, $\mathcal{I}$ does XOR all masked verdict representations which  removes the masking values and XORs are verdicts’ representations. If the result is $0$, then $\mathcal{I}$ concludes that all arbiters must have voted $0$ (with a high probability); so, it sets the final verdict to $0$. However, if the result is a non-zero value, then it checks whether the value is in the Bloom filter. If it is, then it concludes that at least threshold arbiters voted $1$, so it sets the final vector to $1$. Otherwise (if the value is not in the Bloom filter), it concludes that less than threshold arbiters voted $1$; therefore, it sets the final verdict to $0$.  Figures \ref{fig:GPVE} and \ref{fig:GFVD}, in Appendix \ref{sec::Generic-Verdict-Encoding-Decoding-Protocols}, present the  GPVE and GFVD protocols in detail. 



%\item[$\bullet$] for every integer $i$ in the range $[e,n]$, computes the combinations (without repetition) of $i$ elements from set $\{\alpha_{\st 1},..., \alpha_{\st n}\}$, where the combination operation is XOR. Let $W=\{(\alpha_{\st 1}\oplus \alpha_{\st 2}\oplus... \oplus \alpha_{\st e}), (\alpha_{\st 2}\oplus \alpha_{\st 3}\oplus ... \oplus \alpha_{\st e+1}), ..., (\alpha_{\st 1}\oplus \alpha_{\st 2}\oplus... \oplus \alpha_{\st n})\}$ be the result.  
%

%\item[$\bullet$] constructs an empty Bloom filter. Then, it inserts all elements of $W$ into this Bloom filter. Let $\mathtt{BF}$ be the Bloom filter encoding $W$'s elements. 
 
  
  
 
 
 
 



 %and then (ii) masking this parameter by the above pseudorandom value. It sends the result to I. 

%Now, we explain how the GPVE and GFVD work. 





%In Appendix \ref{sec::bloom-filter-}, we explain how the Bloom filter parameters can be set.  




%
%Here, we present two efficient verdict encoding and decoding protocols; namely, Private Verdict Encoding (PVE) and Final Verdict Decoding (FVD) protocols. Their goal is to let a third party $\mathcal{I}$, e.g., $\mathcal{DR}$, find out whether at least one arbiter voted $1$, while satisfying the following  requirements.  The protocols should (1) generate unlinkable verdicts, (2)  not require arbiters to interact with each other for each customer, and (3) be  efficient. Since, the second and third requirements are self-explanatory,  we only explain the first one.  Informally, the first property requires  that the protocols should generate encoded verdicts and final verdict in a way that $\mathcal{I}$,  given the encoded verdicts and final verdict, should not be able to (a)   link a  verdict to an arbiter (except when all arbiters' verdicts are $0$), and (b) find out the total number of $1$ or $0$ verdicts when they provide different verdicts. 
%
%
%
% At a high level, the protocols work as follows.  The arbiters only once for all customers agree on a secret key of a pseudorandom function. This key will allow each of them to generate a pseudorandom masking values such that if all masking values are ``XOR''ed, they would cancel out each other and result $0$.\footnote{This is similar to the idea used in the XOR-based secret sharing \cite{Schneier0078909}.}
% 
% 
% 
% 
% 
%Each arbiter represents its verdict by (i) representing it as a parameter which is set to either $0$ if the verdict is $0$ or to a random value if the verdict is $1$, and then (ii) masking this parameter by the above  pseudorandom value.  It sends the result to $\mathcal{I}$.  To decode the final verdict and find out whether any arbiter voted $1$, $\mathcal{I}$  does XOR all encoded verdicts. This removes the masks and XORs are verdicts' representations.  If the result is $0$, then    all arbiters must have voted $0$; therefore,  the final verdict is $0$. However, if the result is not $0$ (i.e., a random value), then at least one of the arbiters voted $1$, so  the final verdict is $1$. We present the encoding  and decoding protocols in figures \ref{fig:PVE} and \ref{fig:FVD} respectively.
% 
% 
% Not that the protocols' correctness holds, except  a negligible  probability. In particular, if two arbiters  represent their verdict by an identical random value, then when they are XORed they would cannel out each other which can affect the result's correctness. The same holds if the XOR of  multiple verdicts' representations results in a value that can cancel out another verdict's representation. Nevertheless, the probability that such an event occurs is negligible in the security parameter, i.e., at most   $\frac{1}{2^{\st \lambda}}$. It is evident that PVE and FVD protocols meet properties (2) and (3). The primary reason they also meet  property (1) is that each masked verdict reveals nothing about the verdict (and its representation) and  given the final verdict, $\mathcal{I}$ cannot distinguish between the case where there is exactly one arbiter that voted  $1$ and the case where multiple arbiters voted $1$, as in both cases $\mathcal{I}$   extracts only a single random value, which reveals nothing about the number of arbiters which voted $0$ or $1$. 
% 
%




\begin{figure}[!ht]
\setlength{\fboxsep}{0.7pt}
\begin{center}
\begin{boxedminipage}{12.3cm}
\small{
\underline{$\mathtt{PVE}(\bar{k}_{\st 0}, \text{ID},  w_{\st j}, o, n,  j)\rightarrow  \bar{  w}_{\st j}$}\\
%
\begin{itemize}
\item \noindent\textit{Input.} $\bar{k}_{\st 0}$: a key of  pseudorandom function $\mathtt{PRF}(.)$, $\text{ID}$: a unique identifier, $ w_{\st j}$: a  verdict, $o$: an offset, $n$: the total number of  arbiters,  and  $j$: an arbiter's index.
%
\item \noindent\textit{Output.} $\bar{  w}_{\st j}$:  an  encoded verdict.  
%
\end{itemize}
Arbiter $\mathcal{D}_{\st j}$ takes the following steps.
\begin{enumerate}
%
\item\label{ZSPA:val-gen} computes a  pseudorandom  value,  as follows. 
%
\begin{itemize}
%
\item[$\bullet$]$ \text{ if } j< n: r_{\st j}=\mathtt{PRF}(\bar k_{\st 0}, o||j||\text{ID})$.\\
%
\item [$\bullet$] $ \text{ if } j=n: r_{\st j}= \bigoplus\limits^{\st n-1}_{\st i=1} r_{\st i}$.
%
\end{itemize}
%
\item  sets a fresh parameter, $w'_{\st j}$, as below. 
%
\begin{equation*}
   w'_{\st j}= 
\begin{cases}
   0,              & \text{if } w_{\st j}=0\\
   \alpha_{\st j}\stackrel{\st\$}\leftarrow \mathbb{F}_{\st p} ,& \text{if } w_{\st j}=1\\

    %0,              & \text{if } w_{\st j}=0
\end{cases}
\end{equation*}
%
\item encodes  $w'_{\st j}$ as follows. %$\forall i,1\leq i\leq s:$
%
$\bar w_{\st j}= w'_{\st j}\oplus r_{\st j}$.
%
\item outputs $\bar{ w}_{\st j}$.


\
 \end{enumerate}
 
}
\end{boxedminipage}
\end{center}
\caption{Private Verdict Encoding  (PVE) Protocol} 
\label{fig:PVE}
\end{figure}
%
\begin{figure}[!ht]
\setlength{\fboxsep}{0.7pt}
\begin{center}
\begin{boxedminipage}{12.3cm}
\small{
\underline{$\mathtt{FVD}(n,  \bar{\bm w})\rightarrow  v$}\\
%
\begin{itemize}
\item \noindent\textit{Input.} $n$:  the total number of  arbiters,  and  $\bar{\bm w}=[\bar{ w}_{\st 1},..., \bar{ w}_{\st n}]$:  a vector of all arbiters' encodes  verdicts.
%
\item \noindent\textit{Output.} $v$: final verdict.  
%
\end{itemize}
A third-party $\mathcal{I}$ takes the following steps.
\begin{enumerate}
%
\item combines  all arbiters' encoded verdicts, $\bar w_{\st j}\in \bar{\bm {w}}$, as follows. 
%
$c= \bigoplus\limits^{\st n}_{\st j=1} \bar w_{\st j}$
%
\item sets the final verdict $v$ depending on the content of $c$. Specifically, 
%
\begin{equation*}
   v= 
\begin{cases}
    0,              &\text{if } c= 0\\
   1 ,& \text{otherwise }\\

\end{cases}
\end{equation*}
%
\item outputs  $v$. 

\
 \end{enumerate}
 
}
\end{boxedminipage}
\end{center}
\caption{Final Verdict Decoding  (FVD) Protocol} 
\label{fig:FVD}
\end{figure}


  
  
  
  
  
  
  


%In this section, we present a set of subroutines that are  called  by  and help each arbiter to  processes a customer's complaint.  As we stated earlier, in the payment journey the customer may receive a ``pass'', or warning message (or nothing at all). In the PwDR protocol, when a customer has fallen to an APP scam, it is allowed to challenge the validity of these messages. In particular, it can complain that (a) the pass message should have been a warning message  or the bank has not provided any message  and if it provided a warning then the scam would be prevented, or (b) the warning message provided by the bank was ineffective. To let the arbiter deal with the above situations and reach a verdict we define two protocols $\mathtt{verStat}(.)$ and $\mathtt{checkWarning}(.)$, presented below. 
%
%\small{
%\begin{center}
%\fbox{\begin{minipage}{5 in}
%%\caption{Private Verdict Encoding  (PVE) Protocol} 
%\underline{$\mathtt{verStat}(add_{\st\mathcal{S}}, m^{\st\mathcal{(B)}},  \bm l, \Delta, aux)\rightarrow w_{\st 1}$}\\
%%
%\begin{itemize}
%\item\noindent\textit{Input}. $add_{\st\mathcal{S}}$: the address of smart contract $\mathcal{S}$, $m^{\st\mathcal{(B)}}$:  $\mathcal{B}$'s warning message,  $\bm l$:  customer's payees' list, $\Delta$: a time parameter, and $aux$: auxiliary information, e.g., bank's policy. 
%%
%\item\noindent\textit{Output}. $ w_{\st 1}=1$: if the ``pass'' message had been given correctly or the missing message did not play any role in preventing the scam; $ w_{\st 1}=0$: otherwise. 
%\end{itemize}
%\begin{enumerate}
%\item reads the content of   $\mathcal{S}$. It checks if $m^{\st\mathcal{(B)}}=$``pass''  or the encrypted warning message was not sent on time (i.e., never sent or sent after    $t_{\st 0}+\Delta$).  If one of the checks passes, it proceeds to the next step. Otherwise, it aborts. 
%\item checks the validity of  customer's most recent payees' list $\bm l$, with the help of the auxiliary information, $aux$. 
%\begin{itemize}
%\item[$\bullet$]  if $\bm l$ contains an invalid element,  it sets $ w_{\st 1}=0$.
%\item [$\bullet$] otherwise, it sets $ w_{\st 1}=1$.
%\end{itemize}
%\item returns $ w_{\st 1}$.
%\end{enumerate}
%\label{fig:verstat}
%\end{minipage}}
%\end{center}
%}

%%\small{
%\begin{center}
%\fbox{\begin{minipage}{5 in}
%$\mathtt{verPass}(add_{\st\mathcal{S}}, m, l, \text{aux})\rightarrow d\in\{0,1\}$\\
%------------------
%\
%
%\noindent\textbf{Input}. $add_{\st\mathcal{S}}$: the address of smart contract $\mathcal{S}$, $m$: $\mathcal{S}$'s field that corresponds to $\mathcal{B}$'s message, $l$: $\mathcal{S}$'s field related to customer's payees' list, and \text{aux}: auxiliary information, e.g., bank's policy. 
%
%\noindent\textbf{Output}. $d=1$: if the ``pass'' message had been given correctly; $d=0$: otherwise. 
%\begin{enumerate}
%\item reads the content of  $m$ in $\mathcal{S}$.
%\item checks if $m=$``pass''. If the check is passed, it proceeds to the next step. Otherwise, it aborts. 
%\item checks the validity of  customer's most recent payees' list $l$ on  $\mathcal{S}$, with the help of the auxiliary information $\text{aux}$. 
%\begin{itemize}
%\item[$\bullet$]  if $l$ contains an invalid element,  it sets $d=0$.
%\item [$\bullet$] otherwise, it sets $d=1$.
%\end{itemize}
%\item returns $d$.
%\end{enumerate}
%\end{minipage}}
%\end{center}
%%}



%\small{
%\begin{center}
%\fbox{\begin{minipage}{5 in}
%\underline{$\mathtt{checkWarning}(add_{\st\mathcal{S}}, z, m^{\st\mathcal{(B)}},  aux')\rightarrow (w_{\st 2},  w_{\st 3})$}\\
%%
%\begin{itemize}
%%
%\item \noindent\textit{Input}. $add_{\st\mathcal{S}}$: the address of smart contract $\mathcal{S}$, $z$:  $\mathcal{C}$'s complaint, $m^{\st\mathcal{(B)}}$:  $\mathcal{B}$'s warning message,  and $aux'$: auxiliary information, e.g., guideline on warnings' effectiveness. 
%%
%\item\noindent\textit{Output}. $w_{\st 3}=1$: if the certificate in $z$ is valid or no certificate is provided; $ w_{\st 3}=0$: if the certificate is invalid. Also, 
%$w_{\st 2}=1$: if the given warning message is effective; $w_{\st 2}=0$: if the  warning message is ineffective.
%
%\end{itemize}
%
%
%\begin{enumerate}
%%
%\item parse $z= m||sig||pk||\text{``challenge warning''}$. If the certificate $sig$ is  empty, then it  sets $w_{\st 3}=1$ and proceeds to  step \ref{check-m}. Otherwise, it:
%%
%\begin{enumerate} 
%%
%\item verifies the certificate: $\mathtt{Sig.ver}(pk, m, sig)\rightarrow h$. 
%%
%\item if  the certificate is rejected (i.e., $h=0$),  it sets $w_{\st 3}=0$. It  goes to step \ref{return}. 
%%
%\item otherwise (i.e., $h=1$), it sets $w_{\st 3}=1$ and moves onto the next step. 
%\end{enumerate}
%%
%%\item  reads the content of $m$ in $\mathcal{S}$.
%%
%\item\label{check-m} checks if ``warning'' $\in m^{\st\mathcal{(B)}}$.  If the check is passed, it proceeds to the next step. Otherwise, it aborts. 
%%
%\item checks the warning's effectiveness, with the assistance of the evidence $m$ and auxiliary information $aux'$. 
%%
%\begin{itemize}
%\item[$\bullet$]  if it is effective,  it sets $w_{\st 2} = 1$.
%\item [$\bullet$] otherwise, it sets $w_{\st 2} = 0$.
%\end{itemize}
%\item\label{return} returns $(w_{\st 2}, w_{\st 3})$.
%\end{enumerate}
%\end{minipage}}
%\end{center}
%%}
%
%% !TEX root =main.tex

\subsection{Subroutines for  Encoding-Decoding Verdicts}


In this section, we present verdict encoding and decoding protocols. They let a third party $\mathcal{I}$, e.g., $\mathcal{DR}$, find out whether threshold  arbiters voted $1$, while satisfying the following  requirements.  The protocols should (1) generate unlinkable verdicts, (2)  not require arbiters to interact with each other for each customer, and (3) be  efficient. Since, the second and third requirements are self-explanatory,  we only explain the first one.  Informally, the first property requires  that the protocols should generate encoded verdicts and final verdict in a way that $\mathcal{I}$,  given the encoded verdicts and final verdict, should not be able to (a)   link a  verdict to an arbiter (except when all arbiters' verdicts are $0$), and (b) find out the total number of $1$ or $0$ verdicts when they provide different verdicts.  In this section, we present two variants of verdict encoding and decoding protocols, where each variant contains two protocol. The first variant is highly efficient and suitable for the case where the threshold is $1$. The second variant is  generic and works for any threshold. The latter variant is less efficient than the former one, but it is still efficient. Below, we explain each variant. 


\subsubsection{Variant 1: Highly Efficient Verdict  Encoding-Decoding Protocols.}

Here, we present two efficient verdict encoding and decoding protocols; namely, Private Verdict Encoding (PVE) and Final Verdict Decoding (FVD) protocols, that lets $\mathcal{I}$ find out whether at least one arbiter voted $1$, while satisfying the above requirements.  At a high level, the protocols work as follows.  The arbiters only once for all customers agree on a secret key of a pseudorandom function. This key will let each of them   generate a pseudorandom masking values such that if all masking values are ``XOR''ed, they would cancel out each other and result $0$.\footnote{This is similar to the idea used in the XOR-based secret sharing \cite{Schneier0078909}.} Each arbiter represents its verdict by (i) representing it as a parameter which is set to either $0$ if the verdict is $0$ or a random value if the verdict is $1$, and then (ii) masking this parameter by the above  pseudorandom value.  It sends the result to $\mathcal{I}$.  To decode the final verdict and find out whether any arbiter voted $1$, $\mathcal{I}$  does XOR all encoded verdicts. This removes the masks and XORs are verdicts' representations.  If the result is $0$, then    all arbiters must have voted $0$; therefore,  the final verdict is $0$. However, if the result is not $0$ (i.e., a random value), then at least one of the arbiters voted $1$, so  the final verdict is $1$. We present the encoding  and decoding protocols in figures \ref{fig:PVE} and \ref{fig:FVD} respectively.
 
 
 Not that the protocols' correctness holds with an overwhelming probability. In particular, if two arbiters  represent their verdict by an identical random value, then when they are XORed they would cannel out each other which can affect the result's correctness. The same holds if the XOR of  multiple verdicts' representations results in a value that can cancel out another verdict's representation. Nevertheless, the probability that such an event occurs is negligible in the security parameter $|p|=\lambda$, i.e., at most   $\frac{1}{2^{\st \lambda}}$. It is evident that PVE and FVD protocols meet properties (2) and (3). The primary reason they also meet  property (1) is that each masked verdict reveals nothing about the verdict (and its representation) and  given the final verdict, $\mathcal{I}$ cannot distinguish between the case where there is exactly one arbiter that voted  $1$ and the case where multiple arbiters voted $1$, as in both cases $\mathcal{I}$   extracts only a single random value, which reveals nothing about the number of arbiters which voted $0$ or $1$. We will use this variant in the PwDR protocol. 
 

\subsubsection{Variant 2: Generic Verdict  Encoding-Decoding Protocols.} Now, we present two  efficient generic verdict  encoding-decoding protocols, denoted by GPVE and GFVD,  that let $\mathcal{I}$ find out whether at least $e$ arbiters voted $1$, where $e$ can be any integer in the range $[1, n]$. This variant is  built upon the previous one; however, it uses a novel idea that relies on  Bloom filter and combinatorics.  At a high level, the encoding protocol works as follows.  The arbiters (similar to Variant 1) agree on a secret key of a pseudorandom function. As before, each arbiter will use this key to  generate a pseudorandom masking value such that if all arbiters' masking values are ``XOR''ed, they would cancel out each other. Then, each arbiter represents its verdict by a parameter. In particular, if its verdict is $0$, then  it  sets the parameter to either $0$. However, if   its verdict is $1$, it sets the parameter to a fresh \emph{pseudorandom} value $\alpha_{\st j}$ (instead of a random value used in Variant 1),  where this  pseudorandom value is derived from the above key. Therefore, there would be a set $A=\{\alpha_{\st 1},..., \alpha_{\st n}\}$ from which  $\mathcal{D}_{\st j}$ would pick $\alpha_{\st j}$ to represent its verdict if its verdict is $1$. Next, each arbiter masks its verdict representation by its masking  value. It sends the result to 
 $\mathcal{I}$.
 

Arbiter $\mathcal{D}_{\st n}$ also generates a Bloom filter that contains the combinations of set $A$'s elements, regardless of whether a certain arbiter's  vote is $0$ or $1$. More specifically,  for every integer $i$ in the range $[e,n]$, computes the combinations, without repetition, of $i$ elements from set $A=\{\alpha_{\st 1},..., \alpha_{\st n}\}$, where when multiple elements are taken at a time (i.e., $i>1$), the elements are XORed with each other. Let $W=\{(\alpha_{\st 1}\oplus ... \oplus \alpha_{\st e}),  (\alpha_{\st 2}\oplus  ... \oplus \alpha_{\st e+1}), ..., (\alpha_{\st 1}\oplus ... \oplus \alpha_{\st n})\}$ be the result. For instance, when $n=3$ and $e=2$,  we would have $W=\{(\alpha_{\st 1}\oplus \alpha_{\st 2}),  (\alpha_{\st 2}\oplus  \alpha_{\st 3}), (\alpha_{\st 1}\oplus \alpha_{\st 3}), (\alpha_{\st 1}\oplus \alpha_{\st 2} \oplus \alpha_{\st n})\}$. After that, it generates an empty Bloom filter and  inserts all elements of $W$ into this Bloom filter. Let $\mathtt{BF}$ be the Bloom filter that encodes $W$'s elements. It sends $\mathtt{BF}$ to $\mathcal{I}$. To decode and extract the final verdict, as in Variant 1, $\mathcal{I}$ does XOR all masked verdict representations which  removes the masking values and XORs are verdicts’ representations. If the result is $0$, then $\mathcal{I}$ concludes that all arbiters must have voted $0$ (with a high probability); so, it sets the final verdict to $0$. However, if the result is a non-zero value, then it checks whether the value is in the Bloom filter. If it is, then it concludes that at least threshold arbiters voted $1$, so it sets the final vector to $1$. Otherwise (if the value is not in the Bloom filter), it concludes that less than threshold arbiters voted $1$; therefore, it sets the final verdict to $0$.  Figures \ref{fig:GPVE} and \ref{fig:GFVD}, in Appendix \ref{sec::Generic-Verdict-Encoding-Decoding-Protocols}, present the  GPVE and GFVD protocols in detail. 



%\item[$\bullet$] for every integer $i$ in the range $[e,n]$, computes the combinations (without repetition) of $i$ elements from set $\{\alpha_{\st 1},..., \alpha_{\st n}\}$, where the combination operation is XOR. Let $W=\{(\alpha_{\st 1}\oplus \alpha_{\st 2}\oplus... \oplus \alpha_{\st e}), (\alpha_{\st 2}\oplus \alpha_{\st 3}\oplus ... \oplus \alpha_{\st e+1}), ..., (\alpha_{\st 1}\oplus \alpha_{\st 2}\oplus... \oplus \alpha_{\st n})\}$ be the result.  
%

%\item[$\bullet$] constructs an empty Bloom filter. Then, it inserts all elements of $W$ into this Bloom filter. Let $\mathtt{BF}$ be the Bloom filter encoding $W$'s elements. 
 
  
  
 
 
 
 



 %and then (ii) masking this parameter by the above pseudorandom value. It sends the result to I. 

%Now, we explain how the GPVE and GFVD work. 





%In Appendix \ref{sec::bloom-filter-}, we explain how the Bloom filter parameters can be set.  




%
%Here, we present two efficient verdict encoding and decoding protocols; namely, Private Verdict Encoding (PVE) and Final Verdict Decoding (FVD) protocols. Their goal is to let a third party $\mathcal{I}$, e.g., $\mathcal{DR}$, find out whether at least one arbiter voted $1$, while satisfying the following  requirements.  The protocols should (1) generate unlinkable verdicts, (2)  not require arbiters to interact with each other for each customer, and (3) be  efficient. Since, the second and third requirements are self-explanatory,  we only explain the first one.  Informally, the first property requires  that the protocols should generate encoded verdicts and final verdict in a way that $\mathcal{I}$,  given the encoded verdicts and final verdict, should not be able to (a)   link a  verdict to an arbiter (except when all arbiters' verdicts are $0$), and (b) find out the total number of $1$ or $0$ verdicts when they provide different verdicts. 
%
%
%
% At a high level, the protocols work as follows.  The arbiters only once for all customers agree on a secret key of a pseudorandom function. This key will allow each of them to generate a pseudorandom masking values such that if all masking values are ``XOR''ed, they would cancel out each other and result $0$.\footnote{This is similar to the idea used in the XOR-based secret sharing \cite{Schneier0078909}.}
% 
% 
% 
% 
% 
%Each arbiter represents its verdict by (i) representing it as a parameter which is set to either $0$ if the verdict is $0$ or to a random value if the verdict is $1$, and then (ii) masking this parameter by the above  pseudorandom value.  It sends the result to $\mathcal{I}$.  To decode the final verdict and find out whether any arbiter voted $1$, $\mathcal{I}$  does XOR all encoded verdicts. This removes the masks and XORs are verdicts' representations.  If the result is $0$, then    all arbiters must have voted $0$; therefore,  the final verdict is $0$. However, if the result is not $0$ (i.e., a random value), then at least one of the arbiters voted $1$, so  the final verdict is $1$. We present the encoding  and decoding protocols in figures \ref{fig:PVE} and \ref{fig:FVD} respectively.
% 
% 
% Not that the protocols' correctness holds, except  a negligible  probability. In particular, if two arbiters  represent their verdict by an identical random value, then when they are XORed they would cannel out each other which can affect the result's correctness. The same holds if the XOR of  multiple verdicts' representations results in a value that can cancel out another verdict's representation. Nevertheless, the probability that such an event occurs is negligible in the security parameter, i.e., at most   $\frac{1}{2^{\st \lambda}}$. It is evident that PVE and FVD protocols meet properties (2) and (3). The primary reason they also meet  property (1) is that each masked verdict reveals nothing about the verdict (and its representation) and  given the final verdict, $\mathcal{I}$ cannot distinguish between the case where there is exactly one arbiter that voted  $1$ and the case where multiple arbiters voted $1$, as in both cases $\mathcal{I}$   extracts only a single random value, which reveals nothing about the number of arbiters which voted $0$ or $1$. 
% 
%




\begin{figure}[!ht]
\setlength{\fboxsep}{0.7pt}
\begin{center}
\begin{boxedminipage}{12.3cm}
\small{
\underline{$\mathtt{PVE}(\bar{k}_{\st 0}, \text{ID},  w_{\st j}, o, n,  j)\rightarrow  \bar{  w}_{\st j}$}\\
%
\begin{itemize}
\item \noindent\textit{Input.} $\bar{k}_{\st 0}$: a key of  pseudorandom function $\mathtt{PRF}(.)$, $\text{ID}$: a unique identifier, $ w_{\st j}$: a  verdict, $o$: an offset, $n$: the total number of  arbiters,  and  $j$: an arbiter's index.
%
\item \noindent\textit{Output.} $\bar{  w}_{\st j}$:  an  encoded verdict.  
%
\end{itemize}
Arbiter $\mathcal{D}_{\st j}$ takes the following steps.
\begin{enumerate}
%
\item\label{ZSPA:val-gen} computes a  pseudorandom  value,  as follows. 
%
\begin{itemize}
%
\item[$\bullet$]$ \text{ if } j< n: r_{\st j}=\mathtt{PRF}(\bar k_{\st 0}, o||j||\text{ID})$.\\
%
\item [$\bullet$] $ \text{ if } j=n: r_{\st j}= \bigoplus\limits^{\st n-1}_{\st i=1} r_{\st i}$.
%
\end{itemize}
%
\item  sets a fresh parameter, $w'_{\st j}$, as below. 
%
\begin{equation*}
   w'_{\st j}= 
\begin{cases}
   0,              & \text{if } w_{\st j}=0\\
   \alpha_{\st j}\stackrel{\st\$}\leftarrow \mathbb{F}_{\st p} ,& \text{if } w_{\st j}=1\\

    %0,              & \text{if } w_{\st j}=0
\end{cases}
\end{equation*}
%
\item encodes  $w'_{\st j}$ as follows. %$\forall i,1\leq i\leq s:$
%
$\bar w_{\st j}= w'_{\st j}\oplus r_{\st j}$.
%
\item outputs $\bar{ w}_{\st j}$.


\
 \end{enumerate}
 
}
\end{boxedminipage}
\end{center}
\caption{Private Verdict Encoding  (PVE) Protocol} 
\label{fig:PVE}
\end{figure}
%
\begin{figure}[!ht]
\setlength{\fboxsep}{0.7pt}
\begin{center}
\begin{boxedminipage}{12.3cm}
\small{
\underline{$\mathtt{FVD}(n,  \bar{\bm w})\rightarrow  v$}\\
%
\begin{itemize}
\item \noindent\textit{Input.} $n$:  the total number of  arbiters,  and  $\bar{\bm w}=[\bar{ w}_{\st 1},..., \bar{ w}_{\st n}]$:  a vector of all arbiters' encodes  verdicts.
%
\item \noindent\textit{Output.} $v$: final verdict.  
%
\end{itemize}
A third-party $\mathcal{I}$ takes the following steps.
\begin{enumerate}
%
\item combines  all arbiters' encoded verdicts, $\bar w_{\st j}\in \bar{\bm {w}}$, as follows. 
%
$c= \bigoplus\limits^{\st n}_{\st j=1} \bar w_{\st j}$
%
\item sets the final verdict $v$ depending on the content of $c$. Specifically, 
%
\begin{equation*}
   v= 
\begin{cases}
    0,              &\text{if } c= 0\\
   1 ,& \text{otherwise }\\

\end{cases}
\end{equation*}
%
\item outputs  $v$. 

\
 \end{enumerate}
 
}
\end{boxedminipage}
\end{center}
\caption{Final Verdict Decoding  (FVD) Protocol} 
\label{fig:FVD}
\end{figure}


%% !TEX root =main.tex

\subsubsection{Verdict Encoding-Decoding Protocols.}


In this section, we present efficient (verdict) encoding and decoding protocols. The encoding protocol  lets each of the $n$ honest arbiters $\mathcal{D}:\{\mathcal{D}_{\st 1},..., \mathcal{D}_{\st n}\}$ non-interactively encode its verdict such that a third party party $\mathcal{I}$ (where $\mathcal{I}\notin \mathcal{D}$),  given individual encoded verdict,  cannot learn anything about each arbiter's verdict.  The decoding protocol lets  $\mathcal{I}$ combine the encoded verdicts and learn either an specific final verdict  (i.e., $v=0$ or $v=1$) if  all arbiters' verdicts are identical, or  nothing  about the arbiters' inputs if they did not agree on the same specific verdict. The protocols are  primarily  based on ``zero-sum pseudorandom polynomials'' and the techniques often used by private set intersection (PSI) protocols. In particular, the arbiters only once for all customers agree on a secret key of a pseudorandom function. This key will allow each of them to generate a pseudorandom masking polynomial such that if all masking polynomials are summed up, they would cancel out each other and result $0$, i.e., zero-sum pseudorandom polynomials. 



%In this section, we present efficient (verdict) encoding and decoding protocols. The encoding protocol  lets each of the $n$ honest arbiters $\mathcal{D}:\{\mathcal{D}_{\st 1},..., \mathcal{D}_{\st n}\}$ non-interactively encode its verdict such that a third party party $\mathcal{I}$ (where $\mathcal{I}\notin \mathcal{D}$) can extract final verdict if  all arbiters' verdicts are identical with the following security requirements. First,  given individual encoded verdict, $\mathcal{I}$ cannot learn anything about each arbiter's verdict. Second, can find out only final verdict if  all arbiters' verdicts are identical; otherwise, it cannot learn each individual arbiter's verdict.  The decoding protocol lets  $\mathcal{I}$,  combines the encoded verdicts and learn either an specific final verdict  (i.e., $w=0$ or $w=1$) if  all arbiters' verdicts are identical, or  nothing  about the arbiters' inputs if they did not agree on the same specific verdict. The protocols are  primarily  based on ``zero-sum pseudorandom polynomials'' and the techniques often used by private set intersection (PSI) protocols. In particular, the arbiters only once for all customers agree on a secret key of a pseudorandom function. This key will allow each of them to generate a pseudorandom masking polynomial such that if all masking polynomials are summed up, they would cancel out each other and result $0$, i.e., zero-sum pseudorandom polynomials. 

At a high level, the protocols work as follows. To encode a verdict $w$, each arbiter represents it as a polynomial. It randomises this polynomial and then  masks this polynomial with the pseudorandom masking polynomial. It sends the result to $\mathcal{I}$. To decode the final verdict and find out whether all arbiters agreed on the same verdict, i.e., unanimous decision,  $\mathcal{I}$  adds all polynomials up. This removes the masks. Next, it  evaluates the result polynomial at $v=1$ and $v=0$. It considers $v$ as the final verdict if the evaluation is  $0$. We present the encoding  and decoding protocols in figures \ref{fig:PVE} and \ref{fig:FVD} respectively. 









%
%Below, we present ``Zero-sum Pseudorandom Values Generator'' (ZPVG), an algorithm that allows each of the $n$ arbiters  to \emph{efficiently} and \emph{independently}  generate a vector of $m$ pseudorandom values for each customer, such that when all arbiters' vectors are summed up component-wise, it would result in a vector of $m$ zeros. ZPVG is based on  the following idea. Each arbiter $\mathcal{D}_{\st j}\in\{\mathcal{D}_{\st 1},..., \mathcal{D}_{\st n-1}\}$  uses the secret key $\bar k_{\st 0}$ (as defined in Section \ref{Notations-and-Assumptions}), along with the customer's unique ID (e.g., it blockchain account's address) as the inputs of $\mathtt{PRF}(.)$  to derive $m$ pseudorandom values. However,  $\mathcal{D}_{\st n}$ generates each  $j$-th element of  its vector by computing the additive inverse of the sum of the $j$-th elements that the rest of arbiters generated. Even $\mathcal{D}_{\st n}$ does not need to interact with other arbiters, as it can regenerate their values too. Moreover, the arbiters do not need to interact with each other for every   new customer and can   reuse the same key because the new customer would have a new unique ID that would result in a fresh set of pseudorandom values (with a high probability). Figure \ref{fig:ZSPA} presents ZPVG in more detail. 





%It sends the result to all parties which can locally check if the relation holds. Note, in the literature,   there exist protocols that allows parties to agree on zero-sum pseudorandom values, e.g., in \cite{}; however, they are less efficient than $\mathtt{ZSPA}$, as their security requirements are different than ours, i.e., they assume the participants are malicious whereas we assume the arbiters who participate in this protocol  are honest. 


%Next, it commits to each value, where it uses $k_{\st 2}$ to generate the randomness of each commitment. Then, it constructs a Merkel tree on top of the commitments and  stores only the root of the tree  and the hash value of the keys (so in total three values) in  the smart contract.  Then, each party (using the keys) locally checks if the values and commitments have been constructed correctly; if so, each sends  an ``approved" message to the contract. 
%
%
%
%Informally, there are three main security requirements that $\mathtt{ZSPA}$ must meet: (a) privacy, (b)  non-refutability, and (c) indistinguishability. Privacy here means given the state of the  contract, an external party cannot learn any information about any of the (pseudorandom) values:  $z_{\st j}$; while non-refutability  means that if a party sends ``approved" then in future cannot deny the knowledge  of the values whose representation is stored in the contract. Furthermore, indistinguishability means that every $z_{\st j}$ ($1\leq j \leq m$) should be indistinguishable from a truly random value. In Fig. \ref{fig:ZSPA}, we provide $\mathtt{ZSPA}$ that efficiently generates $b$ vectors  where each vector elements is sum to zero. 



%\begin{figure}[ht]
%\setlength{\fboxsep}{0.7pt}
%\begin{center}
%\begin{boxedminipage}{12.3cm}
%\small{
%$\mathtt{ZPVG}(\bar{k}_{\st 0}, \text{ID}, n,  m, j)\rightarrow \bm r_{\st j}$\\
%------------------
%\begin{itemize}
%\item \noindent\textit{Input.} $\bar{k}_{\st 0}$: a key of  pseudorandom function's key $\mathtt{PRF}(.)$, $\text{ID}$: a unique identifier, $n$:  total number of rows of a matrix, and $m$: total number of columns of a matrix, $j$: a row's index in a matrix.
%%
%\item \noindent\textit{Output.} $\bm r_{\st j}$:  $j$-th row of an $n\times m$ matrix, such that  if $i$-th element of $\bm r_{\st j}$ is added with  the rest of elements in $i$-th column of the same matrix, the result would be  $0$. 
%\end{itemize}
%\begin{enumerate}
%%
%\item\label{ZSPA:val-gen} compute $m$ pseudorandom values as follows. 
%
%$\forall i, 1\leq i\leq m:$
%%
%\begin{itemize}
%%
%\item[$\bullet$] if $j< n: r_{\st i,j}=\mathtt{PRF}(\bar k_{\st 0}, i||j||\text{ID})$ 
%%
%\item[$\bullet$]  if $j=n: r_{\st i, n}=\big(-\sum\limits^{\st n-1}_{\st j=1} r_{\st i,j}\big) \bmod p$ 
%%
%\end{itemize}
%%
%\item return $\bm r_{\st j}=[r_{\st 1,j},..., r_{\st m,j}]$
%
%
%
%
%\
% \end{enumerate}
% 
%}
%\end{boxedminipage}
%\end{center}
%\caption{Zero-sum Pseudorandom Values Generator (ZPVG)} 
%\label{fig:ZSPA}
%\end{figure}






\begin{figure}[ht!]
\setlength{\fboxsep}{0.7pt}
\begin{center}
\begin{boxedminipage}{12.3cm}
\small{
$\mathtt{PVE}(\bar{k}_{\st 0}, \text{ID}, w, n, \bm x, \textit{offset}, j)\rightarrow  \bar{\bm w}_{\st j}$\\
------------------
\begin{itemize}
\item \noindent\textit{Input.} $\bar{k}_{\st 0}$: a key of  pseudorandom function $\mathtt{PRF}(.)$, $\text{ID}$: a unique identifier, $w$: binary verdict, $n$: the total number of  arbiters,  $\bm x=[x_{\st 1},...,x_{\st 3}]$:  non-zero distinct public $x$-coordinates, $\textit{offset}$: an offset to avoid generating the same value when this algorithm is called  multiple times by the same party for the same $\text{ID}$,  $j$: an arbiter's index.
%
\item \noindent\textit{Output.} $\bar{\bm w}_{\st j}$: $y$-coordinates of $j$-th arbiter's masked polynomial that encodes $w$.  
%
%$\bm r_{\st j}$:  $j$-th row of an $n\times m$ matrix, such that  if $i$-th element of $\bm r_{\st j}$ is added with  the rest of elements in $i$-th column of the same matrix, the result would be  $0$. 
\end{itemize}
Arbiter $\mathcal{D}_{\st j}$ takes the following steps.
\begin{enumerate}
%
\item\label{ZSPA:val-gen} computes three pseudorandom coefficients of a degree-$2$ polynomial, $\Psi_{\st j}$, as follows. 

$\forall i, \textit{offset}\leq i\leq \textit{offset}+2:$
%
%\begin{itemize}
%
\begin{center}
$\bullet \text{ if } j< n: r_{\st i,j}=\mathtt{PRF}(\bar k_{\st 0}, i||j||\text{ID})$\\
%
\hspace{3.5mm} $\bullet  \text{ if } j=n: r_{\st i, n}=\big(-\sum\limits^{\st n-1}_{\st j=1} r_{\st i,j}\big) \bmod p$
\end{center}
%\end{itemize}
By the end of this phase, a random polynomial of the following form is generated, $\Psi_{\st j}=r_{\st i+2,j}\cdot x^{\st 2}+r_{\st i+1,j}\cdot x+r_{\st i,j}$.
%
\item generates   a polynomial that encodes the verdict, i.e., $\Omega_{\st j}=(x-w)$.  

\item multiplies the polynomial by a fresh random polynomial $\Phi_{\st j}$ of degree $1$ and adds the result with $\Psi_{\st j}$, i.e.,  $\bar\Omega_{\st j}=\Phi_{\st j}\cdot \Omega_{\st j}+\Psi_{\st j}\bmod p$. 


\item  evaluates the result  polynomial, $\bar\Omega_{\st j}$, at every  element $x_{\scriptscriptstyle i}\in {\bm{x}}$. This yields a vector of   $y$-coordinates: $[ \bar w_{\st 1,j},..., \bar w_{\st 3,j}]$.
%

%
\item return $\bar{\bm w}_{\st j}=[ \bar w_{\st 1,j},..., \bar w_{\st 3,j}]$.




\
 \end{enumerate}
 
}
\end{boxedminipage}
\end{center}
\caption{Private Verdict Encoding  (PVE) Protocol} 
\label{fig:PVE}
\end{figure}
%%%%%%%%%%%%%%%%%%%%%%%%%%%%%%%%%%%%%%

\begin{figure}[ht!]
\setlength{\fboxsep}{0.7pt}
\begin{center}
\begin{boxedminipage}{12.3cm}
\small{
$\mathtt{FVD}(n,   \bm x, \bar{\bm w})\rightarrow  v$\\
------------------
\begin{itemize}
\item \noindent\textit{Input.} $n$:  the total number of  arbiters,  $\bm x=[x_{\st 1},...,x_{\st 3}]$: the $x$-coordinates, $\bar{\bm w}=[\bar{\bm w}_{\st 1},..., \bar{\bm w}_{\st n}]$:  $y$-coordinates of all arbiters' masked polynomial that encodes their verdicts.
%
\item \noindent\textit{Output.} $v$: final verdict.  
%
%$\bm r_{\st j}$:  $j$-th row of an $n\times m$ matrix, such that  if $i$-th element of $\bm r_{\st j}$ is added with  the rest of elements in $i$-th column of the same matrix, the result would be  $0$. 
\end{itemize}
A third-party $\mathcal{I}$ takes the following steps.
\begin{enumerate}
%
\item sums all vectors of arbiters' masked polynomials', component-wise, as follows. 
%
 $$\forall i, 1\leq i\leq 3: g_{\st i}=\sum\limits^{\st n}_{\st j=1} \bar{w}_{\st i,j} \bmod p$$
%
\item uses pairs $(x_{\st i}, g_{\st i})$ to interpolate a polynomial, $\Theta$; e.g., using Lagrange interpolation. Note,  this polynomial has the  form: $\Theta=\sum\limits^{\st n}_{\st j=1}\Phi_{\st j}\cdot \Omega_{\st j}$, where $\Omega_{\st j}$'s root is $\mathcal{D}_{\st j}$'s verdict. 
%
\item evaluates $\Theta$ at $v=0$ and $v=1$. It returns $v$, if the result of the evaluation is $0$. 

\
 \end{enumerate}
 
}
\end{boxedminipage}
\end{center}
\caption{Final Verdict Decoding  (FVD) Protocol} 
\label{fig:FVD}
\end{figure}




%%\small{
%\begin{center}
%\fbox{\begin{minipage}{5 in}
%$\mathtt{checkMissedWar}(add_{\st\mathcal{S}},  m, T, l, \text{aux})\rightarrow g\in\{0,1\}$\\
%------------------
%\
%
%\noindent\textbf{Input}. $add_{\st\mathcal{S}}$: the address of smart contract $\mathcal{S}$, $m$:  $\mathcal{S}$'s field that corresponds to $\mathcal{B}$'s message, $T$: $\mathcal{S}$'s field that specifies the time when $\mathcal{B}$ must register the message, $l$: $\mathcal{S}$'s field related to customer's payees' list, and \text{aux}: auxiliary information. 
%
%\noindent\textbf{Output}. $g=1$: if the missing warning could play a role in preventing the scam; $g=0$: otherwise. 
%\begin{enumerate}
%%
%\item reads the bank's signed message $m$ sent to $\mathcal{S}$. If $m$ was not set (to anything) at time $T$, then it proceeds to the next step. Otherwise, it aborts. 
%%
%\item checks the validity of  customer's most recent payees' list $l$ on  $\mathcal{S}$, with the help of the auxiliary information $\text{aux}$. 
%\begin{itemize}
%\item[$\bullet$]  if $l$ contains an invalid element,  it sets $g=1$.
%\item [$\bullet$] otherwise, it sets $g=0$.
%\end{itemize}
%\item returns $g$.
%\end{enumerate}
%\end{minipage}
%}
%\end{center}
%%}
%




\subsection{The PwDR Protocol}
In this section, we present the PwDR protocol in detail. 




%At a high level, the protocol works as follows. 

\begin{enumerate}

\item \underline{\textit{Generating  Certificate Parameters}}.  $\mathtt{keyGen}(1^{\st \lambda})\rightarrow (sk,pk)$

The certificate generator  takes   step \ref{signature-keygen} and an arbiter takes step \ref{encryption-keygen} below. 

\begin{enumerate}
\item\label{signature-keygen} calls $\mathtt{Sig.keyGen}(1^{\st \lambda})\rightarrow (sk_{\st\mathcal{G}}, pk_{\st\mathcal{G}})$ to generate  signing secret key $sk_{\st\mathcal{G}}$ and  verfiying public key $pk_{\st\mathcal{G}}$. It publishes the public key, $pk_{\st\mathcal{G}}$.
%
\item\label{encryption-keygen} calls $\tilde{\mathtt{keyGen}}(1^{\st\lambda})\rightarrow({sk}_{\st\mathcal {D}}, {pk}_{\st\mathcal {D}})$ to generate  decrypting secret key ${sk}_{\st\mathcal {D}}$ and encrypting public key ${pk}_{\st\mathcal {D}}$. It publishes the public key ${pk}_{\st\mathcal {D}}$ and sends ${sk}_{\st\mathcal {D}}$ to the rest of arbiters.
\end{enumerate}

%(1^{\st \lambda})\rightarrow k:=(sk_{\st\mathcal{D}}, pk_{\st\mathcal{D}})%

Let $sk:=(sk_{\st\mathcal{G}}, sk_{\st\mathcal{D}})$ and $pk:=(pk_{\st\mathcal{G}}, pk_{\st\mathcal{D}})$. Note, this   phase takes place only once for all customers.
\vspace{2mm}
\item\label{RCPoRP::Bank-side-Initiation} \underline{\textit{Bank-side Initiation}}. $\mathtt{bankInit}(1^{\st \lambda})\rightarrow (T, pp, \bm l)$

$\mathcal{B}$ takes the following steps. 
\begin{enumerate}

\item\label{RCPoRP::setup} picks   secret keys $\bar k_{\st 1}$ and $\bar k_{\st 2}$ for  symmetric key encryption scheme and  pseudorandom function $\mathtt{PRF}$ respectively. It  sets two private statements as $\pi_{\st 1}=\bar k_{\st 1}$ and $\pi_{\st 2}= \bar k_{\st 2}$.
%
\item\label{RCPoRP::set-qp}  calls $\mathtt{SAP.init}(1^{\st\lambda}, adr_{\st\mathcal{B}}, adr_{\st\mathcal{C}}, \pi_{\st i})\rightarrow(r_{
\st i}, g_{\st i}, adr_{\st\text{SAP}})$ to initiate  agreements on  statements $\pi_{\st i}\in \{\pi_{\st 1}, \pi_{\st 2}\}$  with customer $\mathcal{C}$.  Let $T_{\st i}:=(\ddot{\pi}_{\st i}, g_{\st i})$ and $T:=(T_{\st 1}, T_{\st 2})$,  where  $\ddot{\pi}_{\st i}:=(\pi_{\st i}, r_{\st i})$ is the opening of $g_{\st i}$.  It also sets parameter $\Delta$ as a time window between two specific time points, i.e., $\Delta=t_{\st i} - t_{\st i-1}$. Briefly, it is used to impose an upper bound on a message delay.  %It  constructs a vector, $\bm{x}$, of three non-zero distinct $x$-coordinates, i.e., $\bm{x}=[x_{\st 1},..., x_{\st 3}]$, where $x_{\st i}\in \mathbb{F}_{\st p}$.
%
\item sends $\ddot{\pi}:=(\ddot{\pi}_{\st 1}, \ddot{\pi}_{\st 2})$ to   $\mathcal{C}$ and   sends  public parameter $pp:=(adr_{\st\text{SAP}},\Delta)$ to  smart contract $\mathcal{S}$.
%
\end{enumerate}


\vspace{2mm}
\item \underline{\textit{Customer-side Initiation}}\label{RCSP::Server-side-Initiation}. $\mathtt{customerInit} (1^{\st \lambda}, T, pp)\rightarrow a$

$\mathcal{C}$ takes the following steps. 

\begin{enumerate}
%
\item calls   $\mathtt{SAP.agree}(\pi_{\st i}, r_{\st i}, g_{\st i}, adr_{\st\mathcal{B}}, adr_{\st\text{SAP}})\rightarrow (g'_{\st i}, b_{\st i})$, to check the correctness of parameters in $T_{\st i}\in T$ and (if accepted) to agree on these parameters, where $(\pi_{\st i}, r_{\st i}) \in \ddot{\pi}_{\st i}\in T_{\st i}$ and $1\leq i \leq 2$. Note,  if both $\mathcal{B}$ and $\mathcal{C}$ are honest, then $g_{\st i}=g'_{\st i}$. It also checks $\Delta$ in  $\mathcal{S}$, e.g., to see whether it is sufficiently large.
%
\item if the above checks fail,  it sets $a=0$. Otherwise, it sets $a=1$. It sends $a$ to $\mathcal{S}$. 
\end{enumerate}
%
\vspace{2mm}
\item \underline{\textit{Generating  Update Request}}. $\mathtt{genUpdateRequest}(T, f, {\bm l})\rightarrow \hat m^{\st\mathcal{(C)}}_{\st 1}$

$\mathcal{C}$ takes the following steps. 

\begin{enumerate}
%
\item sets the inputs of algorithm $\mathcal{I}\in \{\mathtt{setupNewPayee}(.), $ $ \mathtt{ammendExistingPayee}(.)\}$ as below. 
%
\begin{itemize}

\item[$\bullet$] if $\mathtt{setupNewPayee}(.)$ is called, then it sets $m_{\st 1}^{\st(\mathcal C)}:=(\phi,f)$, where $f$ is new payee's detail.  %In this case, $\mathcal{S}$ appends $p$ to the payee list, $l$. 
%
\item[$\bullet$] if $\mathtt{ammendExistingPayee}(.)$  is called, then it sets $m_{\st 1}^{\st(\mathcal C)}:=(i, f)$, where $i$ is the index of the element in $\bm l$ that should be changed to $f$.  %In this case, $\mathcal{S}$ replaces $i$-th emement of $l$ with $p$, i.e., $l[i]=p$.

\end{itemize}
\item at time $t_{\st 0}$, sends to $\mathcal{S}$  the encryption of $m_{\st 1}^{\st(\mathcal C)}$, i.e., $\hat m_{\st 1}^{\st(\mathcal C)}=\mathtt{Enc}(\bar k_{\st 1}, m_{\st 1}^{\st(\mathcal C)})$. 
\end{enumerate}
%
% Inserting New Payee
\vspace{2mm}
 \item \underline{\textit{Inserting New Payee}}. $\mathtt{insertNewPayee}(\hat m^{\st\mathcal{(C)}}_{\st 1}, {\bm l})\rightarrow {\hat{\bm l}}$
 
$\mathcal{S}$ takes the following steps. 
  %
 \begin{itemize}
 %
 \item[$\bullet$] if $\hat m_{\st 1}^{\st(\mathcal C)}$  is not empty, it appends $\hat m_{\st 1}^{\st(\mathcal C)}$ to the payee list $\hat{\bm l}$, resulting an updated list, $\hat{\bm l}$. 
 %
  \item[$\bullet$] if $\hat m_{\st 1}^{\st(\mathcal C)}$ is empty,  it does nothing. 
  %
 \end{itemize}
 
 


 % Generating Warning
 \vspace{2mm}
\item \underline{\textit{Generating Warning}}. $\mathtt{genWarning}(T, {\hat{\bm l}}, aux)\rightarrow \hat m^{\st\mathcal{(B)}}_{\st1}$
%

$\mathcal{B}$ takes the following steps. 
\begin{enumerate}
%
\item  checks if the most recent list $\hat{\bm{l}}$ is not empty. If it is empty, it halts. Otherwise, it proceeds to the next step. 
%
\item  decrypts each element of $\hat{\bm l}$ and checks their correctness, e.g., checks whether each element meets its internal policy or CoP requirements stated in $aux$. If the check passes, it sets $m_{\st 1}^{\st(\mathcal{B})}= \text{``pass''}$. Otherwise, it sets $m_{\st 1}^{\st(\mathcal{B})}=\text{warning}$, where $\text{warning}$ is a  string that contains a warning's detail concatenated with the string $\text{``warning''}$.
%
\item at time $t_{\st 1}$, sends to $\mathcal{S}$ the encryption of $m_{\st 1}^{\st(\mathcal{B})}$, i.e.,  $\hat m_{\st 1}^{\st(\mathcal{B})}= \mathtt{Enc}(\bar k_{\st 1}, m_{\st 1}^{\st(\mathcal B)})$. 
\end{enumerate}


\vspace{2mm}
\item\label{clinet-at-T2} \underline{\textit{Generating Payment Request}}. $\mathtt{genPaymentRequest}(T, in_{\st f}, \hat{\bm{l}}, \hat m^{\st\mathcal{(B)}}_{\st1})\rightarrow \hat m^{\st\mathcal{(C)}}_{\st2}$

$\mathcal{C}$ takes the following steps. 
\begin{enumerate}
%
\item at time $t_{\st 2}$, decrypts the content of $\hat{\bm{l}}$ and $\hat m_{\st 1}^{\st(\mathcal{B})}$.  It sets a payment request $m_{\st 2}^{\st(\mathcal C)}$ to  $\phi$ or  $in_{\st f}$, where $in_{\st f}$ contains the payment's detail, e.g., the payee's detail in $\bm l$ and the amount it wants to transfer. 
%
\item at time $t_{\st 3}$, sends  to $\mathcal{S}$ the encryption of $m_{\st 2}^{\st(\mathcal C)}$, i.e., $\hat m_{\st 2}^{\st(\mathcal{C})}= \mathtt{Enc}(\bar k_{\st 1}, m_{\st 2}^{\st(\mathcal C)})$.
\end{enumerate}
%
\vspace{2mm}
\item \underline{\textit{Making Payment}}. $\mathtt{makePayment}(T, \hat m^{\st\mathcal{(C)}}_{\st2})\rightarrow \hat m^{\st\mathcal{(B)}}_{\st2}$

$\mathcal{B}$ takes the following steps. 
\begin{enumerate}
%
 \item at time $t_{\st 4}$, decrypts  the content of $\hat m_{\st 2}^{\st(\mathcal C)}$, i.e.,  $ m_{\st 2}^{\st(\mathcal{C})}= \mathtt{Dec}(\bar k_{\st 1}, \hat m_{\st 2}^{\st(\mathcal C)})$.
 %
 \item at time $t_{\st 5}$, checks the content of $m_{\st 2}^{\st(\mathcal C)}$. If $m_{\st 2}^{\st(\mathcal C)}$ is non-empty, i.e., $m_{\st 2}^{\st(\mathcal C)}=in_{\st f}$, it checks if the payee's detail in $in_{\st f}$ has already been checked and the payment's amount does not exceed the customer's credit. If the checks pass, it  runs the off-chain payment algorithm, $\mathtt{pay}(in_{\st f})$.  In this case, it sets $m_{\st 2}^{\st(\mathcal B)}$=``paid''. Otherwise (i.e., if $m_{\st 2}^{\st(\mathcal C)}=\phi$ or neither checks pass), it sets $m_{\st 2}^{\st(\mathcal B)}=\phi$. It sends  to $\mathcal{S}$ the encryption of  $m_{\st 2}^{\st(\mathcal B)}$, i.e., $\hat m_{\st 2}^{\st(\mathcal{B})}= \mathtt{Enc}(\bar k_{\st 1}, m_{\st 2}^{\st(\mathcal B)})$.  %Otherwise (i.e., if $m_{\st 2}^{\st(\mathcal C)}=\phi$ or neither checks pass),  it halts.
 %
\end{enumerate}
%
\vspace{2mm}
\item  \underline{\textit{Generating Complaint}}. $\mathtt{genComplaint}(\hat m^{\st\mathcal{(B)}}_{\st 1}, \hat m^{\st\mathcal{(B)}}_{\st2}, T, pk, {aux}_{\st f})\rightarrow  (\hat z, \hat{\ddot \pi})$

$\mathcal{C}$ takes the following steps. 
\begin{enumerate}
%
\item\label{DR::C-sends-complaint} decrypts $ \hat m^{\st\mathcal{(B)}}_{\st 1}$ and $\hat m^{\st\mathcal{(B)}}_{\st2}$; this results in $  m^{\st\mathcal{(B)}}_{\st 1}$ and $ m^{\st\mathcal{(B)}}_{\st2}$ respectively. Depending on the content of  the decrypted values, it sets its complaint's parameters $z:=(z_{\st 1}, z_{\st 2}, z_{\st 3})$ as follows.  %sends a complaint: $z:=(k, x, y)$ to $\mathcal{S}$ at time $T_{\st 6}$, where $k, x$ and $y$ are   set as follows.  

\begin{itemize}
%
%\item  [$\bullet$] if $\mathcal{C}$ want to complain that $\mathcal{B}$ has not provided any message (i.e., neither pass nor warning) and 
%
 \item  [$\bullet$] if $\mathcal{C}$ want to make one of the two below statements, it sets  $z_{\st 1}=$``challenge message".
 
 \begin{enumerate}[label=(\roman*)]
  \item the ``pass'' message (in $m^{\st\mathcal{(B)}}_{\st 1}$)  should have been a warning.
  
 \item $\mathcal{B}$ has not provided any message (i.e., neither pass nor warning)   and if $\mathcal{B}$  provided a warning  then  the fraud would have been  prevented.  
 

   \end{enumerate}
   \item [$\bullet$] if $\mathcal{C}$ wants to challenge the effectiveness of the warning (in $m^{\st\mathcal{(B)}}_{\st 1}$),  it sets $z_{\st 2}= m||sig||pk_{\st\mathcal{G}}||$ $\text{``challenge warning''}$, where  $m$ is an evidence,   $sig\in aux_{\st f}$ is the evidence's  certificate (obtained from the certificate generator $\mathcal{G}$), and $pk_{\st\mathcal{G}}\in pk$.  
 
  % $x =$ ``challenge warning''. 

 %
  \item [$\bullet$] if $\mathcal{C}$ wants to complain about the inconsistency of the payment (in $m^{\st\mathcal{(B)}}_{\st 2}$), then it sets  $z_{\st 3} =$ ``challenge payment''. Otherwise, it sets $
z_{\st 3}=\phi $.

    \end{itemize}
    \item at time $t_{\st 6}$, sends to $\mathcal{S}$ the following values: 
    
    \begin{itemize}
     \item[$\bullet$] the encryption of complaint $z$, i.e.,   $\hat z= \mathtt{Enc}(\bar k_{\st 1}, z)$.
     \item[$\bullet$] the encryption of $\ddot{\pi}:=(\ddot{\pi}_{\st 1}, \ddot{\pi}_{\st 2})$, i.e., $\hat{\ddot \pi}=\mathtt{\tilde{Enc}}({pk}_{\st\mathcal {D}}, \ddot \pi)$. Note, $\ddot \pi$ contains the openings of the private statements' commitments (i.e., $g_{\st 1}, g_{\st 2}$), and is encrypted under each $\mathcal{D}_{\st j}$'s public key. 
     \end{itemize}
\end{enumerate}


%\item\label{VerifyingComplaint}  \underline{\textit{Verifying Complaint}}. $\mathtt{verComplaint}(\hat z, \hat{\ddot{\pi}}, g, \hat m^{\st\mathcal{(C)}}_{\st1},\hat m^{\st\mathcal{(C)}}_{\st 2}, \hat m^{\st\mathcal{(B)}}_{\st1}, \hat m^{\st\mathcal{(B)}}_{\st 2}, \hat{\bm{l}},  j, {sk}_{\st\mathcal {D}}, aux, pp)\rightarrow \hat{\bm w}_{\st j}$

\vspace{2mm}
\item\label{VerifyingComplaint}  \underline{\textit{Verifying Complaint}}. $\mathtt{verComplaint}(\hat z, \hat{\ddot{\pi}}, g, \hat {\bm m}, \hat{\bm{l}},  j, {sk}_{\st\mathcal {D}}, aux, pp)\rightarrow \hat{\bm w}_{\st j}$

Every Arbiter, $\mathcal{D}_{\st j}\in\{\mathcal{D}_{\st 1}, ..., \mathcal{D}_{\st n}\}$, takes the following steps.
\begin{enumerate}

%
\item at time $t_{\st 7}$, decrypts $\hat{\ddot{\pi}}$, i.e., ${\ddot{\pi}}=\mathtt{\tilde{Dec}}({sk}_{\st\mathcal {D}}, \hat{\ddot \pi})$. 

\item checks the validity  of $(\ddot{\pi}_{\st 1}, \ddot{\pi}_{\st 2})$ in $\ddot{\pi}$ by locally running  the SAP's verification, i.e., $\mathtt{SAP.verify}(.)$, for each  $\ddot{\pi}_{\st i}$. It   returns  $s$. If $s=0$, it halts. If $s=1$ for both $\ddot{\pi}_{\st 1}$ and  $\ddot{\pi}_{\st 2}$, it proceeds to the next step. 
%
%\item decrypts $\hat m^{\st\mathcal{(C)}}_{\st1}$ and $\hat m^{\st\mathcal{(C)}}_{\st 2}$ and checks if they are not empty. Also, it checks if 
%%%%%%%%%%%%
\item decrypts $\hat{\bm m}=[\hat{m}_{\st 1}^{\st(\mathcal C)}$, $\hat{m}_{\st 2}^{\st(\mathcal C)}, \hat{m}_{\st 1}^{\st(\mathcal B)}, \hat{m}_{\st 2}^{\st(\mathcal B)}]$  using $\mathtt{Dec}(\bar k_{\st 1},.)$, where $\bar k_{\st 1}\in\ddot \pi_{\st 1}$. Let $[{m}_{\st 1}^{\st(\mathcal C)},  {m}_{\st 2}^{\st(\mathcal C)},  {m}_{\st 1}^{\st(\mathcal B)}, {m}_{\st 2}^{\st(\mathcal B)}]$ be the result. 

\item  checks whether $\mathcal C$ made an update request to its payee's list. To do so, it checks if  $m_{\st 1}^{\st(\mathcal C)}$  is non-empty and (its encryption) was registered by $\mathcal{C}$ in $\mathcal{S}$. Also, it checks whether $\mathcal C$ made a payment request, by checking if $m_{\st 2}^{\st(\mathcal C)}$ is non-empty and (its encryption) was registered by $\mathcal{C}$ in $\mathcal{S}$ at time $t_{\st 3}$.  If either check fails, it halts. 
%%%%%%%%%%%%
\item decrypts $\hat z$ and $\hat{\bm{l}}$ using $\mathtt{Dec}(\bar k_{\st 1},.)$, where $\bar k_{\st 1}\in\ddot \pi_{\st 1}$. Let $ z:=(z_{\st 1}, z_{\st 2}, z_{\st 3})$ and ${\bm{l}}$ be the result. 
%
\item\label{arbiters-verdict} sets its verdicts according to  the content of  $z:=(z_{\st 1}, z_{\st 2}, z_{\st 3})$, as follows.  
%
\begin{itemize}

%
\item[$\bullet$]  if  ``challenge message'' $\notin z_{\st 1}$, it sets $w_{\st 1,j}= 0$. Otherwise,  it runs $\mathtt{verStat}(add_{\st\mathcal{S}}, m_{\st 1}^{\st(\mathcal{B})},  \bm{l}, \Delta, {aux})\rightarrow w_{\st 1, j}$, to determine if a warning (in $m_{\st 1}^{\st(\mathcal{B})}$) should have been given (instead of the ``pass'' or no message). %It sends $d_{\st i}$ to $\mathcal{S}$, at time $T_{\st 8}$. %If  $k=\phi$, then it does nothing  in this step.
%
\item[$\bullet$]  if ``challenge warning'' $\notin z_{\st 2}$, it sets $w_{\st 2, j}= w_{\st 3, j}= 0$. Otherwise, it runs $\mathtt{checkWarning}(add_{\st\mathcal{S}}, z_{\st 2}, m_{\st 1}^{\st(\mathcal{B})},$ $  {aux}')\rightarrow (w_{\st 2, j}, w_{\st 3, j})$, to determine the effectiveness of the warning (in $m_{\st 1}^{\st(\mathcal{B})}$). %It sends $(v_{\st i}, \bar v_{\st i})$ to $\mathcal{S}$, at time $T_{\st 9}$.



\item[$\bullet$]  if ``challenge payment'' $\in z_{\st 3}$, it checks whether the  payment has been  made.   If the check passes, it sets  $w_{\st 4, j}=1$. If the check fails,   it sets $w_{\st 4, j}=0$.  If ``challenge payment'' $\notin z_{\st 3}$, it checks if  ``paid'' is in ${m}_{\st 2}^{\st(\mathcal C)}$. If the check passes, it sets $w_{\st 4, j}=1$. Otherwise, it sets $w_{\st 4, j}=0$. 
%
\end{itemize}
%
\item  encodes  its verdicts $(w_{\st 1, j}, w_{\st 2, j}, w_{\st 3, j},  w_{\st 4, j})$ as follows. 
%
\begin{enumerate}
%
\item locally maintains a counter, $o_{\st adr_{\st{c}}}$,  for each $\mathcal{C}$. It sets its initial value to $0$.
%

%\item calls $\mathtt{PVE}(.)$ to represent every verdict as a polynomial. In particular, it performs as follows. $\forall i, 1\leq i \leq 4:$

\item calls $\mathtt{PVE}(.)$ to encode each verdict. In particular, it performs as follows. $\forall i, 1\leq i \leq 4:$
\begin{enumerate}
%

\item[$\bullet$] calls $\mathtt{PVE}(\bar{k}_{\st 0}, adr_{\st\mathcal{C}},  w_{\st i, j}, o_{\st adr_{\st{c}}}, n,  j)\rightarrow  \bar{  w}_{\st i,j}$

%\item[$\bullet$] calls $\mathtt{PVE}(\bar{k}_{\st 0}, adr_{\st\mathcal{C}}, w_{\st i, j}, n, \bm x, \textit{offset}_{\st adr_{\st{c}}}, j)\rightarrow  \bar{\bm w}_{\st i,j}$.
%
\item[$\bullet$] $o_{\st adr_{\st{c}}}=o_{\st adr_{\st{c}}}+1$.
%
\end{enumerate}
By the end of this step, a vector ${\bar {\bm w}}_{\st j}$ of four encoded verdicts is computed, i.e., $\bar {\bm w}_{\st j}=[ \bar{  w}_{\st 1,j},..., \bar{  w}_{\st 4,j}]$.
%
\item uses $\bar k_{\st 2}\in \ddot \pi_{\st 2}$ to further encode/encrypt  $\mathtt{PVE}(.)$'s outputs as follows. %$\forall i, 1\leq i \leq 4:$
%
$ \hat {\bm w}_{\st j}= \mathtt{Enc}(\bar k_{\st 2}, \bar{\bm w}_{\st j})$.


%\
%
%
%\
%
%\item\label{call-ZPVG} generates $4$  pseudorandom values, by calling $\mathtt{ZPVG}(\bar{k}_{\st 0}, adr_{\st\mathcal{C}}, n,  4, j)\rightarrow \bm{r}_{\st j}$. Recall, the sum of each element $ r_{\st i,j}\in \bm r_{\st j}$ and the other arbiters' related values would be $0$.   
%%
%\item\label{derive-PR-from-k2} generates $4$  pseudorandom values using $\bar k_{\st 2}$ (where  $\bar k_{\st 2}\in  \ddot\pi$) as follows. 
%%
%$$\forall i, 1\leq i \leq 4: \alpha_{\st i} = \mathtt{PRF}(\bar k_{\st 2}, i)$$
%
% Note, all arbiters would generate the same set of $\alpha_{\st i}$. 
%
%\item  encodes  each $w_{\st i,j}$ by using the values generated in steps \ref{call-ZPVG} and \ref{derive-PR-from-k2}, as follows. 
%%
% $$\forall i, 1\leq i \leq 4: \hat w_{\st i, j} = \big(\alpha_{\st i}\cdot w_{\st i, j}+r_{\st i, j}\big)\bmod p$$
\end{enumerate}


%%. 
%\item[$\bullet$]  if  ``challenge message'' $\in k$, then it runs $\mathtt{verStat}(add_{\st\mathcal{S}}, m_{\st 1}^{\st(\mathcal{B})}, T, l', \text{aux})\rightarrow d_{\st i}$ to determine whether a warning should have been given (instead of the ``pass'' or no message).  It sends $d_{\st i}$ to $\mathcal{S}$, at time $T_{\st 8}$. %If  $k=\phi$, then it does nothing  in this step.
%%
%\item[$\bullet$]  if  ``challenge warning'' $\in x$, then it runs $\mathtt{checkWarEff}(add_{\st\mathcal{S}}, x, m_{\st 1}^{\st(\mathcal{B})},  \text{aux}')\rightarrow (v_{\st i}, \bar v_{\st i})$ to determine the warning's effectiveness.  It sends $(v_{\st i}, \bar v_{\st i})$ to $\mathcal{S}$, at time $T_{\st 9}$.
%
%
%
%
%\item[$\bullet$]  if $y =$ ``challenge payment'', in the case where the payment has been  made,  it sets its  verdict $w_{\st i}$ to $1$; otherwise, it sets $w_{\st i}$ to $0$. It sends $w_{\st i}$ to $\mathcal{S}$, at time $T_{\st 10}$.  %If  $y=\phi$, then it does nothing  in this step.


\item at time $t_{\st 8}$, sends to $\mathcal S$ the encrypted vector, $\hat {\bm w}_{\st j}$. %=[\hat { w}_{\st 1, j}, \hat {  w}_{\st 2, j}, \hat {  w}_{\st 3, j}, \hat {  w}_{\st 4, j}]$.
\end{enumerate}


\vspace{2mm}
\item\label{DR::DisputeResolution}  \underline{\textit{Resolving Dispute}}. $\mathtt{resDispute}(T_{\st 2}, \hat {\bm w}, pp)\rightarrow \bm v$



 $\mathcal{DR}$ takes the below steps at time $t_{\st 9}$, when it is invoked by $\mathcal{C}$ or  $\mathcal{S}$ which sends $\ddot\pi_{\st 2}\in T_{\st 2}$ to it.

\begin{enumerate}
%
\item checks the validity of $\ddot{\pi}_{\st 2}$ by locally running  the SAP's verification, i.e., $\mathtt{SAP.verify}(.)$, that  returns  $s$. If $s=0$, it halts. Otherwise, it proceeds to the next step. 
%
%\item decrypts $\hat{m}_{\st 1}^{\st(\mathcal C)}$, $\hat{m}_{\st 2}^{\st(\mathcal C)}$, and $\hat{m}_{\st 2}^{\st(\mathcal B)}$  using $\mathtt{Dec}(\bar k_1,.)$, where $\bar k_1\in\ddot \pi$. Let ${m}_{\st 1}^{\st(\mathcal C)}$,  ${m}_{\st 2}^{\st(\mathcal C)}$, and ${m}_{\st 2}^{\st(\mathcal B)}$ be the result respectively. 
%
%\item\label{DR::check-m1c}  checks whether $\mathcal C$ made an update request to its payee's list. To do so, it checks if  $m_{\st 1}^{\st(\mathcal C)}$  is non-empty and (its encryption) was registered by $\mathcal{C}$ in $\mathcal{S}$.
%
%\item\label{DR::check-payment-request}  checks whether $\mathcal C$ made a payment request, by checking if $m_{\st 2}^{\st(\mathcal C)}$ is non-empty and (its encryption) was registered by $\mathcal{C}$ in $\mathcal{S}$ at time $t_{\st 3}$.  

\item computes the final verdicts, as below. 
%
\begin{enumerate}
%
\item uses $\bar k_{\st 2}\in \ddot \pi_{\st 2}$ to decrypt the arbiters' encoded verdicts, as follows. $ \forall j, 1\leq j \leq n:$

$ \bar {\bm w}_{\st j}= \mathtt{Dec}(\bar k_{\st 2}, \hat{\bm w}_{\st j})$, where $\hat{\bm w}_{\st j}\in \hat{\bm w}$.
 % 
 \item constructs four vectors, $[\bm u_{\st 1},...,\bm u_{\st 4}]$, and sets  each vector $\bm u_{\st i}$ as follows. $\forall i, 1\leq i \leq 4:$
 
  $\bm u_{\st i}=[\bar{w}_{\st i,1},...,\bar{w}_{\st i,n}]$, where $\bar{w}_{\st i,j}\in \bar {\bm w}_{\st j}$. 
% $\forall i, 1\leq i \leq 4: \bar {\bm w}_{\st i,j}= \mathtt{Dec}(\bar k_{\st 2}, \hat{\bm w}_{\st i,j})$.
%
\item calls $\mathtt{FVD}(.)$ to extract each final verdict, as follows. $\forall i, 1\leq i \leq 4:$ calls $\mathtt{FVD}(n,  {\bm u}_{\st i})\rightarrow  v_{\st i}$. 


%$\mathtt{FVD}(n,   \bm x, \bar{\bm w})\rightarrow  w$


%regenerates $4$  pseudorandom values using $\bar k_{\st 2}$ (where  $\bar k_{\st 2}\in  \ddot\pi$) as follows. 
%
%$$\forall i, 1\leq i \leq 4: \alpha_{\st i} = \mathtt{PRF}(\bar k_{\st 2}, i)$$
%
%\item  extracts the decoded final verdict (for each $i$) as follows. 
%
% $$\forall i, 1\leq i \leq 4: v_{\st i} = \big((\alpha_{\st i})^{\st -1} \cdot \sum\limits_{\st j=1}^{\st n} \hat {w}_{\st i, j}\big)\bmod p= \sum\limits_{\st j=1}^{\st n} {w}_{\st i, j}$$
%
\end{enumerate}
%

\item outputs $\bm v=[v_{\st 1},...,v_{\st 4}]$.

\end{enumerate}
\end{enumerate}
Customer $\mathcal{C}$ must be reimbursed if the final verdict is that (i)  the ``pass'' message or  missing message should have been a warning or (ii)  the warning was ineffective and the provided evidence was not invalid, and (iii) the payment has been made. Stated formally, the following relation must hold: 
%
\begin{center}
$\Big( \underbrace{(v_{\st 1}=1)}_{\st (\text{i})}\vee  \underbrace{(v_{\st 2}=1\ \wedge\ v_{\st 3}=1)}_{\st (\text {ii})}\Big)\wedge\Big(\underbrace{v_{\st 4}=1}_{\st (\text{iii})}\Big)$.
\end{center}
%
 %where $v_{\st i}\in \bm v$.
%
%\begin{enumerate}
%\item\label{DR::check-warning-related-message}   one of the following  warning-related conditions holds.
%%
%\begin{enumerate}
%%
%\item the final verdict is that the ``pass'' message or a missing message should have been a warning. To do so, it checks  if $v_{\st 1}=0$.
%%
%\item the final verdict is that the warning was ineffective and the provided evidence was not invalid, by checking  if $v_{\st 2}=0$ and $v_{\st 3}=1$. 
%%
%\end{enumerate}
%%
%\item\label{DR::check-payment-related-conditions}  checks if the following  payment-related condition holds.
%%
%\begin{enumerate}
%%
%\item[$\bullet$] the final verdict is that the payment has been made, by checking if $v_{\st 4}=1$. 
%%
%\item  the encryption of  ``paid'' (in ${m}_{\st 2}^{\st(\mathcal C)}$) was registered by $\mathcal{S}$ at time $t_{\st 5}$.
%



%If  the above conditions  in steps  \ref{DR::check-warning-related-message} and \ref{DR::check-payment-related-conditions} hold,  it outputs $1$. Otherwise, it outputs $0$. 
%\end{enumerate}
%\end{enumerate}



Note that in the above PwDR protocol, even $\mathcal{C}$ and $\mathcal{B}$ that know the decryption secret keys, $(\bar k_{\st 1}, \bar k_{\st 2})$, cannot link a certain verdict to an arbiter, for two main reasons; namely,  (a) they do not know the masking random values used by arbiters to mask each verdict and (b) the final verdicts $(v_{\st 1},..., v_{\st 4})$ reveal nothing about the number of $1$ or $0$ verdicts, except when all arbiters vote $0$.  We highlight that we used PVE and FVD in the PwDR protocol only because they are highly efficient. However, it is easy to replace them with GPVE and GFVD, e.g., when the required threshold is greater than one. 



\begin{theorem}\label{theorem::PwDR-security}
The above PwDR scheme is secure, with regard to definition \ref{def::PwDR-security}, if the digital signature is existentially unforgeable under chosen message attacks,  the blockchain, SAP, and pseudorandom function $\mathtt{PRF}(.)$ are secure, and the encryption schemes are semantically secure. 
\end{theorem}


% !TEX root =main.tex


\section{Security Analysis of the PwDR Protocol}

To prove the main theorem (i.e., Theorem \ref{theorem::PwDR-security}), we show that the PwDR scheme satisfies all security properties defined in Section \ref{sec::def}. We first prove that it meets security against a malicious victim. 


\begin{lemma} If the digital signature is existentially unforgeable under chosen message attacks, and the SAP and  blockchain are secure, then the PwDR scheme is secure against a malicious victim, with regard to Definition \ref{def::Security-against-malicious-victim}. 
\end{lemma}

\begin{proof}
 First, we focus on event $\text{\MakeUppercase{\romannumeral 1}}:  \Big((m^{\st\mathcal{(B)}}_{\st1}=warning) \wedge (\sum\limits_{\st j=1}^{\st n}w_{\st 1,j}\geq {e})\Big)$ which considers the case where $\mathcal{B}$ has provided a warning message but $\mathcal{C}$ manages to  convince at least threshold arbiters to  set their verdicts to $1$,  that ultimately results in $\sum\limits_{\st j=1}^{\st n}w_{\st 1,j}\geq e$. We argue that the adversary's success probability in this event is negligible  in the security parameter. In particular, due to the security of SAP, $\mathcal{C}$ cannot convince an arbiter to accept a different decryption key, e.g., $k'\in \ddot\pi'$, that will be used to decrypt $\mathcal{B}$'s encrypted message $\hat {m}^{\st(\mathcal {B})}_{\st 1}$, other than what was agreed between $\mathcal{C}$ and $\mathcal{B}$ in the initiation phase, i.e., $\bar k_{\st 1}\in \ddot \pi_{\st 1}$.  To be more precise, it cannot persuade an arbiter to accept a statement $\ddot \pi'$, where $\ddot \pi'\neq \ddot \pi_{\st 1}$ except with a negligible probability, $\mu(\lambda)$. This ensures that  honest $\mathcal{B}$’s original message (and accordingly the warning) is accessed by every arbiter with a high probability. Next, we consider event  $\text{\MakeUppercase{\romannumeral 2}}:  \Big((\sum\limits_{\st j=1}^{\st n}w_{\st 1,j}<{e}) \wedge ( {v}_{\st 1}=1)\Big)$ that captures the case where only less than  threshold arbiters approved that the pass message was given incorrectly or the missing message could  prevent the APP fraud, but the final verdict that $\mathcal{DR}$ extracts implies that at least threshold arbiters approved that. We argue that the probability that this event occurs is negligible in the security parameter. Specifically, due to the security of the SAP,  $\mathcal{C}$ cannot persuade (a)  an arbiter to accept a different encryption key and (b) $\mathcal{DR}$ to accept a different decryption key other than what was agreed between $\mathcal{C}$ and $\mathcal{B}$ in the initiation phase. Specifically, it cannot persuade them to accept a statement $\ddot \pi'$, where $\ddot \pi'\neq \ddot \pi_{\st 2}$ except with a negligible probability, $\mu(\lambda)$. %Furthermore, as discussed in Section \ref{sec::PwDR-Subroutines}, due to the correctness of the verdict encoding-decoding protocols (i.e., PVE and FVD), the probability that multiple representations of verdict  $1$ cancel out each other is negligible too, $\frac{1}{2^{\st |p|}}$. Thus,  event $\text{\MakeUppercase{\romannumeral 2}}$ occurs only with a negligible probability. 
 
 
 Now, we move on to event   $\text{\MakeUppercase{\romannumeral 3}}: \Big((\mathtt{checkWarning}(m^{\st\mathcal{(B)}}_{\st1})= 1) \wedge (\sum\limits_{\st j=1}^{\st n}w_{\st 2,j}\geq {e})\Big)$. It captures the case where $\mathcal{B}$ has provided an effective warning message but $\mathcal{C}$ manages to make at threshold arbiters  set their verdicts to $1$, that ultimately results in $\sum\limits_{\st j=1}^{\st n}w_{\st 2,j}\geq e$. The same argument provided to event $\text{\MakeUppercase{\romannumeral 1}}$ is applicable to this even too. Briefly, due to the security of SAP, $\mathcal{C}$ cannot persuade an arbiter to accept a different decryption key other than what was agreed between $\mathcal{C}$ and $\mathcal{B}$ in the initiation phase. Therefore, all arbiters will receive the original message of $\mathcal{B}$, including the effective warning message, except a negligible probability, $\mu(\lambda)$. Now, we consider  event $\text{\MakeUppercase{\romannumeral 4}}:  \Big((\sum\limits_{\st j=1}^{\st n}w_{\st 2,j}< {e}) \wedge ({v}_{\st 2}=1)\Big)$, which captures the case where at least threshold arbiters  approved that the warning message was effective but   the final verdict that $\mathcal{DR}$ extracts implies that they approved the opposite. The security argument of  event \text{\MakeUppercase{\romannumeral 2}} applies to this event as well. In short, due to the security of the SAP, $\mathcal{C}$ cannot persuade  an arbiter to accept a different encryption key, and cannot convince $\mathcal{DR}$ to accept a different decryption key other than what was initially agreed between $\mathcal{C}$ and $\mathcal{B}$, except a negligible probability, $\mu(\lambda)$. 
 %
% Furthermore, due to the correctness of PVE and FVD protocols,  the probability that multiple representations of verdict $1$ cancel out each other is negligible, $\frac{1}{2^{\st |p|}}$. Thus, the probability that this event takes place is negligible.
 %
  Now, we analyse event  $\text{\MakeUppercase{\romannumeral 5}}: \Big(u\notin Q \wedge\mathtt{Sig.ver}( pk, u, sig) =1\Big)$. This even captures the case where the malicious victim comes up with a valid signature/certificate on a message that has never been queried to the signing oracle.  Nevertheless, due to the existential unforgeability of the digital signature scheme, the probability that such an event occurs is negligible, $\mu(\lambda)$. Next, we focus on event
 $\text{\MakeUppercase{\romannumeral 6}}: \Big((\sum\limits_{\st j=1}^{\st n}w_{\st 3,j}< {e}) \wedge ( {v}_{\st 3}=1)\Big)$ that considers the case where less than threshold arbiters indicated that the signature (in $\mathcal{C}$'s complaint) is valid, but   the final verdict that $\mathcal{DR}$ extracts implies that at least threshold  arbiters approved the signature. This means the adversary has managed to switch the verdicts of those arbiters which voted $0$ to $1$. However, the probability that this even occurs is negligible as well. Because, due to the SAP's security,  $\mathcal{C}$ cannot convince  an arbiter and $\mathcal{DR}$ to accept   different encryption and decryption keys other than what was initially agreed between $\mathcal{C}$ and $\mathcal{B}$, except a negligible probability, $\mu(\lambda)$.  Therefore, with a negligible probability the adversary can switch a verdict for $0$ to the verdict for $1$. 
 
 
 Moreover, a malicious $\mathcal{C}$ cannot frame an honest $\mathcal{B}$ for providing an invalid message by manipulating the smart contract’s content,  e.g., by replacing an effective warning with an ineffective one in $\mathcal{S}$, or excluding a warning from $\mathcal{S}$. In particular, to do that, it has to either  forge the honest party’s signature, so it can send an invalid message on its behalf, or fork the blockchain so the chain comprising a valid message is discarded. In the former case, the adversary’s probability of success is negligible as long as the signature is secure. The adversary has the same success probability in the latter case, because it has to generate a long enough chain that excludes the valid message which has a negligible success probability, under the assumption that the hash power of the adversary is lower than those of honest miners and due to the blockchain’s liveness property an honestly generated transaction will eventually appear on an honest miner’s chain \cite{GarayKL15}. 
 %
\end{proof}


Now, we first present a lemma   formally stating that the PwDR scheme is secure against a malicious bank and then prove this lemma. 

\begin{lemma} If the  SAP and  blockchain are secure, and the correctness of verdict encoding-decoding protocols (i.e., PVE and FVD) holds, then the PwDR scheme is secure against a malicious bank, with regard to Definition \ref{def::Security-against-malicious-bank}. 
\end{lemma}



\begin{proof}
%
We first focus on event $\text{\MakeUppercase{\romannumeral 1}}: \Big( (\sum\limits_{\st j=1}^{\st n}w_{\st 1,j}\geq e) \wedge ( v_{\st 1}=0)\Big)$ which captures the case where  $\mathcal{DR}$ is convinced that the pass message was correctly given or an important warning message was not missing, despite at least threshold arbiters do not believe so. We argue that the probability that this event takes place is negligible in the security parameter. Because, $\mathcal{B}$ cannot persuade $\mathcal{DR}$ to accept a different decryption key, e.g., $k'\in \ddot\pi'$, other than what was agreed between $\mathcal{C}$ and $\mathcal{B}$ in the initiation phase, i.e., $\bar k_{\st 2}\in \ddot\pi_{\st 2}$, except with a negligible probability. Specifically, it cannot persuade  $\mathcal{DR}$ to accept a statement $\ddot \pi'$, where $\ddot \pi'\neq \ddot \pi_{\st 2}$ except with   probability $\mu(\lambda)$. Furthermore, as discussed in Section \ref{sec::Encoding-Decoding-Verdicts}, due to the correctness of the verdict encoding-decoding protocols, i.e., PVE and FVD, the probability that multiple representations of verdict  $1$ cancel out each other is negligible too, $\frac{1}{2^{\st \lambda}}$. Thus,  event $\text{\MakeUppercase{\romannumeral 1}}$ occurs only with a negligible probability, $\mu(\lambda)$. To  assert that   events $\text{\MakeUppercase{\romannumeral 2}}: \Big(( \sum\limits_{\st j=1}^{\st n}w_{\st 2,j}\geq e) \wedge ( v_{\st 2}=0)\Big), \text{\MakeUppercase{\romannumeral 3}}: \Big(( \sum\limits_{\st j=1}^{\st n}w_{\st 3,j}\geq e) \wedge ( v_{\st 3}=0)\Big)$, and $\text{\MakeUppercase{\romannumeral 4}}: \Big(( \sum\limits_{\st j=1}^{\st n}w_{\st 4,j}\geq e) \wedge ( v_{\st 4}=0)\Big)$ occur only with a  negligible probability, we can directly use the above argument provided for event $\text{\MakeUppercase{\romannumeral 1}}$. To avoid repetition, we do not restate them in this proof.  Moreover, a malicious $\mathcal{B}$ cannot frame an honest $\mathcal{C}$ for providing an invalid message by manipulating the smart contract’s content,  e.g., by replacing its valid signature with an invalid one or sending a message on its behalf, due to the security of the blockchain.
  %
\end{proof}


Next, we prove the PwDR protocol's privacy. As before, we first formally state the related lemma and then prove it. 


\begin{lemma}
If the encryption schemes are semantically secure, and the SAP and encoding-decoding schemes (i.e., PVE and FVD)  are secure, then the PwDR scheme is privacy-preserving with regard to Definition \ref{def::privacy}.  
\end{lemma}

% 

\begin{proof}
We first focus on property 1, i.e., the privacy of the parties' messages from the public.  Due to the privacy-preserving property of the SAP, that relies  on the hiding property of the commitment scheme, given the public commitments, $g:=(g_{\st 1}, g_{\st 2})$,  the adversary learns no information about the
committed values, $(\bar k_{\st 1}, \bar k_{\st 2})$, except with a negligible probability, $\mu(\lambda)$. Thus, it cannot find the encryption-decryption keys used to generate ciphertext  $\hat {\bm m}, \hat{\bm l}, \hat z$, and  $\hat{\bm{w}}$. Moreover, due to the semantically security of the symmetric key and asymmetric key encryption schemes,  given ciphertext $(\hat {\bm m}, \hat{\bm l}, \hat z, \hat{\ddot \pi}, \hat{\bm{w}})$ the adversary cannot learn anything  about the related plaintext, except with a negligible probability, $\mu(\lambda)$. Thus, in experiment  $\mathsf{Exp}_{\st 3}^{\mathcal{A}_{\st 1}}$,  adversary $\mathcal{A}_{\st 1}$ cannot tell the value of $\gamma\in \{0, 1\}$ significantly better than just guessing it, i.e., its success probability is at most $\frac{1}{2}+\mu(\lambda)$. Now we move on to property 2, i.e., the privacy of each verdict from $\mathcal{DR}$. Due to the privacy-preserving property of the SAP, given $g_{\st 1}\in g$, a corrupt $\mathcal{DR}$ cannot learn  $\bar k_{\st 1}$. So,  it cannot find the encryption-decryption key used to generate ciphertext  $\hat {\bm m}, \hat{\bm l}$, and $\hat z$. Also, public parameters $(pk,pp)$ and token $T_{\st 2}$ are independent of $\mathcal{C}$'s and $\mathcal{B}$'s exchanged messages (e.g., payment requests or warning messages)  and $\mathcal{D}_{\st j}$s  verdicts. Furthermore, due to  the semantical security of the symmetric key and asymmetric key encryption schemes,  given ciphertext $(\hat {\bm m}, \hat{\bm l}, \hat z, \hat{\ddot \pi})$ the adversary cannot learn anything  about the related plaintext, except with a negligible probability, $\mu(\lambda)$. Also, due to the security of the  PVE and FVD protocols, the adversary cannot link a verdict to a specific arbiter with a probability significantly better than the maximum probability, $Pr'$, that an arbiter sets its verdict to a certain value, i.e., its success probability is at most $Pr'+\mu(\lambda)$, even if it is given the final verdicts, except when all arbiters' verdicts are $0$. We conclude that, excluding the case where the all verdicts are $0$, given $(T_{\st 2}, pk, pp, g, \hat{\bm m}, \hat{\bm l},  \hat z, \hat{\ddot \pi}, \hat{\bm{w}}, \bm v)$,   adversary $\mathcal{A}_{\st 3}$'s success probability in experiment $\mathsf{Exp}_{\st 4}^{\mathcal{A}_{\st 2}}$ to link a verdict to an arbiter is at most $Pr'+\mu(\lambda)$. 

%\hat {\bm m}, \hat{\bm l}, \hat z, \hat{\ddot \pi}, \hat{\bm{w}}
 %(e.g., o, l, padπ , padq , and k ̄), 
\end{proof}





%\subsection{Extensions}
%
%\subsubsection{Micro-enterprise or Charity Customer.} There is a clause in the CRM code that  allows a bank to refuse reimbursing a victim of an APP scam, if (i) the victim is an organisation (i.e., Micro-enterprise or Charity), (ii) the organisation has internal payment procedures that are effective in preventing the APP scam, and (iii) the victim has not followed those procedures. Below, we outline how the PwDR protocol can be extended to capture these requirements. The modified PwDR allows a customer to prove it has followed those requirements (or to prove there were not such  procedures), which ultimately benefits an honest customer during the dispute resolution phase (i.e., phase \ref{DR::DisputeResolution}).   We present the extension in two phases. In phase I, we provide a subprotocol that determines whether the customer has met the above requirements. In phase II, we show how the subprotocol can be integrated into the PwDR protocol.  
%
%\noindent\textit{{Phase I.}} In this phase, we provide an overview of a  subprotocol that determines if the customer has met the above requirements. At a high level, this subprotocol works as follows.
%
%
% \begin{enumerate}[label=(\Alph*)]
%\item\label{C-side} $\mathcal{C}$ sends the organisation's internal procedure specification and the specification's certificate  to  $\mathcal{S}$. Moreover,   it sends to $\mathcal{S}$  a proof, $p$, asserting that it has followed the above procedure. 
%
%\item\label{D-side} Every  $\mathcal{D}_{\st i}$ takes the following steps:
% 
% \begin{enumerate}
% \item checks the certificate by running procedure $\mathtt{Sig.ver}(.)\rightarrow h_{\st i}$. It sends $h_{\st i}$ to $\mathcal{S}$. It waits until all arbiters provide their inputs. Then, it locally runs $\mathtt{f.verdict}(h_{\st 1},...,h_{\st n})$ to determine if the certificate has been approved (by the majority of arbiters). It only proceeds to the next step if $\mathtt{f.verdict}(.)$ returns $1$.  
%%
% \item\label{subprotocol::evaluate-procedure}  checks whether the  procedure could prevent the APP scam. If the check passes, then it sends $v'_{\st i}=1$  to $\mathcal{S}$; otherwise, it sends $v'_{\st i}=0$  to $\mathcal{S}$. Again, it waits until all arbiters provide their inputs. Next, it locally runs $\mathtt{f.verdict}(v'_{\st 1},...,v'_{\st n})$. It only proceeds to the next step if $\mathtt{f.verdict}(.)$ returns $1$.
% %
% \item verifies  proof $p$. If the check passes, then it sends $v''_{\st i}=1$  to $\mathcal{S}$. Otherwise, it sends $v'_{\st i}=0$  to $\mathcal{S}$.
% 
% \end{enumerate} 
% 
%\item\label{S-side} $\mathcal{S}$ computes and stores the following values:
%  \begin{itemize}
%  %
%  \item [$\bullet$] $h=\mathtt{f.verdict}(h_{\st 1},...,h_{\st n})$.
%  %
%  \item [$\bullet$] if $h=1$, then $v'=\mathtt{f.verdict}(v'_{\st 1},...,v'_{\st n})$.
%  %
%  \item [$\bullet$] if $v'=1$, then $v''=\mathtt{f.verdict}(v''_{\st 1},...,v''_{\st n})$.
%   \item [$\bullet$] sets $g=1$, if one of the  following two conditions holds:  
%   
%   \begin{itemize}
%   \item[*]  $h=0$.
%   \item[*]   $h\wedge v'' =1$.
%  \end{itemize}
% otherwise (if neither condition holds), sets $g=0$. 
%  %
%   \end{itemize} 
% \end{enumerate} 
% 
%% checks whether the  certificate is valid and the procedure could prevent the APP scam. If both checks pass, then $\mathcal{D}_{\st i}$ sends $v'_{\st i}=1$  to $\mathcal{S}$. Otherwise, it sends $v'_{\st i}=0$ to it. Furthermore, in the case where $\mathcal{D}_{\st i}$  sent $v'_{\st i}=1$ to $\mathcal{S}$, it  checks $p$'s validity. If the check passes, then it sends $v''_{\st 1}=1$ to $\mathcal{S}$; if the check fails, it sends $v''_{\st 1}=0$ to $\mathcal{S}$. In the case where 
%%
%%
%%After all arbiters provide their inputs, $\mathcal{S}$ executes    $\mathtt{effective}(v'_{\st 1}, ..., v'_{\st n})\rightarrow b'$ and   $\mathtt{effective}(v''_{\st 1}, ..., v''_{\st n})\rightarrow b''$. 
%
%%
%%Next $\mathcal{C}$ checks $b'$, if it has been set to $1$, then it sends to $\mathcal{S}$  a proof, $p$, asserting that it has followed the above procedure. Otherwise (i.e., $b'=0$), it does not need to send any proof; note that in this case the arbiters reaches to the consensus that there was not effective internal procedure to prevent the APP scam. Given the customer's input, every  arbiter $\mathcal{D}_{\st i}$ checks the proof's validity. If the check passes, then it sends $v''_{\st 1}$ to $\mathcal{S}$; otherwise, it sends $v''_{\st 1}$ to $\mathcal{S}$. Once all arbiter's inputs are provided,  $\mathcal{S}$ runs $\mathtt{effective}(v''_{\st 1}, ..., v''_{\st n})\rightarrow b''$. 
%
%
%
%\noindent\textit{{Phase II.}} Now describe how the  above subprotocol's phases can be integrated into  the PwDR protocol.  First,  phases \ref{C-side} and \ref{D-side},  of the subprotocol, are added to steps \ref{DR::C-sends-complaint} and \ref{arbiters-verdict}, of the PwDR protocol, respectively. Second, the checks in phase \ref{DR::check-m1c}  of the PwDR protocol need to also ensure that $g=1$, by taking the steps of  phase \ref{S-side}   in the subprotocol. 
%
%
%Note that in step \ref{subprotocol::evaluate-procedure} of the subprotocol, it is assumed that each arbiter uses a well-defined process to evaluate whether the customer's internal payment procedures could have prevented the APP scam. Nevertheless,  effective   payment procedures that  prevent the APP fraud have not been appropriately defined  by the CRM code. This leads to a  manual and inefficient evaluation process.    Therefore, one may ask:
%
%
%
%\stepcounter{counter}
%%\arabic{counter}
%
%
%
%
%  \begin{center}\textit{Q\arabic{counter}: which measures exactly should be included in the internal payment procedures of an organisation to prevent  APP scams?}
% \end{center}
% 
%Given an explicit list of payment procedures  that  prevent  APP scams,  we could make the above evaluation procedure autonomous and transparent. In particular, such a list could be encoded into the  smart contract $\mathcal{S}$ which receives inputs from the customer (and possibly arbiters) and check if those internal payment procedures were effective in preventing the APP scam. 
%
%
%\subsubsection{Security against exploitative victim.}
%
%Having in place a transparent deterministic procedure (e.g.,  the PwDR protocol)  for evaluating victims' requests for reimbursement  could potentially create opportunities for exploitations. In particular, an honest victim  of the APP fraud who had been reimbursed in the past due to the payment system's vulnerability, e.g., ineffective payment, may be tempted to  exploit the same  known vulnerability multiple times.  Below, we outline two cases in which the malicious victim may exploit a known vulnerability and describe how they can be dealt with.  
%
%\begin{itemize}
%\item Case 1: A genuine victim, who  previously had been reimbursed,  colludes with a certain payee. In this case, the malicious victim uses the previous working strategy (e.g., the same certificates and complaints) to declare that it has been a victim of the APP fraud but this time it  exploits the weakness (e.g., ignores the ineffective warning). Note, the colluding payee acts exactly the same way as a real scammer does in the APP scam, e.g., after receiving the payment, it transfers the money to another account abroad. 
%
%
%\item Case 2: A genuine victim shares (or sells) its knowledge of the vulnerability to a  malicious customer who colludes with a payee to claim it has been a victim of the APP scam. Therefore, this case is similar to Case 1 with the main difference that the claimant's  identity  changes each time a claim is made. %which makes tracking the victim harder than the one in Case 1. 
%\end{itemize}
%
%
%Ideally, the bank should patch the vulnerability as soon as the first  incident occurs; however, this may not be the case in the real world as the cost of improving a  (payment) system to deal with such an incident may far exceed the  bank's monetary loss in that incident. Currently, neither the CRM code nor the ``Payment Services Regulations'' \cite{Regulations} explicitly offer any solution for the above cases.  To address the issue in Case 1, we propose the following remedy. First, we introduce two (set of) parameters; namely, threshold and counters. We let  bank $\mathcal{B}$  define the value of the threshold  in the smart contract, $\mathcal{S}$. Also, we require the protocol to keep track of the number of times the same customer is reimbursed for the same complaint, e.g., ineffective warning. Now we outline how these parameters can assist the  protocol to rectify the issue. Each time the protocol receives a customer's complaint, it checks whether  the total number of times the same customer is reimbursed for that specific  complaint exceeds the predefined threshold; if so, the protocol discards that complaint. Otherwise, it proceeds as before. More specifically, we amend the PwDR protocol as follows. 
%
%\begin{itemize}
%\item[$\bullet$]  $\mathcal{S}$ maintains a  counter vector, $\bm q^{\st (\mathcal{C})}=[(q_{\st e}, e),(q_{\st  x}, x),(q_{\st y}, y)]$,  for each customer $\mathcal{C}$, where  $q_{\st e}, q_{\st x}$, and $q_{\st y}$ are initially set to $0$ while $e, x$ and $y$ are the types of complaint, as they were defined in the PwDR scheme. 
%
%\item[$\bullet$]    $\mathcal{B}$ defines a fixed threshold $t$ in   $\mathcal{S}$,  where $t >1$. 
%
%\item[$\bullet$] In Phase \ref{DR::DisputeResolution}, $\mathcal{S}$ increments counter $q_{\st j}$ by $1$, each time   $\mathcal{C}$ is reimbursed for complaint $j$, where $j\in \{e,x,y\}$. 
%
%
%%\item[$\bullet$]   $\mathtt{verComplaint}(.)$,  $\mathtt{checkWarEff}(.)$, and $\mathtt{verStat}(.)$ take $\bm q^{\st (\mathcal{C})}$ and $t$ as inputs as well.  
%
%\item[$\bullet$]  Algorithm $\mathtt{verComplaint}(.)$ takes extra parameters $\bm q^{\st (\mathcal{C})}$ and $t$ as inputs.  
%
%
%\item[$\bullet$] In Phase \ref{VerifyingComplaint}, when a committee member, $\mathcal{D}_{\st i}$, wants to examine  $\mathcal{C}$'s  complaint, it reads the content of $\bm q^{\st (\mathcal{C})}$ and $t$ from $\mathcal{S}$ and passes them to $\mathtt{verComplaint}(.)$, which first checks whether the counter in $\bm q^{\st (\mathcal{C})}$ exceeds  $t$ for the complaint that $\mathcal{C}$ is making. If the check passes, it sets $d_{\st i}= v_{\st i}= \bar v_{\st i}=  w_{\st i}=\phi$ and outputs $(d_{\st i}, v_{\st i}, \bar v_{\st i},  w_{\st i})$ without requiring $\mathcal{D}_{\st i}$ to process $\mathcal{C}$'s claim. Otherwise,  $\mathcal{D}_{\st i}$ checks the claim as before.  
%
%\end{itemize}
%
%Now, we move on to the issue in Case 2 which is harder to identify than the one in Case 1. Because, in the former case, the identity of a malicious customer which wrongly claims that it has fallen victim to the APP fraud changes continuously and  is hardly distinguishable from a genuine victim without the assistance of extra information (e.g., phone records, emails) that is not trivial  to attain without  external intervention, e.g., a subpoena. To address this issue, we  propose the following mitigation.  Briefly, we require $\mathcal{B}$ to enhance its system and rectify the issue when the number of complaints related to the same issue exceeds a global threshold. Specifically, $\mathcal{B}$  defines in $\mathcal{S}$ a \emph{global} threshold $gt$ and global counter vector $\bm g=[(g_{\st e}, e),(g_{\st  x}, x),(g_{\st y}, y)]$,   where the vector's elements are defined the same way as those are defined in $\bm q$, with the difference that they are global and are not for a specific customer. Each time a customer is reimbursed for a reason   the related counter in $\bm g$ is incremented by $1$. When a counter exceeds the value of $gt$, $\mathcal{S}$ notifies $\mathcal{B}$ which patches the issue and then sets the related counter to $0$. Note that the value of $gt$ can be set such that when the monetary loss exceeds the upgrade costs, then  $\mathcal{B}$ upgrades the system.  










%$\bm q^{\st (\mathcal{C})}=[(q_{\st e}, e),(q_{\st  x}, x),(q_{\st y}, y)]$,  for each customer $\mathcal{C}$, where  $q_{\st e}, q_{\st x}$, and $q_{\st y}$ are initially set to $0$ while $e, x$ and $y$ are the types of complaint



%we can introduce a threshold for each vulnerability to the system. When it is exceeded, then that vulnerability needs to be dealt with (or in general the system should be upgraded).


%\section{Discussion}
% 
% 
% \subsection{Effective Warning}
%In the protocol, in step \ref{clinet-at-T2}, to which value  $\mathcal{C}$ sets its message $m_{\st 2}^{\st(\mathcal C)}$ is not totally deterministic  and depends on various (external) factors, e.g., warning effectiveness,  human factors. It is very  likely that  if $m_{\st 1}^{\st(\mathcal B)}=$ ``pass'', then $\mathcal{C}$ acts deterministically, by asking $\mathcal{B}$ to make the payment, i.e.,  $\mathcal{C}$ sets  $m_{\st 2}^{\st(\mathcal C)}=in_{\st p}$.   However,  to which value $\mathcal{C}$ sets $m_{\st 2}^{\st(\mathcal C)}$ when (a) $m_{\st 1}^{\st(\mathcal B)}= \text{warning}$ or (b) no message is provided by $\mathcal{B}$ at time $T_{\st 2}$, depends on many factors. For instance, if the warning is not effective and does not concern $\mathcal{C}$, then $\mathcal{C}$ still would  set    $m_{\st 2}^{\st(\mathcal C)}=in_{\st p}$. Similarly, if  $\mathcal B$ does not send $m_{\st 1}^{\st(\mathcal B)}$, it is still possible that $\mathcal{C}$ sets $m_{\st 2}^{\st(\mathcal C)}=in_{\st p}$, for instance when it is distracted and does not pay attention to the absence of  the message that was supposed to be provided by $\mathcal{B}$ on time. On the other hand, if the warning is effective, or $\mathcal{C}$ in general is highly sensitive to warnings and  the absence of $\mathcal B$'s message, then it  does not make any payment, i.e.,  it sets $m_{\st 2}^{\st(\mathcal C)}=\phi$. Despite the above challenges, the proposed protocol ensures that $\mathcal{C}$ will be identified as the party who should be reimbursed,  if it acts according to the protocol, makes a payment but the warning was ineffective or no message was provided by $\mathcal{B}$ in step \ref{clinet-at-T2}. Therefore, the following questions would  follow: 
%
%\stepcounter{counter}
%%\arabic{counter}
%
%  \begin{center}\textit{Q\arabic{counter}: what percentage of customers after encountering the warning do still proceed to make a payment (i.e., set $m_{\st 2}^{\st(\mathcal C)}=in_{\st p}$)?}
%\end{center}
%
%\stepcounter{counter}
%
%  \begin{center}\textit{Q\arabic{counter}: how to make the above rate negligibly small?}
%  \end{center}
%
%
%
%There exists  a comprehensive research line in determining the effectiveness of warnings, e.g., in \cite{laughery2006designing,brinton2016users,felt2014experimenting}. These traditional research line  studies which factors make warnings effective and how a warning recipient is attracted to and follows the warning message.  However, in the context of APP scams, there is a vital  unique factor  that can directly influence a warning effectiveness; the factor is    \emph{the ability of the scammer to interact directly with the victim} (or warning recipient). This lets a scammer to actively try to negate  the effectiveness of banks' warnings and persuade the warning recipient to ignore the warning (and make  payment). Such a factor was not (needed to be) taken into account in the traditional study of warnings. Therefore, one may ask: 
%
%\stepcounter{counter}
%  \begin{center}\textit{Q\arabic{counter}: to what extent can a scammer  negate a warning effectiveness in the onctext of APP scams?}
%  \end{center}
%
%
%One way to answer the above question is to study each individual bank's statistics and find out how successful they have been in combatting the APP scams, as each bank designs its payment system and warnings independent of other banks. Thus, it would be interesting to find out:
%\stepcounter{counter}
%  \begin{center}\textit{Q\arabic{counter}: which bank  does have  the  lowest rate of APP scams and how did its warnings contribute to the low rate?}  
%  \end{center}
%
%
%Furthermore, in step \ref{arbiters-verdict}, it is implicitly assumed that in order for each arbiter, $\mathcal{D}_{\st i}$, to judge  the effectiveness of the warning and reach a verdict, it has access to the payment system and it can interact  with  $\mathcal{C}$ offline, e.g., to obtain further evidence from it.  Nevertheless, this process can be time consuming and a verdict may not be released in the real time once $\mathcal{C}$ sends its complaint to $\mathcal{S}$. Thus, it is natural to ask: 
%
%\stepcounter{counter}
%\begin{center}\textit{Q\arabic{counter}: to what extent can the role of the arbiters  be automated and accurately played by a computer program?}
%\end{center}
%
%%Recall, the  above protocol assumes that the customer does not collude with the APP scammer. However, this assumption may not always hold in the real world. The two parties may collude to exploit  the payment system's weaknesses, e.g., ineffective warning.  In this case,  the malicious customer honestly participates in the protocol but   ignores the warning and makes a payment. Later, it raises a dispute and challenges the warning's effectiveness. It would succeed and be identified as a party who should be reimbursed, if the committee approves that the warning is indeed ineffective. The main difference between this case and the one  where the customer falls victim to an APP scam  is that in the former  the  customer makes extra money while in the latter  it does not. Hence, it is reasonable to ask:
%%
%%\stepcounter{counter}
%%\begin{center}\textit{Q\arabic{counter}: how to relax the above non-colluding assumption while being able to distinguish an honest customer who has fallen victim to  an APP scam from a malicious customer who colludes with the APP scammer?}
%%\end{center}
%
% \subsection{Ensuring the Payment is for Genuine Goods and Services}\label{sec:genuine-goods}
% 
% 
% 
%There are a set of terms in the CRM code  that states: 
%
%\noindent\textit{``In all the circumstances at the time of the payment, in particular the characteristics of the Customer and the complexity and sophistication of the APP scam, the Customer made the payment without a reasonable basis for believing that:}
%\begin{itemize}
%\item[(i)] \textit{the payee was the person the Customer was expecting to pay;}
%\item[(ii)] \textit{the payment was for genuine goods or services; and/or}
%\item[(iii)] \textit{the person or business with whom they transacted was legitimate".}
%\end{itemize}
%
%
%We argue that    clause (ii) plays a minor role in  preventing a APP scam, as it is effective only when the seller wrongly claims it is in possession of a certain goods or can deliver certain services. It is ineffective when   a customer  ensures goods or services it wants to receive are indeed genuine but the seller avoids delivering the goods/services.  In this case, a seller (regardless of whether it is legitimate or not) may prove to the customer that the goods and services are genuine and belong to it (e.g., by sending a copy of related genuine  document), but it   still avoids delivering them once it receives the money. To capture the above issue as well, the above clause should be modified as follows: 
%
%%\begin{center}\textit{``... the payment was made (if and) only if the delivery of genuine good or services are guaranteed''.}
%% \end{center}
% 
%\begin{center}\textit{``... the delivery of genuine goods or services are guaranteed...''.}
% \end{center}
% 
%There are at least two ways to have the above guarantee: (a) involving a reputable trusted third-party intermediary which can compensate the customer if the seller misbehaves, e.g., eBay, amazon, or (b)  using a secure ``contingent service payment" scheme (e.g., in  \cite{CampanelliGGN17}) that supports the  fair exchange of digital goods or services and money without the involvement of the above trusted third-party intermediary. In this case, there would be no need for the customer to ensure if the seller is legitimate, because it pays only if  genuine goods or services are delivered. The existing fair exchange schemes  allow a seller and buyer to trade with each other \emph{outside of the bank payment system} by using a blockchain. However, these schemes can be embedded into the bank payment system such that the bank (or a group of banks) maintains the blockchain and converts internally  a customer's  fiat currency  to a digital one when the customer wants to trade in a fair manner with a party whose authenticity cannot be verified. 
%
%
%\subsection{Ensuring the Legitimacy and Authenticity of Payee}
%
%We highlight that even though clauses (i) and (iii), presented in  Section \ref{sec:genuine-goods}, can prevent a customer from falling to an APP scam, only in certain cases they would   benefit a victim of an APP fraud during the process of allocating liability or dispute resolution.  In particular, if the victim declares that (a)   it has used a   secure authentication mechanism (e.g., digital signature) to ensure  the legitimacy and authenticity of the payee or (b)  it has not performed any authentication, then it would not receive the reimbursement. In the former case, it has transferred the  money to the party it knows. However,   this  is not an APP scam,  according to the scam's general definition.  In the latter case, the firm can avoid reimbursing it, according to the CRM code (as the customer has not performed its part). The only case in which the customer might be reimbursed is when it (a) uses an insecure means for authentication (e.g., phone call) and (b) can prove it has done such a check. For instance, in the case where the victim has been asked by a scammer to make a phone call to its bank while the scammer  answers the phone call that the victim makes later. In this case, the victim needs to  prove that it has made a phone call to the bank, which may not be always possible (e.g., if the scammer uses a method to reply to the call before the real connection between the customer and bank is made).  In this case, in order for the customer to be reimbursed it needs to provide an evidence while there is no evidence left behind by the scammer. Thus, it is natural to ask:
%
%\stepcounter{counter}
%\begin{center}\textit{Q\arabic{counter}: how can the victim prove it has been a victim of an APP scam, in the above case?}
%\end{center}
%
%\subsection{Lack of Gross Negligence Definition}
%
%One of the conditions in the CRM code that allows a bank to avoid reimbursing the customer is clause R2(1)(e) which states: 
%
%
%\begin{center}\textit{``The Customer has been grossly negligent. For the avoidance of doubt the provisions of R2(1)(a)-(d) should not be taken to define gross negligence in this context.''}
% \end{center}
% 
% Nevertheless, neither the CRM code  nor the Payment Services Regulations   explicitly define under which circumstances the customer is considered ``grossly negligent'' in the context of the APP scam. In particular, in the CRM code, the only terms that discuss customer's misbehaviour are  the provisions of R2(1)(a)-(d); however, as stated above, they should be excluded from the definition of the term gross negligence. On the  other hand,  in the Payment Services Regulations, this term is used three times, i.e.,  twice in regulation 75 and once in regulation 77. But in all  three cases it is used for frauds related to \emph{unauthorised payments} which are  different types of frauds from the APP scams. Therefore, even the Payment Services Regulations does not define the term in the context of the APP scam. 
% 
% When an  accurate definition of the term is in place,  its conditions can be encoded into the smart contract of the PwDR protocol. This allows the PwDR protocol to transparently resolve  disputes  between the bank and customer, if the bank claims that the customer has been  grossly negligent. 




\bibliographystyle{splncs03}
\bibliography{ref}
\appendix
% !TEX root =main.tex



\section{Bloom Filter}\label{sec::bloom-filter-}

A Bloom filter \cite{DBLP:journals/cacm/Bloom70} is a compact data structure for probabilistic efficient  elements'  membership checking. A Bloom filter is an array of $\bar  m$ bits that are  initially all set to zero. It  represents $\bar n$  elements.  A Bloom filter comes along with  $\bar k$ independent hash functions. To insert an element, all the  hash values of the element are computed and their corresponding bits in the filter are set to $1$. To check an element's  membership, all its hash values are re-computed and checked whether all are set to one in the filter. If all the corresponding bits are one, then the element is probably in the filter; otherwise, it is not. In Bloom filters false positives are possible, i.e., it is possible that an element is not in the set, but the membership query shows that it is. According to \cite{BoseGKMMMST08}, the upper bound of the false positive probability is: $\bar q=\bar p^{\scriptscriptstyle \bar  k}(1+O(\frac{\bar k}{\bar p}\sqrt{\frac{\ln \bar m - \bar k \ln \bar  p}{\bar m}}))$,  where $\bar p$ is the probability that a particular bit in the filter is set to $1$ and calculated as: $\bar p=1-(1-\frac{1}{\bar m})^{\scriptscriptstyle \bar k\bar n}$. The efficiency of a Bloom filter depends
on  $\bar m$ and $\bar k$. The lower bound of $\bar m$  is $\bar  n \log_{\scriptscriptstyle 2}
\bar e \cdot\log_{\scriptscriptstyle 2} \frac{1}{\bar q}$, where $\bar e$ is the base of natural logarithms,  while the optimal number of hash functions is    $\log_{\scriptscriptstyle 2} \frac{1}{\bar q}$, when $\bar m$ is optimal. In this paper, we only use optimal $\bar k$ and $\bar m$. In practice, we would like to have a predefined acceptable upper bound on false positive probability, e.g., $\bar q=2^{\scriptscriptstyle - 40}$. Given $\bar q$ and $\bar n$, we can determine the rest  of the parameters. 


% !TEX root =main.tex


%\vspace{-3.4mm}
\clearpage

\section{Discussion on PVE-FVD}\label{sec::Variant-1-Theorem-proof}
In this section, we first formally state our observation on which  Variant 1 encoding-decoding protocol relies and then prove it. After that, we explain why this variant meets the three properties we laid out in Section \ref{sec::Encoding-Decoding-Verdicts}, i.e., it should (1) generate unlinkable verdicts, (2) not require auditors to interact with each other for each customer, and (3) be efficient.

\vspace{-3mm}

\begin{theorem}\label{set-xor}
Let set $S=\{s_{\st 1},..., s_{\st m}\}$ be the union of  two disjoint sets $S'$ and $S''$, where $S'$ contains non-zero random values pick uniformly  from a finite field $\mathbb{F}_{\st p}$, $S''$ contains zeros, $|S'|\geq c'=1$, $|S''|\geq c''=0$, and pair $(c',c'')$ is public information. Then, $r= \bigoplus\limits^{\st m}_{\st i=1} s_{\st i}$ reveals nothing beyond the public information.  
\end{theorem}

\vspace{-3mm}
\begin{proof}
Let $s_{\st 1}$ and $s$, be two random values picked uniformly at random from $\mathbb{F}_{\st p}$. Let $\bar s=s_{\st 1}\oplus \underbrace{0\oplus... \oplus 0}_{\st |S''|}$. Since  $\bar s=s_{\st 1}$, two values $\bar s$ and $s$ have identical distribution. Thus, $\bar s$ reveals nothing in this case. Next, let $\tilde s=\underbrace{ s_{\st 1}\oplus s_{\st 2}\oplus... \oplus s_{\st j}}_{\st |S'|}$, where $s_{\st i}\in S'$. Since each $s_{\st i}$ is a uniformly random value,  the XOR of them is a uniformly random value too. That means values $\tilde s$ and $s$ have identical distributions. Thus, $\tilde s$ reveals nothing in this case as well. Also, it is not hard to see that the combination of the above two cases reveals nothing too, i.e., $\bar s\oplus \tilde s$ and $s$ have identical distribution. 
%
\end{proof}

\vspace{-2mm}

The primary reason this variant meets property 1 is that each masked verdict reveals nothing about the verdict (and its representation) and given the final verdict, $\mathcal{I}$ cannot distinguish between the case where there is exactly one auditor that voted  $1$ and the case where multiple auditors voted $1$, as in both cases $\mathcal{I}$   extracts only a single random value, which reveals nothing about the number of auditors which voted $0$ or $1$ (due to Theorem \ref{set-xor}).   Note,  the protocols' correctness holds with an overwhelming probability, i.e., $1-\frac{1}{2^{\st \lambda}}$. Specifically, if two auditors represent their verdict by an identical random value, then when they are XORed they would cancel out each other which can affect the result's correctness. The same holds if the XOR of multiple verdicts' representations results in a value that can cancel out another verdict's representation. Nevertheless, the probability that such an event occurs is negligible in the security parameter $|p|=\lambda$, i.e., the probability is at most   $\frac{1}{2^{\st \lambda}}$. It is evident that this variant meets property 2 as the auditors interact with each other \emph{only once} (to agree on a key) for all customers. It also meets property 3 as it involves pseudorandom function invocations and XOR operations which are highly efficient operations. 


%We will use this variant in the PwDR protocol.
% !TEX root =main.tex


%\vspace{20mm}
\section{Variant 2 Encoding-Decoding Protocol's Main Theorem and Proof}\label{sec::Variant-2-Theorem-proof}

In this section, we first formally state our main observation on which Variant 2 encoding-decoding protocol relies. After that, we prove it.


\begin{theorem}
Let  $S=\{s_{\st 1},..., s_{\st m}\}$ be a set of random values picked uniformly from finite field $\mathbb{F}_{\st p}$, where the cardinality of $S$ is public information. Let $\mathtt{BF}$ be a Bloom filter encoding all elements of   $S$. Then,  $\mathtt{BF}$ reveals nothing about any element of $S$, beyond the public information, except with a negligible probability in the security parameter $\lambda$, i.e., with a probability at most $\frac{|S|}{2^{\st \lambda}}$. 
\end{theorem}

\begin{proof}
First, we consider the simplest case where only a single element of $S$ is encoded in $\mathtt{BF}$. In this case, due to the pre-image resistance of the Bloom filter's hash functions and the fact that the set's element was picked uniformly at random from $\mathbb{F}_{\st p}$, the probability that $\mathtt{BF}$ reveals anything about the original element is at most $\frac{1}{2^{\st \lambda}}$. Now, we move on to the case where all elements of $S$ are encoded in $\mathtt{BF}$. In this case, the probability that $\mathtt{BF}$ reveals anything about at least an element of the set is $\frac{|S|}{2^{\st \lambda}}$, due to the pre-image resistance of the hash functions,  the fact that all elements were selected uniformly at random from the finite field, and the union bound. Nevertheless, when a $\mathtt{BF}$'s size is set appropriately to avoid false-positive without wasting storage, this reveals the number of elements encoded in it, which is public information.  Thus, the only information $\mathtt{BF}$ reveals is the public one.  
 %
\end{proof}
% !TEX root =main.tex

\section{Generic Verdict Encoding-Decoding Protocols}


Here, we present two efficient verdict encoding and decoding protocols; namely, Private Verdict Encoding (PVE) and Final Verdict Decoding (FVD) protocols. Their goal is to let a third party $\mathcal{I}$, e.g., $\mathcal{DR}$, find out whether at least one arbiter voted $1$, while satisfying the following  requirements.  The protocols should (1) generate unlinkable verdicts, (2)  not require arbiters to interact with each other for each customer, and (3) be  efficient. Since, the second and third requirements are self-explanatory,  we only explain the first one.  Informally, the first property requires  that the protocols should generate encoded verdicts and final verdict in a way that $\mathcal{I}$,  given the encoded verdicts and final verdict, should not be able to (a)   link a  verdict to an arbiter (except when all arbiters' verdicts are $0$), and (b) find out the total number of $1$ or $0$ verdicts when they provide different verdicts. 



 At a high level, the protocols work as follows.  The arbiters only once for all customers agree on a secret key of a pseudorandom function. This key will allow each of them to generate a pseudorandom masking values such that if all masking values are ``XOR''ed, they would cancel out each other and result $0$.\footnote{This is similar to the idea used in the XOR-based secret sharing \cite{Schneier0078909}.}
 
 
 
 
 
Each arbiter represents its verdict by (i) representing it as a parameter which is set to either $0$ if the verdict is $0$ or to a random value if the verdict is $1$, and then (ii) masking this parameter by the above  pseudorandom value.  It sends the result to $\mathcal{I}$.  To decode the final verdict and find out whether any arbiter voted $1$, $\mathcal{I}$  does XOR all encoded verdicts. This removes the masks and XORs are verdicts' representations.  If the result is $0$, then    all arbiters must have voted $0$; therefore,  the final verdict is $0$. However, if the result is not $0$ (i.e., a random value), then at least one of the arbiters voted $1$, so  the final verdict is $1$. We present the encoding  and decoding protocols in figures \ref{fig:PVE} and \ref{fig:FVD} respectively.
 
 
 Not that the protocols' correctness holds, except  a negligible  probability. In particular, if two arbiters  represent their verdict by an identical random value, then when they are XORed they would cannel out each other which can affect the result's correctness. The same holds if the XOR of  multiple verdicts' representations results in a value that can cancel out another verdict's representation. Nevertheless, the probability that such an event occurs is negligible in the security parameter, i.e., at most   $\frac{1}{2^{\st \lambda}}$. It is evident that PVE and FVD protocols meet properties (2) and (3). The primary reason they also meet  property (1) is that each masked verdict reveals nothing about the verdict (and its representation) and  given the final verdict, $\mathcal{I}$ cannot distinguish between the case where there is exactly one arbiter that voted  $1$ and the case where multiple arbiters voted $1$, as in both cases $\mathcal{I}$   extracts only a single random value, which reveals nothing about the number of arbiters which voted $0$ or $1$. 
 
%  
% To encode a verdict $w$, each arbiter represents it as a polynomial. It randomises this polynomial and then  masks this polynomial with the pseudorandom masking polynomial. It sends the result to $\mathcal{I}$. To decode the final verdict and find out whether all arbiters agreed on the same verdict, i.e., unanimous decision,  $\mathcal{I}$  adds all polynomials up. This removes the masks. Next, it  evaluates the result polynomial at $v=1$ and $v=0$. It considers $v$ as the final verdict if the evaluation is  $0$. We present the encoding  and decoding protocols in figures \ref{fig:PVE} and \ref{fig:FVD} respectively. 
 



% either an specific final verdict  (i.e., $v=0$ or $v=1$) if  all arbiters' verdicts are identical, or  nothing  about the arbiters' inputs if they did not agree on the same specific verdict. The protocols are  primarily  based on ``zero-sum pseudorandom polynomials'' and the techniques often used by private set intersection (PSI) protocols. In particular, the arbiters only once for all customers agree on a secret key of a pseudorandom function. This key will allow each of them to generate a pseudorandom masking polynomial such that if all masking polynomials are summed up, they would cancel out each other and result $0$, i.e., zero-sum pseudorandom polynomials. 



%In this section, we present efficient (verdict) encoding and decoding protocols. The encoding protocol  lets each of the $n$ honest arbiters $\mathcal{D}:\{\mathcal{D}_{\st 1},..., \mathcal{D}_{\st n}\}$ non-interactively encode its verdict such that a third party party $\mathcal{I}$ (where $\mathcal{I}\notin \mathcal{D}$) can extract final verdict if  all arbiters' verdicts are identical with the following security requirements. First,  given individual encoded verdict, $\mathcal{I}$ cannot learn anything about each arbiter's verdict. Second, can find out only final verdict if  all arbiters' verdicts are identical; otherwise, it cannot learn each individual arbiter's verdict.  The decoding protocol lets  $\mathcal{I}$,  combines the encoded verdicts and learn either an specific final verdict  (i.e., $w=0$ or $w=1$) if  all arbiters' verdicts are identical, or  nothing  about the arbiters' inputs if they did not agree on the same specific verdict. The protocols are  primarily  based on ``zero-sum pseudorandom polynomials'' and the techniques often used by private set intersection (PSI) protocols. In particular, the arbiters only once for all customers agree on a secret key of a pseudorandom function. This key will allow each of them to generate a pseudorandom masking polynomial such that if all masking polynomials are summed up, they would cancel out each other and result $0$, i.e., zero-sum pseudorandom polynomials. 











%
%Below, we present ``Zero-sum Pseudorandom Values Generator'' (ZPVG), an algorithm that allows each of the $n$ arbiters  to \emph{efficiently} and \emph{independently}  generate a vector of $m$ pseudorandom values for each customer, such that when all arbiters' vectors are summed up component-wise, it would result in a vector of $m$ zeros. ZPVG is based on  the following idea. Each arbiter $\mathcal{D}_{\st j}\in\{\mathcal{D}_{\st 1},..., \mathcal{D}_{\st n-1}\}$  uses the secret key $\bar k_{\st 0}$ (as defined in Section \ref{Notations-and-Assumptions}), along with the customer's unique ID (e.g., it blockchain account's address) as the inputs of $\mathtt{PRF}(.)$  to derive $m$ pseudorandom values. However,  $\mathcal{D}_{\st n}$ generates each  $j$-th element of  its vector by computing the additive inverse of the sum of the $j$-th elements that the rest of arbiters generated. Even $\mathcal{D}_{\st n}$ does not need to interact with other arbiters, as it can regenerate their values too. Moreover, the arbiters do not need to interact with each other for every   new customer and can   reuse the same key because the new customer would have a new unique ID that would result in a fresh set of pseudorandom values (with a high probability). Figure \ref{fig:ZSPA} presents ZPVG in more detail. 





%It sends the result to all parties which can locally check if the relation holds. Note, in the literature,   there exist protocols that allows parties to agree on zero-sum pseudorandom values, e.g., in \cite{}; however, they are less efficient than $\mathtt{ZSPA}$, as their security requirements are different than ours, i.e., they assume the participants are malicious whereas we assume the arbiters who participate in this protocol  are honest. 


%Next, it commits to each value, where it uses $k_{\st 2}$ to generate the randomness of each commitment. Then, it constructs a Merkel tree on top of the commitments and  stores only the root of the tree  and the hash value of the keys (so in total three values) in  the smart contract.  Then, each party (using the keys) locally checks if the values and commitments have been constructed correctly; if so, each sends  an ``approved" message to the contract. 
%
%
%
%Informally, there are three main security requirements that $\mathtt{ZSPA}$ must meet: (a) privacy, (b)  non-refutability, and (c) indistinguishability. Privacy here means given the state of the  contract, an external party cannot learn any information about any of the (pseudorandom) values:  $z_{\st j}$; while non-refutability  means that if a party sends ``approved" then in future cannot deny the knowledge  of the values whose representation is stored in the contract. Furthermore, indistinguishability means that every $z_{\st j}$ ($1\leq j \leq m$) should be indistinguishable from a truly random value. In Fig. \ref{fig:ZSPA}, we provide $\mathtt{ZSPA}$ that efficiently generates $b$ vectors  where each vector elements is sum to zero. 



%\begin{figure}[ht]
%\setlength{\fboxsep}{0.7pt}
%\begin{center}
%\begin{boxedminipage}{12.3cm}
%\small{
%$\mathtt{ZPVG}(\bar{k}_{\st 0}, \text{ID}, n,  m, j)\rightarrow \bm r_{\st j}$\\
%------------------
%\begin{itemize}
%\item \noindent\textit{Input.} $\bar{k}_{\st 0}$: a key of  pseudorandom function's key $\mathtt{PRF}(.)$, $\text{ID}$: a unique identifier, $n$:  total number of rows of a matrix, and $m$: total number of columns of a matrix, $j$: a row's index in a matrix.
%%
%\item \noindent\textit{Output.} $\bm r_{\st j}$:  $j$-th row of an $n\times m$ matrix, such that  if $i$-th element of $\bm r_{\st j}$ is added with  the rest of elements in $i$-th column of the same matrix, the result would be  $0$. 
%\end{itemize}
%\begin{enumerate}
%%
%\item\label{ZSPA:val-gen} compute $m$ pseudorandom values as follows. 
%
%$\forall i, 1\leq i\leq m:$
%%
%\begin{itemize}
%%
%\item[$\bullet$] if $j< n: r_{\st i,j}=\mathtt{PRF}(\bar k_{\st 0}, i||j||\text{ID})$ 
%%
%\item[$\bullet$]  if $j=n: r_{\st i, n}=\big(-\sum\limits^{\st n-1}_{\st j=1} r_{\st i,j}\big) \bmod p$ 
%%
%\end{itemize}
%%
%\item return $\bm r_{\st j}=[r_{\st 1,j},..., r_{\st m,j}]$
%
%
%
%
%\
% \end{enumerate}
% 
%}
%\end{boxedminipage}
%\end{center}
%\caption{Zero-sum Pseudorandom Values Generator (ZPVG)} 
%\label{fig:ZSPA}
%\end{figure}






\begin{figure}[!ht]
\setlength{\fboxsep}{0.7pt}
\begin{center}
\begin{boxedminipage}{12.3cm}
\small{
\underline{$\mathtt{GPVE}(\bar{k}_{\st 0}, \text{ID},  w_{\st j}, o, e, n,  j)\rightarrow  (\bar{  w}_{\st j}, \mathtt{BF})$}\\
%
\begin{itemize}
\item \noindent\textit{Input.} $\bar{k}_{\st 0}$: a key of  pseudorandom function $\mathtt{PRF}(.)$, $\text{ID}$: a unique identifier, $ w_{\st j}$: a  verdict, $o$: an offset, $e$: a threshold, $n$: the total number of  arbiters,  and  $j$: an arbiter's index.
%
\item \noindent\textit{Output.} $\bar{  w}_{\st j}$:  an  encoded verdict.  
%
%$\bm r_{\st j}$:  $j$-th row of an $n\times m$ matrix, such that  if $i$-th element of $\bm r_{\st j}$ is added with  the rest of elements in $i$-th column of the same matrix, the result would be  $0$. 
\end{itemize}
Arbiter $\mathcal{D}_{\st j}$ takes the following steps.
\begin{enumerate}
%
\item\label{ZSPA:val-gen} computes a  pseudorandom  value,  as follows. 
%
%$\forall i,1\leq i\leq s:$
%
\begin{itemize}
%
\item[$\bullet$]$ \text{ if } j< n: r_{\st j}=\mathtt{PRF}(\bar k_{\st 0}, 1||o||j||\text{ID})$.\\
%
%\hspace{1mm} 
\item [$\bullet$] $ \text{ if } j=n: r_{\st j}= \bigoplus\limits^{\st n-1}_{\st i=1} r_{\st i}$.
%
\end{itemize}
Note, the above second step is taken only by $\mathcal{D}_{\st n}$.
%By the end of this phase, a random polynomial of the following form is generated, $\Psi_{\st j}=r_{\st i+2,j}\cdot x^{\st 2}+r_{\st i+1,j}\cdot x+r_{\st i,j}$.
%
\item  sets a fresh parameter, $w'_{\st j}$, that represents a verdict, as below. 
%
%$\forall i,1\leq i\leq s:$

%\begin{itemize}
%\item[$\bullet$]  $\text{ if } w_{\st j}=1:$ \text{\ sets \ } $w'_{\st j}= \alpha_{\st j}$, where $\alpha_{\st j}\stackrel{\st\$}\leftarrow \mathbb{F}_{\st p}$.
%
%\item [$\bullet$] $\text{ if } w_{\st j}=0: \text{\ sets \ } w'_{\st j}= 0$.
%
%\end{itemize}
\begin{equation*}
   w'_{\st j}= 
\begin{cases}
   0,              & \text{if } w_{\st j}=0\\
   \alpha_{\st j}=\mathtt{PRF}(\bar k_{\st 0}, 2||o||j||\text{ID}) ,& \text{if } w_{\st j}=1\\

    %0,              & \text{if } w_{\st j}=0
\end{cases}
\end{equation*}
%
\item masks  $w'_{\st j}$ as follows. %$\forall i,1\leq i\leq s:$
%
$\bar w_{\st j}= w'_{\st j}\oplus r_{\st j}$.
%
\item if $j=n$, computes a Bloom filter that encodes the combinations of verdict representations (i.e., $w'_{\st j}$)  for verdict $1$. In particular, it takes the following steps. 
\begin{itemize}
%
\item[$\bullet$] for every integer $i$ in the range $[e,n]$, computes the combinations (without repetition) of $i$ elements from set $\{\alpha_{\st 1},..., \alpha_{\st n}\}$, where the combination operation is XOR. Let $W=\{(\alpha_{\st 1}\oplus \alpha_{\st 2}\oplus... \oplus \alpha_{\st e}), (\alpha_{\st 2}\oplus \alpha_{\st 3}\oplus ... \oplus \alpha_{\st e+1}), ..., (\alpha_{\st 1}\oplus \alpha_{\st 2}\oplus... \oplus \alpha_{\st n})\}$ be the result.  
%
\item[$\bullet$] constructs an empty Bloom filter. Then, it inserts all elements of $W$ into this Bloom filter. Let $\mathtt{BF}$ be the Bloom filter encoding $W$'s elements. 

\end{itemize}
%

%
\item outputs ($\bar{ w}_{\st j}, \mathtt{BF})$.





%generates   a polynomial that encodes the verdict, i.e., $\Omega_{\st j}=(x-w)$.  
%
%\item multiplies the polynomial by a fresh random polynomial $\Phi_{\st j}$ of degree $1$ and adds the result with $\Psi_{\st j}$, i.e.,  $\bar\Omega_{\st j}=\Phi_{\st j}\cdot \Omega_{\st j}+\Psi_{\st j}\bmod p$. 
%
%
%\item  evaluates the result  polynomial, $\bar\Omega_{\st j}$, at every  element $x_{\scriptscriptstyle i}\in {\bm{x}}$. This yields a vector of   $y$-coordinates: $[ \bar w_{\st 1,j},..., \bar w_{\st 3,j}]$.
%%
%
%%
%\item return $\bar{\bm w}_{\st j}=[ \bar w_{\st 1,j},..., \bar w_{\st 3,j}]$.




\
 \end{enumerate}
 
}
\end{boxedminipage}
\end{center}
\caption{Generic Private Verdict Encoding  (GPVE) Protocol} 
\label{fig:PVE}
\end{figure}
%%%%%%%%%%%%%%%%%%%%%%%%%%%%%%%%%%%%%%
%
\begin{figure}[!ht]
\setlength{\fboxsep}{0.7pt}
\begin{center}
\begin{boxedminipage}{12.3cm}
\small{
\underline{$\mathtt{GFVD}(n,  \bar{\bm w}, \mathtt{BF})\rightarrow  v$}\\
%
\begin{itemize}
\item \noindent\textit{Input.} $n$:  the total number of  arbiters,  and  $\bar{\bm w}=[\bar{ w}_{\st 1},..., \bar{ w}_{\st n}]$:  a vector of all arbiters' encodes  verdicts.
%
\item \noindent\textit{Output.} $v$: final verdict.  
%
%$\bm r_{\st j}$:  $j$-th row of an $n\times m$ matrix, such that  if $i$-th element of $\bm r_{\st j}$ is added with  the rest of elements in $i$-th column of the same matrix, the result would be  $0$. 
\end{itemize}
A third-party $\mathcal{I}$ takes the following steps.
\begin{enumerate}
%
%
\item combines  all arbiters' encoded verdicts, $\bar w_{\st j}\in \bar{\bm {w}}$, as follows. 
%
$c= \bigoplus\limits^{\st n}_{\st j=1} \bar w_{\st j}$
% $$\forall i, 1\leq i\leq 3: g_{\st i}=\sum\limits^{\st n}_{\st j=1} \bar{w}_{\st i,j} \bmod p$$
%
%\item if $n$ is odd, then sets $c=c\oplus 1$. 
%
\item checks if $c$ is in the Bloom filter, $\mathtt{BF}$. 
%
\item sets the final verdict $v$ depending on the content of $c$. Specifically, 
%
%\begin{itemize}
%%
%\item[$\bullet$] if $c=0$, sets $v=0$.
%%
%\item[$\bullet$]  otherwise, sets $v=1$.
%%
%\end{itemize}
\begin{equation*}
   v= 
\begin{cases}
    0,              &\text{if } c= 0 \text{ or } c \notin\mathtt{BF}\\
   1 ,& \text{if } c \in\mathtt{BF}\\

\end{cases}
\end{equation*}
%
\item outputs  $v$. 

\
 \end{enumerate}
 
}
\end{boxedminipage}
\end{center}
\caption{Generic Final Verdict Decoding  (GFVD) Protocol} 
\label{fig:FVD}
\end{figure}


% !TEX root =main.tex

\section{Further Discussion on the Verdict Encoding-decoding Protocols}\label{sec:: Further-Discussion-on-the-Encoding-decoding-Protocol}

%In this section, we briefly explain why  existing  solutions are not suitable replacements for the our  verdict encoding-decoding protocols. 



Recall that each variant of our verdict encoding-decoding protocol is a voting mechanism. It  lets a third party, $\mathcal{I}$, find out if a threshold of the auditors voted $1$, while (i) generating unlinkable verdicts, (ii) not requiring auditors to interact with each other for each customer, (iii) hiding the number of $0$ or $1$ verdicts from  $\mathcal{I}$, and (iv) being  efficient. Therefore, it is natural to ask: 

\begin{center}
\emph{Is there   any {e-voting} protocol, in the literature, that can  simultaneously satisfy all the above requirements?}
\end{center}


The short answer is no. Recently, a provably secure  e-voting protocol that can hide the number of $1$ and $0$ votes has been proposed by K{u}sters \textit{et al.} \cite{KustersL00020}. Although this scheme can satisfy the above \emph{security} requirements, it imposes a high computation cost, as  it involves computationally expensive primitives such as zero-knowledge proofs, threshold public-key encryption scheme, and generic multi-party computation. In contrast, our verdict encoding-decoding protocols rely on much more lightweight operations such as XOR and hash function evaluations.  We also  highlight that our verdict encoding-decoding protocols are in a different setting than the one in which most of the e-voting protocols are. Because the former protocols are in the setting where there exists a small number of auditors (or voters) which are trusted and can interact with each other once; whereas, the latter (e-voting) protocols are in a more   generic setting where there is a large number of  voters, some of which might be malicious, and they are not   required to interact with each other. 


Note that each variant of our verdict encoding-decoding protocol requires every auditor to provide an encoded vote  in order for $\mathcal{I}$ to extract the final verdict. To let each variant terminate and $\mathcal{I}$ find out the final verdict in the case where a  set of  auditors do not provide their vote, we can integrate the    following idea into each variant. We define a manager auditor, say $\mathcal{D}_{\st n}$, which is always responsive and keeps track of missing votes. After the voting time elapses and $\mathcal{D}_{\st n}$ realises a certain  number of auditors did not provide their encoded vote, it provides $0$ votes on their behalf and masks them using the  auditors' masking values. 










\end{document}






