% !TEX root =main.tex

\section{Variant 1 Encoding-Decoding Protocols Main Theorem and Proof}\label{sec::Variant-1-Theorem-proof}

\begin{reptheorem}{set-xor}
Let set $S=\{s_{\st 1},..., s_{\st m}\}$ be the union of  two disjoint subsets $S'$ and $S''$, where $S'$ contains non-zero random values pick uniformly  from a finite field $\mathbb{F}_{\st p}$, $S''$ contains zeros, $|S'|\geq c'=1$, $|S''|\geq c''=0$, and pair $(c',c'')$ is public information. Then, $r= \bigoplus\limits^{\st m}_{\st i=1} s_{\st i}$ reveals nothing beyond the public information.  
\end{reptheorem}

\begin{proof}
Let $s_{\st 1}$ and $s$, be two random values picked uniformly at random from $\mathbb{F}_{\st p}$. Let $\bar s=s_{\st 1}\oplus \underbrace{0\oplus... \oplus 0}_{\st |S''|}$. Since  $\bar s=s_{\st 1}$, two values $\bar s$ and $s$ have identical distribution. Thus, $\bar s$ reveals nothing in this case. Next, let $\tilde s=\underbrace{ s_{\st 1}\oplus s_{\st 2}\oplus... \oplus s_{\st j}}_{\st |S'|}$, where $s_{\st i}\in S'$. Since each $s_{\st i}$ is a uniformly random value,  the XOR of them is a uniformly random value too. That means values $\tilde s$ and $s$ have identical distribution. Thus, $\tilde s$ reveals nothing in this case as well. Also, it is not hard to see that the combination of the above two cases reveals nothing too, i.e., $\bar s\oplus \tilde s$ and $s$ have    identical distribution. 
%
\end{proof}