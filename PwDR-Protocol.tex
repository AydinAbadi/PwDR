%!TEX root = main.tex

\subsection{The PwDR Protocol}\label{sec::PwDR-auditorprotocol}

In this section, we present the PwDR protocol in detail. 


%At a high level, the protocol works as follows. 

\begin{enumerate}[leftmargin=.46cm]

\item \underline{\textit{Generating $\mathcal{G}$'s and  $\mathcal{D}_{\st j}$'s  Parameters}}:  $\mathtt{keyGen}(1^{\st \lambda})\rightarrow (sk,pk)$. 

Parties $\mathcal{G}$ and  $\mathcal{D}_{\st j}$   take   steps \ref{signature-keygen} and   \ref{encryption-keygen} respectively. 



\begin{enumerate}
\item\label{signature-keygen} calls $\mathtt{Sig.keyGen}(1^{\st \lambda})\rightarrow (sk_{\st\mathcal{G}}, pk_{\st\mathcal{G}})$ to generate   secret key $sk_{\st\mathcal{G}}$ and   public key $pk_{\st\mathcal{G}}$. It publishes  $pk_{\st\mathcal{G}}$.
%
\item\label{encryption-keygen} calls $\tilde{\mathtt{keyGen}}(1^{\st\lambda})\rightarrow({sk}_{\st\mathcal {D}}, {pk}_{\st\mathcal {D}})$ to generate  decrypting secret key ${sk}_{\st\mathcal {D}}$ and encrypting public key ${pk}_{\st\mathcal {D}}$. It publishes the public key ${pk}_{\st\mathcal {D}}$ and sends ${sk}_{\st\mathcal {D}}$ to the rest of the auditors.
\end{enumerate}


Let $sk:=(sk_{\st\mathcal{G}}, sk_{\st\mathcal{D}})$ and $pk:=(pk_{\st\mathcal{G}}, pk_{\st\mathcal{D}})$. Note, this   phase takes place only once for all customers.

\vspace{2mm}

\item\label{RCPoRP::Bank-side-Initiation}  \underline{\textit{Bank-side Initiation}}: $\mathtt{bankInit}(1^{\st \lambda})\rightarrow (T, pp, \bm l)$.
%

Bank $\mathcal{B}$ takes the following steps. 
\begin{enumerate}

\item\label{RCPoRP::setup} picks   secret keys $\bar k_{\st 1}$ and $\bar k_{\st 2}$ for the symmetric key encryption scheme and  pseudorandom function $\mathtt{PRF}$ respectively. It  sets two private statements as $\pi_{\st 1}=\bar k_{\st 1}$ and $\pi_{\st 2}= \bar k_{\st 2}$.
%
\item\label{RCPoRP::set-qp}  calls $\mathtt{SAP.init}(1^{\st\lambda}, adr_{\st\mathcal{B}}, adr_{\st\mathcal{C}}, \pi_{\st i})\rightarrow(r_{
\st i}, g_{\st i}, adr_{\st\text{SAP}})$ to initiate  agreements on  statements $\pi_{\st i}\in \{\pi_{\st 1}, \pi_{\st 2}\}$  with  $\mathcal{C}$.  Let $T_{\st i}:=(\ddot{\pi}_{\st i}, g_{\st i})$ and $T:=(T_{\st 1}, T_{\st 2})$,  where  $\ddot{\pi}_{\st i}:=(\pi_{\st i}, r_{\st i})$ is the opening of $g_{\st i}$.  It also sets parameter $\Delta$ as a time window between two specific time points, i.e., $\Delta=t_{\st i} - t_{\st i-1}$. Briefly, it is used to impose an upper bound on a message delay.  %It  constructs a vector, $\bm{x}$, of three non-zero distinct $x$-coordinates, i.e., $\bm{x}=[x_{\st 1},..., x_{\st 3}]$, where $x_{\st i}\in \mathbb{F}_{\st p}$.
%
\item sends $\ddot{\pi}:=(\ddot{\pi}_{\st 1}, \ddot{\pi}_{\st 2})$ to   $\mathcal{C}$ and   sends  public parameter $pp:=(adr_{\st\text{SAP}},\Delta)$ to  smart contract $\mathcal{S}$.
%
\end{enumerate}

\vspace{2mm}

\item  \underline{\textit{Customer-side Initiation}}\label{customer-side-Initiation}: $\mathtt{customerInit} (1^{\st \lambda}, T, pp)\rightarrow a$.

Customer $\mathcal{C}$ takes the following steps. 

\begin{enumerate}
%
\item\label{customer-side-Initiation-SAP-agree} calls   $\mathtt{SAP.agree}(\pi_{\st i}, r_{\st i}, g_{\st i}, adr_{\st\mathcal{B}}, adr_{\st\text{SAP}})\rightarrow (g'_{\st i}, b_{\st i})$, to check the correctness of parameters in $T_{\st i}\in T$ and (if accepted) to agree on these parameters, where $(\pi_{\st i}, r_{\st i}) \in \ddot{\pi}_{\st i}\in T_{\st i}$ and $1\leq i \leq 2$. Note,  if both $\mathcal{B}$ and $\mathcal{C}$ are honest, then $g_{\st i}=g'_{\st i}$. It also checks $\Delta$ in  $\mathcal{S}$, e.g., to see whether it is sufficiently large.
%
\item if the above checks fail,  it sets $a=0$ and aborts. Otherwise, it sets $a=1$. It sends $a$ to $\mathcal{S}$. 
\end{enumerate}
%

\vspace{2mm}

\item\label{genUpdateRequest}  {\underline{\textit{Generating  Update Request}}: $\mathtt{genUpdateRequest}(T, f, {\bm l})\rightarrow \hat m^{\st\mathcal{(C)}}_{\st 1}$}.

Customer $\mathcal{C}$ takes the following steps. 

\begin{enumerate}
%
\item sets its request parameter $m_{\st 1}^{\st(\mathcal C)}$ as below. 
%
\begin{itemize}

\item[$\bullet$] if it wants to set up a new payee, then it sets $m_{\st 1}^{\st(\mathcal C)}:=(\phi,f)$, where $f$ is the new payee's detail.  %In this case, $\mathcal{S}$ appends $p$ to the payee list, $l$. 

%\item[$\bullet$] if $\mathtt{setupNewPayee}(.)$ is called, then it sets $m_{\st 1}^{\st(\mathcal C)}:=(\phi,f)$, where $f$ is new payee's detail.  %In this case, $\mathcal{S}$ appends $p$ to the payee list, $l$. 
%
\item[$\bullet$] if it wants to amend the existing payee's detail,   it sets $m_{\st 1}^{\st(\mathcal C)}:=(i, f)$, where $i$ is an index of the element in $\bm l$ that should  change to $f$.  %In this case, $\mathcal{S}$ replaces $i$-th emement of $l$ with $p$, i.e., $l[i]=p$.


%\item[$\bullet$] if $\mathtt{ammendExistingPayee}(.)$  is called, then it sets $m_{\st 1}^{\st(\mathcal C)}:=(i, f)$, where $i$ is the index of the element in $\bm l$ that should be changed to $f$.  %In this case, $\mathcal{S}$ replaces $i$-th emement of $l$ with $p$, i.e., $l[i]=p$.

\end{itemize}
\item\label{send-update-req} at time $t_{\st 0}$, sends to $\mathcal{S}$  the encryption of $m_{\st 1}^{\st(\mathcal C)}$, i.e., $\hat m_{\st 1}^{\st(\mathcal C)}=\mathtt{Enc}(\bar k_{\st 1}, m_{\st 1}^{\st(\mathcal C)})$. 
\end{enumerate}
%
% Inserting New Payee

\vspace{2mm}

 \item \underline{\textit{Inserting New Payee}}: $\mathtt{insertNewPayee}(\hat m^{\st\mathcal{(C)}}_{\st 1}, {\bm l})\rightarrow {\hat{\bm l}}$.
 
Smart contract $\mathcal{S}$ takes the following steps. 
  %
 \begin{itemize}
 %
 \item[$\bullet$] if $\hat m_{\st 1}^{\st(\mathcal C)}$  is not empty, it appends $\hat m_{\st 1}^{\st(\mathcal C)}$ to the payee list $\hat{\bm l}$, resulting in an updated list, $\hat{\bm l}$. 
 %
  \item[$\bullet$] if $\hat m_{\st 1}^{\st(\mathcal C)}$ is empty,  it does nothing. 
  %
 \end{itemize}
 
 % Generating Warning
 
 \vspace{2mm}

\item\label{Generating-Warning} \underline{\textit{Generating Warning}}: $\mathtt{genWarning}(T, {\hat{\bm l}}, aux)\rightarrow \hat m^{\st\mathcal{(B)}}_{\st1}$.

Bank $\mathcal{B}$ takes the following steps. 
\begin{enumerate}
%
\item  checks if the most recent list $\hat{\bm{l}}$ is not empty. If it is empty, it halts. Otherwise, it proceeds to the next step. 
%
\item  decrypts each element of $\hat{\bm l}$ and checks its correctness, e.g., checks whether each element meets its internal policy  stated in $aux$. If the check passes, it sets $m_{\st 1}^{\st(\mathcal{B})}= \text{``pass''}$. Otherwise, it sets $m_{\st 1}^{\st(\mathcal{B})}=\text{warning}$, where the warning is a  string that contains a warning's detail concatenated with the string $\text{``warning''}$.
%
\item at time $t_{\st 1}$, sends to $\mathcal{S}$ the encryption of $m_{\st 1}^{\st(\mathcal{B})}$, i.e.,  $\hat m_{\st 1}^{\st(\mathcal{B})}= \mathtt{Enc}(\bar k_{\st 1}, m_{\st 1}^{\st(\mathcal B)})$. 
\end{enumerate}

\vspace{2mm}

\item\label{clinet-at-T2} \underline{\textit{Generating Payment Request}}: $\mathtt{genPaymentRequest}(T, in_{\st f}, \hat{\bm{l}}, \hat m^{\st\mathcal{(B)}}_{\st1})\rightarrow \hat m^{\st\mathcal{(C)}}_{\st2}$.

Customer $\mathcal{C}$ takes the following steps. 
\begin{enumerate}
%
\item\label{decrypt-warning} at time $t_{\st 2}$, decrypts  $\hat{\bm{l}}$ and $\hat m_{\st 1}^{\st(\mathcal{B})}$.  Depending on the warning's content, it sets a payment request $m_{\st 2}^{\st(\mathcal C)}$ to  $\phi$ or  $in_{\st f}$, where $in_{\st f}$ contains the payment's detail, e.g., the payee's detail in $\hat {\bm l}$ and the amount it wants to transfer. 
%
\item\label{send-payment-req} at time $t_{\st 3}$, sends  to $\mathcal{S}$ the encryption of $m_{\st 2}^{\st(\mathcal C)}$, i.e., $\hat m_{\st 2}^{\st(\mathcal{C})}= \mathtt{Enc}(\bar k_{\st 1}, m_{\st 2}^{\st(\mathcal C)})$.
\end{enumerate}
%

\vspace{2mm}

\item\label{Making-Payment} \underline{\textit{Making Payment}}: $\mathtt{makePayment}(T, \hat m^{\st\mathcal{(C)}}_{\st2})\rightarrow \hat m^{\st\mathcal{(B)}}_{\st2}$.

Bank $\mathcal{B}$ takes the following steps. 
\begin{enumerate}
%
 \item at time $t_{\st 4}$, decrypts   $\hat m_{\st 2}^{\st(\mathcal C)}$, i.e.,  $ m_{\st 2}^{\st(\mathcal{C})}= \mathtt{Dec}(\bar k_{\st 1}, \hat m_{\st 2}^{\st(\mathcal C)})$.
 %
 \item at time $t_{\st 5}$, checks the content of $m_{\st 2}^{\st(\mathcal C)}$. If $m_{\st 2}^{\st(\mathcal C)}$ is non-empty, i.e., $m_{\st 2}^{\st(\mathcal C)}=in_{\st f}$, it checks if the payee's detail in $in_{\st f}$ has already been checked and the payment's amount does not exceed the customer's credit. If the checks pass, it  runs the off-chain payment algorithm, $\mathtt{pay}(in_{\st f})$.  In this case, it sets $m_{\st 2}^{\st(\mathcal B)}$=``paid''. Otherwise (i.e., if $m_{\st 2}^{\st(\mathcal C)}=\phi$ or neither checks pass), it sets $m_{\st 2}^{\st(\mathcal B)}=\phi$. It sends  to $\mathcal{S}$ the encryption of  $m_{\st 2}^{\st(\mathcal B)}$, i.e., $\hat m_{\st 2}^{\st(\mathcal{B})}= \mathtt{Enc}(\bar k_{\st 1}, m_{\st 2}^{\st(\mathcal B)})$.  %Otherwise (i.e., if $m_{\st 2}^{\st(\mathcal C)}=\phi$ or neither checks pass),  it halts.
 %
\end{enumerate}
%

\vspace{2mm}

\item\label{Generating-Complaint}  \underline{\textit{Generating Complaint}}: $\mathtt{genComplaint}(\hat m^{\st\mathcal{(B)}}_{\st 1}, \hat m^{\st\mathcal{(B)}}_{\st2}, T, pk, {aux}_{\st f})\rightarrow  (\hat z, \hat{\ddot \pi})$. 

Customer $\mathcal{C}$ takes the following steps. 
\begin{enumerate}
%
\item\label{DR::C-sends-complaint} decrypts $ \hat m^{\st\mathcal{(B)}}_{\st 1}$ and $\hat m^{\st\mathcal{(B)}}_{\st2}$; this results in $  m^{\st\mathcal{(B)}}_{\st 1}$ and $ m^{\st\mathcal{(B)}}_{\st2}$ respectively. Depending on the content of  the decrypted values, it sets its complaint's parameters $z:=(z_{\st 1}, z_{\st 2}, z_{\st 3})$ as follows.  %sends a complaint: $z:=(k, x, y)$ to $\mathcal{S}$ at time $T_{\st 6}$, where $k, x$ and $y$ are   set as follows.  

\begin{itemize}
%
%\item  [$\bullet$] if $\mathcal{C}$ want to complain that $\mathcal{B}$ has not provided any message (i.e., neither pass nor warning) and 
%
 \item  [$\bullet$] if $\mathcal{C}$ wants to make one of the two below statements, it sets  $z_{\st 1}=$``challenge message''.
 
 \begin{enumerate}[label=(\roman*)]
  \item the pass message (in $m^{\st\mathcal{(B)}}_{\st 1}$)  should have been a warning.
  
 \item $\mathcal{B}$ did not provide any message and if $\mathcal{B}$  provided a warning,    the fraud would have been prevented.  
 

   \end{enumerate}
   \item [$\bullet$] if $\mathcal{C}$ wants to challenge the effectiveness of the warning (in $m^{\st\mathcal{(B)}}_{\st 1}$),  it sets $z_{\st 2}= m||sig||pk_{\st\mathcal{G}}||$ $\text{``challenge warning''}$, where  $m$ is a piece of evidence,   $sig\in aux_{\st f}$ is the evidence's  certificate (obtained from  $\mathcal{G}$), and $pk_{\st\mathcal{G}}\in pk$.  
 
  % $x =$ ``challenge warning''. 

 %
%  \item [$\bullet$] if $\mathcal{C}$ wants to complain about the inconsistency of the payment (in $m^{\st\mathcal{(B)}}_{\st 2}$),  it sets  $z_{\st 3} =$ ``challenge payment''; else, it sets $
%z_{\st 3}=\phi $.


  \item [$\bullet$] if $\mathcal{C}$ wants to complain about the payment's inconsistency,  it sets  $z_{\st 3} =$ ``challenge payment''; else, it sets $
z_{\st 3}=\phi $.

    \end{itemize}
    \item at time $t_{\st 6}$, sends     $\hat z= \mathtt{Enc}(\bar k_{\st 1}, z)$ and 
 $\hat{\ddot \pi}={\tilde{\mathtt{Enc}}}({pk}_{\st\mathcal {D}}, \ddot \pi)$  to $\mathcal{S}$.
    
%    \begin{itemize}
%     \item[$\bullet$] the encryption of complaint $z$, i.e.,   $\hat z= \mathtt{Enc}(\bar k_{\st 1}, z)$.
%     \item[$\bullet$] the encryption of $\ddot{\pi}:=(\ddot{\pi}_{\st 1}, \ddot{\pi}_{\st 2})$, i.e., $\hat{\ddot \pi}={\tilde{\mathtt{Enc}}}({pk}_{\st\mathcal {D}}, \ddot \pi)$. %Note, $\ddot \pi$ contains the openings of the private statements' commitments (i.e., $g_{\st 1}$ and $g_{\st 2}$), and is encrypted under each $\mathcal{D}_{\st j}$'s public key. 
%     \end{itemize}
\end{enumerate}


%\item\label{VerifyingComplaint}  \underline{\textit{Verifying Complaint}}. $\mathtt{verComplaint}(\hat z, \hat{\ddot{\pi}}, g, \hat m^{\st\mathcal{(C)}}_{\st1},\hat m^{\st\mathcal{(C)}}_{\st 2}, \hat m^{\st\mathcal{(B)}}_{\st1}, \hat m^{\st\mathcal{(B)}}_{\st 2}, \hat{\bm{l}},  j, {sk}_{\st\mathcal {D}}, aux, pp)\rightarrow \hat{\bm w}_{\st j}$

\vspace{2mm}

\item\label{VerifyingComplaint}  \underline{\textit{Verifying Complaint}}: $\mathtt{verComplaint}(\hat z, \hat{\ddot{\pi}}, g, \hat {\bm m}, \hat{\bm{l}},  j, {sk}_{\st\mathcal {D}}, aux, pp)\rightarrow \hat{\bm w}_{\st j}$.

Every $\mathcal{D}_{\st j}\in\{\mathcal{D}_{\st 1}, ..., \mathcal{D}_{\st n}\}$ acts as follows.
\begin{enumerate}

%
\item at time $t_{\st 7}$, decrypts $\hat{\ddot{\pi}}$, i.e., ${\ddot{\pi}}={\tilde{\mathtt {Dec}}}({sk}_{\st\mathcal {D}}, \hat{\ddot \pi})$. 

\item checks the validity  of $(\ddot{\pi}_{\st 1}, \ddot{\pi}_{\st 2})$ in $\ddot{\pi}$ by locally running  the SAP's verification, i.e., $\mathtt{SAP.verify}(.)$, for each  $\ddot{\pi}_{\st i}$. It   returns  $s$. If $s=0$, it halts. If $s=1$ for both $\ddot{\pi}_{\st 1}$ and  $\ddot{\pi}_{\st 2}$, it proceeds to the next step. 
%
%\item decrypts $\hat m^{\st\mathcal{(C)}}_{\st1}$ and $\hat m^{\st\mathcal{(C)}}_{\st 2}$ and checks if they are not empty. Also, it checks if 
%%%%%%%%%%%%
\item decrypts $\hat{\bm m}=[\hat{m}_{\st 1}^{\st(\mathcal C)}$, $\hat{m}_{\st 2}^{\st(\mathcal C)}, \hat{m}_{\st 1}^{\st(\mathcal B)}, \hat{m}_{\st 2}^{\st(\mathcal B)}]$  using $\mathtt{Dec}(\bar k_{\st 1},.)$, where $\bar k_{\st 1}\in\ddot \pi_{\st 1}$. Let $[{m}_{\st 1}^{\st(\mathcal C)},  {m}_{\st 2}^{\st(\mathcal C)},  {m}_{\st 1}^{\st(\mathcal B)}, {m}_{\st 2}^{\st(\mathcal B)}]$ be the result. 

\item  checks whether $\mathcal C$ made an update request to its payee's list. To do so, it checks if  $m_{\st 1}^{\st(\mathcal C)}$  is non-empty and (its encryption) was registered by $\mathcal{C}$ in $\mathcal{S}$. Also, it checks whether $\mathcal C$ made a payment request, by checking if $m_{\st 2}^{\st(\mathcal C)}$ is non-empty and (its encryption) was registered by $\mathcal{C}$ in $\mathcal{S}$ at time $t_{\st 3}$.  If either check fails, it halts. 
%%%%%%%%%%%%
\item decrypts $\hat z$ and $\hat{\bm{l}}$ using $\mathtt{Dec}(\bar k_{\st 1},.)$, where $\bar k_{\st 1}\in\ddot \pi_{\st 1}$. Let $ z:=(z_{\st 1}, z_{\st 2}, z_{\st 3})$ and ${\bm{l}}$ be the result. 
%
\item\label{auditors-verdict} sets its verdicts according to  the content of  $z:=(z_{\st 1}, z_{\st 2}, z_{\st 3})$, as follows.  
%
\begin{itemize}

%
\item[$\bullet$]  if  ``challenge message'' $\notin z_{\st 1}$, it sets $w_{\st 1,j}= 0$. Otherwise,  it runs $\mathtt{verStat}(add_{\st\mathcal{S}}, m_{\st 1}^{\st(\mathcal{B})},  \bm{l}, \Delta, {aux})\rightarrow w_{\st 1, j}$, to determine if a warning (in $m_{\st 1}^{\st(\mathcal{B})}$) should have been given (instead of the pass or no message). %It sends $d_{\st i}$ to $\mathcal{S}$, at time $T_{\st 8}$. %If  $k=\phi$, then it does nothing  in this step.
%
\item[$\bullet$]  if ``challenge warning'' $\notin z_{\st 2}$, it sets $w_{\st 2, j}= w_{\st 3, j}= 0$. Otherwise, it runs $\mathtt{checkWarning}(add_{\st\mathcal{S}}, z_{\st 2}, m_{\st 1}^{\st(\mathcal{B})},$ $  {aux}')\rightarrow (w_{\st 2, j}, w_{\st 3, j})$, to determine the effectiveness of the warning (in $m_{\st 1}^{\st(\mathcal{B})}$). %It sends $(v_{\st i}, \bar v_{\st i})$ to $\mathcal{S}$, at time $T_{\st 9}$.



\item[$\bullet$]  if ``challenge payment'' $\in z_{\st 3}$, it checks if the  payment was  made.   If it passes, it sets  $w_{\st 4, j}=1$. If it failes,   it sets $w_{\st 4, j}=0$.  If ``challenge payment'' $\notin z_{\st 3}$, it checks if  ``paid''  $\in{m}_{\st 2}^{\st(\mathcal C)}$. If it passes, it sets $w_{\st 4, j}=1$. Else, it sets $w_{\st 4, j}=0$. 
%
\end{itemize}
%
\item  encodes  its verdicts $(w_{\st 1, j}, w_{\st 2, j}, w_{\st 3, j},  w_{\st 4, j})$ as follows. 
%
\begin{enumerate}
%
\item locally maintains a counter, $o_{\st adr_{\st{c}}}$,  for each $\mathcal{C}$. It sets its initial value to $0$.
%

%\item calls $\mathtt{PVE}(.)$ to represent every verdict as a polynomial. In particular, it performs as follows. $\forall i, 1\leq i \leq 4:$

\item calls $\mathtt{PVE}(.)$ to encode each verdict. In particular, it performs as follows. $\forall i, 1\leq i \leq 4:$
\begin{enumerate}
%

\item[$\bullet$] calls $\mathtt{PVE}(\bar{k}_{\st 0}, adr_{\st\mathcal{C}},  w_{\st i, j}, o_{\st adr_{\st{c}}}, n,  j)\rightarrow  \bar{  w}_{\st i,j}$. %, and  
\item[$\bullet$] sets $o_{\st adr_{\st{c}}}=o_{\st adr_{\st{c}}}+1$.

%\item[$\bullet$] calls $\mathtt{PVE}(\bar{k}_{\st 0}, adr_{\st\mathcal{C}}, w_{\st i, j}, n, \bm x, \textit{offset}_{\st adr_{\st{c}}}, j)\rightarrow  \bar{\bm w}_{\st i,j}$.
%
%\item[$\bullet$] $o_{\st adr_{\st{c}}}=o_{\st adr_{\st{c}}}+1$.
%
\end{enumerate}
By the end of this step, a vector ${\bar {\bm w}}_{\st j}$ of four encoded verdicts is computed, i.e., $\bar {\bm w}_{\st j}=[ \bar{  w}_{\st 1,j},..., \bar{  w}_{\st 4,j}]$.
%
\item uses $\bar k_{\st 2}\in \ddot \pi_{\st 2}$ to further encode/encrypt  $\mathtt{PVE}(.)$'s outputs as follows. %$\forall i, 1\leq i \leq 4:$
%
$ \hat {\bm w}_{\st j}= \mathtt{Enc}(\bar k_{\st 2}, \bar{\bm w}_{\st j})$.


%\
%
%
%\
%
%\item\label{call-ZPVG} generates $4$  pseudorandom values, by calling $\mathtt{ZPVG}(\bar{k}_{\st 0}, adr_{\st\mathcal{C}}, n,  4, j)\rightarrow \bm{r}_{\st j}$. Recall, the sum of each element $ r_{\st i,j}\in \bm r_{\st j}$ and the other auditors' related values would be $0$.   
%%
%\item\label{derive-PR-from-k2} generates $4$  pseudorandom values using $\bar k_{\st 2}$ (where  $\bar k_{\st 2}\in  \ddot\pi$) as follows. 
%%
%$$\forall i, 1\leq i \leq 4: \alpha_{\st i} = \mathtt{PRF}(\bar k_{\st 2}, i)$$
%
% Note, all auditors would generate the same set of $\alpha_{\st i}$. 
%
%\item  encodes  each $w_{\st i,j}$ by using the values generated in steps \ref{call-ZPVG} and \ref{derive-PR-from-k2}, as follows. 
%%
% $$\forall i, 1\leq i \leq 4: \hat w_{\st i, j} = \big(\alpha_{\st i}\cdot w_{\st i, j}+r_{\st i, j}\big)\bmod p$$
\end{enumerate}


%%. 
%\item[$\bullet$]  if  ``challenge message'' $\in k$, then it runs $\mathtt{verStat}(add_{\st\mathcal{S}}, m_{\st 1}^{\st(\mathcal{B})}, T, l', {aux})\rightarrow d_{\st i}$ to determine whether a warning should have been given (instead of the ``pass'' or no message).  It sends $d_{\st i}$ to $\mathcal{S}$, at time $T_{\st 8}$. %If  $k=\phi$, then it does nothing  in this step.
%%
%\item[$\bullet$]  if  ``challenge warning'' $\in x$, then it runs $\mathtt{checkWarEff}(add_{\st\mathcal{S}}, x, m_{\st 1}^{\st(\mathcal{B})},  {aux}')\rightarrow (v_{\st i}, \bar v_{\st i})$ to determine the warning's effectiveness.  It sends $(v_{\st i}, \bar v_{\st i})$ to $\mathcal{S}$, at time $T_{\st 9}$.
%
%
%
%
%\item[$\bullet$]  if $y =$ ``challenge payment'', in the case where the payment has been  made,  it sets its  verdict $w_{\st i}$ to $1$; otherwise, it sets $w_{\st i}$ to $0$. It sends $w_{\st i}$ to $\mathcal{S}$, at time $T_{\st 10}$.  %If  $y=\phi$, then it does nothing  in this step.


\item at time $t_{\st 8}$, sends to $\mathcal S$ the encrypted vector, $\hat {\bm w}_{\st j}$. %=[\hat { w}_{\st 1, j}, \hat {  w}_{\st 2, j}, \hat {  w}_{\st 3, j}, \hat {  w}_{\st 4, j}]$.
\end{enumerate}

\vspace{2mm}

\item\label{DR::DisputeResolution}  \underline{\textit{Resolving Dispute}}: $\mathtt{resDispute}(T_{\st 2}, \hat {\bm w}, pp)\rightarrow \bm v$.

 Party $\mathcal{DR}$ takes the below steps at time $t_{\st 9}$, if it is invoked by $\mathcal{C}$ or  $\mathcal{S}$ which sends $\ddot\pi_{\st 2}\in T_{\st 2}$ to it.

\begin{enumerate}
%
\item checks $\ddot{\pi}_{\st 2}$'s validity by locally running  the SAP's verification, i.e., $\mathtt{SAP.verify}(.)$, that  returns  $s$. If $s=0$, it halts. %Otherwise, it proceeds to the next step. 
%
%\item decrypts $\hat{m}_{\st 1}^{\st(\mathcal C)}$, $\hat{m}_{\st 2}^{\st(\mathcal C)}$, and $\hat{m}_{\st 2}^{\st(\mathcal B)}$  using $\mathtt{Dec}(\bar k_1,.)$, where $\bar k_1\in\ddot \pi$. Let ${m}_{\st 1}^{\st(\mathcal C)}$,  ${m}_{\st 2}^{\st(\mathcal C)}$, and ${m}_{\st 2}^{\st(\mathcal B)}$ be the result respectively. 
%
%\item\label{DR::check-m1c}  checks whether $\mathcal C$ made an update request to its payee's list. To do so, it checks if  $m_{\st 1}^{\st(\mathcal C)}$  is non-empty and (its encryption) was registered by $\mathcal{C}$ in $\mathcal{S}$.
%
%\item\label{DR::check-payment-request}  checks whether $\mathcal C$ made a payment request, by checking if $m_{\st 2}^{\st(\mathcal C)}$ is non-empty and (its encryption) was registered by $\mathcal{C}$ in $\mathcal{S}$ at time $t_{\st 3}$.  

\item computes the final verdicts, as below. 
%
\begin{enumerate}
%
\item uses $\bar k_{\st 2}\in \ddot \pi_{\st 2}$ to decrypt the auditors' encoded verdicts, as follows. $ \forall j, 1\leq j \leq n:$
%
$ \bar {\bm w}_{\st j}= \mathtt{Dec}(\bar k_{\st 2}, \hat{\bm w}_{\st j})$, where $\hat{\bm w}_{\st j}\in \hat{\bm w}$.
 % 
 \item constructs four vectors, $[\bm u_{\st 1},...,\bm u_{\st 4}]$, and sets  each vector $\bm u_{\st i}$ as follows. $\forall i, 1\leq i \leq 4:$
 %
  $\bm u_{\st i}=[\bar{w}_{\st i,1},...,\bar{w}_{\st i,n}]$, where $\bar{w}_{\st i,j}\in \bar {\bm w}_{\st j}$. 
% $\forall i, 1\leq i \leq 4: \bar {\bm w}_{\st i,j}= \mathtt{Dec}(\bar k_{\st 2}, \hat{\bm w}_{\st i,j})$.
%
\item calls $\mathtt{FVD}(.)$ to extract each final verdict, as follows. $\forall i, 1\leq i \leq 4:$ calls $\mathtt{FVD}(n,  {\bm u}_{\st i})\rightarrow  v_{\st i}$. 


%$\mathtt{FVD}(n,   \bm x, \bar{\bm w})\rightarrow  w$


%regenerates $4$  pseudorandom values using $\bar k_{\st 2}$ (where  $\bar k_{\st 2}\in  \ddot\pi$) as follows. 
%
%$$\forall i, 1\leq i \leq 4: \alpha_{\st i} = \mathtt{PRF}(\bar k_{\st 2}, i)$$
%
%\item  extracts the decoded final verdict (for each $i$) as follows. 
%
% $$\forall i, 1\leq i \leq 4: v_{\st i} = \big((\alpha_{\st i})^{\st -1} \cdot \sum\limits_{\st j=1}^{\st n} \hat {w}_{\st i, j}\big)\bmod p= \sum\limits_{\st j=1}^{\st n} {w}_{\st i, j}$$
%
\end{enumerate}
%
\item outputs $\bm v=[v_{\st 1},...,v_{\st 4}]$.
\end{enumerate}
\end{enumerate}

\clearpage           
In this protocol, customer $\mathcal{C}$ must be reimbursed if the final verdict is that (i)  the ``pass'' message or  missing message should have been a warning or (ii)  the warning was ineffective and the provided evidence was not invalid, and (iii) the payment has been made. Stated formally, the following relation must hold: 

$$\Big( \underbrace{(v_{\st 1}=1)}_{\st (\text{i})}\vee  \underbrace{(v_{\st 2}=1\ \wedge\ v_{\st 3}=1)}_{\st (\text {ii})}\Big)\wedge\Big(\underbrace{v_{\st 4}=1}_{\st (\text{iii})}\Big)$$
%
%\begin{center}
%$\Big( \underbrace{(v_{\st 1}=1)}_{\st (\text{i})}\vee  \underbrace{(v_{\st 2}=1\ \wedge\ v_{\st 3}=1)}_{\st (\text {ii})}\Big)\wedge\Big(\underbrace{v_{\st 4}=1}_{\st (\text{iii})}\Big)$.
%\end{center}
%
 %where $v_{\st i}\in \bm v$.
%
%\begin{enumerate}
%\item\label{DR::check-warning-related-message}   one of the following  warning-related conditions holds.
%%
%\begin{enumerate}
%%
%\item the final verdict is that the ``pass'' message or a missing message should have been a warning. To do so, it checks  if $v_{\st 1}=0$.
%%
%\item the final verdict is that the warning was ineffective and the provided evidence was not invalid, by checking  if $v_{\st 2}=0$ and $v_{\st 3}=1$. 
%%
%\end{enumerate}
%%
%\item\label{DR::check-payment-related-conditions}  checks if the following  payment-related condition holds.
%%
%\begin{enumerate}
%%
%\item[$\bullet$] the final verdict is that the payment has been made, by checking if $v_{\st 4}=1$. 
%%
%\item  the encryption of  ``paid'' (in ${m}_{\st 2}^{\st(\mathcal C)}$) was registered by $\mathcal{S}$ at time $t_{\st 5}$.
%
%If  the above conditions  in steps  \ref{DR::check-warning-related-message} and \ref{DR::check-payment-related-conditions} hold,  it outputs $1$. Otherwise, it outputs $0$. 
%\end{enumerate}
%\end{enumerate}
%
Note, in the above PwDR protocol, even $\mathcal{C}$ and $\mathcal{B}$ that know the decryption secret keys, $(\bar k_{\st 1}, \bar k_{\st 2})$, cannot link a certain verdict to an auditor, for two main reasons; namely,  (a) they do not know the masking random values used by auditors to mask each verdict and (b) the final verdicts $(v_{\st 1},..., v_{\st 4})$ reveal nothing about the number of $1$ or $0$ verdicts, except when all auditors vote $0$.   In the PwDR, we used PVE and FVD only because they are highly efficient. However, it is easy to replace them with GPVE and GFVD. %We present the PwDR protocol's main theorem and its proof in Appendix \ref{sec::proof}. 