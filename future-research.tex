% !TEX root =main.tex


\section{Future Research}


In this section, we highlight a set of future research directions in the context of APP frauds. 

\subsection{Improving Warnings' Effectiveness}

As we stated in  Section \ref{sec::Lack-of-Effective-Warning-Definition},  one of the determining factors in the process of allocating liability to APP fraud victims is if they follow warnings. However,  there exists no publicly available study on the effectiveness of warnings in  the context of APP frauds. There exists a comprehensive research line in determining the effectiveness of warnings in general, e.g., in \cite{brinton2016users,felt2014experimenting,laughery2006designing}. These traditional research line studies which factors make warnings effective and how a warning recipient is attracted to and follows the warning message. Nevertheless, in the context of APP frauds, there is a vital unique factor that can directly influence a warning's effectiveness. The factor is the ability of  fraudsters to interact directly with their victims. This lets a fraudster to actively try to negate the effectiveness of a bank's warning and persuade the warning recipient to ignore the warning (and make payment). Such a factor was not (needed to be) taken into account in the traditional study of warnings. The current high rate of  APP frauds suggests that there exists a huge room for improving warnings' effectiveness. Thus, future research can investigate (via  users studies) to identify  key factors that can improve the effectiveness of warnings in this context. 





\subsection{Protecting APP Fraud Victims of Alternative Payment Platforms}

Currently, there is no official report on the occurrence of APP frauds on any other payment platforms (e.g., cryptocurrency) than  regular banking. This can be due to a lack of an organisation which oversights and collects fraud-related data or due to a low rate of such fraud taking place in these platforms because they are not popular enough. Nevertheless,  with the increase in the popularity of  alternative payment platforms (including CBDC), it is likely that APP fraudsters will target these platforms' users. Hence, another interesting future research direction would be to design secure dispute resolution protocols  to protect  APP frauds victims in these platforms as well. 

\subsection{Studying Users' Compliance with the CRM Code Guidelines}

%User Studies on Users’ Compliance with the CRM Code Guidelines

\subsection{Security Against Exploitative Victims}


Future research in this field could also investigate how to design a secure protocol that could help lower the rate of APP frauds occurrence.


