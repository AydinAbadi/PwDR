%!TEX root = main.tex

\section{Background}


\subsection{Contingent Reimbursement Model Code}

In  2016, the UK's consumer protection organisation, called ``Which?'', submitted a super-complaint to the ``Financial Conduct Authority'' (FCA) about the APP frauds. It raised its concerns that, despite the APP frauds' victims rate is significantly growing, the victims do not have enough protection \cite{Which?-super-complaint}.  Since then, the FCA has been collaborating with financial institutes  to develop a number of initiatives that
could help prevent these frauds, and improve the response when they  occur. As a result,  the ``Contingent Reimbursement Model'' (CRM) code has been proposed. The  CRM code  lays out a set of requirements and explains under which circumstances customers should be reimbursed by their trading financial institutes, when they fall victim to an APP fraud \cite{CRM-code}. So far,  there are at least nine firms, comprising nineteen brands (e.g., Barclays, HSBC,  Lloyds) signed up to the CRM code. One of the tangible outcomes of the code is a service called ``Confirmation of Payee" (CoP)  offered by the CRM code signatories \cite{CoP}. This service checks the money recipient's account name once it is inserted by the sender customer into the online banking platform. If there is not an exact match, CoP provides a warning to the  customers about the risks of making the payment. In the case where a customer ignores such a warning, makes a payment, and later falls to an APP fraud, then it may not be reimbursed, according to the CRM code.

Although the CRM code is a vital guideline  towards reducing the occurrence of such frauds and protecting  the frauds' victims, it is still too vague and leaves a huge room for interpretation. For instance, in 2020, the ``Financial Ombudsman Service'' (that settles complaints between consumers and businesses)  highlighted that firms  are applying the CRM code inconsistently and in some cases incorrectly, which resulted in failing to reimburse victims in circumstances anticipated by this code \cite{Financial-Ombudsman-Service-response}. Hence, there is a pressing need for an accurate solution to help and protect the APP frauds victims. 

\subsection{Central Bank Digital Currency}