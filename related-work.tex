%!TEX root = main.tex

\section{Related Work}


\subsection{Authorised Push Payment Fraud}

Since, in Section \ref{sec::background}, we have already covered two most important work  with regard to APP frauds so far (i.e.,  the CRM code and FCA's  report), below we briefly  review other works related to this type of fraud. Anderson \textit{et al.} \cite{anderson2019measuring} provide an overviews APP frauds and highlight that although  the (CRM) code would urge  banks to accept more liability for APP frauds, it remains to be seen how this will evolve as   fraudsters will continuously try to figure out how bank systems can facilitate misdirection attacks. Taylor \textit{et al.} \cite{taylor2020new} reviews the CRM code from legal and practical perspectives. They  state that this code's proper implementation would make considerable advances to protect victims of APP frauds. They also highlight this code's shortcomings. For instance, they argue that this code is still ambiguous. Specifically, they refer to a part of the  CRM code  stating that for a victim to be liable for reimbursement,  it needs to take an appropriate level of  care when an APP fraud has been taking place, and argue that   this  code does not  clarify (i)  the level of care needed,  and (ii) how to verify that the victim has in fact taken adequate steps. Our PwDR  can be considered as a remedy for the above issue.   Moreover, Kjorven \cite{kjorven2020pays} investigates whether banks or customers should be liable for customers' financial loss to online  frauds including APP ones under Scandinavian and European law. The author highlights  that consumers are often left to deal with the losses caused by APP frauds. She concludes that this should change and  a larger portion of the losses should be allocated to financial institutions.


\subsection{Dispute Resolution}


\subsection{Central Bank Digital Currency}

Now, we briefly discuss a different but related area; namely, ``Central Bank Digital Currency" (CBDC). Due to the growing interest in digital payments, some central banks around the world have been exploring or even piloting the idea of CBDC \cite{CBDC}.  The idea behind CBDC is that a central bank issues digital money/token (i.e., a representation of banknotes and coins) to the public where this digital money is regulated by the nation's monetary authority or central bank, similar to regular fiat currencies. CBDC can offer various features such as efficiency, transparency, programmable money, transactions' traceability, or financial inclusion to name a few \cite{CBDC,CBDC-core-features}. Researchers have already discussed that (in CBDC) users transactions' privacy and regulatory oversight can coexist, e.g., in \cite{abs-2103-00254,WustKCC19}. In such a setting, users' amount of payments and even with whom they transact can remain confidential, while  various regulations can be accurately encoded  into cryptographic algorithms (and ultimately into a software) which are executed on users' transaction history by  authorities, e.g., to ensure compliance with ``Anti-Money Laundering'' or ``Countering the Financing of Terrorism''.  CBDC is still in its infancy and has not been adopted by banks yet. The adoption of such a model requires fundamental changes to and further digitalisations of the current banking  infrastructure. Therefore, unlike CBDC which is more futuristic, the focus of our work is on the existing regular  online banking systems.  Nevertheless, when CBDC becomes mainstream, APP frauds  might happen to the CBDC users as well. In this case, (a variant of) our result can be integrated into  such a framework to protect  APP frauds victims.  





