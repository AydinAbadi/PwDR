% !TEX root =main.tex
\vspace{-3mm}
\section{More Discussion on Definition \ref{def-a::Security-against-malicious-victim}}\label{sec::Further-Discussion-on-Definition-Sec-against-victim}
\vspace{-1mm}

Definition  \ref{def-a::Security-against-malicious-victim}
%
 %aim to be as generic as possible (so they can capture randomised and deterministic protocols). Moreover, they 
 captures the case where $\mathtt{resDispute}(.)$ takes its inputs $\hat{\bm w}$ and  $pp$  from a smart contract $\mathcal{S}$. Since there is a (negligible) probability that a smart contract’s state can be tampered with, the definitions also capture such an event. 
 %
 Also, the definition aims to be as generic as possible, to capture the case where $\mathtt{resDispute}(.)$ can be either randomised/probabilistic or deterministic (in this paper, our PwDR that realises the definition relies on a deterministic dispute resolution algorithm). 
 %
 However, in a simpler case, where the parties send their messages directly to a  fully trusted party (and the probability of tampering with its state $0$) and/or $\mathtt{resDispute}(.)$ is deterministic, the definition can be slightly simplified. 












