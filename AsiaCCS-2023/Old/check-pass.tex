%!TEX root = main.tex

%\subsection{The PwDR Protocol's Subroutines}\label{sec::PwDR-Subroutines}


%In this section, we present a few subroutines that will be  called  by  the PwDR protocol. 



%\subsubsection{Determining Bank's Message Status.}
%
%  As we stated earlier, in the payment journey the customer may receive a ``pass'' message or even nothing at all, e.g., due to the system failure. In such cases,  a victim of  an APP fraud may complain that the pass or missing message should have been a warning one that  ultimately would have prevented the victim from falling into the APP fraud. To assist the committee members to deal with and process such complaints deterministically, we propose a process  called $\mathtt{verStat}(.)$ that outputs $1$ in the case where a pass message was given correctly or the missing message could not play prevent the scam, and output $0$, otherwise. The process is defined in figure \ref{fig:verstat}.
  
  
% !TEX root =main.tex


\subsection{A Subroutine for Determining Bank's Message Status}

As we stated earlier, in the payment journey the customer may receive a ``pass'' message or even nothing at all, e.g., due to a system failure. In such cases,  a victim of  an APP fraud must be able to complain that if the pass or missing message was   a warning message, then it   would have prevented the victim from falling to the APP fraud. To assist the committee members to deal with  such complaints deterministically, we propose an algorithm,  called $\mathtt{verStat}$. It is run locally by each committee member. It outputs $0$ if a pass message was given correctly or the missing message could not  prevent the scam, and outputs $1$ otherwise. The algorithm is presented in figure \ref{fig:verStat}.


\begin{figure}[!htbp]
\setlength{\fboxsep}{0.7pt}
\begin{center}
\begin{boxedminipage}{12.3cm}
\small{
\underline{$\mathtt{verStat}(add_{\st\mathcal{S}}, m^{\st\mathcal{(B)}},  \bm l, \Delta, aux)\rightarrow w_{\st 1}$}\\
%
\begin{itemize}
\item\noindent\textit{Input}. $add_{\st\mathcal{S}}$: the address of smart contract $\mathcal{S}$, $m^{\st\mathcal{(B)}}$:  $\mathcal{B}$'s warning message,  $\bm l$:  customer's payees' list, $\Delta$: a time parameter, and $aux$: auxiliary information, e.g., bank's policy. 
%
\item\noindent\textit{Output}. $ w_{\st 1}=0$: if the ``pass'' message had been given correctly or the missing message did not play any role in preventing the scam; $ w_{\st 1}=1$: otherwise. 
\end{itemize}
\begin{enumerate}
\item reads the content of   $\mathcal{S}$. It checks if $m^{\st\mathcal{(B)}}=$``pass''  or the encrypted warning message was not sent on time (i.e., never sent or sent after    $t_{\st 0}+\Delta$).  If one of the checks passes, it proceeds to the next step. Otherwise, it aborts. 
\item checks the validity of  customer's most recent payees' list $\bm l$, with the help of the auxiliary information, $aux$. 
\begin{itemize}
\item[$\bullet$]  if $\bm l$ contains an invalid element,  it sets $ w_{\st 1}=1$.
\item [$\bullet$] otherwise, it sets $ w_{\st 1}=0$.
\end{itemize}
\item returns $ w_{\st 1}$.

\

\end{enumerate}

}
\end{boxedminipage}
\end{center}
\caption{Algorithm to Determine a Bank's Message Status} 
\label{fig:verStat}
\end{figure}
%%%%%%%%%%%%%%%%%%%%%%%%%%%%%%%%%%%%%%

  % !TEX root =main.tex

\vspace{-4mm}

\subsection{A Subroutine for Checking a Warning's Effectiveness}
\vspace{-1mm}


To help the auditors deterministically compile a victim's complaint about a warning's effectiveness, we propose an algorithm,  called $\mathtt{checkWarning}(.)$ which is run locally by each auditor. It also allows the victims to provide (to the auditors) a certificate/evidence as part of their complaints.   This algorithm is presented in Figure \ref{fig:checkWarning}.

%The algorithm outputs a pair $(w_{\st 2}, w_{\st 3})$. It  sets $w_{\st 2}=0$  if the given warning message is effective, and sets $w_{\st 2}=1$, if it is not. It sets $w_{\st 3}=1$ if the certificate that the victim provided is valid (or empty) and sets $w_{\st 3}=0$ if it is invalid.  



\vspace{-1mm}
\begin{figure}[!htbp]
\setlength{\fboxsep}{.9pt}
\begin{center}
    \begin{tcolorbox}[enhanced,width=81mm, height=78mm, left=0mm,
    drop fuzzy shadow southwest,
    colframe=black,colback=white]
    {\small{
    \vspace{-2.5mm}
 \underline{$\mathtt{checkWarning}(add_{\st\mathcal{S}}, z, m^{\st\mathcal{(B)}},  aux')\rightarrow (w_{\st 2},  w_{\st 3})$}\\
%
\vspace{-2.2mm}
\begin{itemize}[leftmargin=4.2mm]
%
\item \noindent\textit{Input}. $add_{\st\mathcal{S}}$: the address of smart contract $\mathcal{S}$, $z$:  $\mathcal{C}$'s complaint, $m^{\st\mathcal{(B)}}$:  $\mathcal{B}$'s warning message,  and $aux'$: auxiliary information, e.g., guideline on warnings' effectiveness. 
%
\item\noindent\textit{Output}. $w_{\st 2}=0$: if the given warning message is effective; $w_{\st 2}=1$: if the  warning message is ineffective. Also, $w_{\st 3}=1$: if the certificate in $z$ is valid or no certificate is provided; $ w_{\st 3}=0$: if the certificate is invalid. 
%
\end{itemize}
%
\begin{enumerate}[leftmargin=5.2mm]
%
\item parse $z= m||sig||pk||\text{``challenge warning''}$. If  $sig$ is  empty,  it  sets $w_{\st 3}=0$ and goes to  step \ref{check-m}. Otherwise, it:
%
\begin{enumerate} 
%
\item verifies the certificate: $\mathtt{Sig.ver}(pk, m, sig)\rightarrow h$. 
%
\item if  the certificate is rejected (i.e., $h=0$),  it sets $w_{\st 3}=0$. It  goes to step \ref{return}. 
%
\item otherwise (i.e., $h=1$), it sets $w_{\st 3}=1$ and moves onto the next step. 
\end{enumerate}
%
%\item  reads the content of $m$ in $\mathcal{S}$.
%
\item\label{check-m} checks if ``warning'' $\in m^{\st\mathcal{(B)}}$.  If the check is passed, it proceeds to the next step. Otherwise, it aborts. 
%
\item checks the warning's effectiveness, with the assistance of the evidence $m$ and auxiliary information $aux'$. 
%
\begin{itemize}
\item[$\bullet$]  if it is effective,  it sets $w_{\st 2} = 0$. Otherwise, it sets $w_{\st 2} = 1$.
%\item [$\bullet$] otherwise, it sets $w_{\st 2} = 1$.
\end{itemize}
\item\label{return} returns $(w_{\st 2}, w_{\st 3})$.
\vspace{-1mm}
\end{enumerate}
}   }
\end{tcolorbox}
\end{center}
\vspace{-3mm}
\caption{Algorithm to Check Warning's Effectiveness.} 
\vspace{-5mm}
\label{fig:checkWarning}
\end{figure}
%\vspace{-2mm}
%%%%%%%%%%%%%%%%%%%%%
%
%\begin{figure}[!htb]
%\setlength{\fboxsep}{0.7pt}
%\begin{center}
%\begin{boxedminipage}{13.3cm}
%\small{
%\underline{$\mathtt{checkWarning}(add_{\st\mathcal{S}}, z, m^{\st\mathcal{(B)}},  aux')\rightarrow (w_{\st 2},  w_{\st 3})$}\\
%%
%\begin{itemize}
%%
%\item \noindent\textit{Input}. $add_{\st\mathcal{S}}$: the address of smart contract $\mathcal{S}$, $z$:  $\mathcal{C}$'s complaint, $m^{\st\mathcal{(B)}}$:  $\mathcal{B}$'s warning message,  and $aux'$: auxiliary information, e.g., guideline on warnings' effectiveness. 
%%
%\item\noindent\textit{Output}. $w_{\st 2}=0$: if the given warning message is effective; $w_{\st 2}=1$: if the  warning message is ineffective. Also, $w_{\st 3}=1$: if the certificate in $z$ is valid or no certificate is provided; $ w_{\st 3}=0$: if the certificate is invalid. 
%%
%\end{itemize}
%%
%\begin{enumerate}
%%
%\item parse $z= m||sig||pk||\text{``challenge warning''}$. If  $sig$ is  empty,  it  sets $w_{\st 3}=0$ and goes to  step \ref{check-m}. Otherwise, it:
%%
%\begin{enumerate} 
%%
%\item verifies the certificate: $\mathtt{Sig.ver}(pk, m, sig)\rightarrow h$. 
%%
%\item if  the certificate is rejected (i.e., $h=0$),  it sets $w_{\st 3}=0$. It  goes to step \ref{return}. 
%%
%\item otherwise (i.e., $h=1$), it sets $w_{\st 3}=1$ and moves onto the next step. 
%\end{enumerate}
%%
%%\item  reads the content of $m$ in $\mathcal{S}$.
%%
%\item\label{check-m} checks if ``warning'' $\in m^{\st\mathcal{(B)}}$.  If the check is passed, it proceeds to the next step. Otherwise, it aborts. 
%%
%\item checks the warning's effectiveness, with the assistance of the evidence $m$ and auxiliary information $aux'$. 
%%
%\begin{itemize}
%\item[$\bullet$]  if it is effective,  it sets $w_{\st 2} = 0$.
%\item [$\bullet$] otherwise, it sets $w_{\st 2} = 1$.
%\end{itemize}
%\item\label{return} returns $(w_{\st 2}, w_{\st 3})$.
%\vspace{1mm}
%
%\end{enumerate}
%}
%\end{boxedminipage}
%\end{center}
%\caption{Algorithm to Check Warning's Effectiveness} 
%\label{fig:checkWarning}
%\end{figure}
%%%%%%%%%%%%%%%%%%%%%%%%%%%%%%%%%%%%%%

  % !TEX root =main.tex

\subsubsection{Verdict Encoding-Decoding Protocols.}


Here, we present two efficient verdict encoding and decoding protocols; namely, Private Verdict Encoding (PVE) and Final Verdict Decoding (FVD) protocols. Their goal is to let a third party $\mathcal{I}$, e.g., $\mathcal{DR}$, find out whether at least one arbiter voted $1$, while satisfying the following  requirements.  The protocols should (1) generate unlinkable verdicts, (2)  not require arbiters to interact with each other for each customer, and (3) be  efficient. Since, the second and third requirements are self-explanatory,  we only explain the first one.  Informally, the first property requires  that the protocols should generate encoded verdicts and final verdict in a way that $\mathcal{I}$,  given the encoded verdicts and final verdict, should not be able to (a)   link a  verdict to an arbiter (except when all arbiters' verdicts are $0$), and (b) find out the total number of $1$ or $0$ verdicts when they provide different verdicts. 



 At a high level, the protocols work as follows.  The arbiters only once for all customers agree on a secret key of a pseudorandom function. This key will allow each of them to generate a pseudorandom masking values such that if all masking values are ``XOR''ed, they would cancel out each other and result $0$.\footnote{This is similar to the idea used in the XOR-based secret sharing \cite{Schneier0078909}.}
 
 
 
 
 
Each arbiter represents its verdict by (i) representing it as a parameter which is set to either $0$ if the verdict is $0$ or to a random value if the verdict is $1$, and then (ii) masking this parameter by the above  pseudorandom value.  It sends the result to $\mathcal{I}$.  To decode the final verdict and find out whether any arbiter voted $1$, $\mathcal{I}$  does XOR all encoded verdicts. This removes the masks and XORs are verdicts' representations.  If the result is $0$, then    all arbiters must have voted $0$; therefore,  the final verdict is $0$. However, if the result is not $0$ (i.e., a random value), then at least one of the arbiters voted $1$, so  the final verdict is $1$. We present the encoding  and decoding protocols in figures \ref{fig:PVE} and \ref{fig:FVD} respectively.
 
 
 Not that the protocols' correctness holds, except  a negligible  probability. In particular, if two arbiters  represent their verdict by an identical random value, then when they are XORed they would cannel out each other which can affect the result's correctness. The same holds if the XOR of  multiple verdicts' representations results in a value that can cancel out another verdict's representation. Nevertheless, the probability that such an event occurs is negligible in the security parameter, i.e., at most   $\frac{1}{2^{\st \lambda}}$. It is evident that PVE and FVD protocols meet properties (2) and (3). The primary reason they also meet  property (1) is that each masked verdict reveals nothing about the verdict (and its representation) and  given the final verdict, $\mathcal{I}$ cannot distinguish between the case where there is exactly one arbiter that voted  $1$ and the case where multiple arbiters voted $1$, as in both cases $\mathcal{I}$   extracts only a single random value, which reveals nothing about the number of arbiters which voted $0$ or $1$. 
 
%  
% To encode a verdict $w$, each arbiter represents it as a polynomial. It randomises this polynomial and then  masks this polynomial with the pseudorandom masking polynomial. It sends the result to $\mathcal{I}$. To decode the final verdict and find out whether all arbiters agreed on the same verdict, i.e., unanimous decision,  $\mathcal{I}$  adds all polynomials up. This removes the masks. Next, it  evaluates the result polynomial at $v=1$ and $v=0$. It considers $v$ as the final verdict if the evaluation is  $0$. We present the encoding  and decoding protocols in figures \ref{fig:PVE} and \ref{fig:FVD} respectively. 
 



% either an specific final verdict  (i.e., $v=0$ or $v=1$) if  all arbiters' verdicts are identical, or  nothing  about the arbiters' inputs if they did not agree on the same specific verdict. The protocols are  primarily  based on ``zero-sum pseudorandom polynomials'' and the techniques often used by private set intersection (PSI) protocols. In particular, the arbiters only once for all customers agree on a secret key of a pseudorandom function. This key will allow each of them to generate a pseudorandom masking polynomial such that if all masking polynomials are summed up, they would cancel out each other and result $0$, i.e., zero-sum pseudorandom polynomials. 



%In this section, we present efficient (verdict) encoding and decoding protocols. The encoding protocol  lets each of the $n$ honest arbiters $\mathcal{D}:\{\mathcal{D}_{\st 1},..., \mathcal{D}_{\st n}\}$ non-interactively encode its verdict such that a third party party $\mathcal{I}$ (where $\mathcal{I}\notin \mathcal{D}$) can extract final verdict if  all arbiters' verdicts are identical with the following security requirements. First,  given individual encoded verdict, $\mathcal{I}$ cannot learn anything about each arbiter's verdict. Second, can find out only final verdict if  all arbiters' verdicts are identical; otherwise, it cannot learn each individual arbiter's verdict.  The decoding protocol lets  $\mathcal{I}$,  combines the encoded verdicts and learn either an specific final verdict  (i.e., $w=0$ or $w=1$) if  all arbiters' verdicts are identical, or  nothing  about the arbiters' inputs if they did not agree on the same specific verdict. The protocols are  primarily  based on ``zero-sum pseudorandom polynomials'' and the techniques often used by private set intersection (PSI) protocols. In particular, the arbiters only once for all customers agree on a secret key of a pseudorandom function. This key will allow each of them to generate a pseudorandom masking polynomial such that if all masking polynomials are summed up, they would cancel out each other and result $0$, i.e., zero-sum pseudorandom polynomials. 











%
%Below, we present ``Zero-sum Pseudorandom Values Generator'' (ZPVG), an algorithm that allows each of the $n$ arbiters  to \emph{efficiently} and \emph{independently}  generate a vector of $m$ pseudorandom values for each customer, such that when all arbiters' vectors are summed up component-wise, it would result in a vector of $m$ zeros. ZPVG is based on  the following idea. Each arbiter $\mathcal{D}_{\st j}\in\{\mathcal{D}_{\st 1},..., \mathcal{D}_{\st n-1}\}$  uses the secret key $\bar k_{\st 0}$ (as defined in Section \ref{Notations-and-Assumptions}), along with the customer's unique ID (e.g., it blockchain account's address) as the inputs of $\mathtt{PRF}(.)$  to derive $m$ pseudorandom values. However,  $\mathcal{D}_{\st n}$ generates each  $j$-th element of  its vector by computing the additive inverse of the sum of the $j$-th elements that the rest of arbiters generated. Even $\mathcal{D}_{\st n}$ does not need to interact with other arbiters, as it can regenerate their values too. Moreover, the arbiters do not need to interact with each other for every   new customer and can   reuse the same key because the new customer would have a new unique ID that would result in a fresh set of pseudorandom values (with a high probability). Figure \ref{fig:ZSPA} presents ZPVG in more detail. 





%It sends the result to all parties which can locally check if the relation holds. Note, in the literature,   there exist protocols that allows parties to agree on zero-sum pseudorandom values, e.g., in \cite{}; however, they are less efficient than $\mathtt{ZSPA}$, as their security requirements are different than ours, i.e., they assume the participants are malicious whereas we assume the arbiters who participate in this protocol  are honest. 


%Next, it commits to each value, where it uses $k_{\st 2}$ to generate the randomness of each commitment. Then, it constructs a Merkel tree on top of the commitments and  stores only the root of the tree  and the hash value of the keys (so in total three values) in  the smart contract.  Then, each party (using the keys) locally checks if the values and commitments have been constructed correctly; if so, each sends  an ``approved" message to the contract. 
%
%
%
%Informally, there are three main security requirements that $\mathtt{ZSPA}$ must meet: (a) privacy, (b)  non-refutability, and (c) indistinguishability. Privacy here means given the state of the  contract, an external party cannot learn any information about any of the (pseudorandom) values:  $z_{\st j}$; while non-refutability  means that if a party sends ``approved" then in future cannot deny the knowledge  of the values whose representation is stored in the contract. Furthermore, indistinguishability means that every $z_{\st j}$ ($1\leq j \leq m$) should be indistinguishable from a truly random value. In Fig. \ref{fig:ZSPA}, we provide $\mathtt{ZSPA}$ that efficiently generates $b$ vectors  where each vector elements is sum to zero. 



%\begin{figure}[ht]
%\setlength{\fboxsep}{0.7pt}
%\begin{center}
%\begin{boxedminipage}{12.3cm}
%\small{
%$\mathtt{ZPVG}(\bar{k}_{\st 0}, \text{ID}, n,  m, j)\rightarrow \bm r_{\st j}$\\
%------------------
%\begin{itemize}
%\item \noindent\textit{Input.} $\bar{k}_{\st 0}$: a key of  pseudorandom function's key $\mathtt{PRF}(.)$, $\text{ID}$: a unique identifier, $n$:  total number of rows of a matrix, and $m$: total number of columns of a matrix, $j$: a row's index in a matrix.
%%
%\item \noindent\textit{Output.} $\bm r_{\st j}$:  $j$-th row of an $n\times m$ matrix, such that  if $i$-th element of $\bm r_{\st j}$ is added with  the rest of elements in $i$-th column of the same matrix, the result would be  $0$. 
%\end{itemize}
%\begin{enumerate}
%%
%\item\label{ZSPA:val-gen} compute $m$ pseudorandom values as follows. 
%
%$\forall i, 1\leq i\leq m:$
%%
%\begin{itemize}
%%
%\item[$\bullet$] if $j< n: r_{\st i,j}=\mathtt{PRF}(\bar k_{\st 0}, i||j||\text{ID})$ 
%%
%\item[$\bullet$]  if $j=n: r_{\st i, n}=\big(-\sum\limits^{\st n-1}_{\st j=1} r_{\st i,j}\big) \bmod p$ 
%%
%\end{itemize}
%%
%\item return $\bm r_{\st j}=[r_{\st 1,j},..., r_{\st m,j}]$
%
%
%
%
%\
% \end{enumerate}
% 
%}
%\end{boxedminipage}
%\end{center}
%\caption{Zero-sum Pseudorandom Values Generator (ZPVG)} 
%\label{fig:ZSPA}
%\end{figure}






\begin{figure}[!ht]
\setlength{\fboxsep}{0.7pt}
\begin{center}
\begin{boxedminipage}{12.3cm}
\small{
\underline{$\mathtt{PVE}(\bar{k}_{\st 0}, \text{ID},  w_{\st j}, o, n,  j)\rightarrow  \bar{  w}_{\st j}$}\\
%
\begin{itemize}
\item \noindent\textit{Input.} $\bar{k}_{\st 0}$: a key of  pseudorandom function $\mathtt{PRF}(.)$, $\text{ID}$: a unique identifier, $ w_{\st j}$: a  verdict, $o$: an offset, $n$: the total number of  arbiters,  and  $j$: an arbiter's index.
%
\item \noindent\textit{Output.} $\bar{  w}_{\st j}$:  an  encoded verdict.  
%
%$\bm r_{\st j}$:  $j$-th row of an $n\times m$ matrix, such that  if $i$-th element of $\bm r_{\st j}$ is added with  the rest of elements in $i$-th column of the same matrix, the result would be  $0$. 
\end{itemize}
Arbiter $\mathcal{D}_{\st j}$ takes the following steps.
\begin{enumerate}
%
\item\label{ZSPA:val-gen} computes a  pseudorandom  value,  as follows. 
%
%$\forall i,1\leq i\leq s:$
%
\begin{itemize}
%
\item[$\bullet$]$ \text{ if } j< n: r_{\st j}=\mathtt{PRF}(\bar k_{\st 0}, o||j||\text{ID})$.\\
%
%\hspace{1mm} 
\item [$\bullet$] $ \text{ if } j=n: r_{\st j}= \bigoplus\limits^{\st n-1}_{\st j=1} r_{\st j}$.
%
\end{itemize}
%By the end of this phase, a random polynomial of the following form is generated, $\Psi_{\st j}=r_{\st i+2,j}\cdot x^{\st 2}+r_{\st i+1,j}\cdot x+r_{\st i,j}$.
%
\item  sets a fresh parameter, $w'_{\st j}$, as below. 
%
%$\forall i,1\leq i\leq s:$

%\begin{itemize}
%\item[$\bullet$]  $\text{ if } w_{\st j}=1:$ \text{\ sets \ } $w'_{\st j}= \alpha_{\st j}$, where $\alpha_{\st j}\stackrel{\st\$}\leftarrow \mathbb{F}_{\st p}$.
%
%\item [$\bullet$] $\text{ if } w_{\st j}=0: \text{\ sets \ } w'_{\st j}= 0$.
%
%\end{itemize}
\begin{equation*}
   w'_{\st j}= 
\begin{cases}
   0,              & \text{if } w_{\st j}=0\\
   \alpha_{\st j}\stackrel{\st\$}\leftarrow \mathbb{F}_{\st p} ,& \text{if } w_{\st j}=1\\

    %0,              & \text{if } w_{\st j}=0
\end{cases}
\end{equation*}
%
\item encodes  $w'_{\st j}$ as follows. %$\forall i,1\leq i\leq s:$
%
$\bar w_{\st j}= w'_{\st j}\oplus r_{\st j}$.
%
\item outputs $\bar{ w}_{\st j}$.





%generates   a polynomial that encodes the verdict, i.e., $\Omega_{\st j}=(x-w)$.  
%
%\item multiplies the polynomial by a fresh random polynomial $\Phi_{\st j}$ of degree $1$ and adds the result with $\Psi_{\st j}$, i.e.,  $\bar\Omega_{\st j}=\Phi_{\st j}\cdot \Omega_{\st j}+\Psi_{\st j}\bmod p$. 
%
%
%\item  evaluates the result  polynomial, $\bar\Omega_{\st j}$, at every  element $x_{\scriptscriptstyle i}\in {\bm{x}}$. This yields a vector of   $y$-coordinates: $[ \bar w_{\st 1,j},..., \bar w_{\st 3,j}]$.
%%
%
%%
%\item return $\bar{\bm w}_{\st j}=[ \bar w_{\st 1,j},..., \bar w_{\st 3,j}]$.




\
 \end{enumerate}
 
}
\end{boxedminipage}
\end{center}
\caption{Private Verdict Encoding  (PVE) Protocol} 
\label{fig:PVE}
\end{figure}
%%%%%%%%%%%%%%%%%%%%%%%%%%%%%%%%%%%%%%
%
\begin{figure}[!ht]
\setlength{\fboxsep}{0.7pt}
\begin{center}
\begin{boxedminipage}{12.3cm}
\small{
\underline{$\mathtt{FVD}(n,  \bar{\bm w})\rightarrow  v$}\\
%
\begin{itemize}
\item \noindent\textit{Input.} $n$:  the total number of  arbiters,  and  $\bar{\bm w}=[\bar{ w}_{\st 1},..., \bar{ w}_{\st n}]$:  a vector of all arbiters' encodes  verdicts.
%
\item \noindent\textit{Output.} $v$: final verdict.  
%
%$\bm r_{\st j}$:  $j$-th row of an $n\times m$ matrix, such that  if $i$-th element of $\bm r_{\st j}$ is added with  the rest of elements in $i$-th column of the same matrix, the result would be  $0$. 
\end{itemize}
A third-party $\mathcal{I}$ takes the following steps.
\begin{enumerate}
%
%
\item combines  all arbiters' encoded verdicts, $\bar w_{\st j}\in \bar{\bm {w}}$, as follows. 
%
$c= \bigoplus\limits^{\st n}_{\st j=1} \bar w_{\st j}$
% $$\forall i, 1\leq i\leq 3: g_{\st i}=\sum\limits^{\st n}_{\st j=1} \bar{w}_{\st i,j} \bmod p$$
%
%\item if $n$ is odd, then sets $c=c\oplus 1$. 
%
\item sets the final verdict $v$ depending on the content of $c$. Specifically, 
%
%\begin{itemize}
%%
%\item[$\bullet$] if $c=0$, sets $v=0$.
%%
%\item[$\bullet$]  otherwise, sets $v=1$.
%%
%\end{itemize}
\begin{equation*}
   v= 
\begin{cases}
    0,              &\text{if } c= 0\\
   1 ,& \text{otherwise }\\

\end{cases}
\end{equation*}
%
\item outputs  $v$. 

\
 \end{enumerate}
 
}
\end{boxedminipage}
\end{center}
\caption{Final Verdict Decoding  (FVD) Protocol} 
\label{fig:FVD}
\end{figure}


  
  
  
  
  
  
  


%In this section, we present a set of subroutines that are  called  by  and help each arbiter to  processes a customer's complaint.  As we stated earlier, in the payment journey the customer may receive a ``pass'', or warning message (or nothing at all). In the PwDR protocol, when a customer has fallen to an APP scam, it is allowed to challenge the validity of these messages. In particular, it can complain that (a) the pass message should have been a warning message  or the bank has not provided any message  and if it provided a warning then the scam would be prevented, or (b) the warning message provided by the bank was ineffective. To let the arbiter deal with the above situations and reach a verdict we define two protocols $\mathtt{verStat}(.)$ and $\mathtt{checkWarning}(.)$, presented below. 
%
%\small{
%\begin{center}
%\fbox{\begin{minipage}{5 in}
%%\caption{Private Verdict Encoding  (PVE) Protocol} 
%\underline{$\mathtt{verStat}(add_{\st\mathcal{S}}, m^{\st\mathcal{(B)}},  \bm l, \Delta, aux)\rightarrow w_{\st 1}$}\\
%%
%\begin{itemize}
%\item\noindent\textit{Input}. $add_{\st\mathcal{S}}$: the address of smart contract $\mathcal{S}$, $m^{\st\mathcal{(B)}}$:  $\mathcal{B}$'s warning message,  $\bm l$:  customer's payees' list, $\Delta$: a time parameter, and $aux$: auxiliary information, e.g., bank's policy. 
%%
%\item\noindent\textit{Output}. $ w_{\st 1}=1$: if the ``pass'' message had been given correctly or the missing message did not play any role in preventing the scam; $ w_{\st 1}=0$: otherwise. 
%\end{itemize}
%\begin{enumerate}
%\item reads the content of   $\mathcal{S}$. It checks if $m^{\st\mathcal{(B)}}=$``pass''  or the encrypted warning message was not sent on time (i.e., never sent or sent after    $t_{\st 0}+\Delta$).  If one of the checks passes, it proceeds to the next step. Otherwise, it aborts. 
%\item checks the validity of  customer's most recent payees' list $\bm l$, with the help of the auxiliary information, $aux$. 
%\begin{itemize}
%\item[$\bullet$]  if $\bm l$ contains an invalid element,  it sets $ w_{\st 1}=0$.
%\item [$\bullet$] otherwise, it sets $ w_{\st 1}=1$.
%\end{itemize}
%\item returns $ w_{\st 1}$.
%\end{enumerate}
%\label{fig:verstat}
%\end{minipage}}
%\end{center}
%}

%%\small{
%\begin{center}
%\fbox{\begin{minipage}{5 in}
%$\mathtt{verPass}(add_{\st\mathcal{S}}, m, l, \text{aux})\rightarrow d\in\{0,1\}$\\
%------------------
%\
%
%\noindent\textbf{Input}. $add_{\st\mathcal{S}}$: the address of smart contract $\mathcal{S}$, $m$: $\mathcal{S}$'s field that corresponds to $\mathcal{B}$'s message, $l$: $\mathcal{S}$'s field related to customer's payees' list, and \text{aux}: auxiliary information, e.g., bank's policy. 
%
%\noindent\textbf{Output}. $d=1$: if the ``pass'' message had been given correctly; $d=0$: otherwise. 
%\begin{enumerate}
%\item reads the content of  $m$ in $\mathcal{S}$.
%\item checks if $m=$``pass''. If the check is passed, it proceeds to the next step. Otherwise, it aborts. 
%\item checks the validity of  customer's most recent payees' list $l$ on  $\mathcal{S}$, with the help of the auxiliary information $\text{aux}$. 
%\begin{itemize}
%\item[$\bullet$]  if $l$ contains an invalid element,  it sets $d=0$.
%\item [$\bullet$] otherwise, it sets $d=1$.
%\end{itemize}
%\item returns $d$.
%\end{enumerate}
%\end{minipage}}
%\end{center}
%%}



%\small{
%\begin{center}
%\fbox{\begin{minipage}{5 in}
%\underline{$\mathtt{checkWarning}(add_{\st\mathcal{S}}, z, m^{\st\mathcal{(B)}},  aux')\rightarrow (w_{\st 2},  w_{\st 3})$}\\
%%
%\begin{itemize}
%%
%\item \noindent\textit{Input}. $add_{\st\mathcal{S}}$: the address of smart contract $\mathcal{S}$, $z$:  $\mathcal{C}$'s complaint, $m^{\st\mathcal{(B)}}$:  $\mathcal{B}$'s warning message,  and $aux'$: auxiliary information, e.g., guideline on warnings' effectiveness. 
%%
%\item\noindent\textit{Output}. $w_{\st 3}=1$: if the certificate in $z$ is valid or no certificate is provided; $ w_{\st 3}=0$: if the certificate is invalid. Also, 
%$w_{\st 2}=1$: if the given warning message is effective; $w_{\st 2}=0$: if the  warning message is ineffective.
%
%\end{itemize}
%
%
%\begin{enumerate}
%%
%\item parse $z= m||sig||pk||\text{``challenge warning''}$. If the certificate $sig$ is  empty, then it  sets $w_{\st 3}=1$ and proceeds to  step \ref{check-m}. Otherwise, it:
%%
%\begin{enumerate} 
%%
%\item verifies the certificate: $\mathtt{Sig.ver}(pk, m, sig)\rightarrow h$. 
%%
%\item if  the certificate is rejected (i.e., $h=0$),  it sets $w_{\st 3}=0$. It  goes to step \ref{return}. 
%%
%\item otherwise (i.e., $h=1$), it sets $w_{\st 3}=1$ and moves onto the next step. 
%\end{enumerate}
%%
%%\item  reads the content of $m$ in $\mathcal{S}$.
%%
%\item\label{check-m} checks if ``warning'' $\in m^{\st\mathcal{(B)}}$.  If the check is passed, it proceeds to the next step. Otherwise, it aborts. 
%%
%\item checks the warning's effectiveness, with the assistance of the evidence $m$ and auxiliary information $aux'$. 
%%
%\begin{itemize}
%\item[$\bullet$]  if it is effective,  it sets $w_{\st 2} = 1$.
%\item [$\bullet$] otherwise, it sets $w_{\st 2} = 0$.
%\end{itemize}
%\item\label{return} returns $(w_{\st 2}, w_{\st 3})$.
%\end{enumerate}
%\end{minipage}}
%\end{center}
%%}
%
%% !TEX root =main.tex

\subsubsection{Verdict Encoding-Decoding Protocols.}


Here, we present two efficient verdict encoding and decoding protocols; namely, Private Verdict Encoding (PVE) and Final Verdict Decoding (FVD) protocols. Their goal is to let a third party $\mathcal{I}$, e.g., $\mathcal{DR}$, find out whether at least one arbiter voted $1$, while satisfying the following  requirements.  The protocols should (1) generate unlinkable verdicts, (2)  not require arbiters to interact with each other for each customer, and (3) be  efficient. Since, the second and third requirements are self-explanatory,  we only explain the first one.  Informally, the first property requires  that the protocols should generate encoded verdicts and final verdict in a way that $\mathcal{I}$,  given the encoded verdicts and final verdict, should not be able to (a)   link a  verdict to an arbiter (except when all arbiters' verdicts are $0$), and (b) find out the total number of $1$ or $0$ verdicts when they provide different verdicts. 



 At a high level, the protocols work as follows.  The arbiters only once for all customers agree on a secret key of a pseudorandom function. This key will allow each of them to generate a pseudorandom masking values such that if all masking values are ``XOR''ed, they would cancel out each other and result $0$.\footnote{This is similar to the idea used in the XOR-based secret sharing \cite{Schneier0078909}.}
 
 
 
 
 
Each arbiter represents its verdict by (i) representing it as a parameter which is set to either $0$ if the verdict is $0$ or to a random value if the verdict is $1$, and then (ii) masking this parameter by the above  pseudorandom value.  It sends the result to $\mathcal{I}$.  To decode the final verdict and find out whether any arbiter voted $1$, $\mathcal{I}$  does XOR all encoded verdicts. This removes the masks and XORs are verdicts' representations.  If the result is $0$, then    all arbiters must have voted $0$; therefore,  the final verdict is $0$. However, if the result is not $0$ (i.e., a random value), then at least one of the arbiters voted $1$, so  the final verdict is $1$. We present the encoding  and decoding protocols in figures \ref{fig:PVE} and \ref{fig:FVD} respectively.
 
 
 Not that the protocols' correctness holds, except  a negligible  probability. In particular, if two arbiters  represent their verdict by an identical random value, then when they are XORed they would cannel out each other which can affect the result's correctness. The same holds if the XOR of  multiple verdicts' representations results in a value that can cancel out another verdict's representation. Nevertheless, the probability that such an event occurs is negligible in the security parameter, i.e., at most   $\frac{1}{2^{\st \lambda}}$. It is evident that PVE and FVD protocols meet properties (2) and (3). The primary reason they also meet  property (1) is that each masked verdict reveals nothing about the verdict (and its representation) and  given the final verdict, $\mathcal{I}$ cannot distinguish between the case where there is exactly one arbiter that voted  $1$ and the case where multiple arbiters voted $1$, as in both cases $\mathcal{I}$   extracts only a single random value, which reveals nothing about the number of arbiters which voted $0$ or $1$. 
 
%  
% To encode a verdict $w$, each arbiter represents it as a polynomial. It randomises this polynomial and then  masks this polynomial with the pseudorandom masking polynomial. It sends the result to $\mathcal{I}$. To decode the final verdict and find out whether all arbiters agreed on the same verdict, i.e., unanimous decision,  $\mathcal{I}$  adds all polynomials up. This removes the masks. Next, it  evaluates the result polynomial at $v=1$ and $v=0$. It considers $v$ as the final verdict if the evaluation is  $0$. We present the encoding  and decoding protocols in figures \ref{fig:PVE} and \ref{fig:FVD} respectively. 
 



% either an specific final verdict  (i.e., $v=0$ or $v=1$) if  all arbiters' verdicts are identical, or  nothing  about the arbiters' inputs if they did not agree on the same specific verdict. The protocols are  primarily  based on ``zero-sum pseudorandom polynomials'' and the techniques often used by private set intersection (PSI) protocols. In particular, the arbiters only once for all customers agree on a secret key of a pseudorandom function. This key will allow each of them to generate a pseudorandom masking polynomial such that if all masking polynomials are summed up, they would cancel out each other and result $0$, i.e., zero-sum pseudorandom polynomials. 



%In this section, we present efficient (verdict) encoding and decoding protocols. The encoding protocol  lets each of the $n$ honest arbiters $\mathcal{D}:\{\mathcal{D}_{\st 1},..., \mathcal{D}_{\st n}\}$ non-interactively encode its verdict such that a third party party $\mathcal{I}$ (where $\mathcal{I}\notin \mathcal{D}$) can extract final verdict if  all arbiters' verdicts are identical with the following security requirements. First,  given individual encoded verdict, $\mathcal{I}$ cannot learn anything about each arbiter's verdict. Second, can find out only final verdict if  all arbiters' verdicts are identical; otherwise, it cannot learn each individual arbiter's verdict.  The decoding protocol lets  $\mathcal{I}$,  combines the encoded verdicts and learn either an specific final verdict  (i.e., $w=0$ or $w=1$) if  all arbiters' verdicts are identical, or  nothing  about the arbiters' inputs if they did not agree on the same specific verdict. The protocols are  primarily  based on ``zero-sum pseudorandom polynomials'' and the techniques often used by private set intersection (PSI) protocols. In particular, the arbiters only once for all customers agree on a secret key of a pseudorandom function. This key will allow each of them to generate a pseudorandom masking polynomial such that if all masking polynomials are summed up, they would cancel out each other and result $0$, i.e., zero-sum pseudorandom polynomials. 











%
%Below, we present ``Zero-sum Pseudorandom Values Generator'' (ZPVG), an algorithm that allows each of the $n$ arbiters  to \emph{efficiently} and \emph{independently}  generate a vector of $m$ pseudorandom values for each customer, such that when all arbiters' vectors are summed up component-wise, it would result in a vector of $m$ zeros. ZPVG is based on  the following idea. Each arbiter $\mathcal{D}_{\st j}\in\{\mathcal{D}_{\st 1},..., \mathcal{D}_{\st n-1}\}$  uses the secret key $\bar k_{\st 0}$ (as defined in Section \ref{Notations-and-Assumptions}), along with the customer's unique ID (e.g., it blockchain account's address) as the inputs of $\mathtt{PRF}(.)$  to derive $m$ pseudorandom values. However,  $\mathcal{D}_{\st n}$ generates each  $j$-th element of  its vector by computing the additive inverse of the sum of the $j$-th elements that the rest of arbiters generated. Even $\mathcal{D}_{\st n}$ does not need to interact with other arbiters, as it can regenerate their values too. Moreover, the arbiters do not need to interact with each other for every   new customer and can   reuse the same key because the new customer would have a new unique ID that would result in a fresh set of pseudorandom values (with a high probability). Figure \ref{fig:ZSPA} presents ZPVG in more detail. 





%It sends the result to all parties which can locally check if the relation holds. Note, in the literature,   there exist protocols that allows parties to agree on zero-sum pseudorandom values, e.g., in \cite{}; however, they are less efficient than $\mathtt{ZSPA}$, as their security requirements are different than ours, i.e., they assume the participants are malicious whereas we assume the arbiters who participate in this protocol  are honest. 


%Next, it commits to each value, where it uses $k_{\st 2}$ to generate the randomness of each commitment. Then, it constructs a Merkel tree on top of the commitments and  stores only the root of the tree  and the hash value of the keys (so in total three values) in  the smart contract.  Then, each party (using the keys) locally checks if the values and commitments have been constructed correctly; if so, each sends  an ``approved" message to the contract. 
%
%
%
%Informally, there are three main security requirements that $\mathtt{ZSPA}$ must meet: (a) privacy, (b)  non-refutability, and (c) indistinguishability. Privacy here means given the state of the  contract, an external party cannot learn any information about any of the (pseudorandom) values:  $z_{\st j}$; while non-refutability  means that if a party sends ``approved" then in future cannot deny the knowledge  of the values whose representation is stored in the contract. Furthermore, indistinguishability means that every $z_{\st j}$ ($1\leq j \leq m$) should be indistinguishable from a truly random value. In Fig. \ref{fig:ZSPA}, we provide $\mathtt{ZSPA}$ that efficiently generates $b$ vectors  where each vector elements is sum to zero. 



%\begin{figure}[ht]
%\setlength{\fboxsep}{0.7pt}
%\begin{center}
%\begin{boxedminipage}{12.3cm}
%\small{
%$\mathtt{ZPVG}(\bar{k}_{\st 0}, \text{ID}, n,  m, j)\rightarrow \bm r_{\st j}$\\
%------------------
%\begin{itemize}
%\item \noindent\textit{Input.} $\bar{k}_{\st 0}$: a key of  pseudorandom function's key $\mathtt{PRF}(.)$, $\text{ID}$: a unique identifier, $n$:  total number of rows of a matrix, and $m$: total number of columns of a matrix, $j$: a row's index in a matrix.
%%
%\item \noindent\textit{Output.} $\bm r_{\st j}$:  $j$-th row of an $n\times m$ matrix, such that  if $i$-th element of $\bm r_{\st j}$ is added with  the rest of elements in $i$-th column of the same matrix, the result would be  $0$. 
%\end{itemize}
%\begin{enumerate}
%%
%\item\label{ZSPA:val-gen} compute $m$ pseudorandom values as follows. 
%
%$\forall i, 1\leq i\leq m:$
%%
%\begin{itemize}
%%
%\item[$\bullet$] if $j< n: r_{\st i,j}=\mathtt{PRF}(\bar k_{\st 0}, i||j||\text{ID})$ 
%%
%\item[$\bullet$]  if $j=n: r_{\st i, n}=\big(-\sum\limits^{\st n-1}_{\st j=1} r_{\st i,j}\big) \bmod p$ 
%%
%\end{itemize}
%%
%\item return $\bm r_{\st j}=[r_{\st 1,j},..., r_{\st m,j}]$
%
%
%
%
%\
% \end{enumerate}
% 
%}
%\end{boxedminipage}
%\end{center}
%\caption{Zero-sum Pseudorandom Values Generator (ZPVG)} 
%\label{fig:ZSPA}
%\end{figure}






\begin{figure}[!ht]
\setlength{\fboxsep}{0.7pt}
\begin{center}
\begin{boxedminipage}{12.3cm}
\small{
\underline{$\mathtt{PVE}(\bar{k}_{\st 0}, \text{ID},  w_{\st j}, o, n,  j)\rightarrow  \bar{  w}_{\st j}$}\\
%
\begin{itemize}
\item \noindent\textit{Input.} $\bar{k}_{\st 0}$: a key of  pseudorandom function $\mathtt{PRF}(.)$, $\text{ID}$: a unique identifier, $ w_{\st j}$: a  verdict, $o$: an offset, $n$: the total number of  arbiters,  and  $j$: an arbiter's index.
%
\item \noindent\textit{Output.} $\bar{  w}_{\st j}$:  an  encoded verdict.  
%
%$\bm r_{\st j}$:  $j$-th row of an $n\times m$ matrix, such that  if $i$-th element of $\bm r_{\st j}$ is added with  the rest of elements in $i$-th column of the same matrix, the result would be  $0$. 
\end{itemize}
Arbiter $\mathcal{D}_{\st j}$ takes the following steps.
\begin{enumerate}
%
\item\label{ZSPA:val-gen} computes a  pseudorandom  value,  as follows. 
%
%$\forall i,1\leq i\leq s:$
%
\begin{itemize}
%
\item[$\bullet$]$ \text{ if } j< n: r_{\st j}=\mathtt{PRF}(\bar k_{\st 0}, o||j||\text{ID})$.\\
%
%\hspace{1mm} 
\item [$\bullet$] $ \text{ if } j=n: r_{\st j}= \bigoplus\limits^{\st n-1}_{\st j=1} r_{\st j}$.
%
\end{itemize}
%By the end of this phase, a random polynomial of the following form is generated, $\Psi_{\st j}=r_{\st i+2,j}\cdot x^{\st 2}+r_{\st i+1,j}\cdot x+r_{\st i,j}$.
%
\item  sets a fresh parameter, $w'_{\st j}$, as below. 
%
%$\forall i,1\leq i\leq s:$

%\begin{itemize}
%\item[$\bullet$]  $\text{ if } w_{\st j}=1:$ \text{\ sets \ } $w'_{\st j}= \alpha_{\st j}$, where $\alpha_{\st j}\stackrel{\st\$}\leftarrow \mathbb{F}_{\st p}$.
%
%\item [$\bullet$] $\text{ if } w_{\st j}=0: \text{\ sets \ } w'_{\st j}= 0$.
%
%\end{itemize}
\begin{equation*}
   w'_{\st j}= 
\begin{cases}
   0,              & \text{if } w_{\st j}=0\\
   \alpha_{\st j}\stackrel{\st\$}\leftarrow \mathbb{F}_{\st p} ,& \text{if } w_{\st j}=1\\

    %0,              & \text{if } w_{\st j}=0
\end{cases}
\end{equation*}
%
\item encodes  $w'_{\st j}$ as follows. %$\forall i,1\leq i\leq s:$
%
$\bar w_{\st j}= w'_{\st j}\oplus r_{\st j}$.
%
\item outputs $\bar{ w}_{\st j}$.





%generates   a polynomial that encodes the verdict, i.e., $\Omega_{\st j}=(x-w)$.  
%
%\item multiplies the polynomial by a fresh random polynomial $\Phi_{\st j}$ of degree $1$ and adds the result with $\Psi_{\st j}$, i.e.,  $\bar\Omega_{\st j}=\Phi_{\st j}\cdot \Omega_{\st j}+\Psi_{\st j}\bmod p$. 
%
%
%\item  evaluates the result  polynomial, $\bar\Omega_{\st j}$, at every  element $x_{\scriptscriptstyle i}\in {\bm{x}}$. This yields a vector of   $y$-coordinates: $[ \bar w_{\st 1,j},..., \bar w_{\st 3,j}]$.
%%
%
%%
%\item return $\bar{\bm w}_{\st j}=[ \bar w_{\st 1,j},..., \bar w_{\st 3,j}]$.




\
 \end{enumerate}
 
}
\end{boxedminipage}
\end{center}
\caption{Private Verdict Encoding  (PVE) Protocol} 
\label{fig:PVE}
\end{figure}
%%%%%%%%%%%%%%%%%%%%%%%%%%%%%%%%%%%%%%
%
\begin{figure}[!ht]
\setlength{\fboxsep}{0.7pt}
\begin{center}
\begin{boxedminipage}{12.3cm}
\small{
\underline{$\mathtt{FVD}(n,  \bar{\bm w})\rightarrow  v$}\\
%
\begin{itemize}
\item \noindent\textit{Input.} $n$:  the total number of  arbiters,  and  $\bar{\bm w}=[\bar{ w}_{\st 1},..., \bar{ w}_{\st n}]$:  a vector of all arbiters' encodes  verdicts.
%
\item \noindent\textit{Output.} $v$: final verdict.  
%
%$\bm r_{\st j}$:  $j$-th row of an $n\times m$ matrix, such that  if $i$-th element of $\bm r_{\st j}$ is added with  the rest of elements in $i$-th column of the same matrix, the result would be  $0$. 
\end{itemize}
A third-party $\mathcal{I}$ takes the following steps.
\begin{enumerate}
%
%
\item combines  all arbiters' encoded verdicts, $\bar w_{\st j}\in \bar{\bm {w}}$, as follows. 
%
$c= \bigoplus\limits^{\st n}_{\st j=1} \bar w_{\st j}$
% $$\forall i, 1\leq i\leq 3: g_{\st i}=\sum\limits^{\st n}_{\st j=1} \bar{w}_{\st i,j} \bmod p$$
%
%\item if $n$ is odd, then sets $c=c\oplus 1$. 
%
\item sets the final verdict $v$ depending on the content of $c$. Specifically, 
%
%\begin{itemize}
%%
%\item[$\bullet$] if $c=0$, sets $v=0$.
%%
%\item[$\bullet$]  otherwise, sets $v=1$.
%%
%\end{itemize}
\begin{equation*}
   v= 
\begin{cases}
    0,              &\text{if } c= 0\\
   1 ,& \text{otherwise }\\

\end{cases}
\end{equation*}
%
\item outputs  $v$. 

\
 \end{enumerate}
 
}
\end{boxedminipage}
\end{center}
\caption{Final Verdict Decoding  (FVD) Protocol} 
\label{fig:FVD}
\end{figure}


%% !TEX root =main.tex

\subsubsection{Verdict Encoding-Decoding Protocols.}


In this section, we present efficient (verdict) encoding and decoding protocols. The encoding protocol  lets each of the $n$ honest arbiters $\mathcal{D}:\{\mathcal{D}_{\st 1},..., \mathcal{D}_{\st n}\}$ non-interactively encode its verdict such that a third party party $\mathcal{I}$ (where $\mathcal{I}\notin \mathcal{D}$),  given individual encoded verdict,  cannot learn anything about each arbiter's verdict.  The decoding protocol lets  $\mathcal{I}$ combine the encoded verdicts and learn either an specific final verdict  (i.e., $v=0$ or $v=1$) if  all arbiters' verdicts are identical, or  nothing  about the arbiters' inputs if they did not agree on the same specific verdict. The protocols are  primarily  based on ``zero-sum pseudorandom polynomials'' and the techniques often used by private set intersection (PSI) protocols. In particular, the arbiters only once for all customers agree on a secret key of a pseudorandom function. This key will allow each of them to generate a pseudorandom masking polynomial such that if all masking polynomials are summed up, they would cancel out each other and result $0$, i.e., zero-sum pseudorandom polynomials. 



%In this section, we present efficient (verdict) encoding and decoding protocols. The encoding protocol  lets each of the $n$ honest arbiters $\mathcal{D}:\{\mathcal{D}_{\st 1},..., \mathcal{D}_{\st n}\}$ non-interactively encode its verdict such that a third party party $\mathcal{I}$ (where $\mathcal{I}\notin \mathcal{D}$) can extract final verdict if  all arbiters' verdicts are identical with the following security requirements. First,  given individual encoded verdict, $\mathcal{I}$ cannot learn anything about each arbiter's verdict. Second, can find out only final verdict if  all arbiters' verdicts are identical; otherwise, it cannot learn each individual arbiter's verdict.  The decoding protocol lets  $\mathcal{I}$,  combines the encoded verdicts and learn either an specific final verdict  (i.e., $w=0$ or $w=1$) if  all arbiters' verdicts are identical, or  nothing  about the arbiters' inputs if they did not agree on the same specific verdict. The protocols are  primarily  based on ``zero-sum pseudorandom polynomials'' and the techniques often used by private set intersection (PSI) protocols. In particular, the arbiters only once for all customers agree on a secret key of a pseudorandom function. This key will allow each of them to generate a pseudorandom masking polynomial such that if all masking polynomials are summed up, they would cancel out each other and result $0$, i.e., zero-sum pseudorandom polynomials. 

At a high level, the protocols work as follows. To encode a verdict $w$, each arbiter represents it as a polynomial. It randomises this polynomial and then  masks this polynomial with the pseudorandom masking polynomial. It sends the result to $\mathcal{I}$. To decode the final verdict and find out whether all arbiters agreed on the same verdict, i.e., unanimous decision,  $\mathcal{I}$  adds all polynomials up. This removes the masks. Next, it  evaluates the result polynomial at $v=1$ and $v=0$. It considers $v$ as the final verdict if the evaluation is  $0$. We present the encoding  and decoding protocols in figures \ref{fig:PVE} and \ref{fig:FVD} respectively. 









%
%Below, we present ``Zero-sum Pseudorandom Values Generator'' (ZPVG), an algorithm that allows each of the $n$ arbiters  to \emph{efficiently} and \emph{independently}  generate a vector of $m$ pseudorandom values for each customer, such that when all arbiters' vectors are summed up component-wise, it would result in a vector of $m$ zeros. ZPVG is based on  the following idea. Each arbiter $\mathcal{D}_{\st j}\in\{\mathcal{D}_{\st 1},..., \mathcal{D}_{\st n-1}\}$  uses the secret key $\bar k_{\st 0}$ (as defined in Section \ref{Notations-and-Assumptions}), along with the customer's unique ID (e.g., it blockchain account's address) as the inputs of $\mathtt{PRF}(.)$  to derive $m$ pseudorandom values. However,  $\mathcal{D}_{\st n}$ generates each  $j$-th element of  its vector by computing the additive inverse of the sum of the $j$-th elements that the rest of arbiters generated. Even $\mathcal{D}_{\st n}$ does not need to interact with other arbiters, as it can regenerate their values too. Moreover, the arbiters do not need to interact with each other for every   new customer and can   reuse the same key because the new customer would have a new unique ID that would result in a fresh set of pseudorandom values (with a high probability). Figure \ref{fig:ZSPA} presents ZPVG in more detail. 





%It sends the result to all parties which can locally check if the relation holds. Note, in the literature,   there exist protocols that allows parties to agree on zero-sum pseudorandom values, e.g., in \cite{}; however, they are less efficient than $\mathtt{ZSPA}$, as their security requirements are different than ours, i.e., they assume the participants are malicious whereas we assume the arbiters who participate in this protocol  are honest. 


%Next, it commits to each value, where it uses $k_{\st 2}$ to generate the randomness of each commitment. Then, it constructs a Merkel tree on top of the commitments and  stores only the root of the tree  and the hash value of the keys (so in total three values) in  the smart contract.  Then, each party (using the keys) locally checks if the values and commitments have been constructed correctly; if so, each sends  an ``approved" message to the contract. 
%
%
%
%Informally, there are three main security requirements that $\mathtt{ZSPA}$ must meet: (a) privacy, (b)  non-refutability, and (c) indistinguishability. Privacy here means given the state of the  contract, an external party cannot learn any information about any of the (pseudorandom) values:  $z_{\st j}$; while non-refutability  means that if a party sends ``approved" then in future cannot deny the knowledge  of the values whose representation is stored in the contract. Furthermore, indistinguishability means that every $z_{\st j}$ ($1\leq j \leq m$) should be indistinguishable from a truly random value. In Fig. \ref{fig:ZSPA}, we provide $\mathtt{ZSPA}$ that efficiently generates $b$ vectors  where each vector elements is sum to zero. 



%\begin{figure}[ht]
%\setlength{\fboxsep}{0.7pt}
%\begin{center}
%\begin{boxedminipage}{12.3cm}
%\small{
%$\mathtt{ZPVG}(\bar{k}_{\st 0}, \text{ID}, n,  m, j)\rightarrow \bm r_{\st j}$\\
%------------------
%\begin{itemize}
%\item \noindent\textit{Input.} $\bar{k}_{\st 0}$: a key of  pseudorandom function's key $\mathtt{PRF}(.)$, $\text{ID}$: a unique identifier, $n$:  total number of rows of a matrix, and $m$: total number of columns of a matrix, $j$: a row's index in a matrix.
%%
%\item \noindent\textit{Output.} $\bm r_{\st j}$:  $j$-th row of an $n\times m$ matrix, such that  if $i$-th element of $\bm r_{\st j}$ is added with  the rest of elements in $i$-th column of the same matrix, the result would be  $0$. 
%\end{itemize}
%\begin{enumerate}
%%
%\item\label{ZSPA:val-gen} compute $m$ pseudorandom values as follows. 
%
%$\forall i, 1\leq i\leq m:$
%%
%\begin{itemize}
%%
%\item[$\bullet$] if $j< n: r_{\st i,j}=\mathtt{PRF}(\bar k_{\st 0}, i||j||\text{ID})$ 
%%
%\item[$\bullet$]  if $j=n: r_{\st i, n}=\big(-\sum\limits^{\st n-1}_{\st j=1} r_{\st i,j}\big) \bmod p$ 
%%
%\end{itemize}
%%
%\item return $\bm r_{\st j}=[r_{\st 1,j},..., r_{\st m,j}]$
%
%
%
%
%\
% \end{enumerate}
% 
%}
%\end{boxedminipage}
%\end{center}
%\caption{Zero-sum Pseudorandom Values Generator (ZPVG)} 
%\label{fig:ZSPA}
%\end{figure}






\begin{figure}[ht!]
\setlength{\fboxsep}{0.7pt}
\begin{center}
\begin{boxedminipage}{12.3cm}
\small{
$\mathtt{PVE}(\bar{k}_{\st 0}, \text{ID}, w, n, \bm x, \textit{offset}, j)\rightarrow  \bar{\bm w}_{\st j}$\\
------------------
\begin{itemize}
\item \noindent\textit{Input.} $\bar{k}_{\st 0}$: a key of  pseudorandom function $\mathtt{PRF}(.)$, $\text{ID}$: a unique identifier, $w$: binary verdict, $n$: the total number of  arbiters,  $\bm x=[x_{\st 1},...,x_{\st 3}]$:  non-zero distinct public $x$-coordinates, $\textit{offset}$: an offset to avoid generating the same value when this algorithm is called  multiple times by the same party for the same $\text{ID}$,  $j$: an arbiter's index.
%
\item \noindent\textit{Output.} $\bar{\bm w}_{\st j}$: $y$-coordinates of $j$-th arbiter's masked polynomial that encodes $w$.  
%
%$\bm r_{\st j}$:  $j$-th row of an $n\times m$ matrix, such that  if $i$-th element of $\bm r_{\st j}$ is added with  the rest of elements in $i$-th column of the same matrix, the result would be  $0$. 
\end{itemize}
Arbiter $\mathcal{D}_{\st j}$ takes the following steps.
\begin{enumerate}
%
\item\label{ZSPA:val-gen} computes three pseudorandom coefficients of a degree-$2$ polynomial, $\Psi_{\st j}$, as follows. 

$\forall i, \textit{offset}\leq i\leq \textit{offset}+2:$
%
%\begin{itemize}
%
\begin{center}
$\bullet \text{ if } j< n: r_{\st i,j}=\mathtt{PRF}(\bar k_{\st 0}, i||j||\text{ID})$\\
%
\hspace{3.5mm} $\bullet  \text{ if } j=n: r_{\st i, n}=\big(-\sum\limits^{\st n-1}_{\st j=1} r_{\st i,j}\big) \bmod p$
\end{center}
%\end{itemize}
By the end of this phase, a random polynomial of the following form is generated, $\Psi_{\st j}=r_{\st i+2,j}\cdot x^{\st 2}+r_{\st i+1,j}\cdot x+r_{\st i,j}$.
%
\item generates   a polynomial that encodes the verdict, i.e., $\Omega_{\st j}=(x-w)$.  

\item multiplies the polynomial by a fresh random polynomial $\Phi_{\st j}$ of degree $1$ and adds the result with $\Psi_{\st j}$, i.e.,  $\bar\Omega_{\st j}=\Phi_{\st j}\cdot \Omega_{\st j}+\Psi_{\st j}\bmod p$. 


\item  evaluates the result  polynomial, $\bar\Omega_{\st j}$, at every  element $x_{\scriptscriptstyle i}\in {\bm{x}}$. This yields a vector of   $y$-coordinates: $[ \bar w_{\st 1,j},..., \bar w_{\st 3,j}]$.
%

%
\item return $\bar{\bm w}_{\st j}=[ \bar w_{\st 1,j},..., \bar w_{\st 3,j}]$.




\
 \end{enumerate}
 
}
\end{boxedminipage}
\end{center}
\caption{Private Verdict Encoding  (PVE) Protocol} 
\label{fig:PVE}
\end{figure}
%%%%%%%%%%%%%%%%%%%%%%%%%%%%%%%%%%%%%%

\begin{figure}[ht!]
\setlength{\fboxsep}{0.7pt}
\begin{center}
\begin{boxedminipage}{12.3cm}
\small{
$\mathtt{FVD}(n,   \bm x, \bar{\bm w})\rightarrow  v$\\
------------------
\begin{itemize}
\item \noindent\textit{Input.} $n$:  the total number of  arbiters,  $\bm x=[x_{\st 1},...,x_{\st 3}]$: the $x$-coordinates, $\bar{\bm w}=[\bar{\bm w}_{\st 1},..., \bar{\bm w}_{\st n}]$:  $y$-coordinates of all arbiters' masked polynomial that encodes their verdicts.
%
\item \noindent\textit{Output.} $v$: final verdict.  
%
%$\bm r_{\st j}$:  $j$-th row of an $n\times m$ matrix, such that  if $i$-th element of $\bm r_{\st j}$ is added with  the rest of elements in $i$-th column of the same matrix, the result would be  $0$. 
\end{itemize}
A third-party $\mathcal{I}$ takes the following steps.
\begin{enumerate}
%
\item sums all vectors of arbiters' masked polynomials', component-wise, as follows. 
%
 $$\forall i, 1\leq i\leq 3: g_{\st i}=\sum\limits^{\st n}_{\st j=1} \bar{w}_{\st i,j} \bmod p$$
%
\item uses pairs $(x_{\st i}, g_{\st i})$ to interpolate a polynomial, $\Theta$; e.g., using Lagrange interpolation. Note,  this polynomial has the  form: $\Theta=\sum\limits^{\st n}_{\st j=1}\Phi_{\st j}\cdot \Omega_{\st j}$, where $\Omega_{\st j}$'s root is $\mathcal{D}_{\st j}$'s verdict. 
%
\item evaluates $\Theta$ at $v=0$ and $v=1$. It returns $v$, if the result of the evaluation is $0$. 

\
 \end{enumerate}
 
}
\end{boxedminipage}
\end{center}
\caption{Final Verdict Decoding  (FVD) Protocol} 
\label{fig:FVD}
\end{figure}




%%\small{
%\begin{center}
%\fbox{\begin{minipage}{5 in}
%$\mathtt{checkMissedWar}(add_{\st\mathcal{S}},  m, T, l, \text{aux})\rightarrow g\in\{0,1\}$\\
%------------------
%\
%
%\noindent\textbf{Input}. $add_{\st\mathcal{S}}$: the address of smart contract $\mathcal{S}$, $m$:  $\mathcal{S}$'s field that corresponds to $\mathcal{B}$'s message, $T$: $\mathcal{S}$'s field that specifies the time when $\mathcal{B}$ must register the message, $l$: $\mathcal{S}$'s field related to customer's payees' list, and \text{aux}: auxiliary information. 
%
%\noindent\textbf{Output}. $g=1$: if the missing warning could play a role in preventing the scam; $g=0$: otherwise. 
%\begin{enumerate}
%%
%\item reads the bank's signed message $m$ sent to $\mathcal{S}$. If $m$ was not set (to anything) at time $T$, then it proceeds to the next step. Otherwise, it aborts. 
%%
%\item checks the validity of  customer's most recent payees' list $l$ on  $\mathcal{S}$, with the help of the auxiliary information $\text{aux}$. 
%\begin{itemize}
%\item[$\bullet$]  if $l$ contains an invalid element,  it sets $g=1$.
%\item [$\bullet$] otherwise, it sets $g=0$.
%\end{itemize}
%\item returns $g$.
%\end{enumerate}
%\end{minipage}
%}
%\end{center}
%%}
%

