% !TEX root =main.tex

\subsection{Subroutines for  Encoding-Decoding Verdicts}\label{sec::Encoding-Decoding-Verdicts}

In this section, we present verdict encoding and decoding protocols. They let a third party $\mathcal{I}$, e.g., $\mathcal{DR}$, learn if a threshold of the auditors voted $1$, while satisfying the following requirements.  The protocols  (1) generate unlinkable verdicts, (2) do not require auditors to interact with each other for each customer, and (3) are efficient. Since the second and third requirements are self-explanatory,  we only explain the first one.  Informally, the first property states that the protocols generate encoded verdicts and final verdict in a way that $\mathcal{I}$,  given these values, cannot (a)   link a  verdict to an auditor (except when all verdicts are $0$), and (b) learn the total number of $1$ or $0$ verdicts when they provide different verdicts.  Shortly, we present two variants of verdict encoding and decoding protocol. The first variant is highly efficient and suitable when the threshold is $1$. The second variant is generic and works for any threshold (but is less efficient). 



%In this section, we present verdict encoding and decoding protocols. They let a third party $\mathcal{I}$, e.g., $\mathcal{DR}$, find out whether threshold  arbiters voted $1$, while satisfying the following  requirements.  The protocols should (1) generate unlinkable verdicts, (2)  not require arbiters to interact with each other for each customer, and (3) be  efficient. Since the second and third requirements are self-explanatory,  we only explain the first one.  Informally, the first property requires  that the protocols should generate encoded verdicts and final verdict in a way that $\mathcal{I}$,  given the encoded verdicts and final verdict, should not be able to (a)   link a  verdict to an arbiter (except when all arbiters' verdicts are $0$), and (b) find out the total number of $1$ or $0$ verdicts when they provide different verdicts.  In this section, we present two variants of verdict encoding and decoding protocol. The first variant is highly efficient and suitable for the case where the threshold is $1$. The second variant is  generic and works for any threshold. The latter variant is slightly less efficient than the former one. These two variants might be of independent interest. Below, we explain each variant. 


\subsubsection{Variant 1:  Efficient Verdict  Encoding-Decoding Protocol.}


%Below, we present  efficient verdict encoding and decoding protocols; namely, Private Verdict Encoding (PVE) and Final Verdict Decoding (FVD). They  let $\mathcal{I}$ find out whether at least one arbiter voted $1$, while satisfying the above requirements.   


This variant has two protocols,  Private Verdict Encoding (PVE) and Final Verdict Decoding (FVD). They let $\mathcal{I}$ learn if at least one auditor voted $1$. This variant relies on our observation that if a set of random values and $0$s are XORed, then the result reveals nothing, e.g.,  about the number of non-zero and zero values. Below, we formally state it.  We refer readers to Appendix \ref{sec::Variant-1-Theorem-proof} for the proof.


% In Appendix \ref{sec::Variant-1-Theorem-proof}, we present the above observation's formal statement and its proof. 




%This variant  mainly relies on our observation that   if   a set of random values and $0$s are XORed, then the result   reveals nothing, e.g.,  about the number of non-zero and  zero values. Below, we formally state it.  We refer readers to Appendix \ref{sec::Variant-1-Theorem-proof} for the proof.

\begin{theorem}\label{set-xor}
Let set $S=\{s_{\st 1},..., s_{\st m}\}$ be the union of  two disjoint sets $S'$ and $S''$, where $S'$ contains non-zero random values pick uniformly  from a finite field $\mathbb{F}_{\st p}$, $S''$ contains zeros, $|S'|\geq c'=1$, $|S''|\geq c''=0$, and pair $(c',c'')$ is public information. Then, $r= \bigoplus\limits^{\st m}_{\st i=1} s_{\st i}$ reveals nothing beyond the public information.  
\end{theorem}

%\begin{proof}
%Let $s_{\st 1}$ and $s$, be two random values picked uniformly at random from $\mathbb{F}_{\st p}$. Let $\bar s=s_{\st 1}\oplus \underbrace{0\oplus... \oplus 0}_{\st |S''|}$. Since  $\bar s=s_{\st 1}$, two values $\bar s$ and $s$ have identical distribution. Thus, $\bar s$ reveals nothing in this case. Next, let $\tilde s=\underbrace{ s_{\st 1}\oplus s_{\st 2}\oplus... \oplus s_{\st j}}_{\st |S'|}$, where $s_{\st i}\in S'$. Since each $s_{\st i}$ is a uniformly random value,  the XOR of them is a uniformly random value too. That means values $\tilde s$ and $s$ have identical distribution. Thus, $\tilde s$ reveals nothing in this case as well. Also, it is not hard to see that the combination of the above two cases reveals nothing too, i.e., $\bar s\oplus \tilde s$ and $s$ have    identical distribution. 
%%
%\end{proof}


At a high level, PVE and FVD work as follows.  The auditors only once (for all future customers) agree on a secret key. This key will let each of them, in PVE,  generate a pseudorandom masking value such that if all masking values are XORed, they would cancel out each other.\footnote{This is similar to the idea used in the XOR-based secret sharing \cite{Schneier0078909}.} In PVE, each auditor encodes its verdict by (i) representing it as a parameter which is set to  $0$ if the verdict is $0$, or to a random value if the verdict is $1$, and then (ii) ``masking'' this parameter with the above pseudorandom value.  It sends the result to $\mathcal{I}$.  In FVD,  $\mathcal{I}$   XORs all encoded verdicts. This removes the masks and XORs all verdicts' representations.  If the result is $0$,   it concludes that all auditors voted $0$; so,  the final verdict is $0$. But, if the result is not $0$,  it knows that at least one of the auditors voted $1$, so the final verdict is $1$. Figures \ref{fig:PVE} and \ref{fig:FVD} present  PVE and FVD respectively.  



%At a high level, PVE and  FVD work as follows.  The arbiters only once for all customers agree on a secret key of a pseudorandom function. This key will let each of them, in PVE,  generate a pseudorandom masking value such that if all masking values are XORed, they would cancel out each other and result in $0$.\footnote{This is similar to the idea used in the XOR-based secret sharing \cite{Schneier0078909}.} In PVE, each arbiter encodes its verdict by (i) representing it as a parameter which is set to  $0$ if the verdict is $0$, or to a random value if the verdict is $1$, and then (ii) masking this parameter by the above  pseudorandom value.  It sends the result to $\mathcal{I}$.  In FVD, to decode the final verdict and find out whether any arbiter voted $1$, $\mathcal{I}$   XORs all encoded verdicts. This removes the masks and XORs are verdicts' representations.  If the result is $0$, then  it concludes that  all arbiters must have voted $0$; therefore,  the final verdict is $0$. However, if the result is not $0$ (i.e., a random value), then it knows that at least one of the arbiters voted $1$, so  the final verdict is $1$. %In summary, by requiring each arbiter to represent  its verdict $1$ with a distinct random value and add a masking layer to each verdict representation, the protocols can meet the above requirements. 


Intuitively, the primary reason this variant meets property 1 is that each masked verdict reveals nothing about the verdict (and its representation) and given the final verdict, $\mathcal{I}$ cannot distinguish between the case where there is exactly one auditor that voted  $1$ and the case where multiple auditors voted $1$, as in both cases $\mathcal{I}$   extracts only a single random value, which reveals nothing about the number of auditors which voted $0$ or $1$ (due to Theorem \ref{set-xor}). It is evident that this variant meets property 2 as the auditors interact with each other \emph{only once} (to agree on a key) for all customers. It also meets property 3 as it involves pseudorandom function invocations and XOR operations which are highly efficient operations. Note,  the protocols' correctness holds with an overwhelming probability, i.e., $1-\frac{1}{2^{\st \lambda}}$. Specifically, if two auditors represent their verdict by an identical random value, then when they are XORed they would cancel out each other which can affect the result's correctness. The same holds if the XOR of multiple verdicts' representations results in a value that can cancel out another verdict's representation. Nevertheless, the probability that such an event occurs is negligible in the security parameter $|p|=\lambda$, i.e., the probability is at most   $\frac{1}{2^{\st \lambda}}$. 




%%%%%%%%%%%%%%%%%%%%%%%%%%%%%%


% \vspace{-1.7mm}
\begin{figure}[!htb]
\setlength{\fboxsep}{1pt}
\begin{center}
    \begin{tcolorbox}[enhanced,width=5.5in, height=85mm,
    drop fuzzy shadow southwest,
    colframe=black,colback=white]
   %{ \small{
    %\vspace{-2.1mm}
\underline{$\mathtt{PVE}(\bar{k}_{\st 0}, \text{ID},  w_{\st j}, o, n,  j)\rightarrow  \bar{  w}_{\st j}$}\\
%
    %\vspace{-2.1mm}
\begin{itemize}
\item \noindent\textit{Input.} $\bar{k}_{\st 0}$: a key of  pseudorandom function $\mathtt{PRF}(.)$, $\text{ID}$: a unique identifier, $ w_{\st j}$: a  verdict, $o$: an offset, $n$: the total number of  auditors,  and  $j$: an auditor's index.
%
\item \noindent\textit{Output.} $\bar{  w}_{\st j}$:  an  encoded verdict.  
%
\end{itemize}
Auditor $\mathcal{D}_{\st j}$ takes the following steps.
    %\vspace{-1.2mm}
\begin{enumerate}
%
\item\label{ZSPA:val-gen} computes a  pseudorandom  value,  as follows. 
%
\begin{itemize}
%
\item[$\bullet$]$ \text{ if } j< n: r_{\st j}=\mathtt{PRF}(\bar k_{\st 0}, o||j||\text{ID})$.\\
%
    %\vspace{-1.6mm}
\item [$\bullet$] $ \text{ if } j=n: r_{\st j}= \bigoplus\limits^{\st n-1}_{\st i=1} r_{\st i}$.
%
\end{itemize}
%
\item  sets a fresh parameter, $w'_{\st j}$, as below. 
%
    %\vspace{-1.5mm}
\begin{equation*}
   w'_{\st j}= 
\begin{cases}
   0,              & \text{if } w_{\st j}=0\\
   \alpha_{\st j}\stackrel{\st\$}\leftarrow \mathbb{F}_{\st p} ,& \text{if } w_{\st j}=1\\

    %0,              & \text{if } w_{\st j}=0
\end{cases}
\end{equation*}
%

\item encodes  $w'_{\st j}$ as follows. %$\forall i,1\leq i\leq s:$
%
$\bar w_{\st j}= w'_{\st j}\oplus r_{\st j}$.
%
\item outputs $\bar{ w}_{\st j}$.
%
%\vspace{-1.4mm}
 \end{enumerate}
 
 %%%%%%%%%%%%%%%%%%%%%%%%%%%%%
 
 
 %%%%%%%%%%%%%%%%%%%%%%%%%%%%
%}}
\end{tcolorbox}
\end{center}
%\vspace{-1mm}
\caption{Private Verdict Encoding  (PVE) Protocol} 
\label{fig:PVE}
\end{figure}
%
%\vspace{-3mm} 

%%%%%%%%%%%%%%%%%%%%%%%%
 %\vspace{-2mm}
\begin{figure}[!htb]
\setlength{\fboxsep}{1pt}
\begin{center}
    \begin{tcolorbox}[enhanced,width=5.5in, height=67mm,
    drop fuzzy shadow southwest,
    colframe=black,colback=white]
   % \small{
   % \vspace{-1.3mm}
\underline{$\mathtt{FVD}(n,  \bar{\bm w})\rightarrow  v$}\\
%
%\vspace{-2.1mm}
\begin{itemize}
\item \noindent\textit{Input.} $n$:  the total number of  auditors,  and  $\bar{\bm w}=[\bar{ w}_{\st 1},..., \bar{ w}_{\st n}]$:  a vector of all auditors' encoded  verdicts.
%
\item \noindent\textit{Output.} $v$: final verdict.  
%
\end{itemize}
A third-party $\mathcal{I}$ takes the following steps.
    %\vspace{-1.3mm}
\begin{enumerate}
%
\item combines  all auditors' encoded verdicts, $\bar w_{\st j}\in \bar{\bm {w}}$, as follows. 
%
$c= \bigoplus\limits^{\st n}_{\st j=1} \bar w_{\st j}$
%
\item sets the final verdict $v$ depending on the content of $c$. Specifically, 
%
%\vspace{-1mm}
\begin{equation*}
   v= 
\begin{cases}
    0,              &\text{if } c= 0\\
   1 ,& \text{otherwise }\\
\end{cases}
\end{equation*}
%
\item outputs  $v$. 
%\vspace{-1.5mm}
 \end{enumerate}
%}
\end{tcolorbox}
\end{center}
%\vspace{-2mm}
\caption{Final Verdict Decoding  (FVD) Protocol} 
\label{fig:FVD}
\end{figure}



%%%%%%%%%%%%%%%%%%%%%%%%%%%%%%%%




 
 

 
 







\subsubsection{Variant 2: Generic Verdict  Encoding-Decoding Protocol.} This variant also includes two protocols, Generic Private Verdict Encoding (GPVE) and Generic Final Verdict Decoding (GFVD) which let $\mathcal{I}$ learn if at least $e$ auditors voted $1$, where $e$ is an integer in  $[1, n]$. It uses a novel combination of   Bloom filter and combinatorics. It relies on our observation that a Bloom filter encoding a set of random values reveals nothing about the set's elements.  We formally state it in Theorem \ref{set-bf} whose proof is given in Appendix \ref{sec::Variant-2-Theorem-proof}. 

%Appendix \ref{sec::Variant-2-Theorem-proof} presents the above observation's formal statement and proof.  

%
%Now, we present  an efficient \emph{generic} verdict  encoding-decoding protocol, denoted by GPVE and GFVD. They   let $\mathcal{I}$ find out whether at least $e$ arbiters voted $1$, where $e$ can be set to an integer in the range $[1, n]$. This variant is  built upon the previous one;  however, it also uses a novel combination of   Bloom filter and combinatorics. It  relies on our observation that a Bloom filter  encoding a set of random values reveals nothing about the set's elements, except with a negligible probability. We formally state it in Theorem \ref{set-bf} whose proof is given in Appendix \ref{sec::Variant-2-Theorem-proof}. 


% however, it also relies on our observation that a Bloom filter that encodes a set of random values reveals nothing about the set's elements, except with a negligible probability. Below, we formally state and prove it. 


\begin{theorem}\label{set-bf}
Let  $S=\{s_{\st 1},..., s_{\st m}\}$ be a set of random values picked uniformly from finite field $\mathbb{F}_{\st p}$, where the cardinality of $S$ is public information. Let $\mathtt{BF}$ be a Bloom filter encoding all elements of   $S$. Then,  $\mathtt{BF}$ reveals nothing about any element of $S$, beyond the public information, except with a negligible probability in the security parameter $\lambda$, i.e., with a probability at most $\frac{|S|}{2^{\st \lambda}}$. 
\end{theorem}

%\begin{proof}
%
%
%First, we consider the simplest case where only a single element  of $S$ is encoded in $\mathtt{BF}$. In this case, due to the pre-image resistance of the Bloom filter's hash functions and the fact that the set's element was picked uniformly at random from $\mathbb{F}_{\st p}$, the probability that $\mathtt{BF}$ reveals anything about the original element is at most $\frac{1}{2^{\st \lambda}}$. Now, we move on to the  case where all elements of $S$ are encoded in $\mathtt{BF}$. In this case, the probability that $\mathtt{BF}$ reveals anything about at least an element of the set is $\frac{|S|}{2^{\st \lambda}}$, due to  the pre-image resistance of the hash functions,  the fact that all elements were selected  uniformly at random from the finite field, and the union bound. Nevertheless, when a $\mathtt{BF}$'s size is set appropriately to avoid false-positive without wasting storage, this reveals the number of elements encoded in it, which is public information.  Thus, the only information $\mathtt{BF}$ reveals is the public one.  
% %
%\end{proof}

Now we present an overview of this variant. In this variant also, the auditors initially agree on a secret key used to generate a pseudorandom masking value. Each auditor $\mathcal{D}_{\st j}$ represents its verdict by a parameter, such that if its verdict is $0$, it sets the parameter to  $0$; but, if the verdict is $1$, it sets the parameter to a fresh \emph{pseudorandom} value $\alpha_{\st j}$  (instead of a random value used in Variant 1) where this pseudorandom value is also derived from the above key.  Thus, there would be a set $A=\{\alpha_{\st 1},..., \alpha_{\st n}\}$ from which   $\mathcal{D}_{\st j}$ would pick $\alpha_{\st j}$ to represent its verdict $1$. Each $\mathcal{D}_{\st j}$ masks its verdict representation by its masking  value. It sends the result to $\mathcal{I}$. Also, (only) auditor $\mathcal{D}_{\st n}$  generates a set $W$  of all  combinations of auditors' verdict $1$'s representations that satisfy the threshold, $e$.  Specifically,  for every integer $i$ in $[e, n]$, it computes the combinations (without repetition) of $i$ elements from  $A=\{\alpha_{\st 1},..., \alpha_{\st n}\}$. If  multiple elements are taken at a time (i.e., $i>1$), they are XORed with each other.   Note, $\mathcal{D}_{\st n}$ computes these values    regardless of what a specific  auditor  votes.   Let $W=\{(\alpha_{\st 1}\oplus ... \oplus \alpha_{\st e}),  (\alpha_{\st 2}\oplus  ... \oplus \alpha_{\st e+1}), ..., (\alpha_{\st 1}\oplus ... \oplus \alpha_{\st n})\}$ be the result. For instance, if the total number of arbiters, $n$, is $3$ and the threshold, $e$, is $2$,  then $W=\{(\alpha_{\st 1}\oplus \alpha_{\st 2}),  (\alpha_{\st 2}\oplus  \alpha_{\st 3}), (\alpha_{\st 1}\oplus \alpha_{\st 3}), (\alpha_{\st 1}\oplus \alpha_{\st 2} \oplus \alpha_{\st n})\}$. After that, it generates an empty Bloom filter and  inserts all elements of $W$ into this Bloom filter. Let $\mathtt{BF}$ be the Bloom filter that encodes $W$'s elements. It sends $\mathtt{BF}$ to $\mathcal{I}$. Note that inserting the combinations into $\mathtt{BF}$ ensures that the privacy of  each vote's representation (e.g., $\alpha_{\st j}$) is protected from $\mathcal{I}$. 



%In GFVD, to decode and extract the final verdict, as in Variant 1, party $\mathcal{I}$  XORs all masked verdict representations which  removes the masking values and XORs all verdicts’ representations. Let $c$ be the result.  If $c=0$, then $\mathcal{I}$ concludes that all arbiters must have voted $0$ (with a high probability); so, it sets the final verdict to $0$. However, if $c$ is a non-zero value, then it checks whether $c$ is in the Bloom filter. If it is, then it concludes that at least threshold arbiters voted $1$, so it sets the final vector to $1$. Otherwise (if $c$ is not in the Bloom filter), it concludes that less than threshold arbiters voted $1$; therefore, it sets the final verdict to $0$.  


In GFVD, to decode and extract the final verdict,  $\mathcal{I}$  XORs all masked verdict representations which removes the masking values and XORs the representations. Let $c$ be the result.  If $c=0$, then $\mathcal{I}$ concludes that all auditors voted $0$; so, it sets the final verdict to $0$. If $c\neq 0$, then it checks if $c\in \mathtt{BF}$. If it is, then it concludes that at least the threshold of the auditors voted $1$, so it sets the final verdict to $1$. Otherwise ($c\notin \mathtt{BF}$), it learns that less than the threshold of the auditors voted $1$; so, it sets the final verdict to $0$. Figures \ref{fig:GPVE} and \ref{fig:GFVD} present the  GPVE and GFVD protocols in detail. 
%
% !TEX root =main.tex

\section{Generic Verdict Encoding-Decoding Protocols}


Here, we present two efficient verdict encoding and decoding protocols; namely, Private Verdict Encoding (PVE) and Final Verdict Decoding (FVD) protocols. Their goal is to let a third party $\mathcal{I}$, e.g., $\mathcal{DR}$, find out whether at least one arbiter voted $1$, while satisfying the following  requirements.  The protocols should (1) generate unlinkable verdicts, (2)  not require arbiters to interact with each other for each customer, and (3) be  efficient. Since, the second and third requirements are self-explanatory,  we only explain the first one.  Informally, the first property requires  that the protocols should generate encoded verdicts and final verdict in a way that $\mathcal{I}$,  given the encoded verdicts and final verdict, should not be able to (a)   link a  verdict to an arbiter (except when all arbiters' verdicts are $0$), and (b) find out the total number of $1$ or $0$ verdicts when they provide different verdicts. 



 At a high level, the protocols work as follows.  The arbiters only once for all customers agree on a secret key of a pseudorandom function. This key will allow each of them to generate a pseudorandom masking values such that if all masking values are ``XOR''ed, they would cancel out each other and result $0$.\footnote{This is similar to the idea used in the XOR-based secret sharing \cite{Schneier0078909}.}
 
 
 
 
 
Each arbiter represents its verdict by (i) representing it as a parameter which is set to either $0$ if the verdict is $0$ or to a random value if the verdict is $1$, and then (ii) masking this parameter by the above  pseudorandom value.  It sends the result to $\mathcal{I}$.  To decode the final verdict and find out whether any arbiter voted $1$, $\mathcal{I}$  does XOR all encoded verdicts. This removes the masks and XORs are verdicts' representations.  If the result is $0$, then    all arbiters must have voted $0$; therefore,  the final verdict is $0$. However, if the result is not $0$ (i.e., a random value), then at least one of the arbiters voted $1$, so  the final verdict is $1$. We present the encoding  and decoding protocols in figures \ref{fig:PVE} and \ref{fig:FVD} respectively.
 
 
 Not that the protocols' correctness holds, except  a negligible  probability. In particular, if two arbiters  represent their verdict by an identical random value, then when they are XORed they would cannel out each other which can affect the result's correctness. The same holds if the XOR of  multiple verdicts' representations results in a value that can cancel out another verdict's representation. Nevertheless, the probability that such an event occurs is negligible in the security parameter, i.e., at most   $\frac{1}{2^{\st \lambda}}$. It is evident that PVE and FVD protocols meet properties (2) and (3). The primary reason they also meet  property (1) is that each masked verdict reveals nothing about the verdict (and its representation) and  given the final verdict, $\mathcal{I}$ cannot distinguish between the case where there is exactly one arbiter that voted  $1$ and the case where multiple arbiters voted $1$, as in both cases $\mathcal{I}$   extracts only a single random value, which reveals nothing about the number of arbiters which voted $0$ or $1$. 
 
%  
% To encode a verdict $w$, each arbiter represents it as a polynomial. It randomises this polynomial and then  masks this polynomial with the pseudorandom masking polynomial. It sends the result to $\mathcal{I}$. To decode the final verdict and find out whether all arbiters agreed on the same verdict, i.e., unanimous decision,  $\mathcal{I}$  adds all polynomials up. This removes the masks. Next, it  evaluates the result polynomial at $v=1$ and $v=0$. It considers $v$ as the final verdict if the evaluation is  $0$. We present the encoding  and decoding protocols in figures \ref{fig:PVE} and \ref{fig:FVD} respectively. 
 



% either an specific final verdict  (i.e., $v=0$ or $v=1$) if  all arbiters' verdicts are identical, or  nothing  about the arbiters' inputs if they did not agree on the same specific verdict. The protocols are  primarily  based on ``zero-sum pseudorandom polynomials'' and the techniques often used by private set intersection (PSI) protocols. In particular, the arbiters only once for all customers agree on a secret key of a pseudorandom function. This key will allow each of them to generate a pseudorandom masking polynomial such that if all masking polynomials are summed up, they would cancel out each other and result $0$, i.e., zero-sum pseudorandom polynomials. 



%In this section, we present efficient (verdict) encoding and decoding protocols. The encoding protocol  lets each of the $n$ honest arbiters $\mathcal{D}:\{\mathcal{D}_{\st 1},..., \mathcal{D}_{\st n}\}$ non-interactively encode its verdict such that a third party party $\mathcal{I}$ (where $\mathcal{I}\notin \mathcal{D}$) can extract final verdict if  all arbiters' verdicts are identical with the following security requirements. First,  given individual encoded verdict, $\mathcal{I}$ cannot learn anything about each arbiter's verdict. Second, can find out only final verdict if  all arbiters' verdicts are identical; otherwise, it cannot learn each individual arbiter's verdict.  The decoding protocol lets  $\mathcal{I}$,  combines the encoded verdicts and learn either an specific final verdict  (i.e., $w=0$ or $w=1$) if  all arbiters' verdicts are identical, or  nothing  about the arbiters' inputs if they did not agree on the same specific verdict. The protocols are  primarily  based on ``zero-sum pseudorandom polynomials'' and the techniques often used by private set intersection (PSI) protocols. In particular, the arbiters only once for all customers agree on a secret key of a pseudorandom function. This key will allow each of them to generate a pseudorandom masking polynomial such that if all masking polynomials are summed up, they would cancel out each other and result $0$, i.e., zero-sum pseudorandom polynomials. 











%
%Below, we present ``Zero-sum Pseudorandom Values Generator'' (ZPVG), an algorithm that allows each of the $n$ arbiters  to \emph{efficiently} and \emph{independently}  generate a vector of $m$ pseudorandom values for each customer, such that when all arbiters' vectors are summed up component-wise, it would result in a vector of $m$ zeros. ZPVG is based on  the following idea. Each arbiter $\mathcal{D}_{\st j}\in\{\mathcal{D}_{\st 1},..., \mathcal{D}_{\st n-1}\}$  uses the secret key $\bar k_{\st 0}$ (as defined in Section \ref{Notations-and-Assumptions}), along with the customer's unique ID (e.g., it blockchain account's address) as the inputs of $\mathtt{PRF}(.)$  to derive $m$ pseudorandom values. However,  $\mathcal{D}_{\st n}$ generates each  $j$-th element of  its vector by computing the additive inverse of the sum of the $j$-th elements that the rest of arbiters generated. Even $\mathcal{D}_{\st n}$ does not need to interact with other arbiters, as it can regenerate their values too. Moreover, the arbiters do not need to interact with each other for every   new customer and can   reuse the same key because the new customer would have a new unique ID that would result in a fresh set of pseudorandom values (with a high probability). Figure \ref{fig:ZSPA} presents ZPVG in more detail. 





%It sends the result to all parties which can locally check if the relation holds. Note, in the literature,   there exist protocols that allows parties to agree on zero-sum pseudorandom values, e.g., in \cite{}; however, they are less efficient than $\mathtt{ZSPA}$, as their security requirements are different than ours, i.e., they assume the participants are malicious whereas we assume the arbiters who participate in this protocol  are honest. 


%Next, it commits to each value, where it uses $k_{\st 2}$ to generate the randomness of each commitment. Then, it constructs a Merkel tree on top of the commitments and  stores only the root of the tree  and the hash value of the keys (so in total three values) in  the smart contract.  Then, each party (using the keys) locally checks if the values and commitments have been constructed correctly; if so, each sends  an ``approved" message to the contract. 
%
%
%
%Informally, there are three main security requirements that $\mathtt{ZSPA}$ must meet: (a) privacy, (b)  non-refutability, and (c) indistinguishability. Privacy here means given the state of the  contract, an external party cannot learn any information about any of the (pseudorandom) values:  $z_{\st j}$; while non-refutability  means that if a party sends ``approved" then in future cannot deny the knowledge  of the values whose representation is stored in the contract. Furthermore, indistinguishability means that every $z_{\st j}$ ($1\leq j \leq m$) should be indistinguishable from a truly random value. In Fig. \ref{fig:ZSPA}, we provide $\mathtt{ZSPA}$ that efficiently generates $b$ vectors  where each vector elements is sum to zero. 



%\begin{figure}[ht]
%\setlength{\fboxsep}{0.7pt}
%\begin{center}
%\begin{boxedminipage}{12.3cm}
%\small{
%$\mathtt{ZPVG}(\bar{k}_{\st 0}, \text{ID}, n,  m, j)\rightarrow \bm r_{\st j}$\\
%------------------
%\begin{itemize}
%\item \noindent\textit{Input.} $\bar{k}_{\st 0}$: a key of  pseudorandom function's key $\mathtt{PRF}(.)$, $\text{ID}$: a unique identifier, $n$:  total number of rows of a matrix, and $m$: total number of columns of a matrix, $j$: a row's index in a matrix.
%%
%\item \noindent\textit{Output.} $\bm r_{\st j}$:  $j$-th row of an $n\times m$ matrix, such that  if $i$-th element of $\bm r_{\st j}$ is added with  the rest of elements in $i$-th column of the same matrix, the result would be  $0$. 
%\end{itemize}
%\begin{enumerate}
%%
%\item\label{ZSPA:val-gen} compute $m$ pseudorandom values as follows. 
%
%$\forall i, 1\leq i\leq m:$
%%
%\begin{itemize}
%%
%\item[$\bullet$] if $j< n: r_{\st i,j}=\mathtt{PRF}(\bar k_{\st 0}, i||j||\text{ID})$ 
%%
%\item[$\bullet$]  if $j=n: r_{\st i, n}=\big(-\sum\limits^{\st n-1}_{\st j=1} r_{\st i,j}\big) \bmod p$ 
%%
%\end{itemize}
%%
%\item return $\bm r_{\st j}=[r_{\st 1,j},..., r_{\st m,j}]$
%
%
%
%
%\
% \end{enumerate}
% 
%}
%\end{boxedminipage}
%\end{center}
%\caption{Zero-sum Pseudorandom Values Generator (ZPVG)} 
%\label{fig:ZSPA}
%\end{figure}






\begin{figure}[!ht]
\setlength{\fboxsep}{0.7pt}
\begin{center}
\begin{boxedminipage}{12.3cm}
\small{
\underline{$\mathtt{GPVE}(\bar{k}_{\st 0}, \text{ID},  w_{\st j}, o, e, n,  j)\rightarrow  (\bar{  w}_{\st j}, \mathtt{BF})$}\\
%
\begin{itemize}
\item \noindent\textit{Input.} $\bar{k}_{\st 0}$: a key of  pseudorandom function $\mathtt{PRF}(.)$, $\text{ID}$: a unique identifier, $ w_{\st j}$: a  verdict, $o$: an offset, $e$: a threshold, $n$: the total number of  arbiters,  and  $j$: an arbiter's index.
%
\item \noindent\textit{Output.} $\bar{  w}_{\st j}$:  an  encoded verdict.  
%
%$\bm r_{\st j}$:  $j$-th row of an $n\times m$ matrix, such that  if $i$-th element of $\bm r_{\st j}$ is added with  the rest of elements in $i$-th column of the same matrix, the result would be  $0$. 
\end{itemize}
Arbiter $\mathcal{D}_{\st j}$ takes the following steps.
\begin{enumerate}
%
\item\label{ZSPA:val-gen} computes a  pseudorandom  value,  as follows. 
%
%$\forall i,1\leq i\leq s:$
%
\begin{itemize}
%
\item[$\bullet$]$ \text{ if } j< n: r_{\st j}=\mathtt{PRF}(\bar k_{\st 0}, 1||o||j||\text{ID})$.\\
%
%\hspace{1mm} 
\item [$\bullet$] $ \text{ if } j=n: r_{\st j}= \bigoplus\limits^{\st n-1}_{\st i=1} r_{\st i}$.
%
\end{itemize}
Note, the above second step is taken only by $\mathcal{D}_{\st n}$.
%By the end of this phase, a random polynomial of the following form is generated, $\Psi_{\st j}=r_{\st i+2,j}\cdot x^{\st 2}+r_{\st i+1,j}\cdot x+r_{\st i,j}$.
%
\item  sets a fresh parameter, $w'_{\st j}$, that represents a verdict, as below. 
%
%$\forall i,1\leq i\leq s:$

%\begin{itemize}
%\item[$\bullet$]  $\text{ if } w_{\st j}=1:$ \text{\ sets \ } $w'_{\st j}= \alpha_{\st j}$, where $\alpha_{\st j}\stackrel{\st\$}\leftarrow \mathbb{F}_{\st p}$.
%
%\item [$\bullet$] $\text{ if } w_{\st j}=0: \text{\ sets \ } w'_{\st j}= 0$.
%
%\end{itemize}
\begin{equation*}
   w'_{\st j}= 
\begin{cases}
   0,              & \text{if } w_{\st j}=0\\
   \alpha_{\st j}=\mathtt{PRF}(\bar k_{\st 0}, 2||o||j||\text{ID}) ,& \text{if } w_{\st j}=1\\

    %0,              & \text{if } w_{\st j}=0
\end{cases}
\end{equation*}
%
\item masks  $w'_{\st j}$ as follows. %$\forall i,1\leq i\leq s:$
%
$\bar w_{\st j}= w'_{\st j}\oplus r_{\st j}$.
%
\item if $j=n$, computes a Bloom filter that encodes the combinations of verdict representations (i.e., $w'_{\st j}$)  for verdict $1$. In particular, it takes the following steps. 
\begin{itemize}
%
\item[$\bullet$] for every integer $i$ in the range $[e,n]$, computes the combinations (without repetition) of $i$ elements from set $\{\alpha_{\st 1},..., \alpha_{\st n}\}$, where the combination operation is XOR. Let $W=\{(\alpha_{\st 1}\oplus \alpha_{\st 2}\oplus... \oplus \alpha_{\st e}), (\alpha_{\st 2}\oplus \alpha_{\st 3}\oplus ... \oplus \alpha_{\st e+1}), ..., (\alpha_{\st 1}\oplus \alpha_{\st 2}\oplus... \oplus \alpha_{\st n})\}$ be the result.  
%
\item[$\bullet$] constructs an empty Bloom filter. Then, it inserts all elements of $W$ into this Bloom filter. Let $\mathtt{BF}$ be the Bloom filter encoding $W$'s elements. 

\end{itemize}
%

%
\item outputs ($\bar{ w}_{\st j}, \mathtt{BF})$.





%generates   a polynomial that encodes the verdict, i.e., $\Omega_{\st j}=(x-w)$.  
%
%\item multiplies the polynomial by a fresh random polynomial $\Phi_{\st j}$ of degree $1$ and adds the result with $\Psi_{\st j}$, i.e.,  $\bar\Omega_{\st j}=\Phi_{\st j}\cdot \Omega_{\st j}+\Psi_{\st j}\bmod p$. 
%
%
%\item  evaluates the result  polynomial, $\bar\Omega_{\st j}$, at every  element $x_{\scriptscriptstyle i}\in {\bm{x}}$. This yields a vector of   $y$-coordinates: $[ \bar w_{\st 1,j},..., \bar w_{\st 3,j}]$.
%%
%
%%
%\item return $\bar{\bm w}_{\st j}=[ \bar w_{\st 1,j},..., \bar w_{\st 3,j}]$.




\
 \end{enumerate}
 
}
\end{boxedminipage}
\end{center}
\caption{Generic Private Verdict Encoding  (GPVE) Protocol} 
\label{fig:PVE}
\end{figure}
%%%%%%%%%%%%%%%%%%%%%%%%%%%%%%%%%%%%%%
%
\begin{figure}[!ht]
\setlength{\fboxsep}{0.7pt}
\begin{center}
\begin{boxedminipage}{12.3cm}
\small{
\underline{$\mathtt{GFVD}(n,  \bar{\bm w}, \mathtt{BF})\rightarrow  v$}\\
%
\begin{itemize}
\item \noindent\textit{Input.} $n$:  the total number of  arbiters,  and  $\bar{\bm w}=[\bar{ w}_{\st 1},..., \bar{ w}_{\st n}]$:  a vector of all arbiters' encodes  verdicts.
%
\item \noindent\textit{Output.} $v$: final verdict.  
%
%$\bm r_{\st j}$:  $j$-th row of an $n\times m$ matrix, such that  if $i$-th element of $\bm r_{\st j}$ is added with  the rest of elements in $i$-th column of the same matrix, the result would be  $0$. 
\end{itemize}
A third-party $\mathcal{I}$ takes the following steps.
\begin{enumerate}
%
%
\item combines  all arbiters' encoded verdicts, $\bar w_{\st j}\in \bar{\bm {w}}$, as follows. 
%
$c= \bigoplus\limits^{\st n}_{\st j=1} \bar w_{\st j}$
% $$\forall i, 1\leq i\leq 3: g_{\st i}=\sum\limits^{\st n}_{\st j=1} \bar{w}_{\st i,j} \bmod p$$
%
%\item if $n$ is odd, then sets $c=c\oplus 1$. 
%
\item checks if $c$ is in the Bloom filter, $\mathtt{BF}$. 
%
\item sets the final verdict $v$ depending on the content of $c$. Specifically, 
%
%\begin{itemize}
%%
%\item[$\bullet$] if $c=0$, sets $v=0$.
%%
%\item[$\bullet$]  otherwise, sets $v=1$.
%%
%\end{itemize}
\begin{equation*}
   v= 
\begin{cases}
    0,              &\text{if } c= 0 \text{ or } c \notin\mathtt{BF}\\
   1 ,& \text{if } c \in\mathtt{BF}\\

\end{cases}
\end{equation*}
%
\item outputs  $v$. 

\
 \end{enumerate}
 
}
\end{boxedminipage}
\end{center}
\caption{Generic Final Verdict Decoding  (GFVD) Protocol} 
\label{fig:FVD}
\end{figure}



Note also that the total number of the combinations (i.e., the cardinality of $W$) is relatively small when the number of arbiters is not very high. In general,  due to the  binomial theorem, the cardinality of $W$ is determined as follows:
%

\begin{equation}\label{w-cardinality}
|W|=\sum\limits_{\st i=e}^{\st n}\frac{n!}{i!(n- i)!}
 \end{equation}
 
 %
 For instance, when $n=10$ and  $e=6$, then $W$'s cardinality is only $386$, according to Equation \ref{w-cardinality}.  In the above scheme, instead of inserting the combinations into $\mathtt{BF}$ we could simply hash  the combinations and give the hash values to $\mathcal{I}$. However, using a Bloom filter allows us to save considerable communication costs. For instance, when $n=10$, $e=6$, and SHA-256 is used, then $\mathcal{D}_{\st n}$ needs to send  $98816=386\times 256$ bits to $\mathcal{I}$, whereas  if they are inserted into a Bloom filter, then it only needs to send $22276 $ bits to $\mathcal{I}$. Thus, by using a Bloom filter,  it  can save  communication costs by at least a factor of  $4$. We refer readers to Appendix \ref{sec:: Further-Discussion-on-the-Encoding-decoding-Protocol} for  further discussion on the verdict encoding-decoding protocols. 
 
 
 
 

 
 




%Arbiter $\mathcal{D}_{\st n}$ also generates a Bloom filter that contains the combinations of set $A$'s elements, regardless of whether a certain arbiter's  vote is $0$ or $1$. More specifically,  for every integer $i$ in the range $[e,n]$, computes the combinations, without repetition, of $i$ elements from set $A=\{\alpha_{\st 1},..., \alpha_{\st n}\}$, where when multiple elements are taken at a time (i.e., $i>1$), the elements are XORed with each other. Let $W=\{(\alpha_{\st 1}\oplus ... \oplus \alpha_{\st e}),  (\alpha_{\st 2}\oplus  ... \oplus \alpha_{\st e+1}), ..., (\alpha_{\st 1}\oplus ... \oplus \alpha_{\st n})\}$ be the result. For instance, when $n=3$ and $e=2$,  we would have $W=\{(\alpha_{\st 1}\oplus \alpha_{\st 2}),  (\alpha_{\st 2}\oplus  \alpha_{\st 3}), (\alpha_{\st 1}\oplus \alpha_{\st 3}), (\alpha_{\st 1}\oplus \alpha_{\st 2} \oplus \alpha_{\st n})\}$. After that, it generates an empty Bloom filter and  inserts all elements of $W$ into this Bloom filter. Let $\mathtt{BF}$ be the Bloom filter that encodes $W$'s elements. It sends $\mathtt{BF}$ to $\mathcal{I}$. To decode and extract the final verdict, as in Variant 1, $\mathcal{I}$ does XOR all masked verdict representations which  removes the masking values and XORs are verdicts’ representations. If the result is $0$, then $\mathcal{I}$ concludes that all arbiters must have voted $0$ (with a high probability); so, it sets the final verdict to $0$. However, if the result is a non-zero value, then it checks whether the value is in the Bloom filter. If it is, then it concludes that at least threshold arbiters voted $1$, so it sets the final vector to $1$. Otherwise (if the value is not in the Bloom filter), it concludes that less than threshold arbiters voted $1$; therefore, it sets the final verdict to $0$.  Figures \ref{fig:GPVE} and \ref{fig:GFVD}, in Appendix \ref{sec::Generic-Verdict-Encoding-Decoding-Protocols}, present the  GPVE and GFVD protocols in detail. 



%\item[$\bullet$] for every integer $i$ in the range $[e,n]$, computes the combinations (without repetition) of $i$ elements from set $\{\alpha_{\st 1},..., \alpha_{\st n}\}$, where the combination operation is XOR. Let $W=\{(\alpha_{\st 1}\oplus \alpha_{\st 2}\oplus... \oplus \alpha_{\st e}), (\alpha_{\st 2}\oplus \alpha_{\st 3}\oplus ... \oplus \alpha_{\st e+1}), ..., (\alpha_{\st 1}\oplus \alpha_{\st 2}\oplus... \oplus \alpha_{\st n})\}$ be the result.  
%

%\item[$\bullet$] constructs an empty Bloom filter. Then, it inserts all elements of $W$ into this Bloom filter. Let $\mathtt{BF}$ be the Bloom filter encoding $W$'s elements. 
 
  
  
 
 
 
 



 %and then (ii) masking this parameter by the above pseudorandom value. It sends the result to I. 

%Now, we explain how the GPVE and GFVD work. 





%In Appendix \ref{sec::bloom-filter-}, we explain how the Bloom filter parameters can be set.  




%
%Here, we present two efficient verdict encoding and decoding protocols; namely, Private Verdict Encoding (PVE) and Final Verdict Decoding (FVD) protocols. Their goal is to let a third party $\mathcal{I}$, e.g., $\mathcal{DR}$, find out whether at least one arbiter voted $1$, while satisfying the following  requirements.  The protocols should (1) generate unlinkable verdicts, (2)  not require arbiters to interact with each other for each customer, and (3) be  efficient. Since, the second and third requirements are self-explanatory,  we only explain the first one.  Informally, the first property requires  that the protocols should generate encoded verdicts and final verdict in a way that $\mathcal{I}$,  given the encoded verdicts and final verdict, should not be able to (a)   link a  verdict to an arbiter (except when all arbiters' verdicts are $0$), and (b) find out the total number of $1$ or $0$ verdicts when they provide different verdicts. 
%
%
%
% At a high level, the protocols work as follows.  The arbiters only once for all customers agree on a secret key of a pseudorandom function. This key will allow each of them to generate a pseudorandom masking values such that if all masking values are ``XOR''ed, they would cancel out each other and result $0$.\footnote{This is similar to the idea used in the XOR-based secret sharing \cite{Schneier0078909}.}
% 
% 
% 
% 
% 
%Each arbiter represents its verdict by (i) representing it as a parameter which is set to either $0$ if the verdict is $0$ or to a random value if the verdict is $1$, and then (ii) masking this parameter by the above  pseudorandom value.  It sends the result to $\mathcal{I}$.  To decode the final verdict and find out whether any arbiter voted $1$, $\mathcal{I}$  does XOR all encoded verdicts. This removes the masks and XORs are verdicts' representations.  If the result is $0$, then    all arbiters must have voted $0$; therefore,  the final verdict is $0$. However, if the result is not $0$ (i.e., a random value), then at least one of the arbiters voted $1$, so  the final verdict is $1$. We present the encoding  and decoding protocols in figures \ref{fig:PVE} and \ref{fig:FVD} respectively.
% 
% 
% Not that the protocols' correctness holds, except  a negligible  probability. In particular, if two arbiters  represent their verdict by an identical random value, then when they are XORed they would cannel out each other which can affect the result's correctness. The same holds if the XOR of  multiple verdicts' representations results in a value that can cancel out another verdict's representation. Nevertheless, the probability that such an event occurs is negligible in the security parameter, i.e., at most   $\frac{1}{2^{\st \lambda}}$. It is evident that PVE and FVD protocols meet properties (2) and (3). The primary reason they also meet  property (1) is that each masked verdict reveals nothing about the verdict (and its representation) and  given the final verdict, $\mathcal{I}$ cannot distinguish between the case where there is exactly one arbiter that voted  $1$ and the case where multiple arbiters voted $1$, as in both cases $\mathcal{I}$   extracts only a single random value, which reveals nothing about the number of arbiters which voted $0$ or $1$. 
% 
%



%
%\begin{figure}[!ht]
%\setlength{\fboxsep}{0.7pt}
%\begin{center}
%\begin{boxedminipage}{12cm}
%\small{
%\underline{$\mathtt{PVE}(\bar{k}_{\st 0}, \text{ID},  w_{\st j}, o, n,  j)\rightarrow  \bar{  w}_{\st j}$}\\
%%
%\begin{itemize}
%\item \noindent\textit{Input.} $\bar{k}_{\st 0}$: a key of  pseudorandom function $\mathtt{PRF}(.)$, $\text{ID}$: a unique identifier, $ w_{\st j}$: a  verdict, $o$: an offset, $n$: the total number of  arbiters,  and  $j$: an arbiter's index.
%%
%\item \noindent\textit{Output.} $\bar{  w}_{\st j}$:  an  encoded verdict.  
%%
%\end{itemize}
%Arbiter $\mathcal{D}_{\st j}$ takes the following steps.
%\begin{enumerate}
%%
%\item\label{ZSPA:val-gen} computes a  pseudorandom  value,  as follows. 
%%
%\begin{itemize}
%%
%\item[$\bullet$]$ \text{ if } j< n: r_{\st j}=\mathtt{PRF}(\bar k_{\st 0}, o||j||\text{ID})$.\\
%%
%\item [$\bullet$] $ \text{ if } j=n: r_{\st j}= \bigoplus\limits^{\st n-1}_{\st i=1} r_{\st i}$.
%%
%\end{itemize}
%%
%\item  sets a fresh parameter, $w'_{\st j}$, as below. 
%%
%\begin{equation*}
%   w'_{\st j}= 
%\begin{cases}
%   0,              & \text{if } w_{\st j}=0\\
%   \alpha_{\st j}\stackrel{\st\$}\leftarrow \mathbb{F}_{\st p} ,& \text{if } w_{\st j}=1\\
%
%    %0,              & \text{if } w_{\st j}=0
%\end{cases}
%\end{equation*}
%%
%\item encodes  $w'_{\st j}$ as follows. %$\forall i,1\leq i\leq s:$
%%
%$\bar w_{\st j}= w'_{\st j}\oplus r_{\st j}$.
%%
%\item outputs $\bar{ w}_{\st j}$.
%
%
%\
% \end{enumerate}
%}
%\end{boxedminipage}
%\end{center}
%\caption{Private Verdict Encoding  (PVE) Protocol} 
%\label{fig:PVE}
%\end{figure}
%%
%%\vspace{-3mm} 
%\begin{figure}[!ht]
%\setlength{\fboxsep}{0.7pt}
%\begin{center}
%\begin{boxedminipage}{12cm}
%\small{
%\underline{$\mathtt{FVD}(n,  \bar{\bm w})\rightarrow  v$}\\
%%
%\begin{itemize}
%\item \noindent\textit{Input.} $n$:  the total number of  arbiters,  and  $\bar{\bm w}=[\bar{ w}_{\st 1},..., \bar{ w}_{\st n}]$:  a vector of all arbiters' encodes  verdicts.
%%
%\item \noindent\textit{Output.} $v$: final verdict.  
%%
%\end{itemize}
%A third-party $\mathcal{I}$ takes the following steps.
%\begin{enumerate}
%%
%\item combines  all arbiters' encoded verdicts, $\bar w_{\st j}\in \bar{\bm {w}}$, as follows. 
%%
%$c= \bigoplus\limits^{\st n}_{\st j=1} \bar w_{\st j}$
%%
%\item sets the final verdict $v$ depending on the content of $c$. Specifically, 
%%
%\begin{equation*}
%   v= 
%\begin{cases}
%    0,              &\text{if } c= 0\\
%   1 ,& \text{otherwise }\\
%
%\end{cases}
%\end{equation*}
%%
%\item outputs  $v$. 
%
%\
% \end{enumerate}
% 
%}
%\end{boxedminipage}
%\end{center}
%\caption{Final Verdict Decoding  (FVD) Protocol} 
%\label{fig:FVD}
%\end{figure}

